\input{../article_base.tex}
\title{פתרון מטלה 06 --- אנליזה על יריעות, 80426}

\DeclareMathOperator{\vol}{vol}

\begin{document}
\maketitle
\maketitleprint[blue]

\question{}
תהי $U \subseteq \RR^k$ פתוחה ותהי $\varphi : U \to \RR^n$ כך ש־$(\varphi(U), \varphi)$ היא יריעה פרמטרית רגולרית $k$־מימדית.
נראה שלכל $x_0 \in U$ קיימת סביבה $x_0 \in U' \subseteq U$ כך ש־$\varphi(U')$ היא יריעה.
\begin{proof}
	יהי $x_0 \in U$ ונניח ש־$U'$ סביבה פתוחה $U' \subseteq U$ בה $\varphi \mid_{U'}$ היא רגולרית אף היא (נבחין כי אנו יכולים לבחור $U' = U$ בחלק נרחב מהמקרים), וכן חד־חד ערכית, אנו יודעים שקיימת כזו מהגדרת הרגולריות.
	כדי להראות ש־$\varphi(U')$ היא יריעה, עלינו להראות שלכל בחירת $x \in \varphi(U')$ נוכל לבחור את $U'$ כסביבה פתוחה בה יש פרמטריזציה מקומית.
	תהי $x \in \varphi(U')$ כזו, ונבחר את $\varphi_0 = \varphi \mid_{U'}$, נראה כי היא אכן פרמטריזציה.
	$x \in \varphi_0(U')$ ישירות מבחירת הסביבה, ולכן עלינו רק להראות ש־$\varphi_0$ היא חד־חד ערכית, על ופתוחה, נבחין כי היא חלקה כצמצום של פונקציה חלקה, ורגולרית מאותה הסיבה.
	הגדרנו את $U'$ כך שיתקיים $U_0$ חד־חד ערכית, והגדרנו את $\varphi(U')$ כצמצום של פונקציה חד־חד ערכית ולכן היא על, ונוכל להסיק ש־$\varphi_0$ היא הפיכה.
	כדי להראות שהיא גם דיפאומורפיזם, כלומר שהפיכתה גזירה (וחלקה) נרצה להשתמש במשפט הפונקציה ההפוכה, ולשם כך נשתמש בשיטה מההרצאה.
	קיימת העתקה לינארית אורתוגונלית כך שנקבל $T \varphi(U_0) \subseteq \RR^k$ במובן $\RR^k = \{ u \in \RR^n \mid \forall i > k, u_i = 0 \}$, ונוכל להשתמש במשפט הפונקציה ההפוכה על ההרכבה הזו.
	העתקות לינאריות רגולריות הן דיפאומורפיזם ולכן נוכל להסיק שגם ההרכבה דיפאומורפיזם ולכן גם $\varphi_0$ עצמה.
	נבחין כי $\varphi_0^{-1}$ גזירה ולכן רציפה ובפרט נובע ש־$\varphi_0$ הומיאומורפיזם כפי שרצינו להראות.
\end{proof}

\question{}
תהי $M \subseteq \RR^n$ יריעה $k$־מימדית,
ותהי $\alpha : U \to W$ פרמטריזציה מקומית סביב $p \in M$. \\
נראה שקיימת קבוצה פתוחה $\tilde{W} \subseteq \RR^n$ ו־$\psi : \tilde{W} \to U$ חלקה,
כך ש־$p \in \tilde{W}$ וגם,
\[
	\forall q \in \alpha^{-1}(\tilde{W}),\ 
	\psi(\alpha(q)) = q
\]
\begin{proof}
	נניח ש־$\alpha(x_0) = p$ עבור $x_0 \in U$.
	נניח ללא הגבלת הכלליות ש־$k$ השורות הראשונות של $D \varphi \mid_{x_0}$ בלתי־תלויות לינארית,
	מותר לנו להניח כן שכן אחרת נוכל לכפול בהעתקה אורתוגונלית מתאימה, וכי נתון כי $M$ יריעה ולכן ההעתקה $\alpha$ היא רגולרית.
	נגדיר את הקבוצה $\tilde{U} = U \times \RR^{n - k}$ וכן נגדיר את ההעתקה $\tilde{\alpha} : \tilde{U} \to \RR^n$ על־ידי,
	\[
		\tilde{\alpha}(x_1, \ldots, x_n)
		= \alpha(x_1, \ldots, x_k) + \sum_{i = k + 1}^n x_i \cdot e_i
	\]
	עבור $e_i$ איבר הבסיס הסטנדרטי עבור $1 \le i \le n$.
	מבדיקה ישירה מתקיים,
	\[
		D \tilde{\varphi} \mid_{(x_0, 0)}
		= \begin{pmatrix}
			D \varphi \mid_{x_0} & 0 \\
			\vdots & \id
		\end{pmatrix}
	\]
	כלומר זוהי מטריצה כך שהיא מרחיבה את $D \varphi \mid_{x_0}$ על־ידי מטריצת היחידה.
	נבחין כי החישוב מתקבל ישירות מההגדרה בהתאם למימד.
	אנו יודעים כי דרגת המטריצה $k + (n - k) = n$, כלומר זוהי מטריצה רגולרית, ולכן הנגזרת היא בעלת דטרמיננטה לא אפס, ונוכל להסיק שתנאי משפט הפונקציה ההפוכה חלים.
	נסיק כי קיימת $U_0 \times V_0 \subseteq \tilde{U}$ כך ש־$x_0 \in U_0 \subseteq U$, וקיימת $p \in \tilde{W} \subseteq \RR^n$, כך ש־$\tilde{\alpha} : U_0 \times V_0 \to \tilde{W}$ דיפאומורפיזם.
	נגדיר את $\psi : \tilde{W} \to U$ על־ידי $\psi = \pi_k \circ \tilde{\alpha}^{-1}$ עבור $\pi_k : \RR^n \to \RR^k$ ההטלה של $k$ האיברים הראשונים.
	נבחין כי ממשפט ההעתקה ההפוכה $\psi$ היא העתקה גזירה, ומהפעלה חוזרת ונשנית נקבל שהיא גזירה מכל סדר, קרי היא פונקציה חלקה.
	לבסוף נובע שעבור $q \in U_0 = \alpha^{-1}(\tilde{W})$ מתקיים $\psi(\alpha(q)) = q$.
\end{proof}

\question{}
תהי $M$ יריעה $k$־מימדית ב־$\RR^n$, נגדיר את האגד המשיק,
\[
	TM
	= \{ (p, v) \in \RR^n \times \RR^n \mid p \in M, v \in T_p M \}
\]

\subquestion{}
נראה ש־$TM$ היא יריעה $2k$־מימדית.
\begin{proof}
	תהי נקודה $p \in M$ כלשהי, נבחין כי $M$ יריעה ולכן קיימת פרמטריזציה מקומית $\alpha : U \to M$ כך ש־$\alpha(0) = p$ (ללא הגבלת הכלליות).
	נגדיר את הפונקציה $\overline{\alpha} : U \times \RR^k \to \RR^{2n}$ על־ידי $\overline{\alpha}(q, v) = (\alpha(q), d \alpha \mid_p(v))$.
	לכל $u \in T_p M$ נבחין כי קיים $v \in \RR^k$ כך ש־$u = T_p(M) \cdot v$, ולכן $\overline{\alpha}(0, v) = (p, u)$, כלומר $\overline{\alpha}$ חשודה כפרמטריזציה של $(p, u)$.

	אנו יודעים כי $\alpha$ הפיכה, וכן $d \alpha \mid_p$ היא העתקה לינארית רגולרית ולכן הפיכה אף היא ב־$T_p M$, ולכן נסיק שגם $\overline{\alpha}$ היא הפיכה.
	אנו גם יודעים כי $\alpha$ חלקה ורגולרית, וכן כל העתקה לינארית היא חלקה, ומרגולריות $\alpha$ נסיק שנגזרתה רגולרית אף היא בתחום, ולכן נסיק שגם $\overline{\alpha}$ היא חלקה ורגולרית.
	עלינו להראות שההעתקה היא פתוחה, אך גם הפעם, $\alpha$ היא העתקה פתוחה, והעתקות לינאריות תמיד פתוחות, ולכן נסיק ש־$\overline{\alpha}$ היא אכן פרמטריזציה מקומית של $(p, u)$.

	נסיק אם כן ש־$TM$ היא יריעה $2k$־מימדית ב־$\RR^{2n}$.
\end{proof}

\subquestion{}
נראה שאם $M$ היא תת־קבוצה פתוחה של $\RR^n$, אז $TM = M \times \RR^n$.
\begin{proof}
	נבהיר תחילה שכמסקנה מהטענה מהתרגול $M$ היא יריעה $n$־מימדית, ולכן $TM$ יריעה $2n$־מימדית ב־$\RR^{2n}$.
	לכל $p \in M$ נניח ש־$\alpha : U \to M$ פרמטריזציה מקומית, אז $\alpha$ רגולרית, דהינו $\dim D \alpha = n$ ולכן היא הפיכה (ממשפט הפונקציה ההפוכה), ובפרט לכל $u \in \RR^n$ נסיק ש־$u \in T_p M$.
	נובע אם כך ש־$\{ p \} \times \RR^n \in TM$ לכל $p \in M$, ונסיק ש־$TM = M \times \RR^n$.
\end{proof}

\question{}
תהיינה $M \subseteq \RR^m$ ו־$N \subseteq \RR^n$ שתי יריעות.
תהי $f : M \to N$.
נניח ש־$\alpha : U \to W$ פרמטריזציה מקומית של $M$ סביב $p \in M$,
ונניח ש־$\beta : A \to B$ פרמטריזציה מקומית של $N$ סביב $f(p)$. \\
נראה שאם קיימת סביבה פתוחה $U' \subseteq U$ של $\alpha^{-1}(p)$ כך ש־$\beta^{-1} \circ f \circ \alpha : U' \to A$ היא חלקה, אז $f$ חלקה ב־$p$.
\begin{proof}
	נניח שקיימת סביבה פתוחה $U'$ כך שהתנאי מתקיים.
	נבחין תחילה כי $\beta^{-1}(f(\alpha(x_0))) = \beta^{-1}(f(p))$ היא נקודה פנימית בהרכבה.
	אנו גם יודעים שהרכבת פונקציות חלקות היא פונקציה חלקה, לכן גם $\beta \circ \beta^{-1} \circ f \circ \alpha = f \circ \alpha : U' \to N$ היא העתקה חלקה ב־$\alpha^{-1}(p)$.
	מאותה סיבה בדיוק גם $f \circ \alpha \circ \alpha^{-1} = f : M \to N$ היא חלקה ב־$\alpha^{-1}(\alpha(x_0)) = p$ ונקבל ש־$f$ חלקה ב־$p$, כפי שרצינו.
	נבהיר ונאמר ש־$\alpha, \beta, \alpha^{-1}, \beta^{-1}$ כולן דיפאומורפיזמים חלקים, ולכן הפיכות, חלקות, ומוגדרות בסביבות הנתונות.
\end{proof}

\end{document}
