\input{../article_base.tex}
\title{פתרון מטלה 05 --- אנליזה על יריעות, 80426}

\DeclareMathOperator{\vol}{vol}

\begin{document}
\maketitle
\maketitleprint[teal]

\question{}
\subquestion{}
תהי ההעתקה הלינארית $T : \RR^3 \to \RR^4$ המוגדרת על־ידי,
\[
	T(x, y, z)
	= (x + 3z, y + x + z, 2y + x, z)
\]
ויהי תת־המרחב $H = \{(x, y, z) \in \RR^3 \mid x = y \}$.
נגדיר $S = T \restriction H$, נחשב את $V(S)$.
\begin{solution}
	נגזור ונקבל,
	\[
		D T
		= \begin{pmatrix}
			1 & 0 & 3 \\
			1 & 1 & 1 \\
			1 & 2 & 0 \\
			0 & 0 & 1
		\end{pmatrix}
	\]
	אבל במרחב $H$ מתקיים $x = y$ ולכן נובע,
	\[
		S(x, x, z)
		= (x + 3z, 2x + z, 3x, z)
	\]
	כלומר קיבלנו כי $S : H \to \RR^4$ היא העתקה ממימד 2 למימד 4, ונסיק,
	\[
		D S
		= \begin{pmatrix}
			1 & 3 \\
			2 & 1 \\
			3 & 0 \\
			0 & 1
		\end{pmatrix}
	\]
	ולכן נובע שמתקיים,
	\[
		V(S) = \sqrt{\begin{vmatrix}
				14 & 5 \\
				5 & 11
		\end{vmatrix}}
		= \sqrt{129}
	\]
\end{solution}

\subquestion{}
עבור $i \in \{1, 2, 3\}$ יהיו $U_i$ מרחבי וקטוריים ממימד $k$, ותהינה ההעתקות $T : U_1 \to U_2, S : U_2 \to U_3$.
נראה שמתקיים
\[
	V(ST) = V(S) V(T)
\]
\begin{proof}
	\[
		V(ST)
		= \sqrt{|{(ST)}^t ST|}
		= \sqrt{|T^t S^t ST|}
		= \sqrt{|T^t| \cdot |S^t S| \cdot |T|}
		= \sqrt{|T^t|} V(S) \sqrt{|T|}
		= V(S) V(T)
	\]
	כאשר השתמשנו בכפליות הדטרמיננטה, בחילופיות הכפל, וכן באפיון שחלוף כפל.
\end{proof}

\question{}
\subquestion{}
תהי $(X, \varphi : U \to \RR^n)$ יריעה פרמטרית $k$־מימדית כך ש־$U \subseteq \RR^k$ וכן $X \subseteq \RR^n$ עבור $n \ge k$ כלשהם. \\
נניח ש־$k$ השורות הראשונות של $D \varphi \mid_u$ בלתי תלויות לינארית.
נראה שקיימת קבוצה פתוחה $V \subseteq U$, דיפאומורפיזם $\psi : W \to V$, עבור $W \subseteq \RR^k$ פתוחה, ופונקציה חלקה $h : W \to \RR^{n - k}$, כך שמתקיים,
\[
	\forall w \in W,\ 
	(w, h(w)) = (\varphi \circ \psi)(w)
\]
\begin{proof}
	נגדיר $\varphi = (\varphi_1, \varphi_2)$ עבור $\varphi_1 : \RR^k \to \RR^k$ ו־$\varphi_2 : \RR^k \to \RR^{n - k}$.
	נתון כי $k$ השורות הראשונות של $D \varphi$ בלתי תלויות לינארית, כלומר $\varphi_1$ בעלת נגזרת ממימד $k$, ובהתאם היא הפיכה בכל נקודה ודיפאומורפיזם ממשפט הפונקציה ההפוכה.
	נגדיר את $\psi$ להיות פונקציה הפוכה של $\varphi_1$ בסביבה פתוחה ב־$X$, נסמנה $W$, ונסמן את תמונתה ב־$V$.
	נגדיר את הפונקציה $h : W \to \RR^{n - k}$ כפונקציה המקיימת,
	\[
		\forall w \in W,\ 
		(w, h(w)) = (\varphi \circ \psi)(w)
	\]
	קיימת כזו כצמצום מקומי של $\varphi_2$.
	עלינו להראות ש־$h$ היא חלקה, אך זה נובע מהפעלה חוזרת ונשנית של משפט הפונקציה ההפוכה (ובהתאם, צמצום ראשוני של $W, V$ בהתאם) בדומה להוכחה עבור מסילות.
\end{proof}

\subquestion{}
תהי $(X, \varphi : U \to \RR^k)$ יריעה פרמטרית $k$־מימדית רגולרית ב־$u \in U$.
נראה שקיימים $V, W, \psi, h$ כבסעיף הקודם, ו־$A \in O_n(\RR)$ (כלומר $A$ העתקה אורתוגונלית) כך שמתקיים,
\[
	\forall w \in W,\ 
	(w, h(w)) = A \varphi(\psi(w))
\]
\begin{proof}
	נתון כי $\dim(D\varphi \mid_u) = k$ כהגדרת הרגולריות בנקודה, ואנו רוצים למצוא העתקה כך ש־$A D\varphi \mid_u$ תהיה מטריצה ש־$k$ השורות הראשונות שלה בלתי־תלויות לינארית.
	נוכל לבחור אם כך את $A$ כך שתזיז את $k$ השורות שמובטח שבלתי־תלויות מהרגולריות לשורות הראשונות, ונקבל ש־$A \varphi$ תקיים את תנאי הסעיף הקודם, כמבוקש.
\end{proof}

\question{}
\subquestion{}
תהי יריעה פרמטרית $n$־מימדית $(X, \varphi)$ עבור $\varphi : U \to \RR^{n + 1}$ כך ש־$U \subseteq \RR^n$, כך שמתקיים,
\[
	\varphi(x_1, \ldots, x_n)
	= (x_1, \ldots, x_n, u(x_1, \ldots, x_n))
.\]
עבור $u : U \to \RR$ פונקציה חלקה כלשהי.
\begin{solution}
	היא אכן קיימת.
\end{solution}

\subquestion{}
נראה ש־$(X, \varphi)$ היא רגולרית.
\begin{proof}
	מתקיים,
	\[
		D\varphi
		= \begin{pmatrix}
			1 & 0 & \cdots & 0 \\
			0 & 1 & \cdots & 0 \\
			\vdots & \cdots & \ddots & \vdots \\
			0 & \cdots & 0 & 1 \\
			\frac{\partial u}{d x_1} & \frac{\partial u}{d x_2} & \cdots & \frac{\partial u}{d x_n}
		\end{pmatrix}
	\]
	כלומר מטריצת היחידה ולאחריה $\nabla u$.
	לכל $x \in U$ נקבל ש־$\dim D\varphi \mid_x = n$, זאת שכן עמודות המטריצה בלתי־תלויות לינארית (ואף קנוניות).
	נסיק אם כך ש־$(X, \varphi)$ רגולרית בכל נקודה ולכן רגולרית.
\end{proof}

\subquestion{}
נראה שמתקיים,
\[
	\vol_n(X)
	= \int_U \sqrt{1 + {|\nabla u|}^2}\ dx
\]
\begin{proof}
	נתחיל בחישוב $V(D\varphi \mid_x)$,
	\[
		V(D \varphi \mid_x)
		= \sqrt{\det\left(\begin{pmatrix}
				u_1 u_1 & \cdots & u_1 u_n \\
				\vdots & \ddots & \vdots \\
				u_n u_1 & \cdots & u_n u_n
		\end{pmatrix} + I_n\right)}
		= \sqrt{\det\left({(\nabla u)}^t (\nabla u) + I_n\right)}
	\]
	נבחין כי מתקיים,
	\[
		({(\nabla u)}^t (\nabla u)) \cdot {(\nabla u)}^t
		= {(\nabla u)}^t \cdot {\lVert \nabla u \rVert}^2
	\]
	ולכן זהו וקטור עצמי של המטריצה, ונוכל להסיק כי זהו הערך העצמי היחיד של המטריצה הסימטרית הזו, זאת שכן,
	\[
		({(\nabla u)}^t (\nabla u)) \cdot v = \alpha v
		\iff \forall i, 
		u_i (\nabla u \cdot v) = \alpha v_i
		\iff v = \frac{\nabla u \cdot v}{\alpha} u
	\]
	אבל מטריצה סימטרית תמיד לכסינה, ולכן גם ${(\nabla u)}^t (\nabla u) + I_n$ לכסינה ומתקיים,
	\[
		\sqrt{\det\left({(\nabla u)}^t (\nabla u) + I_n\right)}
		= \sqrt{{|\nabla u|}^2 + 1}
	\]
	כנביעה מנוסחה לחישוב דטרמיננטת מטריצה לכסינה אורתוגונלית.
\end{proof}

\subquestion{}
תהי נקודה $x \in U$, נראה שמתקיים $T_x(X) = \{v \in \RR^{n + 1} \mid \langle v, g(x) \rangle = 0 \}$, כאשר,
\[
	g(x)
	= \frac{1}{\sqrt{1 + {|\nabla u|}^2}} \left( \frac{-\partial u}{\partial x_1}, \ldots, \frac{-\partial u}{\partial x_n}, 1 \right)
\]
\begin{proof}
	מהגדרה,
	\[
		T_x(X)
		= \im(D \varphi \mid_x)
		= \{ D \varphi \mid_x(v) \mid v \in \RR^n \}
		= \{ (v_1, \ldots, v_n, \langle \nabla u, v \rangle) \mid v \in \RR^n \}
		= \{ v \in \RR^{n + 1} \mid v_{n + 1} = \langle \nabla v, f(x) \rangle \}
	\]
	עבור $f(x) = \left( \frac{-\partial u}{\partial x_1}, \ldots, \frac{-\partial u}{\partial x_n}, 0 \right)$.
	בהעברת אגפים נקבל,
	\[
		T_x(X)
		= \{ v \in \RR^{n + 1} \mid 0 = \langle \nabla v, g(x) \rangle \}
	\]
	עבור $g(x) = \left( \frac{-\partial u}{\partial x_1}, \ldots, \frac{-\partial u}{\partial x_n}, 1 \right)$, כאשר נוכל לכפוך בסקלר $\frac{1}{\sqrt{1 + {|\nabla u|}^2}}$ בשל לינאריות המכפלה הפנימית.
\end{proof}

\subquestion{}
נגדיר את טורוס הסיבוב $(X, \varphi : (0, \pi) \times (0, \pi) \to \RR^4)$ המוגדר על־ידי,
\[
	\varphi(\theta, \nu)
	= (2 + \sin \nu) (\cos \theta, \sin \theta, 0) + (0, 0, \cos \nu)
\]
ונחשב את $T_{(\frac{\pi}{2}, \frac{\pi}{2})}(X)$.
\begin{solution}
	לא ממש ברור מה ההגדרה עבור המימד הרביעי, לכן נניח שיש רק שלושה.
	נבחין כי,
	\[
		D\varphi
		= \begin{pmatrix}
			(2 + \sin \nu) (- \sin \theta) & \cos \theta \cos \nu \\
			(2 + \sin \nu) (\cos \theta) & \sin \theta \cos \nu \\
			0 & -\sin \nu
		\end{pmatrix}
	\]
	וכן $\varphi(\frac{\pi}{2}, \frac{\pi}{2}) = (0, 3, 0)$, לכן,
	\[
		T_{(\frac{\pi}{2}, \frac{\pi}{2})}(X)
		= T_{(0, 3, 0)}(X)
		= \im(D\varphi \mid_{(0, 3, 0)})
		= \im \begin{pmatrix}
			-3 & 0 \\
			0 & 0 \\
			0 & -1
		\end{pmatrix}
		= \Sp\{(1, 0, 0), (0, 0, 1)\}
	\]
\end{solution}

\subquestion{}
נמצא נוסחה עבור $\vol_2 X$.
\begin{solution}
	נחשב את אלמנט הנפח.
	\[
		V(D \varphi)
		= \sqrt{{(D\varphi)}^t D \varphi}
		= \sqrt{\begin{vmatrix}
				{(2 + \sin \nu)}^2 & 0 \\
				0 & 1
		\end{vmatrix}}
		= 2 + \sin \nu
	\]
	ולכן,
	\[
		\int_X 1\ d\vol_2
		\int_{0}^{2\pi} \int_{-\pi}^{\pi} \int_{1}^{3} 2 + \sin \nu\ dr\ d\nu\ d\theta
	\]
\end{solution}

\end{document}
