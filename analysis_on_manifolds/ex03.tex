\input{../article_base.tex}
\title{פתרון מטלה 03 --- אנליזה על יריעות, 80426}

\begin{document}
\maketitle
\maketitleprint{}

\question{}
יהי $\vec{F} : U \to \RR^2$ עבור $U \subseteq \RR^2$ שדה וקטורי.
נניח שלכל מלבן $R \subseteq U$ האינטגרל בין שני הקצוות שווה לאינטגרל בין שני הקצוות האחרים.
נראה ש־$\vec{F}$ משמר.
\begin{proof}
	האסטרטגיה שלנו היא להשתמש בחלוקה למלבנים לכל מסילה כללית $\gamma : [a, b] \to U$ במטרה להראות שהאינטגרל על המסילה תלוי רק בנקודות קצה.

	תהיינה $\gamma_1, \gamma_2 : [a, b] \to U$ מסילות כלשהן כך שהן חולקות נקודות קצה.
	עבור $N \in \NN$ נגדיר את הסדרה ${\{ x_n \}}_{n = 0}^N \subseteq U$ כך ש־$x_n = \gamma_1(a + (b - a)\frac{n}{N})$, כלומר זוהי חלוקה של המסילה.
	נבנה סדרת מלבנים כך שבין $x_i, x_{i + 1}$ נמצא מלבן לכל $i$, כך שנקודות אלה הן אלכסון המלבן, מטעמי נוחות נבחר את האלכסון הנותר כך שלכל המלבנים תהיה אוריינטציה זהה.
	נניח גם ש־$ABCD$ מלבן שאלכסונו הוא $\gamma_1(a), \gamma_1(b)$ עם אוריינטציה זהה לזו של המלבנים בחלוקה.
	לבסוף נאריך את המלבנים בסדרה כך שצלע שלהם תשב על המלבן $ABCD$.
	מהנתון נובע שאם $B_i$ הקודקודים של המלבנים בסדרה כפי שבחרנו, אז,
	\[
		\int_{\gamma_{ADC}} \vec{F}\ dl
		= \sum_{i = 0}^N \int_{\gamma_{x_i B_i x_{i + 1}}} \vec{F}\ dl
	\]
	ונקבל,
	\[
		\int_{\gamma_{ADC}} \vec{F}\ dl
		\lim_{N \to \infty} \int_{\gamma_{ADC}} \vec{F}\ dl
		= \lim_{N \to \infty} \sum_{i = 0}^N \int_{\gamma_{x_i B_i x_{i + 1}}} \vec{F}\ dl
		= \int_{\gamma_1} \vec{F}\ dl
	\]
	אבל גם,
	\[
		\int_{\gamma_{ADC}} \vec{F}\ dl
		= \int_{\gamma_2} \vec{F}\ dl
	\]
	משיקולים ודמים, ולכן נובע ש־$\vec{F}$ משמר.
\end{proof}

\question{}
נניח ש־$\vec{F}(x, y, z) = (\frac{x}{x^2 + y^2}, \frac{y}{x^2 + y^2}, z)$, בתחום $\RR^3 \setminus \{ (0, 0, z) \}$.

\subquestion{}
נראה ש־$\vec{F}$ משמר.
\begin{proof}
	אילו קיימת $\varphi : \RR^3 \to \RR$ המעידה על שימור $\vec{F}$, אז מתקיים,
	\[
		\frac{\partial}{\partial x} \varphi = \frac{x}{x^2 + y^2}
		\implies \varphi = \frac{1}{2} \ln(x^2 + y^2) + \varphi_{yz}
	\]
	באותו אופן נסיק כי גם,
	\[
		\frac{\partial}{\partial y} \varphi = \frac{y}{x^2 + y^2}
		\implies \varphi = \frac{1}{2} \ln(x^2 + y^2) + \varphi_{xz}
	\]
	ולבסוף שגם,
	\[
		\frac{\partial}{\partial z} \varphi = z
		\implies \varphi = \frac{1}{2} z^2 + \varphi_{xy}
	\]
	נבחין אם כך שאם נגדיר $\varphi(x, y, z) = \frac{1}{2}(\ln(x^2 + y^2) + z^2)$
	אז נקבל בתחום ש־$\varphi$ היא פונקציה רציפה וגזירה, וכן ש־$\nabla \varphi = \vec{F}$, ולכן מהתנאי השקול לשימור נובע ש־$\vec{F}$ אכן משמר.
\end{proof}

\subquestion{}
נחשב את $\int_{\gamma} \vec{F}\ dl$ עבור $\gamma(t) = (\cos t, \sin t, t^2)$ כאשר $t \in [0, 2 \pi]$.
\begin{solution}
	נבחין כי התחום של המסילה הוא לא פשוט קשר, ולכן עלינו לחשב את האינטגרל ישירות מהגדרה,
	\[
		\int_{\gamma} \vec{F}\ dl
		= \int_{0}^{2 \pi} \vec{F}(\gamma(t)) \cdot \gamma'(t)\ dt
		= \int_{0}^{2 \pi} (\frac{\cos t}{1}, \frac{\sin t}{1}, t^2) \cdot (-\sin t, \cos t, 2t)\ dt
		= \int_{0}^{2 \pi} 0 + 2t^2\ dt
		= {\left. \frac{2}{3} t^3 \right|}_{t = 0}^{t = 2 \pi}
		= \frac{2^4 \pi}{3}
	\]
\end{solution}

\question{}
נניח ש־$U \subseteq \RR^n$ תחום פתוח.

\subquestion[2]
נראה שמחלקות הקשירות המסילתית של $U$ הן קבוצות פתוחות, וכן שכל רכיב קשירות מסילתית הוא קשיר מסילתית.
\begin{proof}
	נניח ש־${\{ A_i \}}_{i \in I} \subseteq U$ הן מחלקות הקשירות המסילתית, אז בהכרח $U = \biguplus_{i \in I} A_i$, כלומר זוהי חלוקה של $U$.
	אילו נניח שקיימת $A_i$ לא פתוחה נקבל סתירה ישירה לפתיחות של $U$, לכן נסיק ש־$A_i$ פתוחה לכל $i \in I$.

	נניח עתה ש־$A \subseteq U$ הוא רכיב קשירות מסילתית, ונראה שהוא קשיר מסילתית.
	לכל $x, y \in A$ מתקיים $x \sim y$, כלומר יש מסילה ביניהם, ולכן מצאנו שהקבוצה $A$ קשירה מסילתית.
\end{proof}

\subquestion{}
נראה שמחלקות הקשירות של $U$ הן קבוצות פתוחות, וכן שכל רכיב קשירות הוא קשיר.
\begin{proof}
	החלק הראשון זהה בהוכחתו לזה שבסעיף הקודם.

	נניח ש־$A \subseteq U$ רכיב קשירות.
	בהוכחה של יחס שקילות בתרגול ראינו כי אם $A, B$ קבוצות קשירות כך שהן לא זרות, אז איחודן קשיר.
	נקבע אם כך $x \in A$, ונגדיר גם $B_y \subseteq A$ הקבוצה הקשירה המעידה על $x \simeq y$ לכל $y \in A$.
	נבחין כי בהכרח $U = \bigcup_{y \in A} B_y$, זאת שכן כל נקודה של $U$ מופיעה באיחוד, וכן אילו יש נקודה באיחוד שלא ב־$A$ אז קיבלנו סתירה להיותה רכיב קשירות.
	נקבל מהטענה הקודמת ש־$U$ היא איחוד לא זר של קבוצות קשירות ולכן קשירה.
\end{proof}

\subquestion{}
נראה שקבוצת מחלקות הקשירות (המסילתית) של $U$ היא בת־מניה.
\begin{proof}
	נניח ש־${\{ A_i \}}_{i \in I}$ רכיבי הקשירות (המסילתית), אז לכל $i \in I$ גם $A_i$ היא פתוחה וקשירה (מסילתית).
	יהי $\epsilon > 0$ כך ש־$\operatorname{diam}(A_i) < \epsilon$ לכל $i \in I$.
	נניח בשלילה ש־$|I| > \aleph_0$, ולכן נובע ש־$B_{\epsilon}(x_i)$ עבור כדורים המוכלים במחלקות הקשירות (המסילתית) היא בעצמה קבוצה של כדורים זרים, אבל ידוע כי $\epsilon$־כיסוי של $\RR^n$ הוא בן־מניה בסתירה.
\end{proof}

\question{}
עבור $a, b > 0$ תהי $E_{a, b} = \{ {(x / a)}^2 + {(y / b)}^2 = 1 \} \subseteq \RR^2$ האליפסה עם הפרמטרים $a, b$.
נשתמש במשפט גרין לתחומים פשוטים כדי להראות שמתקיים $\operatorname{Area}(E_{a, b}) = \pi a b$.
\begin{solution}
	TODO
\end{solution}

\end{document}
