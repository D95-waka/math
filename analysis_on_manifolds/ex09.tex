\input{../article_base.tex}
\title{פתרון מטלה 09 --- אנליזה על יריעות, 80426}
% chktex-file 17
% chktex-file 9

\DeclareMathOperator{\vol}{vol}

\begin{document}
\maketitle
\maketitleprint[blue]

\question{}
תהי $M^k \subseteq \RR^n$ יריעה עם שפה.
נסמן $M^\circ = M \setminus \partial M$.

\subquestion{}
נראה ש־$M^\circ$ היא צפופה ב־$M$.
\begin{proof}
	$M^\circ$ צפופה ב־$M$ אם ורק אם כל נקודה ב־$M$ היא ערך גבולי של סדרה ב־$M \setminus \partial M$.
	עבור נקודות ב־$M$ הטענה נכונה באופן טריוויאלי, לכן מספיק שנמצא סדרה ${\{ x_n \}}_{n = 1}^\infty \subseteq M^\circ$ המתכנסת ל־$x \in \partial M$ לכל $x$ כזה.
	נניח ש־$x \in \partial M$ ונניח ש־$\alpha : U \to M$ פרמטריזציה מקומית של $x$, כלומר $p \in U \subseteq \HH^k$ כאשר $p = \alpha^{-1}(x)$.
	פרמטריזציה משמרת שפה, כלומר $y \in \partial U \iff \alpha(y) \in \partial M$, אבל $U$ פתוחה ולכן קיים $\epsilon > 0$ כך ש־$B(p, \epsilon) \cap \HH^k \subseteq U$.
	נבחר נקודה $x_1 \in B(p, \epsilon) \cap U \cap {(\HH^k)}^\circ$ וכן נקודה $x_n \in B(p, \frac{\epsilon}{n}) \cap U \cap {(\HH^k)}^\circ$ לכל $n \in \NN$.
	נקבל סדרה כך ש־$x_n \to x$ וכן $x_n \in M^\circ$ לכל $n \in \NN$, כפי שרצינו.
\end{proof}

\subquestion{}
נראה שאם $k = n$ אז $M^\circ$ היא קבוצה פתוחה ב־$\RR^n$.
\begin{proof}
	ראינו בהרצאה ש־$M^\circ$ היא יריעה ללא שפה ממימד $k = n$, וכן ראינו שיריעות $M^n \subseteq \RR^n$ הן קבוצות פתוחות ב־$\RR^n$.

	הטענה נובעת באופן ישיר מקיום פרמטריזציה מקומית.
	אם $x \in M^\circ$ אז קיימת פרמטריזציה $\alpha : U \to M^\circ$ כך ש־$U \subseteq \RR^n$ פתוחה, ו־$\alpha$ הומיאומורפיזם ולכן בפרט פתוחה ו־$\alpha(U) \subseteq M$ פתוחה במובן המטרי.
	נסיים את ההוכחה עם סגירות פתוחות לאיחוד ואיחוד כלל הסביבות הפתוחות לכל נקודה ב־$M^\circ$.
\end{proof}

\subquestion{}
נראה שאם $U \subseteq M$ קבוצה פתוחה ב־$\RR^n$ אז $U \subseteq M^\circ$.
\begin{proof}
	נתון כי $U \subseteq M$ ו־$\deg U = n$ ולכן $\deg M = n$.
	לכל נקודה $x \in U$ יש פרמטריזציה מקומית וכמו בסעיף הקודם היא פתוחה ולכן מעידה ש־$x \in M^\circ$ כקבוצה פתוחה (הנחה שמותר לעשות בשל תוצאת הסעיף הקודם).
\end{proof}

\subquestion{}
נסיק שאם $M$ תת־קבוצה סגורה של $\RR^n$ ו־$\dim M = n$ אז ההגדרות של שפה ופנים של יריעות ושל קבוצות מזדהות.
\begin{proof}
	נוכל להסיק ישירות מהסעיף הקודם שפנים של יריעה וקבוצה מזדהים כאשר $\dim M = n$.
	ביתר פירוט, אם $U \subseteq M$ פתוחה אז בפרט $U \subseteq M^\circ$ ולכן $M^\circ$ תת־קבוצה פתוחה מקסימלית של $M$, כלומר היא עומדת בהגדרה המטרית המדויקת של פנים.
	נקבל על־ידי חיסור קבוצות והעובדה ש־$M = \overline{M}$ ש־$\partial M = \overline{M} \setminus M^\circ$, כלומר גם השפה מזדהה עם ההגדרה המטרית.
\end{proof}

\question{}
תהי $M^k \subseteq \mathbb{R}^n$ יריעה עם שפה,
נניח ש־$q \in \partial M$ ותהי פרמטריזציה $\alpha : V \to W$ כך ש־$V \subseteq \HH^k$ פתוחה ו־$q \in W \subseteq M$.
נניח גם ש־$\tilde{\alpha} : \tilde{V} \to \tilde{W}$ הרחבה חלקה שלה, כלומר $\tilde{V} \subseteq \RR^k$ ו־$\dim \tilde{W} = k$. \\
נניח ש־$y : [0, \varepsilon) \to \tilde{V}$ היא מסילה חלקה כך ש־$y(0) = y_0 = \alpha^{-1}(q)$, וש־$d \tilde{\alpha}_{y_0}(y'(0)) = X(q)$ מקיימת $\langle X(q), \nu(q) \rangle < 0$,
כאשר $\nu$ הנורמל החיצוני של $M$. \\
נראה שקיים $\delta > 0$ כך ש־$y(t) \in \HH^k$ עבור $t \in [0, \delta]$,
ובפרט ש־$\tilde{\alpha}(y(t)) = \alpha(y(t)) \in M$.
\begin{proof}
	אנו נראה שהקורדינטה האחרונה של $y'(0)$ חיובית.
	נפעיל תהליך גרם־שמידט על $X(q)$ ונקבל,
	\[
		X(q)
		= \langle X(q), \nu(q) \rangle \nu(q) + Y(q)
	\]
	כאשר $Y(q) \in T_q \partial M$ מאורתוגונליות הנורמל החיצוני והמשיק לשפה.
	בהתאם הקורדינטה האחרונה של $d Y(q)$ אפס כווקטור בנגזרת השפה, $\langle X(q), \nu(q) \rangle$ ערך שלילי ו־$\nu(q)$ שלילי בקורדינטה האחרונה, ולכן $X(q)$ חיובי בקורדינטה האחרונה.

	עתה נסיק כי קיים $\delta > 0$ כך ש־$y(t) \in \HH^k$, זאת שכן הוא נע מעלה בקורדינטה האחרונה ומוגדר בכל אלו שאינן האחרונות.
	בפרט $\tilde{\alpha}(y(t)) = \alpha(y(t))$, שכן מהנתון ההרחבה היא עבור ערכים שליליים.
\end{proof}

\question{}
\subquestion{}
תהי $M^k \subseteq \RR^n$ יריעה קומפקטית עם שפה.
נניח ש־$f, g : M \to \RR$ רציפות וש־$c \in \RR$.
נראה שמתקיים,
\[
	\int_M (\alpha f + g)\ d\vol_k
	\alpha \int_M f\ d\vol_k + \int_M g\ d\vol_k
\]
\begin{proof}
	נניח ש־${\{ \alpha_i : U_i \to W_i \}}_{i = 1}^N$ פרמטריזציות מקומיות המכסות את $M$, כלומר $M = \bigcup_{i = 1}^N W_i$.
	נניח גם ש־${\{ \varphi_i : M \to W_i \}}_{i = 1}^N$ חלוקת יחידה עבור $\{ W_i \}$.
	אז מהגדרת האינטגרל,
	\[
		\int_M (c f + g)\ d\vol_k
		= \sum_{i = 1}^N \int_{U_i} \varphi_i(\alpha_i(x)) (c f + g)(\alpha_i(x)) V(D \alpha_i |_x)\ dx
	\]
	נבחין כי זהו אינטגרל רב משתני רגיל ולכן לינארי מתכונות האינטגרל, כלומר,
	\begin{multline*}
		\sum_{i = 1}^N \int_{U_i} \varphi_i(\alpha_i(x)) (c f + g)(\alpha_i(x)) V(D \alpha_i |_x)\ dx \\
		= \sum_{i = 1}^N \left( c \int_{U_i} \varphi_i(\alpha_i(x)) f(\alpha_i(x)) V(D \alpha_i |_x)\ dx + \int_{U_i} \varphi_i(\alpha_i(x)) g(\alpha_i(x)) V(D \alpha_i |_x)\ dx \right)
	\end{multline*}
	אבל גם סכומים סופיים הם לינאריים ונסיק,
	\begin{multline*}
		\sum_{i = 1}^N \left( c \int_{U_i} \varphi_i(\alpha_i(x)) f(\alpha_i(x)) V(D \alpha_i |_x)\ dx + \int_{U_i} \varphi_i(\alpha_i(x)) g(\alpha_i(x)) V(D \alpha_i |_x)\ dx \right) \\
		= \left( c \sum_{i = 1}^N \int_{U_i} \varphi_i(\alpha_i(x)) f(\alpha_i(x)) V(D \alpha_i |_x)\ dx \right) + \sum_{i = 1}^N \left( \int_{U_i} \varphi_i(\alpha_i(x)) g(\alpha_i(x)) V(D \alpha_i |_x)\ dx \right)
	\end{multline*}
	והביטוי האחרון איננו אלא,
	\[
		\alpha \int_M f\ d\vol_k + \int_M g\ d\vol_k
	\]
	כפי שרצינו.
\end{proof}

\subquestion{}
משפט פוביני,
נניח ש־$M^m$ ו־$N^n$ יריעות, ונניח כי $f, g : M \times N \to \RR$ פונקציה רציפה,
אז,
\[
	\int_{M \times N} f\ d\vol_{m + n}
	= \int_{M} \left( \int_N f\ d\vol_n \right)\ d\vol_m
	= \int_{N} \left( \int_M f\ d\vol_m \right)\ d\vol_n
\]
\begin{proof}
	נניח ש־${\{ \alpha_i : U_i \to W_i \}}_{i = 1}^k$ פרמטריזציות מקומיות המכסות את $M \times N$, כלומר $M \times N = \bigcup_{i = 1}^k W_i$. \\
	נניח גם ש־${\{ \varphi_i : M \times N \to W_i \}}_{i = 1}^k$ חלוקת יחידה עבור $\{ W_i \}$.
	נניח כי $U_i = U_i^M \times U_i^N$ עבור $U_i^M \subseteq \HH^m, U_i^N \subseteq \HH^n$ לכל $i \le k$. \\
	נסמן פירוק דומה ל־$W_i$, וכן עבור $\alpha_i, \varphi_i$ לכל $i$. \\
	מהגדרה,
	\begin{align*}
		\int_{M \times N} f\ d\vol_{m + n}
		& = \sum_{i = 1}^k \int_{U_i} \varphi_i(\alpha_i(x)) f(\alpha_i(x)) V(D \alpha_i |_x)\ dx \\
		& = \sum_{i = 1}^k \int_{U_i^M \times U_i^N} \varphi_i(\alpha_i(x, y)) f(\alpha_i(x, y)) V(D \alpha_i |_{(x, y)})\ d(x, y)
	\end{align*}
	כאשר $x \in \RR^m, y \in \RR^n$ בביטוי האחרון.
	זהו סכום סופי של אינטגרלים רב־משתניים ולכן משפט פוביני חל בהם, נראה את אחת הגרסות כדי לחסוך בנייר,
	\begin{align*}
		& \sum_{i = 1}^k \int_{U_i^M \times U_i^N} \varphi_i(\alpha_i(x, y)) f(\alpha_i(x, y)) V(D \alpha_i |_{(x, y)})\ dx\ dy \\
		\overset{(1)}{=}  & \sum_{i = 1}^k \int_{U_i^M} \left( \int_{U_i^N} \varphi_i(\alpha_i(x, y)) f(\alpha_i(x, y)) V(D \alpha_i |_{(x, y)})\ dy \right)\ dx \\
		\overset{(2)}{=}  & \sum_{i = 1}^k \int_{U_i^M} \left( \sum_{j = 1}^k \int_{U_i^N} \varphi_i^M(\alpha_i^M(x)) \varphi_j^N(\alpha_j^N(y)) f(\alpha_i(x, y)) V(D \alpha_i |_{(x, y)})\ dy \right)\ dx \\
		\overset{(3)}{=}  & \sum_{i = 1}^k \int_{U_i^M} \left( \varphi_i^M(\alpha_i^M(x)) \sum_{j = 1}^k \int_{U_i^N} \varphi_j^N(\alpha_j^N(y)) f(\alpha_i(x, y)) V(D \alpha_i^N |_{y})\ dy \right)V(D \alpha_i^M |_{(x)})\ dx \\
		= & \sum_{i = 1}^k \int_{U_i^M} \left( \varphi_i^M(\alpha_i^M(x)) \int_{N} f(x, y)\ d\vol_n \right) V(D \alpha_i^M |_{(x)})\ dx \\
		= & \int_{M} \left( \int_{N} f\ d\vol_n \right)\ d\vol_m
	\end{align*}
	כאשר,
	\begin{enumerate}
		\item משפט פוביני לאינטגרלים רב־משתניים
		\item הוספת סכום ומציין יחידה
		\item כפליות דטרמיננטת בלוקים
	\end{enumerate}
\end{proof}

\question{}
תהי $M^k$ יריעה קומפקטית ונניח ש־$M_1^k, M_2^k$ שתי יריעות קומפקטיות עם שפה המקיימות,
\[
	M_1 \cup M_2 = M,
	\quad
	M_1 \cap M_2
	= \partial M_1
	= \partial M_2
	= \Gamma
\]
נניח ש־$f : M \to \RR$ רציפה,
אנו נראה כי,
\[
	\int_M f\ d\vol_k
	= \int_{M_1} f\ d\vol_k
	+ \int_{M_2} f\ d\vol_k
\]
\begin{proof}
	נניח ש־${\{ \alpha_i : U_i \to W_i \}}_{i = 1}^l$ פרמטריזציות מקומיות של $M$ המכסות אותה.
	נניח גם כי כל פרמטריזציה מושרית מנקודה ב־$\Gamma$ או זרה ל־$\Gamma$.
	מותר לנו להניח כך שכן נוכל לקחת כיסוי פתוח מפרמטריזציות של $\Gamma$ (היא קומפקטית) והרחבת הכיסוי לכל $M$ ושימוש בחיתוכים סופיים.
	נניח ש־${\{ \varphi_i : M \to W_i \}}_{i = 1}^l$ כיסוי יחידה הכפוף ל־$\{ W_i \}$.
	מההנחה הנוספת נוכל גם להניח בלי הגבלת הכלליות כי קיימים $l_1 < l_2 < l$ כך ש־$W_i \subseteq M_1$ עבור $1 \le i \le l_1$,
	$W_i \subseteq M_2$ עבור $l_1 < i \le l_2$ ולכל $l_2 < i \le l$ ש־$\alpha_i$ מושרית מנקודה ב־$\Gamma$.

	ממשפט מהתרגול לכל $l_2 < i \le l$, הפרמטריזציה $\alpha_i$ מקיימת,
	\[
		\alpha_i(\HH^k \cap U_i) \subseteq M_1,
		\qquad
		\alpha_i((-\HH^k) \cap U_i) \subseteq M_2
	\]
	ונסמן את שתי הפרמטריזציות המקומיות המושרות הללו $\alpha_i^1, \alpha_i^2$ בהתאמה. \\
	נגדיר גם פירוק של $\varphi_i$ ל־$\varphi_i^1, \varphi_i^2$ על־ידי מכפלה במציין $1_{M_1}$ ובמציין $1_{M_2}$ בהתאמה.
	עתה אנו מוכנים ויכולים לבחון את האינטגרל,
	\begin{align*}
		& \int_M f\ d\vol_k \\
		= & \sum_{i = 1}^l \int_{U_i} \varphi_i(\alpha_i(x)) f(\alpha_i(x)) V(D \alpha_i |_{x})\ dx \\
		= & \sum_{i = 1}^{l_1} \int_{U_i} \varphi_i(\alpha_i(x)) f(\alpha_i(x)) V(D \alpha_i |_{x})\ dx \\
		  & + \sum_{i = l_1 + 1}^{l_2} \int_{U_i} \varphi_i(\alpha_i(x)) f(\alpha_i(x)) V(D \alpha_i |_{x})\ dx \\
		  & + \sum_{i = l_2 + 1}^{l} \int_{U_i} \varphi_i^1(\alpha_i(x)) f(\alpha_i(x)) V(D \alpha_i |_{x})\ dx \\
		  & + \sum_{i = l_2 + 1}^{l} \int_{U_i} \varphi_i^2(\alpha_i(x)) f(\alpha_i(x)) V(D \alpha_i |_{x})\ dx \\
		= & \sum_{1 \le i \le l_1, l_2 < i \le l} \int_{U_i} \varphi_i^1(\alpha_i(x)) f(\alpha_i(x)) V(D \alpha_i |_{x})\ dx \\
		  & + \sum_{l_1 < i \le l} \int_{U_i} \varphi_i^2(\alpha_i(x)) f(\alpha_i(x)) V(D \alpha_i |_{x})\ dx \\
		= & \sum_{1 \le i \le l_1, l_2 < i \le l} \int_{U_i^1} \varphi_i^1(\alpha_i^1(x)) f(\alpha_i^1(x)) V(D \alpha_i^1 |_{x})\ dx \\
		  & + \sum_{l_1 < i \le l} \int_{U_i^2} \varphi_i^2(\alpha_i^2(x)) f(\alpha_i^2(x)) V(D \alpha_i^2 |_{x})\ dx \\
		= & \int_{M_1} f\ d\vol_k + \int_{M_2} f\ d\vol_k
	.\end{align*}
\end{proof}

\end{document}
