\input{../article_base.tex}
\title{פתרון מטלה 07 --- אנליזה על יריעות, 80426}

\DeclareMathOperator{\vol}{vol}

\begin{document}
\maketitle
\maketitleprint[blue]

\question{}
ניזכר כי $TM$ הוא האגד המשיק של היריעה $M$.
תהי $\pi : TM \to M$ ההעתקה $(x, v) \mapsto x$, ונגדיר $s : M \to TM$ על־ידי $s(x) = (x, 0)$.
תהי $f : M \to N$ חלקה, אז נקבל ש־$df : TM \to TN$ העתקה המוגדרת על־ידי $df(x, v) = (f(x), Df |_x(v))$.

\subquestion{}
נראה ש־$\pi, s$ חלקות.
\begin{proof}
	נבחין כי שתי הפונקציות הן לינאריות ולכן בפרט חלקות, אבל נוכיח את הטענה ישירות מהגדרה.

	נניח ש־$M^k \subseteq \RR^m$.
	אז $TM \subseteq \RR^{m + k}$.
	תהי $(p, v) \in TM$ ונבחן את $\overline{\pi} : \RR^{m + k} \to \RR^m$ הרחבה רציפה של $\pi : TM \to M$.
	כלומר, נגדיר ש־$\overline{\pi} \restriction TM = \pi$, וכן $\overline{\pi}(x, u) = x$, זוהי העתקה לינארית $\RR^{m + k} \to \RR^m$, כלומר העתקה לינארית אוקלידית, ולכן היא חלקה.

	נבצע תהליך דומה ל־$s$, נגדיר $\overline{s} : \RR^m \to \RR^{m + k}$ על־ידי $\overline{s}(x) = (x, 0)$, אז בבירור $\overline{s} \restriction M = s$, ו־$\overline{s}$ חלקה ומעידה על $s$ כהעתקה חלקה.
\end{proof}

\subquestion{}
נראה ש־$df$ היא חלקה.
\begin{proof}
	נניח ש־$N^l \subseteq \RR^n$. אז $f \subseteq {(\RR^n)}^{\RR^m}$ וכן $D f |_x : T_x(M) \to T_{f(x)}(N)$, כלומר $Df |_x \subseteq {(\RR^l)}^{\RR^k}$, לכל $x \in M$. \\
	הפעם נרצה לבנות את ההרחבה כך שקיימת פונקציה חלקה $\overline{df} : U \to \RR^{n + l}$ כך ש־$U \subseteq \RR^{m + k}$ פתוחה, וכן ש־$\overline{df} \restriction TM = df$.
	נניח ש־$(p, v) \in TM$ ונרצה להראות ש־$df$ חלקה ב־$(x, v)$.
	ידוע כי $f$ חלקה ולכן קיימת $\overline{f} : U_0 \to \RR^n$ חלקה, עבור $p \in U_0 \subseteq \RR^m$ פתוחה.
	לכל $x \in U_0$ נבחין כי $D \overline{f} |_x$ היא העתקה לינארית ולכן חלקה אף היא.
	נגדיר אם כך את $\overline{df}(x, v) = (\overline{f}(x), D \overline{f}|_x(v))$ ונקבל שזו העתקה חלקה, אבל $\overline{df} \restriction TM = df$ ישירות מהגדרת הנגזרת על יריעה, ולכן פונקציה זו מעידה כי $df$ בעצמה חלקה.
\end{proof}

\question{}
ניזכר כי,
\[
	\operatorname{SL}_n(\RR)
	= \{ A \in M_n(\RR) \mid \det(A) = 1 \}
\]
היא קבוצת ההעתקות הלינאריות $\RR^n \to \RR^n$ ההפיכות והאורתוגונליות.

\subquestion{}
נראה ש־$\operatorname{SL}_n(\RR)$ היא יריעה.
\begin{proof}
	צריך להראות ש־$\operatorname{GL}_n(\RR)$ פתוחה ב־$M_n(\RR)$.
	נגדיר $f : \operatorname{GL}_n(\RR) \to \RR$ על־ידי $f(A) = \det(A)$.
	נשים לב כי מתקיים,
	\begin{align*}
		D f |_{I_n}(B)
		& = \lim_{h \to 0} \frac{f(I_n + h B) - f(I_n)}{h} \\
		& = \lim_{h \to 0} \frac{\det(I_n + h B) - \det(I_n)}{h} \\
		& = \lim_{h \to 0} \frac{(1 + h B_{1, 1}) \det(I_{n - 1} + h B_{i > 1, j > 1}) + h(\ldots) - 1}{h} \\
		& = \lim_{h \to 0} \frac{(1 + h B_{1, 1}) \cdots (1 + h B_{n, n}) - 1}{h} + o(h) \\
		& = \lim_{h \to 0} \frac{1 + h \tr(B) - 1}{h} + o(h) \\
		& = \tr(B)
	\end{align*}
	נבחין כי $\det(A + h B) = \det(A) \cdot \det(I_n + h A^{-1} B)$ עבור כל $A \in \operatorname{GL}_n(\RR)$, ונוכל להסיק מהתוצאה האחרונה כי,
	\[
		D f |_{A}
		= \det(A) \cdot \tr(A^{-1} B)
	\]
	לכן לכל $A$ כזו נוכל להסיק שהנגזרת על, נראה זאת על־ידי בחירת $B = A \cdot \diag(\frac{r}{\det(A)}, 1, \ldots, 1)$ לכל $r \in \RR$.
	נוכל להסיק אם כך שכל מטריצה הפיכה היא נקודה רגולרית של ההעתקה, ובפרט $1$ ערך רגולרי, זאת שכן אם $\det(A) \ne 0$ אז היא הפיכה, בפרט במקרה $\det(A) = 1$.
	נבחין גם כי $\operatorname{SL}_n(\RR) = f^{-1}(\{1\})$ ולכן ממשפט הפונקציה הסתומה ליריעות נקבל שזו אכן יריעה.
\end{proof}

\subquestion{}
נחשב את $\dim \operatorname{SL}_n(\RR)$.
\begin{solution}
	נבחין כי $\dim \operatorname{GL}_n(\RR) = n^2$ וכן $\dim \RR = 1$ ולכן מהמשפט שהשתמשנו בו זה עתה נובע ש־$\dim \operatorname{SL}_n(\RR) = n^2 - 1$.
\end{solution}

\subquestion{}
נבדוק אם $\operatorname{SL}_n(\RR)$ קומפקטית.
\begin{solution}
	%אנחנו יודעים ש־$f$ פונקציה רציפה, ואנו יודעים גם כי $\{ 1 \}$ הוא קבוצה סגורה ב־$\RR$, מתנאי שקול לרציפות טופולוגית נסיק שגם $\operatorname{SL}_n(\RR)$. \\
	נגדיר את המטריצה $A_r = \diag(r, \frac{1}{r}, 1, \ldots, 1)$, זוהי מטריצה אלכסונית והפיכה לכל $r \in \RR \setminus \{ 0 \}$.
	נבחין כי $\lVert A_r \rVert_\infty = |r|$ ולכן לכל $r > 0$ נוכל למצוא מטריצה $A_{r + 1} \in \operatorname{SL}_n(\RR)$ לא חסומה ב־$B_r(0)$.
	נסיק ש־$\operatorname{SL}_n(\RR)$ לא קומפקטית.
\end{solution}

\question[4]
נגדיר $M_a^2 = \{ (x, y, z, w) \in \RR^4 \mid x^2 + y^2 = 1, z^2 + w^2 = {(x + a)}^2 \}$.

\subquestion{}
נראה ש־$M_3^2$ היא יריעה.
\begin{proof}
	נגדיר את הפונקציה $g : \RR^4 \to \RR^2$ על־ידי,
	\[
		g(x, y, z, w)
		= (x^2 + y^2, z^2 + w^2 - {(x + a)}^2)
	\]
	אז נקבל ש־$g^{-1}(\{ (1, 0) \}) = M_a^2$.
	ממשפט הפונקציה הסתומה ליריעות מספיק שנוכיח ש־$(1, 0)$ ערך רגולרי של $f$.
	נגזור את הפונקציה,
	\[
		D g |_{(x, y, z, w)}
		= \begin{pmatrix}
			2x & 2y & 0 & 0 \\
			-2(x + a) & 0 & 2z & 2w
		\end{pmatrix}
	\]
	ונקבל כי לכל הצבת ערך יש לפחות שתי שורות בלתי תלויות־לינארית, כלומר הנקודה רגולרית לכל נקודה, \\
	ובפרט אם $(x, y, z, w) \in f^{-1}(\{ (1, 0) \})$.
	נסיק ממשפט הפונקציה הסתומה שאכן $M_3^2$ יריעה.
\end{proof}

\subquestion{}
נראה ש־$M_0^2$ איננה יריעה.
\begin{proof}
	נניח בשלילה ש־$M_0^2$ יריעה, ונגדיר,
	\[
		f : M \to S^1,
		\qquad
		f(x, y, z, w) = (x, y)
	\]
	זוהי העתקה חלקה כהעתקת צמצום ולכל $q \in S^1$ ערך רגולרי גם $f^{-1}(q)$ יריעה.
	אבל מהגדרת $f$ נקבל שכל ערך ב־$S^1$ היא רגולרית, זאת שכן,
	\[
		D f \mid_{(x, y, z, w)}
		= \begin{pmatrix}
			1 & 0 & 0 & 0 \\
			0 & 1 & 0 & 0
		\end{pmatrix} 
	\]
	ולכן בפרט $f^{-1}(\{(0, 1)\})$ יריעה, ממימד 2.
	אבל,
	\[
		f(x, y, z, w) = (0, 1)
		\iff x^2 + y^2 = 1, w^2 + z^2 = x^2, x = 0, y = 1
		\iff x = 0, y = 1, z = 0, w = 0
	\]
	כלומר זוהי יריעה ממימד 0, בסתירה למשפט הפונקציה הסתומה.
\end{proof}

\question{}
נגדיר את העתקת הופ כ־$h : S^3(1) \to S^2(\frac{1}{2})$ המוגדרת על־ידי,
\[
	h(z, w)
	= (z \overline{w}, \frac{1}{2} ({|z|}^2 - {|w|}^2))
\]
נראה כי אכן תמונתה היא ב־$S^2(\frac{1}{2})$ ושכל $q \in S^2(\frac{1}{2})$ היא ערך רגולרי של $h$.
\begin{proof}
	נבחין כי $\frac{1}{2} ({|z|}^2 - {|w|}^2) \in \RR$ וכן ש־$z \overline{w} \in \CC$ ולכן $h$ משוכנת ב־$\RR^3$.
	עוד נבחין כי,
	\[
		{|h(z, w)|}^2
		= z \overline{w} w \overline{z} + \frac{1}{4} {|z|}^4 - \frac{1}{2} {|zw|}^2 + \frac{1}{4} {|w|}^4
		= \frac{1}{4} {|z|}^4 + \frac{1}{2} {|zw|}^2 + \frac{1}{4} {|w|}^4
		= {\left(\frac{1}{2} ({|z|}^2 + {|w|}^2)\right)}^2
		= {\left(\frac{1}{2} \cdot 1\right)}^2
	\]
	ונובע שתמונת $h$ היא ב־$S^2(\frac{1}{2})$ בלבד.

	נעבור לבדיקת רגולריות.
	נגזור את איברי המספרים המרוכבים,
	\[
		D h |_{(z, w)}
		= D h |_{(z_r, z_i, w_r, w_i)}
		= \begin{pmatrix}
			w_r & - w_i & z_r & -z_i \\
			w_i & w_r & z_i & z_r \\
			\frac{1}{2} z_r & \frac{1}{2} z_i & \frac{1}{2} w_r & \frac{1}{2} w_i
		\end{pmatrix} 
		= \begin{pmatrix}
			\overline{w} & \overline{z} \\
			i \overline{w} & i \overline{z} \\
			\frac{1}{2} z & \frac{1}{2} w
		\end{pmatrix} 
	\]
	והנקודה $(z, w)$ רגולרית אם ההעתקה היא על $\RR^2 \simeq \CC \simeq T_{h(z, w)}(S^2(\frac{1}{2}))$.
	אבל נבחין כי עמודותיה בלתי־תלויות לינארית עבור כל $(z, w) \ne 0$, וכן $0 \notin S^3(1)$, ולכן נסיק שהיא רגולרית בכל נקודה.
	נובע אם כך שלכל $q \in S^2(\frac{1}{2})$ אכן כל מקורותיה רגולריים וכן היא ערך רגולרי.
\end{proof}

\end{document}
