\input{../article_base.tex}
\title{פתרון מטלה 10 --- אנליזה על יריעות, 80426}
% chktex-file 17
% chktex-file 9

\DeclareMathOperator{\vol}{vol}

\begin{document}
\maketitle
\maketitleprint[blue]

\question{}
תהי $M \subseteq \RR^n$ יריעה קומפקטית, ונניח ש־$f : M \to \RR$ חלקה.
יהי $\xi : M \to \RR^n$ שדה וקטורי.

\subquestion{}
נראה שקיימת סביבה פתוחה $U \subseteq \RR^n$ של $M$ והעתקה חלקה $\tilde{f} : U \to \RR$ כך ש־$\tilde{f} |_{M} = f$.
\begin{proof}
	לכל $x \in M$ קיימת פרמטריזציה מקומית של $M$, כלומר $\alpha_x : U_x \to M$ חלקה, ויש לה הרחבה חלקה לקבוצה פתוחה $x \in V_x \subseteq \RR^n$.
	מהפיכות משמאל נסיק שגם $f |_{U_x}$ ניתנת להרחבה, ונסמן $g_x$ כהרחבה זו.
	$M$ קומפקטית ו־$M \subseteq \bigcup_{x \in M} U_x$ ולכן קיים תת־כיסוי סופי $M \subseteq \bigcup_{i = 1}^N U_x$.
	זוהי משפחת קבוצות פתוחות חסומה ולכן יש לה חלוקת יחידה ${\{ h_i \}}_{i = 1}^N$, ונגדיר את הפונקציה $\tilde{f} : U \to \RR$ עבור $U = \bigcup_{i = 1}^N U_{x_i}$ על־ידי,
	\[
		\tilde{f}(x)
		= \sum_{i = 1}^N h_i(x) g_{x_i}(x)
	\]
	ונקבל ש־$\tilde{f} |_M = f$ ישירות מהגדרתה מחלוקת היחידה והשוויון בסביבות המקומיות.
\end{proof}

\subquestion{}
נסיק שקיימת סביבה פתוחה $U \subseteq \RR^n$ של $M$ ושדה וקטורי $\tilde{\xi} : U \to \RR^n$ כך ש־$\tilde{\xi} |_M = \xi$.
\begin{proof}
	נבחין כי קיים פירוק $\xi(x) = (\xi_1(x), \ldots, \xi_n(x))$ כך ש־$\xi_i : M \to \RR$ חלקה לכל $1 \le i \le n$.
	מהסעיף הקודם נובע שקיימת סביבה פתוחה $U_i$ כך ש־$\xi_i$ ניתנת להרחבה חלקה ל־$U_i$.
	עתה נבחר $U = \bigcap_{i = 1}^n U_i$ ונקבל קבוצה פתוחה ב־$\RR^n$ כך ש־$\xi$ ניתנת להרחבה קורדינטה קורדינטה בה לפונקציה $\tilde{\xi} : U \to \RR^n$. \\
	נבחין כי טענה זו נכונה רק עבור מימד סופי.
\end{proof}

\subquestion{}
נראה שקיימות הרחבות $F : \RR^n \to \RR, X : \RR^n \to \RR^n$ המרחיבות את $f, \xi$ בהתאמה.
\begin{proof}
	בסעיף א' הגדרנו חלוקת יחידה של קבוצה פתוחה $M \subseteq U$.
	עתה נגדיר חלוקת יחידה של $\RR^n$ על־ידי הרחבת חלוקת היחידה שהגדרנו בסעיף א', יחד עם הקבוצה $W = \RR^n \setminus C$ עבור $C \subseteq U$ סגורה כך ש־$M \subsetneq C$.
	נגדיר את הפונקציה $F$ על־ידי,
	\[
		F(x)
		= \left( \sum_{i = 1}^N \tilde{f} |_{U_i}(x) \alpha_i(x) \right) + \alpha_W(x) \cdot 0
	\]
	ונקבל פונקציה חלקה כך שהיא מתאפסת מחוץ לסביבה של $U$.
	התהליך עבור $X$ זהה.
\end{proof}

\question{}
\subquestion{}
יהיו $0 < a < b < \pi$,
ונגדיר את היריעה $N_{a, b} = S^{n - 1} \times [a, b] \subseteq \RR^{n + 1}$. \\
נראה ש־$\varphi : N_{a, b} \to S^n$ הנתונה על־ידי $\varphi(x, y) = (x \sin y, \cos y)$ דיפאומורפיזם על תמונתה, כאשר $x \in \RR^{n - 2}, y \in \RR$.
\begin{proof}
	מתקיים,
	\[
		{\lVert \varphi(x, y) \rVert}^2
		= {\lVert (x \sin y, \cos y) \rVert}^2
		= {\lVert x \rVert}^2 \sin^2 y + \cos^2 y
		= 1
	\]
	ישירות מהעובדה כי $x \in S^{n - 2}$.
	לכן $\varphi(x, y) \in S^n$ והפונקציה מוגדרת היטב.
	זוהי גם פונקציה חלקה כהרכבת חלקות, ונותר לבדוק חד־חד ערכיות, ממשפט הפונקציה ההפוכה נסיק דיפאומורפיזם לתמונה. \\
	נניח ש־$x \ne x', y \ne y'$, אם $\cos y \ne \cos y'$ אז סיימנו, אחרת $y = y' = \frac{\pi}{2}$ בלבד (ישירות מתחום ההגדרה).
	במקרה זה נקבל $\sin y = \sin y' = 1$ ולכן $x \sin y \ne x' \sin y'$ וקיבלנו חד־חד ערכיות.
	אם גם $x = -x'$
\end{proof}

\subquestion{}
נסמן $M_{a, b} = \varphi(N_{a, b})$.
נראה שאם $f : S^n \to \RR$ רציפה אז,
\[
	\int_{M_{a, b}} f(x)\ dx
	= \int_{a}^{b} {(\sin y)}^{n - 1} \int_{S^{n - 1}} f(x \sin y, \cos y)\ d\vol_{n - 1}(x)\ dy
\]
\begin{proof}
	ישירות מהגדרת אינטגרל, שימוש ב־$\varphi$ ומשפט פוביני נקבל,
	\[
		\int_{M_{a, b}} f(x)\ dx
		= \int_{a}^{b} \int_{S^{n - 1}} f(x \sin y, \cos y) V(D \varphi |_{(x, y)})\ d\vol_{n - 1}(x)\ dy
	\]
	ולכן מספיק שנראה ש־$V(D \varphi |_{(x, y)}) = {(\sin y)}^{n - 1}$.
	נובע ישירות מחישוב.
\end{proof}

\subquestion{}
נסמן $N = S^{n - 1} \times (0, \pi)$ ונגדיר $\varphi : N \to S^n$ כמקודם.
נראה ש־$S^n \setminus \varphi(N)$ ממידה 0.
\begin{proof}
	נבחין ש־$\varphi(S^{n - 1} \times [0, \pi]) = S^n$ (אילו היינו מרחיבים את ההגדרה) וכן,
	\[
		L
		= \varphi(S^{n - 1} \times \{0, \pi\})
		= \varphi(S^{n - 1} \times \{ 0 \})
		= \{ (x \sin 0, \cos 0) \mid x \in S^{n - 2} \}
		= S^{n - 2} \times \{ 0 \}
	\]
	כלומר $S^n \setminus \varphi(N) = L$ עבור $L$ יריעה ממימד $n - 1$, ובפרט קבוצה ממידה 0.
\end{proof}

\subquestion{}
נסיק שאם $f : S^n \to \RR$ רציפה אז,
\[
	\int_{M_{a, b}} f(x)\ dx
	= \int_{a}^{b} {(\sin y)}^{n - 1} \int_{S^{n - 1}} f(x \sin y, \cos y)\ d\vol_{n - 1}(x)\ dy
\]
\begin{proof}
	מתקבל באופן זהה לחלוטין לסעיף ב'.
\end{proof}

\question[4]
\subquestion{}
תהי $M^k \subseteq \RR^n$ יריעה קומפקטית עם שפה.
נניח ש־$x \in \RR^n \setminus M$ נקודה ו־$p \in M$ כך שמתקיים,
\[
	\lVert p - x \rVert
	= \min_{q \in M} \lVert q - x \rVert
\]
נראה ש־$p - x$ היא אנכית ל־$T_p M$.
\begin{proof}
	אנו רוצים להראות שלכל $v \in T_p M$ מתקיים $\langle v, p - x \rangle = 0$.
	באופן שקול נרצה להראות שלכל מסילה $\gamma : (-\delta, \delta) \to M$ כך ש־$\gamma(0) = p$ מתקיים $\langle \gamma'(0), v - x \rangle = 0$.
	נניח שקיימת מסילה כזו כך שמתקבל ערך שונה מאפס.
	אז מהגדרת מכפלה פנימית קיים $0 < \varepsilon < \delta$ (בלי הגבלת הכלליות) כך ש־$\langle \gamma(\varepsilon), v - x \rangle < \langle \gamma(0), v - x \rangle$, וזו סתירה להגדרת $\gamma(0) = p$.
\end{proof}

\subquestion{}
נמצא דוגמה ליריעה עם שפה $M^k \subseteq \RR^n$ כך שהטענה שהוכחנו זה עתה לא חלה בה.
\begin{solution}
	נבחר את חצי הספירה $\HH^n \cap S^{n + 1}$ ואת הנקודה $x = (1, 0, \ldots, -\epsilon)$ נקודה קרובה מאוד לקוטב.
	בבירור עבור המישור המשיק המורחב נקבל ש־$p - x$ נמצאת במישור המשיק, אבל אם נבחן את המישור המשיק עם התייחסות למשיק חד־צדדי בשפה נקבל שאכן יש אנכיות.
\end{solution}

\subquestion{}
במטלה קודמת ראינו כי אם $M = \partial U$ יריעה חלקה המהווה שפה עבור קבוצה פתוחה, קיים $\varepsilon > 0$ כך ש־$\varphi(t, p) = t \nu(p) + p$ דיפאומורפיזם. \\
עתה נראה כי לכל $0 < \delta < \varepsilon$ מתקיים,
\[
	\varphi([-\delta, 0] \times M)
	= \{ x \in \overline{U} \mid \operatorname{dist}(x, M) \le \delta \}
\]
\begin{proof}
	יהי $0 < \delta < \varepsilon$, ותהי $p \in M$, כלומר $p \in \partial U$ ובפרט $p \in \overline{U}$.
	אז נובע ישירות ש־$p \in \varphi([-\delta, 0] \times M)$, ועתה נגדיר,
	\[
		x = \varphi(-\delta, p) = p - \delta \nu(p)
	\]
	אז מתקיים מהגדרת המרחק,
	\[
		\operatorname{dist}(x, M)
		\le \operatorname{dist}(x, p)
		= \langle x, p \rangle
		= \lVert \delta \nu(p) \rVert
		= \delta \cdot 1
	\]
	ולכן בפרט $x \in \varphi([-\delta, 0] \times M)$ וסיימנו על־ידי בחירת $\delta' < \delta$.
\end{proof}

\subquestion{}
נארה ש־$V(D \varphi |_{(0, x)}) = 1$ לכל $x \in M$.
\begin{proof}
	נניח בלי הגבלת הכלליות שאנו עובדים בבסיס כך שרק המימדים הראשונים תלויים במישור המשיק של $M$ ב־$x$.
	במקרה זה,
	\[
		D \varphi |_{(t, x)}
		I_n + \begin{pmatrix}
			t I_k &  \\
			 & 0
		\end{pmatrix} 
	\]
	ולכן בפרט $D \varphi |_{(0, x)} = I_n$ לכל $x$, ומפה הטענה נובעת ישירות.
\end{proof}

\subquestion{}
נסיק שמתקיים,
\[
	\vol_{n - 1}(M)
	= \lim_{\delta \to 0} \frac{1}{\delta} \vol_n(N_{\delta})
	= - \frac{d}{d \delta} |_{\delta = 0} \vol_n(N_{\delta})
\]
\begin{proof}
	
	\[
		\vol_{n - 1}(M)
		= \sum_{i = 1}^N \int_{U_i} \varphi_i(\alpha_i(x)) \cdot 1 \cdot D(V 1)\ dx
		= \sum_{i = 1}^N \int_{U_i} \varphi_i(\alpha_i(x))\ dx
	\]
	מהצד השני,
	\[
		\frac{1}{\delta} \vol_n(N_{\delta})
		= \sum_{i = 1}^N \int_{U_i} \varphi_i(\varphi([-\delta, 0] \times \alpha_i(x)))\ dx 
		= \sum_{i = 1}^N \int_{U_i} \delta \varphi_i(\alpha_i(x))\ dx 
	\]
	מפוביני ולכן,
	\[
		\lim_{\delta \to 0} \frac{1}{\delta} \vol_n(N_{\delta})
		= \sum_{i = 1}^N \int_{U_i} \varphi_i(\alpha_i(x))\ dx
	\]
	וזאת הנגזרת ישירות מההגדרה.
\end{proof}

\subquestion{}
נראה שהטענה שמצאנו זה עתה מתקיימת עבור מעגלים וספירות.
\begin{solution}
	עבור מעגל $s = 2 \pi$ מצד אחד, ומהצד השני,
	\[
		- \frac{d}{d \delta} |_{\delta = 0} \vol_n(N_{\delta})
		= - \frac{d}{d \delta} |_{\delta = 0} (\pi - \pi \delta^2)
		= 2 \pi \delta |_{\delta = 0}
		= 2 \pi
	\]
	והחישוב לספירה דומה.
\end{solution}

\end{document}
