\input{../article_base.tex}
\title{פתרון מטלה 04 --- אנליזה על יריעות, 80426}

\DeclareMathOperator{\vol}{vol}

\begin{document}
\maketitle
\maketitleprint{}

\question{}
תהי $U = {(0, 1)}^2$ ותהי ההעתקה $\varphi : U \to \RR^3$ המוגדרת על־ידי $\varphi(x, y) = (x, y, e^{x + y})$. \\
נסמן $X = \im \varphi$ וב־$d \vol_2$ את אלמנט הנפח, נחשב את,
\[
	\int_X \sqrt{2z^2 + 1}\ d\vol_2
\]
\begin{solution}
	אם נסמן $f(x, y, z) = \sqrt{2z^2 + 1}$ אז נקבל שמתקיים,
	\[
		\int_X \sqrt{2z^2 + 1}\ d\vol_2
		= \int_U f(\varphi(u)) V(D \varphi \mid_u)\ du
	\]
	נחשב את הנגזרת של $\varphi$,
	\[
		D\varphi
		= \begin{pmatrix}
			1 & 0 \\
			0 & 1 \\
			e^{x + y} & e^{x + y}
		\end{pmatrix} 
	\]
	מתוך כוונה להשתמש בלמה לחישוב $V(D \varphi)$ נחשב,
	\[
		V(D\varphi)
		= \sqrt{{(D\varphi)}^t D \varphi}
		= \sqrt{\begin{vmatrix}
				1 + e^{{(x + y)}^2} & e^{{(x + y)}^2} \\
				e^{{(x + y)}^2} & 1 + e^{{(x + y)}^2}
		\end{vmatrix}}
		= \sqrt{1 + 2e^{{(x + y)}^2}}
	\]
	ולכן נציב,
	\[
		\int_{(0, 1)} \int_{(0, 1)} \sqrt{2 e^{{(x + y)}^2} + 1} \cdot \sqrt{1 + 2 e^{{(x + y)}^2}}\ dx\ dy
		= \int_{(0, 1)} \int_{(0, 1)} 2 e^{{(x + y)}^2} + 1\ dx\ dy
	\]
	ואין לי מושג איך לחשב את זה.
\end{solution}

\question{}
\subquestion{}
תהי פונקציה $f : [a, b] \to (0, \infty)$ גזירה ברציפות, ויהי גוף הסיבוב שלה,
\[
	\Sigma_f
	= \{ (x, y, z) \in \RR^3 \mid z \in [a, b], \sqrt{x^2 + y^2} = f(z) \}
\]
נראה ש־$(\varphi, \Sigma_f)$ היא יריעה דו־מימדית, כאשר,
\[
	\varphi(t, z)
	= (f(z) \cos t, f(z) \sin t, z)
\]
עבור $(t, z) \in [0, 2\pi] \times [a, b] = K$.
\begin{proof}
	נבחין כי הקבוצה עליה $\varphi$ מוגדרת היא קבוצה סגורה (למעשה דיסק סגור), בסתירה להגדרה של יריעה, כשנדבר על היריעה נוכל להתכוון לדיסק הפתוח ובכך לפתור את הבעיה.
	עלינו להראות ש־$\varphi(K) = \Sigma_f$.
	\[
		(x, y, z) = \varphi(t, z)
		\implies \sqrt{x^2 + y^2}
		= \sqrt{f^2(z) \cos^2 t + f^2(z) + f^2(z) \sin^2 t}
		= \sqrt{f^z(z)}
		= f(z)
	\]
	כאשר המהלך האחרון נובע מהעובדה ש־$f$ חיובית לחלוטין.
	נובע ש־$\varphi(K) \subseteq \Sigma_f$.
	תהי $(x, y, z) \in \Sigma_f$, אז נבחר $t = \Arg(x + iy)$ ונקבל מהגדרת הארגומנט ש־$\varphi(t, z) = (x, y, z)$, לכן גם $\varphi(K) \supseteq \Sigma_f$, ונסיק $\varphi(K) = \Sigma_f$ כפי שרצינו.
\end{proof}
נחשב את $\vol_2(\Sigma_f)$.
\begin{solution}
	נחשב את $V(D\varphi)$,
	\[
		D\varphi
		= \begin{pmatrix}
			- f(z) \sin t & f'(z) \cos t \\
			f(z) \cos t & f'(z) \sin t \\
			0 & 1
		\end{pmatrix} 
	\]
	ולכן,
	\begin{align*}
		V(D\varphi)
		& = \sqrt{\begin{vmatrix}
				f^2(z) \sin^2 t + f^2(z) \cos^2 t & -f(z) f'(z) \sin t \cos t + f(z) f'(z) \sin t \cos t \\
				-f(z) f'(z) \sin t \cos t + f(z) f'(z) \sin t \cos t & {(f'(z))}^2 \cos^2 t + {(f'(z))}^2 \sin^2 t + 1
		\end{vmatrix}} \\
		& = \sqrt{\begin{vmatrix}
				f^2(z) & 0 \\
				0 & {(f'(z))}^2 + 1
		\end{vmatrix}}
		= f(z) \sqrt{{(f'(z))}^2 + 1}
	\end{align*}
	מהגדרת הנפח,
	\[
		\vol(\Sigma_f)
		= \int_K V(D \varphi \mid_u)\ du
		= 2\pi \int_a^b f(z) \sqrt{{(f'(z))}^2 + 1}\ dz
	\]
\end{solution}

\subquestion{}
נחשב את,
\[
	\int_{\Sigma_f} x^2 + y^2\ d\vol_2
\]
\begin{solution}
	אנו כבר יודעים את ערך $V(D \varphi)$ ולכן נותר להציב,
	\[
		\int_{\Sigma_f} x^2 + y^2\ d\vol_2
		= \int_K f^2(z) \cdot f(z) \sqrt{{(f'(z))}^2 + 1}\ dz\ dt
		= 2\pi \int_a^b f^3(z) \sqrt{{(f'(z))}^2 + 1}\ dz
	\]
\end{solution}

\question{}
\subquestion{}
נחשב את מרכז המסה של,
\[
	S_+^2
	= \{ (x, y, z) \in \RR^3 \mid x^2 + y^2 + z^2 = 1, z, y, z \ge 0 \}
\]
\begin{solution}
	נתחיל בחישוב השטח של העקומה,
	נגדיר $\varphi(x, y) = (x, y, \sqrt{1 - x^2 - y^2})$,
	נחשב את $D\varphi$,
	\[
		D\varphi
		= \begin{pmatrix}
			1 & 0 \\
			0 & 1 \\
			\frac{-x}{\sqrt{1 - x^2 - y^2}} & \frac{-y}{\sqrt{1 - x^2 - y^2}}
		\end{pmatrix}
	\]
	ובהתאם נקבל שגם,
	\[
		V(D\varphi)
		= \sqrt{{(D\varphi)}^t D\varphi}
		= \sqrt{\begin{vmatrix}
				1 + \frac{x^2}{1 - x^2 - y^2} & \frac{xy}{1 - x^2 - y^2} \\
				\frac{xy}{1 - x^2 - y^2} & 1 + \frac{y^2}{1 - x^2 - y^2}
		\end{vmatrix}}
		= \frac{1}{\sqrt{1 - x^2 - y^2}}
	\]
	השטח מתקבל אם כך על־ידי,
	\[
		\vol_2(S_+^2)
		= \int_0^1 \int_0^{\sqrt{1 - x^2}} \frac{1}{\sqrt{1 - x^2 - y^2}}\ dy\ dx
		= \frac{\pi}{4} \int_0^1 \frac{1}{\sqrt{1 - r^2}} \cdot r\ dr
		= \frac{\pi}{4} \int_{1}^{0} \frac{1}{\sqrt{t}}\ dt
		= \frac{\pi}{4} 2 \sqrt{1}
		= \frac{\pi}{2}
	\]
	מטעמי סימטריה מספיק שנבדוק את אחד מהצירים בלבד, נבחר את ציר ה־$x$.
	אנו יודעים ש־$x_{cm} = \frac{1}{\vol_2(S_+^2)} \int_{S_+^2} x\ d\vol_2$, ולכן,
	\begin{align*}
		x_{cm}
		& = \frac{2}{\pi} \int_0^1 \int_0^{\sqrt{1 - x^2}} \frac{x}{\sqrt{1 - x^2 - y^2}}\ dy\ dx \\
		& = \frac{2}{\pi} \int_{0}^{\frac{\pi}{2}} \int_{0}^{1} r \cdot \frac{r \cos \theta}{\sqrt{1 - r^2}}\ dr\ d\theta \\
		& = \frac{2}{\pi} {\left[ \sin \theta \right]}_{\theta = 0}^{\theta = \frac{\pi}{2}} \cdot \int_0^1 \frac{r^2}{\sqrt{1 - r^2}}\ dr \\
		& = \frac{2}{\pi} \cdot 1 \cdot \frac{\pi}{4} \\
		& = \frac{1}{2}
	\end{align*}
	ולכן מרכז המסה הוא $(\frac{1}{2}, \frac{1}{2}, \frac{1}{2})$.
\end{solution}

\subquestion{}
נראה שאם $M, N \subseteq \RR^n$ יריעות פרמטריות זרות עם אותו מימד,
אז $M \cup N$ היא יריעה פרמטרית כך שמרכז המסה שלה הוא הממוצע המשוכלל,
\[
	\vec{c}_{M \cup N}
	= \frac{\vol(M) \vec{c}_M + \vol(N) \vec{c}_N}{\vol(M) + \vol(N)}
\]
\begin{proof}
	נניח ש־$\varphi_M, \varphi_N$ הפרמטריזציה של היריעות $M, N$ בהתאמה.
	נרצה לבחור פרמטריזציה $\varphi : U \to \RR^d$ כך ש־$\varphi(U) = M \cup N$, אך $\dom \varphi_M, \dom \varphi_N$ לא בהכרח זרות.

	נוכיח אם כן שקיים דיפאומורפיזם $\psi : \RR^d \to {(-1, 1)}^d$.
	נגדיר $\psi_i(x_i) = \frac{2}{\pi} \arctan x_i$ ונקבל מתכונות $\tan$ שאכן $\psi$ פונקציה רציפה, הפיכה וכן מגזירות $\tan$ שהיא דיפאומורפיזם.

	נוכל אם כן להניח ש־$\dom \varphi_M \cap \dom \varphi_N = \emptyset$, שאם לא כן, נוכל לבחור $\psi \circ \varphi_M$ ו־$1^d + \psi \circ \varphi_N$ כאשר $1^d = (1, \ldots, 1)$.

	נסיק שקיימים תחומים זרים $U_M, U_N$ כך ש־$\varphi \restriction U_M = \varphi_M, \varphi \restriction U_N = \varphi_N$.
	בהתאם נוכל לחשב את נפח $N \cup M$,
	\begin{align*}
		\vol_d(M \cup N)
		& = \int_{U_N \uplus U_M} V(D\varphi)\ d\vol_d u \\
		& = \int_{U_N} V(D\varphi)\ d\vol_d + \int_{U_M} V(D\varphi)\ d\vol_d u \\
		& = \int_{U_N} V(D\varphi_N)\ d\vol_d + \int_{U_M} V(D\varphi_M)\ d\vol_d u \\
		& = \vol_d(M) + \vol_d(N)
	\end{align*}
	נעבור לחישוב מרכז המסה $\vec{c}_{M \cup N}$,
	\[
		\vec{c}_{M \cup N}
		= \frac{1}{\vol_d(M \cup N)} \int_{M \uplus N} \vec{u}\ d\vol_d \vec{u}
		= \frac{1}{\vol_d(M \cup N)} \left( \int_{M} \vec{u}\ d\vol_d \vec{u} + \int_{M} \vec{u}\ d\vol_d \vec{u} \right)
		= \frac{\vol(M) \vec{c}_M + \vol(N) \vec{c}_N}{\vol(M) + \vol(N)}
	\]
	כאשר המעבר האחרון נובע ישירות מההגדרה של $\vec{c}_M$ ו־$\vec{c}_N$.
\end{proof}

\subquestion{}
נחשב את מרכז המסה של החרוט הסגור,
\[
	\{ \sqrt{x^2 + y^2} = 1 - z \mid 0 < z < 1 \} \cup \{ x^2 + y^2 \le 1, z = 0 \}
\]
\begin{solution}
	נבחין כי החרוט הסגור אינו אלא יריעה פרמטרית של חרוט פתוח ושל עיגול, שתיהן יריעות ממימד 2 ב־$\RR^3$.
	נסמן את החרוט ב־$M$ ואת העיגול ב־$N$, וכן נבחין שמטעמי סימטריה $\vec{c}_N = 0$, וש־$\vol_2(N) = \pi$.
	נעבור אם כך לחישוב הנפח של $M$.
	נגדיר $\varphi : B_1(0) \to M$ על־ידי $\varphi(x, y) = (x, y, 1 - \sqrt{x^2 + y^2})$.
	נעבור לחישובים הכרחיים עבור חישוב האינטגרל,
	\[
		D\varphi
		= \begin{pmatrix}
			1 & 0 \\
			0 & 1 \\
			-\frac{x}{\sqrt{x^2 + y^2}} & -\frac{y}{\sqrt{x^2 + y^2}}
		\end{pmatrix}
	\]
	וכן,
	\[
		V(D\varphi)
		= \sqrt{\begin{vmatrix}
				1 + \frac{x^2}{x^2 + y^2} & \frac{xy}{x^2 + y^2} \\
				\frac{xy}{x^2 + y^2} & 1 + \frac{y^2}{x^2 + y^2}
		\end{vmatrix}}
		= \sqrt{1 + \frac{x^2 + y^2}{x^2 + y^2}}
		= \sqrt{2}
	\]
	ועתה נעבור לחישוב הנפח,
	\[
		\vol_2(M)
		= \int_{B_1(0)} V(D\varphi \mid_u)\ d\vol_2 u
		= \int_{B_1(0)} \sqrt{2}\ du
		= \sqrt{2} \cdot \pi
	\]
	נחשב את מרכז המסה, נבחין שמטעמי סימטריה $c_M^x = c_M^y = 0$ ועלינו לחשב רק את ציר ה־$z$.
	\[
		c_M^z
		= \frac{1}{\sqrt{2} \pi} \int_M z\ d\vol_2 u
		= \frac{1}{\pi} \int_{B(0, 1)} 1 - \sqrt{x^2 + y^2}\ dx\ dy
		= \frac{1}{\pi} (\pi - \int_{0}^{2 \pi} \int_{0}^{1} r^2\ dr\ d \theta)
		= \frac{1}{\pi} (\pi - 2 \pi \cdot \frac{1}{3})
		= \frac{1}{3}
	\]
	ולבסוף נשתמש בתוצאת סעיף ב' ונקבל,
	\[
		\vec{c}_{M \cup N}
		= \frac{\sqrt{2}\pi \cdot (0, 0, \frac{1}{3}) + \pi (0, 0, 0)}{\sqrt{2}\pi + \pi}
		= \left(0, 0, \frac{\sqrt{2}}{3(\sqrt{2} + 1)}\right)
	\]
\end{solution}

\end{document}
