\input{../article_base.tex}
\title{פתרון מטלה 11 --- אנליזה על יריעות, 80426}
% chktex-file 17
% chktex-file 9

\DeclareMathOperator{\vol}{vol}
\DeclareMathOperator{\Div}{div}

\begin{document}
\maketitle
\maketitleprint[blue]

\question{}
תהי $M^k \subseteq \RR^n$ יריעה ו־$f : M \to \RR$ פונקציה חלקה ו־$X : M \to \RR^n$ שדה וקטורי חלק.

\subquestion{}
נראה שמתקיים,
\[
	\nabla f(p)
	= \sum_{i = 1}^l df_p(E_i) E_i
\]
כאשר ${\{ E_i \}}_{i = 1}^l$ הוא איזשהו בסיס אורתונורמלי ל־$T_p(M)$, כאשר $\dim T_p(M) = l \le k$.
\begin{proof}
	מהגדרה,
	\[
		\nabla f(p)
		= \sum_{i = 1}^k df_p(e_i) e_i
	\]
	עבור ${\{ e_i \}}_{i = 1}^k$ הבסיס הסטנדרטי ל־$\RR^k$.
	נניח שמתקיים,
	\[
		E_i = \sum_{j = 1}^k \alpha_j^i e_j
	\]
	לכל $i$.
	ישירות מהגדרת נגזרת והעובדה ש־$\alpha_i^j = 0$ לכל $l < i \le k$ ולכל $j$,
	\[
		\sum_{i = 1}^l df_p(E_i) E_i
		= \sum_{i = 1}^l \sum_{j = 1}^k df_p(\alpha_j^i e_j) \alpha_j^i e_j
		= \sum_{j = 1}^k \sum_{i = 1}^l df_p(\alpha_j^i e_j) \alpha_j^i e_j
		= \sum_{j = 1}^k df_p(e_j) e_j
	\]
	כאשר המעבר האחרון נובע מהנתון כי $\{ E_i \}$ בסיס אורתונורמלי.
\end{proof}

\subquestion{}
נראה שמתקיים,
\[
	\Div_M(f X)
	= f \Div_M X + \langle \nabla f, X \rangle
\]
\begin{proof}
	ניזכר בהגדרה,
	\[
		\Div_M(f(p) X(p))
		= \sum_{i = 1}^k \langle D_{e_i} f X |_p, e_i \rangle
		= \sum_{i = 1}^k \langle f_{e_i}(p) X(p) + f(p) D_{e_i} X |_p, e_i \rangle
	\]
	מהרחבה לדיפרנציאל של מכפלת פונקציות מקורסים קודמים.
	תוך שימוש בסעיף הקודם ובלי הגבלת הכלליות נוכל להניח שהבסיס פורש את $T_p M$ ולכן,
	\[
		\sum_{i = 1}^k \langle f_{e_i}(p) X(p) + f(p) D_{e_i} X |_p, e_i \rangle
		= \sum_{i = 1}^k \langle f_{e_i}(p) X(p), e_i \rangle + \langle f(p) D_{e_i} X |_p, e_i \rangle
		= \sum_{i = 1}^k f_{e_i}(p) \langle X(p), e_i \rangle + f(p) \langle D_{e_i} X |_p, e_i \rangle
	\]
	אבל ישירות מהגדרת ייצוג לפי בסיסים אורתונורמליים נקבל,
	\[
		\Div_M(f(p) X(p))
		= f(p) \Div_M X(p) + \langle \nabla f(p), X(p) \rangle
	\]
	וקיבלנו את המבוקש.
\end{proof}

\subquestion{}
תהי $N^{k - 1} \subseteq M$ תת־יריעה.
יהי $p \in N$, ונגדיר $\nu \in T_p M$ איזשהו וקטור יחידה נורמלי ל־$T_p N$.
נראה שמתקיים,
\[
	\Div_N X(p)
	= \Div_M X(p) - \langle D X_p(\nu), \nu \rangle
\]
\begin{proof}
	נניח בלי הגבלת הכלליות ש־$\nu = e_k$, מותר לנו להניח כך מסעיף א' (אחרת נבנה בסיס אורתונורמלי של $T_p N$ ונרחיב אותו עם $\nu$).
	מתקיים,
	\[
		\Div_M X(p)
		= \sum_{i = 1}^k \langle D_{e_i} X |_p, e_i \rangle
		= \sum_{i = 1}^{k - 1} \langle D_{e_i} X |_p, e_i \rangle + \langle D_{e_k} X |_p, e_k \rangle 
		= \Div_N X(p) + \langle D_{\nu} X |_p, \nu \rangle 
	\]
	ומעשה זוהי הטענה עצמה.
\end{proof}

\question{}
\subquestion{}
יהי השדה הווקטורי $Y : S^1 \to \RR^2$ המוגדר על־ידי $Y(x) = x$.
נחשב את $\Div_{S^1} Y$ ישירות ובאמצעות הסעיף הקודם.
\begin{solution}
	\textbf{ישירות},
	מתקיים,
	\[
		Y(\cos \alpha, \sin \alpha)
		= (\cos \alpha, \sin \alpha),
		\quad
		D Y |_{(\cos \alpha, \sin \alpha)}
		= \begin{pmatrix}
			-\sin \alpha & 0 \\
			0 & \cos \alpha
		\end{pmatrix}
	\]
	ולכן,
	\[
		\Div_{S^1} Y(p)
		= \sum_{i = 1}^2 \langle D_{e_i} Y |_p, e_i \rangle
		= -\sin \alpha + \cos \alpha
	.\]

	\textbf{שאלה 1},
	היריעה $S^1$ היא חד־מימדית ולכן מספיק שנמצא $\nu \perp T_p S^1$, כמובן שאם,
	\[
		p = (\cos \alpha, \sin \alpha)
	\]
	אז,
	\[
		\nu
		= (-\sin \alpha, \cos \alpha)
	\]
	הוא וקטור נורמלי כזה, ולכן,
	\[
		\Div_{S^1} Y(p)
		= 0 + \langle D X_p(\nu), \nu \rangle
		= -\sin \alpha + \cos \alpha
	\]
\end{solution}

\subquestion{}
ניזכר בדיפאומורפיזם $\varphi : S^{n - 1} \times (0, \pi) \to S^n \setminus \{(0, \ldots, \pm 1)\}$ המוגדר על־ידי,
\[
	\varphi(x, y)
	= (x \sin y, \cos y)
\]
ונגדיר שדה וקטורי $X : S^n \setminus \{(0, \ldots, \pm 1)\} \to \RR^{n + 1}$ על־ידי,
\[
	(X \circ \varphi)(x, y)
	= (- x \sin y \cos y, \sin^2 y)
\]
נחשב את $\Div_{S^n} X$.
\begin{solution}
	השאלה לא מוגדרת היטב
\end{solution}

\question{}
\subquestion{}
תהי $U \subseteq \RR^n$ פתוחה ויהי $X : U \to \RR^n$ שדה וקטורי חלק.
נגדיר $p \in U$ ו־$\varphi_t^X(p) : I_{\max}^{p, X} \to U$ הזרימה שלה.
נגדיר $Y = -X$ ו־$\varphi_t^Y(p) : I_{\max}^{p, Y} \to U$ הזרימה המתאימה.
נראה כי $I_{\max}^{p, Y} = - I_{\max}^{p, X}$ וכן שמתקיים,
\[
	\varphi_t^Y(p)
	= \varphi_{-t}^X(p)
\]
\begin{proof}
	אילו נניח בשלילה ש־$I_{\max}^{p, Y} < - I_{\max}^{p, X}$ שונים (בלי הגבלת הכלליות),
	אז נקבל ש־$\varphi_{-\delta}^Y(p)$ ניתנת להרחבה.
	טענה זו נכונה לשני הכיוונים ולכן נקבל,
	\[
		I_{\max}^{p, Y} \le - I_{\max}^{p, X},
		\quad
		I_{\max}^{p, Y} \ge - I_{\max}^{p, X}
	\]
	ובהתאם שוויון בלבד.
	החלק השני נובע ישירות מהצבה $t_0$ ו־$-t_0$ נקבל את הטענה ממשפט 11.6 בסיכום הקורס.
\end{proof}

\subquestion{}
יהי $X : \RR \to \RR$ שדה וקטורי המוגדר על־ידי $X(x) = x^2$.
נראה שמתקיים,
\[
	\varphi_t(p) = \frac{p}{1 - pt}
\]
ונסיק ש־$I_{\max}^p \ne \RR$ כל תנאי ש־$p \ne 0$.
\begin{proof}
	תהי $p \in \RR$ כלשהי, אז $X(p) = p^2$ ולכן $\varphi_0(p) = p$ ו־$\frac{d}{dt} \varphi_t(p)|_{t = 0} = p^2$.
	מתקיים,
	\[
		\frac{d}{dt} \frac{p}{1 - pt}
		= p \cdot (-p) \cdot (-1) \cdot \frac{1}{{(1 - pt)}^2}
		= {(\varphi_t(p))}^2
		= X(\varphi_t(p))
	\]
	כלומר התנאי מתקיים ולכן זוהי אכן הזרימה.
	ממשפט היחידות זוהי גם ההרחבה היחידה.

	במקרה $p = 0$ נקבל $\varphi_t(0) = 0$ ולכן $I_{\max}^0 = \RR$.
	אחרת, $\varphi_{\frac{1}{p}}(p)$ לא מוגדר, ולכן בפרט $I_{\max}^p \subseteq (-\frac{1}{p}, \frac{1}{p})$.
\end{proof}

\question{}
תהי $M^k \subseteq \RR^n$ יריעה קומפקטית עם שפה, ו־$X : \RR^n \to \RR^n$ שדה וקטורי כך שלכל $p \in M$ מתקיים $X(p) \in T_p(M)$,
ולכל $q \in \partial M$, מתקיים $\langle X(q), \nu(q) \rangle < 0$.
תהי $p \in M$ ותהי ההעתקה $t \mapsto \varphi_t(p)$ עבור $t \in (-\varepsilon, \varepsilon)$ הזרימה של $X$ המתחילה ב־$p$.

\subquestion{}
נראה ש־$\varphi_t(p) \in M$ או שעבור,
\[
	s_0
	= \sup\{ t \in (-\varepsilon, 0] \mid \varphi_t(p) \notin M \}
\]
מתקיים $\varphi_{s_0}(p) \in \partial M$.
\begin{proof}
	עבור $t \in [0, \varepsilon)$ הטענה נובעת ישירות ממשפט שהוכח בהרצאה.
	לכל $s_0 < t$ נקבל ממקומיות ומשאלה 2 שגם מתקיים $\varphi_t(p) \in M$, ולכן נותר לבדוק את $t = s_0$.
	אנו יודעים כי לכל $t < s_0$ מתקיים $\varphi_t(p) \notin M$ וכן מתקיים $\varphi_{s_0}(p) \in M$ מקומפקטיות, ולכן מהגדרה שקולה לשפת יריעה נקבל $s_0 \in \partial M$.
\end{proof}

\subquestion{}
נראה שלכל $t \in (0, \varepsilon)$ מתקיים $\varphi_t(p) \in M \setminus \partial M$.
\begin{proof}
	נובע מהמשפט לגבי יריעות ללא שפה.
	אם ניקח $N = M^\circ$ ו־$\phi_t(p)$ הזרימה על $N$, אז מיחידות נקבל $\varphi \equiv \phi$.
	בהתאם גם $\phi_t(p)$ מוגדר על $[0, \varepsilon)$ מההגדרה.
\end{proof}

\end{document}
