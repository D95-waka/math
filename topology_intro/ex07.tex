\input{../article_base.tex}
\title{פתרון מטלה 07 --- מבוא לטופולוגיה, 80516}
% chktex-file 9
% chktex-file 17

\begin{document}
\maketitle
\maketitleprint[purple]

\question{}
תהי $f : X = [0, 1] \to {[0, 1]}^2 = Y$ רציפה ועל.
נראה ש־$f$ אינה חד־חד ערכית.
\begin{proof}
	נניח בשלילה ש־$f$ גם חד־חד ערכית.
	נבחין כי שני המרחבים הם מרחבי בייר ממשפט הקטגוריה של בייר, כלומר אין לקבוצות מקטגוריה שנייה פנים.
	בנוסף אנו יודעים כי שני המרחבים הם האוסדורף וקומפקטיים ולכן $f$ סגורה ונובע שהיא הומיאומורפיזם.

	נגדיר $Z = \{ \frac{1}{2} \} \times [0, 1] \subseteq Y$, זוהי קבוצה דלילה, זאת שכן $\overline{Z} = Z$ אבל אין בה אף כדור פתוח.
	לכן גם $f^{-1}(Z) \subseteq X$ היא קבוצה דלילה.
	בפרט היא מקטגוריה ראשונה ולכן $X_0 = X \setminus f^{-1}(Z)$ היא מקטגוריה שנייה ובעלת פנים ריק.
	אבל $Y \setminus Z$ מורכבת משני רכיבי קשירות מסילתית ולכן גם $X_0$ מורכבת משני רכיבי קשירות מסילתית ובפרט מורכבת משני קטעים פתוחים ב־$[0, 1]$.
	בפרט הפנים של קטעים אלה לא ריק, וקיבלנו סתירה.
\end{proof}

\question[3]
יהי $X$ מרחב טופולוגי,
$A \subseteq X$ תיקרא $G_{\delta}$ ב־$X$ אם היא חיתוך בן־מניה של פתוחות מ־$X$.

\subquestion{}
יהי $X$ מרחב בייר ותהי $A \subseteq X$ קבוצה $G_{\delta}$ צפופה ב־$X$.
נראה ש־$A$ מרחב בייר.
\begin{proof}
	נניח ש־$B \subseteq A$ פתוחה, אז $B' \subseteq X$ פתוחה כך ש־$B' \cap A = B$.
	אבל בהתאם $B'$ היא קבוצה $G_{\delta}$ ב־$X$.
	עתה נניח ש־$B_{\alpha}$ קבוצות פתוחות וצפופות ב־$A$, אז $B_{\alpha}' \subseteq X$ פתוחות וצפופות ב־$X$ כך ש־$B_{\alpha} = A \cap B_{\alpha}'$, לכל $\alpha \in \NN$.
	$X$ הוא מרחב בייר ולכן $\bigcap_{\alpha \in \NN} B_{\alpha}'$ צפופה ב־$X$ ובפרט $A \cap \bigcap_{\alpha \in \NN} B_{\alpha}' = \bigcap_{\alpha \in \NN} B_{\alpha}$ צפופה ב־$A$, ולכן $A$ מרחב בייר.
\end{proof}

\subquestion{}
נמצא דוגמה למרחב בייר $X$ ו־$A \subseteq X$ קבוצת $G_{\delta}$ שאינה מרחב בייר.
\begin{solution}
	נגדיר $X = (\RR \times (0, \infty)) \cup (\QQ \times \{0\})$.
	המרחב $\HH = \{ (x, y) \in \RR^2 \mid y \ge 0 \}$ הוא מרחב מטרי שלם ולכן בייר.
	הקבוצה $\HH$ כמובן פתוחה וכן לכל רציונלי גם $(q, 0) \in B((q, 1), 1)$ ולכן נוכל להסיק ש־$X$ היא $G_{\delta}$ ב־$\HH$ וצפופה ולכן מרחב בייר.

	נגדיר עתה $A = \QQ \times \{ 0 \}$.
	זוהי קבוצה סגורה ב־$X$ כ־$\partial \HH$, ומאפיון קבוצות סגורות במרחבים מטריים נובע שגם $G_{\delta}$.
	לעומת זאת $A$ היא לא קבוצת בייר כטענה מהתרגול, ולכן מהווה דוגמה כפי שרצינו.
\end{solution}

\question{}
\subquestion{}
יהי $X$ מרחב בייר $T_1$ עם כמות בת־מניה של נקודות.
נראה שקיימת ב־$X$ נקודה מבודדת.
\begin{proof}
	אם קיים יחידון פתוח ב־$X$ אז סיימנו, לכן נניח שאין כאלה.
	יהי $x_0 \in X$, אז לכל $y \ne x_0$ קיימת סביבה פתוחה $y \in U_y \subseteq X$ כך ש־$x \notin U_y$, זאת כנביעה מ־$T_1$.
	מסגירות לאיחודים נסיק ש־$\bigcup_{y \ne x_0} U_y$ קבוצה פתוחה, ולכן $\{ x_0 \}$ קבוצה סגורה.
	אז נובע ש־$\{ x_0 \}$ היא קבוצה דלילה לכל $x_0 \in X$.
	נסיק ש־$X = \bigcup_{x \in X} \{ x \}$ והעובדה ש־$X$ בת־מניה כי היא קבוצה מקטגוריה ראשונה.
	אבל $X$ מרחב בייר ולקבוצות מקטגוריה ראשונה אין פנים, כלומר $X^\circ = X = \emptyset$ בלבד.
	אבל $|X| = \aleph_0$ בסתירה ל־$X = \emptyset$, ולכן נסיק שקיים יחידון פתוח.
\end{proof}

\subquestion{}
נראה ש־$\RR$ אינו איחוד בן־מניה זר של קטעים סגורים וחסומים. \\
כלומר עבור קטעים זרים וחסומים $[a_n, b_n]$ ל־$n \in \NN$ מתקיים $\RR \setminus \bigcup_{n \in \NN} [a_n, b_n] \ne \emptyset$.
\begin{proof}
	נניח בשלילה שקיימים קטעים כאלה, ונגדיר $X = \bigcup_{n \in \NN} \{ a_n, b_n \}$.
	הקבוצה $X$ היא קבוצה בת־מניה ו־$T_1$ מזרות הקטעים.
	זהו גם מרחב מטרי כמרחב המושרה מ־$\RR$, ומדיסקרטיות הוא אף שלם, ולכן ממשפט הקטגוריה של בייר נסיק שהוא מרחב בייר.
	נובע אם כך מסעיף א' שקיימת ב־$X$ נקודה מבודדת.
	נניח בלי הגבלת הכלליות ש־$a_l$ היא הנקודה הזו, לכן נובע שיש סביבה פתוחה של $a_l$ בה אין נקודות $b_n$ לאף $n$, כלומר $(a_l - \epsilon, a_l) \cap [a_l, b_l]$ קבוצה ריקה, ובפרט $a_l - \epsilon$ לא נמצאת באף קטע.
\end{proof}

\end{document}
