\input{../article_base.tex}
\title{פתרון מטלה 04 --- מבוא לטופולוגיה, 80516}
% chktex-file 9
% chktex-file 17

\begin{document}
\maketitle
\maketitleprint{}

\question{}
יהי $(X, \rho)$ מרחב מטרי.
נוכיח כי התנאים הבאים שקולים,
\begin{enumerate}
	\item $X$ מקיים את אקסיומת המנייה השנייה
	\item $X$ הוא מרחב לינדלוף
	\item $X$ ספרבילי
\end{enumerate}
\begin{proof}
	$1 \implies 2$.
	אקסיומת המנייה גוררת את כל אקסיומות המנייה האחרות, לכן בפרט $X$ מרחב לינדלוף.

	$2 \implies 3$.
	נרצה למצוא קבוצה צפופה בת־מנייה ב־$X$.
	נוכל לכסות את המרחב על־ידי כדורי $\epsilon > 0$ לכל נקודה במרחב, לכן בפרט יש כיסוי בן־מניה של כל המרחב על־ידי קבוצות פתוחות לכל $\epsilon$ כזה.
	נניח ש־${\{ B(x_n^m, \epsilon_m) \}}_{n = 1}^\infty$ קבוצה כזו עבור $\epsilon_m = \frac{1}{m}$ לכל $m$.
	נגדיר את הקבוצה $A = \{ x_n^m \mid n, m \in \NN \}$, זוהי כמובן קבוצה בת־מניה, ונטען ש־$A$ צפופה ב־$X$.
	נניח ש־$x \in X$, אם $x \in A$ אז סיימנו, אחרת לכל $m \in \NN$ קיים $n(m) \in \NN$ כך ש־$x \in B(x_{n(m)}^m, \frac{1}{m})$ מהגדרת הכיסויים, ולכן הסדרה ${( x_{n(m)}^m )}_{m = 1}^\infty$ מתכנסת ל־$x$ ולכן $x \in \overline{A}$.
	נסיק אם כך שאכן $A$ צפופה ובת־מניה ומעידה כי $X$ ספרבילי.

	$3 \implies 1$.
	נגדיר את הקבוצה $\Bb = \{ B(a, \frac{1}{n}) \mid a \in A, n \in \NN \}$ עבור $A$ קבוצה צפופה ב־$X$ שקיימת מספרביליות.
	אנו טוענים כי $\Bb$ בסיס בן־מניה ל־$X$.
	$|\Bb| = |\NN \times A| = \aleph_0$, ולכן נשאר להראות כי זהו אכן בסיס לטופולוגיה של $X$, אך למעשה זו טענה שהוכחה בהרצאה, ולכן נוכל להסיק ש־$X$ מקיימת את אקסיומת המנייה השנייה.
\end{proof}

\question{}
\subquestion{}
יהי $X$ מרחב מנייה שנייה ויהי $A \subseteq X$,
נראה שגם $A$ מרחב מנייה שנייה.
\begin{proof}
	יהי $\Bb$ בסיס בן־מניה של $X$,
	ויהי $\Bb_0 = \{ B \cap A \mid B \in \Bb \}$, זוהי קבוצה בת־מנייה, ואנו טוענים שאף בסיס של $A$.
	נניח ש־$U \subseteq A$ קבוצה פתוחה, אז קיימת $V \subseteq X$ כך ש־$U = V \cap A$ וכן $V$ פתוחה ב־$X$.
	בהתאם $V = \bigcup_{\alpha \in I} V_{\alpha}$ עבור $I$ קבוצת אינדקסים כך ש־$V_{\alpha} \in \Bb$ לכל $\alpha \in I$.
	לכן נובע ש־$U = (\bigcup_{\alpha \in I} V_{\alpha}) \cap A = \bigcup_{\alpha \in I} V_{\alpha} \cap A = \bigcup_{\alpha \in I} U_{\alpha}$ עבור $U_{\alpha} \in \Bb_0$.
	נסיק אם כך שאכן $\Bb_0$ בסיס לטופולוגיה של $A$, וראינו כבר כי הוא בן מנייה, ולכן $A$ מרחב מנייה שנייה.
\end{proof}

\subquestion{}
נבחן את המרחב המטרי $X = {[0, 1]}^\NN$ עם המטריקה $\rho(x, y) = \sup \{ |x(n) - y(n)| \mid n \in \NN \}$. \\
נראה שהוא אינו ספרבילי, אינו לינדלוף ואינו מנייה שנייה.
\begin{proof}
	נבחן את הקבוצה $X \supseteq A = \{ f \in X \mid \forall n, f(n) \in \{0, 1\} \}$, כלומר הסדרות שתמונתן חלקית ל־$\{0, 1\}$.
	לכל $x, y \in A$, מתקיים $\rho(x, y) = 0 \iff x = y$, מהגדרת המטריקה אך גם $\rho(x, y) = 1 \iff x \ne y$, כלומר זוהי המטריקה הדיסקרטית, ולכן היא משרה טופולוגיה דיסקרטית.

	נניח בשלילה ש־$X$ מרחב מנייה שנייה (ולכן שקול לשאר ההגדרות כנביעה משאלה 1), ולכן מסעיף א' נובע שגם $A$ מרחב מנייה שנייה.
	לכן קיים בסיס בן־מנייה ל־$A$, אבל כל יחידון צריך להופיע בבסיס, זאת שכן כל איבר בטופולוגיה הוא איחוד של איברי הבסיס, לכן נסיק ש־$A$ הוא הבסיס היחיד של $A$.
	נבחין כי $|A| = 2^\NN > \aleph_0$, ולכן ל־$A$ אין בסיס בן־מנייה בסתירה להיותו מרחב מנייה שנייה.
	נסיק שאכן $X$ איננו מרחב מנייה שנייה, אינו לינדלוף, ואינו ספרבילי.
\end{proof}

\question{}
נגדיר טופולוגיה על $\NN^2$ כך ש־$U$ פתוחה אם ורק אם $(0, 0) \notin U$ או שקיים $n$ כך שלכל $m > n$ הקבוצה $\{ k \in \NN \mid (m, k) \notin U \}$ היא סופית. \\
מרחב זה נקרא מרחב ארנס־פורט, נראה כי הוא לא מרחב מנייה שנייה.
\begin{proof}
	תהי $M \subseteq \NN$, נגדיר את הקבוצה $A = \{ (m, k) \in \NN^2 \mid m \in M, k \in \NN \} \cup \{ (m, k + 1) \in \NN^2 \mid m \notin M, k \in \NN \}$.
	לכל $m \in \NN$ הקבוצה $\{ k \in \NN \mid (m, k) \notin A \} \in \{ \emptyset, \{ 0 \} \}$ ובפרט סופית ולכן $A$ קבוצה פתוחה.
	נגדיר את הפונקציה $f : \Pp(\NN) \to \tau$ עבור $\tau$ הטופולוגיה שהגדרנו על־ידי מיפוי כל קבוצה $M$ לקבוצה $A$ המתאימה לה.
	זוהי פונקציה חד־חד ערכית, זאת שכן אם $M_0 \ne M_1$ אז ללא הגבלת הכלליות $m \in M_0 \setminus M_1$ ואז $(m, 0) \in f(M_0)$ אבל מההגדרה $(m, 0) \notin f(M_1)$.
	נסיק ש־$\aleph_0 < |\Pp(\NN)| \le |\tau|$.

	נניח ש־$\Bb$ בסיס בן־מנייה לנקודה $(0, 0)$.
	לכל $f(M)$ יש קבוצה פתוחה $U \in \Bb$ כך ש־$(0, 0) \in U \subseteq f(M)$.
	$\Bb$ בת־מנייה ולכן קיימת $U$ כך שיש אינסוף ערכי $f(M)$ המכילים את $U$.
	כלומר אם נקבע $A = f(M)$ ונגדיר את הסדרה $A_0 = A, A_1 = A_0 \setminus \{ (1, k) \mid k \in \NN \}, A_2 = \ldots$ אז נקבל ש־$U \subseteq A_i$ לכל $i \in \NN$, בפרט $U \subseteq \bigcap_{i \in \NN} A_i$.
	אבל $\bigcap_{i \in \NN} A_i$ לא מקיימת את התנאי של הסופיות ובפרט לא קבוצה פתוחה, נסיק ש־$U$ אף היא לא פתוחה, בסתירה ל־$U \in \Bb$.
	נסיק שמרחב זה הוא לא מנייה שנייה.
\end{proof}

\question{}
נוכיח שמכפלה סופית של מרחבים ספרביליים היא ספרבילית.
\begin{proof}
	נניח ש־$X, Y$ מרחבים ספרביליים ונבחן את $Z = X \times Y$.
	נניח גם ש־$A \subseteq X, B \subseteq Y$ קבוצות צפופות בנות־מניה המעידות על הספרביליות.
	נגדיר $C = A \times B$, זוהי קבוצה בת־מניה כמכפלה סופית של קבוצות בנות־מניה, אנו נראה שהיא צפופה ב־$Z$.
	ניעזר בטענה שקבוצה היא צפופה אם ורק אם היא חותכת כל קבוצה פתוחה במרחב, ונניח ש־$U = U_X \times U_Y \subseteq Z$ פתוחה, לכן מהגדרת מכפלות סופיות למרחבים טופולוגיים $U_X, U_Y$ פתוחות ב־$X, Y$.
	בהתאם $U_X \cap A, U_Y \cap B$ קבוצות לא ריקות, נניח ש־$x \in X, y \in Y$ איברים המעידים על כך.
	אז $C \cap U = (A \times B) \cap (U_X \cap U_Y) = (A \cap U_X) \times (B \cap U_Y) \ni (x, y)$, כלומר החיתוך של כל קבוצה פתוחה עם $C$ לא ריק, ולכן היא צפופה ב־$Z$.
\end{proof}

\question[6]
נוכיח שטופולוגיית הקופסה על $\RR^\NN$ אינה קשירה.
\begin{proof}
	תהי $A \subseteq \RR^\NN$ קבוצת הפונקציות המתכנסות.
	לכל $f \in A$ נוכל לבחור $f \in \prod_{n = 1}^\infty B_{\frac{1}{n}}(f(n))$ קבוצה פתוחה, ולכן $A$ קבוצה פתוחה.
	כדי להראות שהמרחב איננו קשיר נצטרך להראות שאף קבוצת הפונקציות שלא מתכנסות, קבוצה שזרה ל־$A$, היא קבוצה פתוחה.
	תהי $f \in A^C$, אז נוכל לבחור קבוצה פתוחה באותו אופן, ונבחין שמהגדרת התכנסות כלל הפונקציות בקבוצה הפתוחה שנבנה כלל הפונקציות אכן לא מתכנסות.
	אז $A$ וגם $A^C$ פתוחות וזרות ומכסות את המרחב, ובהתאם המרחב אינו קשיר.
\end{proof}

\question{}
\subquestion{}
נמצא דוגמה למרחב קשיר מקומית שאיננו קשיר.
\begin{solution}
	נגדיר $X = \{0, 1\}$ עם הטופולוגיה הדיסקרטית.
	המרחב איננו קשיר שכן $X = \{ 0 \} \cup \{ 1 \}$ ובמרחב דיסקרטי כל יחידון הוא פתוח.
	מן הצד השני המרחב קשיר מקומית ב־$0$, זאת שכן כל קבוצה כך ש־$A \subseteq X$ וגם $0 \in A$, אז $\{ 0 \} \subseteq A$ וכן $\{ 0 \}$ פתוחה.
\end{solution}

\subquestion{}
לכל $0 < n \in \NN$ נסמן $l_n = \{ (t, \frac{t}{n}) \mid t \in [0, 1] \}$ ונסמן $l = [0, 1] \times \{ 0 \}$.
נגדיר את המרחב $X = l \cup \bigcup_{n > 0} l_n$ ונראה שהוא קשיר אך לא קשיר מקומית.
\begin{proof}
	נבחין כי לכל $n \in \NN$ הקבוצה $l_n$ היא קטע ולכן קשירה כמסקנה מהתרגול.
	גם $l$ הוא קטע ולכן גם הוא קשיר.
	נבחין כי לכל $n \in \NN$, מתקיים $(0, 0) \in l_n$ וכן ש־$(0, 0) \in l$, ולכן מלמת כוכב המרחב הוא קשיר (ואף קשיר מסילתית).

	נראה שהמרחב איננו קשיר מקומית.
	נבחן את הקבוצה הפתוחה $U = B((\frac{1}{2}, \frac{1}{2}), \frac{1}{4}) \cap X$, זוהי קבוצה פתוחה כצמצום קבוצה פתוחה במרחב המטרי $\RR^2$.
	מהצד השני נבחין כי $U = \{ (t, t) \mid t \in (\frac{1}{3}, \frac{2}{3}) \} \cup \{ (t, \frac{t}{2}) \mid t \in (\frac{1}{2}, \frac{7}{10}) \}$.
	כלומר $U$ הוא איחוד קטעים זרים ופתוחים ולכן לא קשיר מקומית.
\end{proof}

\end{document}
