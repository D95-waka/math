\input{../article_base.tex}
\title{פתרון מטלה 11 --- מבוא לטופולוגיה, 80516}
% chktex-file 9
% chktex-file 17

\usepackage{pgfplots}
\pgfplotsset{width=7cm,compat=1.18}

\begin{document}
\maketitle
\maketitleprint[purple]

\question{}
נניח ש־$g : S^2 \to S^2$ רציפה כך שלכל $x \in S^2$ מתקיים $g(x) \ne g(-x)$.
נראה ש־$g$ היא על.
\begin{proof}
	נניח ש־$x_0 \in S^2$ נקודה כך ש־$x_0 \notin g(S^2)$, לכן נוכל לצמצם את טווח $g$, כלומר מתקיים,
	\[
		g : S^2 \to S^2 \setminus \{ x_0 \}
		\simeq \RR^n
	\]
	כלומר אם $h : S^2 \setminus \{ x_0 \} \to \RR^n$ הומיאומורפיזם (שאנו יודעים שקיים) אז $g_0 = h \circ g : S^2 \to \RR^n$ פונקציה רציפה.
	ממשפט בורסוק־אולם נובע שקיימת נקודה $x_1 \in S^2$ כך שמתקיים,
	\[
		g_0(x_1) = g_0(-x_1)
		\iff h(g(x_1)) = h(g(-x_1))
		\iff g(x_1) = g(-x_1)
	\]
	אבל זוהי סתירה ישירה להנחה שלנו, ולכן לא קיימת $x_0$ כזו, כלומר $g$ היא על.
\end{proof}

\question{}
תהי $A \in M_2(\RR)$ מטריצה ממשית עם כניסות אי־שליליות,
נראה כי ל־$A$ יש וקטור עצמי.
\begin{proof}
	תחילה, אם $\det A = 0$ אז בהכרח $0$ ערך עצמי וקיים וקטור לא טריוויאלי בגרעין, ולכן סיימנו.
	נניח אם כן ש־$A$ רגולרית.
	נגדיר,
	\[
		X = \{ (x, y, z) \in \RR^3 \mid x, y, z \ge 0 \}
	.\]
	ונגדיר את ההעתקה $f : S^2 \cap X \to S^2 \cap X$ על־ידי,
	\[
		f(x, y, z)
		= \frac{A {(x, y, z)}^t}{\lVert A {(x, y, z)}^t \rVert}
	\]
	$f$ מוגדרת היטב שכן $A$ רגולרית ואי־שלילית.

	אנו יודעים שיש הומיאומורפיזם $\varphi : S^2 \cap X \to D^1$, ולכן נוכל להגדיר $g : D^1 \to D^1$ על־ידי $g = \varphi \circ f \circ \varphi^{-1}$.
	אנו יודעים כי $g$ רציפה כהרכבת רציפות, ולכן משפט נקודת השבת של בראוור מתקיים ויש נקודת שבת ל־$g$, כלומר קיים $x_0 \in D^1$ כך שמתקיים,
	\[
		g(x_0) = x_0
		\iff f(\varphi^{-1}(x_0)) = \varphi^{-1}(x_0)
	\]
	ואם נסמן $u = \varphi^{-1}(x_0)$ כאשר $u \in S^2 \cap X$, אז,
	\[
		f(u) = u
		\iff A u = \lVert A u \rVert u
	\]
	כלומר $u$ וקטור עצמי של $A$.
\end{proof}

\question{}
בכל סעיף נגדיר מרחב ונקבע האם $S^1$ הוא נסג עיוות שלו.

\subquestion{}
נבחן את הספרה $S^2$.
\begin{solution}
	אנו יודעים כי $\pi_1(S^1) = \ZZ$, וכן שאם יש נסג עיוות מ־$S^2$ ל־$S^1$ אז $\pi_1(S^2) \simeq \pi_1(S^1) = \ZZ$.
	מהצד השני ראינו כי לכל $n \ge 2$ הספירה $S^n$ היא פשוטת קשר ולכן $\pi_1(S^2)$ טריוויאלית, ובפרט לא $\ZZ$.
\end{solution}

\subquestion{}
נגדיר את הצילינדר,
\[
	C
	= \{ (x, y, z) \in \RR^3 \mid x^2 + y^2 = 1 \}
\]
\begin{solution}
	נגדיר את ההעתקה $H : [0, 1] \times C \to S^1$ על־ידי,
	\[
		H(t, (x, y, z))
		= (x, y, (1 - t) \cdot z)
	\]
	ונראה שזהו נסג עיוות. \\
	לכל $(x, y, 0) \in S^1$ מתקיים,
	\[
		H(t, (x, y, 0))
		= (x, y, 0)
	\]
	בנוסף,
	\[
		H(0, (x, y, z))
		= (x, y, (1 - 0) z)
		= (x, y, z)
	\]
	ולבסוף גם,
	\[
		H(1, (x, y, z))
		= (x, y, 0)
		\in S^1
	\]
	כלומר $H$ היא אכן נסג עיוות מ־$C$ ל־$S^1$ (ל־$S^1 \times \{ 0 \}$ אשר הומיאומורפי ל־ $S^1$).
\end{solution}

\end{document}
