\input{../article_base.tex}
\title{מבוא לטופולוגיה --- סיכום}
\setcounter{secnumdepth}{2}
% chktex-file 9
% chktex-file 17

\usepackage{fancyhdr}
\pagestyle{fancy}
\renewcommand{\headrulewidth}{0pt}

\begin{document}
\maketitle
\maketitleprint{}

\tableofcontents

\section{שיעור 1 --- 24.3.2025}
\subsection{מבוא}
בעבר דיברנו על מרחבים מטריים, באינפי 1 מתבוננים ב־$\RR$ והגדרנו את מושג הגבול של סדרות, ולאחריו את המושג של פונקציה רציפה $f : \RR \to \RR$.
ההגדרה הייתה ש־$f$ תיקרא רציפה אם לכל $x \in \RR$ ולכל $\lim_{n \to \infty} x_n = x$ מתקיים $\lim_{n \to \infty} f(x_n) = f(x)$.
באינפי 3 כבר ראינו את המושג הכללי והרחב יותר של רציפות במרחבים מטריים.
ניזכר בהגדרה של מרחב מטרי.
\begin{definition}[מרחב מטרי]
	מרחב מטרי הוא זוג $(X, d)$ כאשר $X$ קבוצה לא ריקה ו־$d : X \times X \to \RR$ פונקציה (הנקראת מטריקה) המקיימת,
	\begin{enumerate}
		\item $d(x, y) = d(y, x)$ לכל $x, y \in X$
		\item $\forall x, y \in X, d(x, y) \ge 0$ וכן $d(x, y) = 0 \iff x = y$
		\item אי־שוויון המשולש, $\forall x, y, z \in X, d(x, y) \le d(x, y) + d(y, z)$
	\end{enumerate}
\end{definition}
\begin{example}
	נראה דוגמות למרחבים מטריים,
	\begin{enumerate}
		\item $\RR$ יחד עם $d(x, y) = |x - y|$
		\item $(\RR^n, d_2)$ המוגדרת על־ידי $d_2(\bar{x}, \bar{y}) = \sqrt{\sum_{i = 1}^{n} {|x_i - y_i|}^2}$
		\item נוכל עבור $\RR^n$ להגדיר את $d_p(\bar{x}, \bar{y}) = {(\sum_{i = 1}^{n} {|x_i - y_i|}^p)}^\frac{1}{p}$ ואת מטריקת אינסוף, $d_\infty(\bar{x}, \bar{y}) = \max_{1 \le i \le n} |x_i - y_i|$
		\item עבור $C([a, b])$ קבוצת הפונקציות הרציפות $[a, b] \to \RR$ עבור $a < b$, ונגדיר את המטריקה $\rho(f, g) = \sup_{x \in [a, b]} |f(x) - g(x)|$
	\end{enumerate}
\end{example}
נראה את ההגדרה הפורמלית של רציפות,
\begin{definition}[רציפות]\label{function_continuous_definition}
	תהי $f : X \to Y$ עבור $(X, d), (Y, \rho)$ מרחבים מטריים, אז נאמר ש־$f$ רציפה אם ורק אם לכל $\epsilon > 0$ קיים $\delta > 0$ כך שאם $d(x', x) < \delta$ אז $\rho(f(x'), f(x)) < \epsilon$.
\end{definition}
אבל יותר קל לדבר במונחים של קבוצות פתוחות.
\begin{definition}[כדור]
	עבור $(X, d)$ מרחב מטרי,
	נסמן $B(r, x) = B_r(x) = \{ z \in X \mid d(x, z) < r \}$.
\end{definition}
\begin{definition}[קבוצה פתוחה]
	יהי $(X, d)$ מרחב מטרי, תת־קבוצה $U \subseteq X$ תיקרא פתוחה אם לכל $x \in U$ קיים $r > 0$ כך ש־$x \in B(x, r) \subseteq U$.
\end{definition}
\begin{definition}[הגדרה שקולה לרציפות]
	$f : X \to Y$ תיקרא רציפה אם לכל $V \subseteq Y$ קבוצה פתוחה ב־$Y$ מתקיים $f^{-1}(V) = \{ x \in X \mid f(x) \in V \}$ קבוצה פתוחה ב־$X$.
\end{definition}
\begin{definition}[טופולוגיה]
	תהי $X$ קבוצה (לא ריקה), \textbf{טופולוגיה} על $X$ היא אוסף $\tau \subseteq \Pp(X)$, כך שמתקיימים התנאים הבאים,
	\begin{enumerate}
		\item $X, \emptyset \in \tau$
		\item $\tau$ סגור לאיחוד, כלומר אם ${\{U_\alpha\}}_{\alpha \in I}$ לקבוצת אינדקסים $I$, כך ש־$\forall \alpha \in I, U_\alpha \in \tau$ אז $\bigcup_{\alpha \in I} U_\alpha \in \tau$
		\item $\tau$ סגור לחיתוכים סופיים, כלומר לכל $U, V \in \tau$ מתקיים $U \cap V \in \tau$
	\end{enumerate}
\end{definition}
\begin{definition}[מרחב טופולוגי]
	זוג $(X, \tau)$ כאשר $X$ קבוצה לא ריקה ו־$\tau$ טופולוגיה על $X$, יקרא מרחב טופולוגי.
\end{definition}
\begin{remark}
	בעצם הגדרנו כבר מתי פונקציה $f : X \to Y$ עבור מרחבים טופולוגיים $(X, \tau), (Y, \Omega)$ היא רציפה, כאשר $f^{-1}(U) \in \tau$ לכל $U \in \Omega$.
\end{remark}
\begin{notation}
	איברי $\tau$ יקראו קבוצות פתוחות.
\end{notation}
\begin{definition}[קבוצה סגורה]
	אם $(X, \tau)$ מרחב טופולוגי אז תת־קבוצה $A \subseteq X$ תיקרא סגורה אם $X \setminus A \in \tau$, כלומר המשלים של $A$ היא קבוצה פתוחה.
\end{definition}
\begin{example}
	יהי $(X, d)$ מרחב מטרי, נגדיר $\tau = \{ U \subseteq X \mid \forall x \in U \exists r > 0, B(x, r) \subseteq U \}$, כלומר נגדיר טופולוגיה באופן טריוויאלי כנביעה מהמרחב המטרי.
\end{example}
\begin{exercise}
	הוכיחו כי אכן זהו מרחב טופולוגי.
\end{exercise}
\begin{example}
	יהי $X$ קבוצה כלשהי, אז ניתן להגדיר על $X$ טופולוגיה $\tau_0 = \{ \emptyset, X \}$.
	טופולוגיה זו נקראת טופולוגיה טריוויאלית.
\end{example}
\begin{example}
	נגדיר $\tau_1 = \Pp(X)$ עבור קבוצה $X$, גם קבוצה זו היא טופולוגיה, והיא נקראת הטופולוגיה הדיסקרטית.
\end{example}
\begin{example}
	נניח ש־$(Y, \tau)$ מרחב טופולוגי, ותהי $f : (Y, \tau) \to (X, \tau_0)$, מתי $f$ היא רציפה? התשובה היא שהיא רציפה תמיד.
	מתי $f : (Y, \tau) \to (X, \tau_1)$ רציפה? תלוי בהגדרת הפונקציה, אבל במקרה שבו היא אכן רציפה, אז היא רציפה בכל טופולוגיה שהיא.
	לעומת זאת כל $f : (X, \tau_1) \to (Y, \tau)$ היא רציפה.
\end{example}
\begin{remark}
	לא כל טופולוגיה נובעת ממטריקה.
	לדוגמה הטופולוגיה הטריוויאלית על מרחב עם לפחות 2 נקודות.
\end{remark}
\begin{remark}
	נניח $x, y \in X$ אז נבחר $r = \frac{1}{2} d(x, y)$ ואז $y \notin B(x, r)$ ולכן $\emptyset \ne B(x, r) \ne X$, קל לראות שביחס לטופולוגיה שמושרית מהמטריקה $d$, הקבוצה $B(x, r)$ קבוצה פתוחה.
\end{remark}
\begin{example}
	נגדיר $X = \CC^n$ עבור איזשהו $n \in \NN$ ונגדיר $\Ff = \{ A \subseteq \CC^n \mid \exists {\{f_i\}}_{i \in I} \subseteq \CC[x_1, \dots, x_n], A = \{(p_1, \dots, p_n) \mid \forall i \in I, f_i(p_1, \dots, p_n) = 0 \}\}$.
\end{example}
\begin{definition}[בסיס לטופולוגיה]\label{topology_basis_definition}
	בסיס לטופולוגיה הוא אוסף $\Bb$ של תתי־קבוצות של $X$ כך שמתקיים,
	\begin{enumerate}
		\item לכל $x \in X$ יש $B \in \Bb$ כך ש־$x \in B$
		\item לכל $A, B \in \Bb$ ולכל $x \in A \cap B$ יש $C \in \Bb$ כך ש־$x \in C \subseteq A \cap B$
	\end{enumerate}
\end{definition}
\begin{proposition}
	עבור בסיס $\Bb$ האוסף $\tau_\Bb = \{ U \subseteq X \mid U \text{ is a union of elements of } \Bb \}$ היא טופולוגיה,
	\[
		\forall \alpha \in I, B_\alpha \in \Bb, U = \bigcup_{\alpha \in I} B_\alpha
	\]
\end{proposition}
\begin{proof}
	מכיוון ש־$\tau_\Bb$ סגורה לחיתוך סופי, אז אם $U, V \in \tau_\Bb$ אז $U = \bigcup_{\alpha \in I} B_\alpha \in \Bb$ וכן $V = \bigcup_{\beta \in J} A_\beta, A_\beta \in \Bb$, אז מתקיים,
	\[
		U \cap V
		= (\bigcup_{\alpha \in I} B_\alpha) \cap (\bigcup_{\beta \in J} A_\beta)
		= \bigcup_{\alpha, \beta \in I \times J} B_\alpha \cap A_\beta
		= D
	\]
	לכן לכל $x \in U \cap V$ ישנם $\alpha_0 \in I, \beta_0 \in J$ כך ש־$x \in B_{\alpha_0} \cap A_{\beta_0}$,
	אבל מהגדרת הבסיס קיימת קבוצה $C_{\alpha_0, \beta_0} \in \Bb$ כך ש־$C_{\alpha_0, \beta_0} \subseteq B_{\alpha_0} \cap A_{\beta_0}$.
	לכן $D \subseteq \bigcup_{(x, \alpha, \beta)} C_{x, \alpha, \beta}$.
	בהתאם מצאנו סגירות לחיתוך סופי.
\end{proof}
\begin{remark}
	יהי $(X, d)$ מרחב מטרי, אז $\{ B(x, r) \subseteq X \mid x \in X, r > 0 \}$ הוא טופולוגיה.
	אבל עכשיו נוכל להגדיר גם את $\{ B(x, \frac{1}{n}) \subseteq X \mid x \in X, n \in \NN \}$, זהו בסיס לטופולוגיה לאותה הטופולוגיה שהגדרנו למרחב המטרי.
\end{remark}
\begin{exercise}
	הוכיחו שזהו אכן בסיס עבור המרחב הטופולוגי הנתון.
\end{exercise}
\begin{example}
	נניח ש־$X = \ZZ$, ונגדיר את הבסיס $C$ להיות אוסף הסדרות האריתמטיות הדו־צדדיות, כלומר $C = \{ a + d \ZZ \mid a, d \in \ZZ, d \ne 0 \}$.
	אנו טוענים כי זהו אכן בסיס (לטופולוגיה).
	נתבונן בזוג קבוצות ב־$C$, $a + d \ZZ, b + q \ZZ$, ונניח ש־$p \in (a + d \ZZ) \cap (b + q \ZZ)$ אז $p \in p + dq \ZZ \subseteq (a + d \ZZ) \cap (b + q \ZZ)$.
	נגדיר טופולוגיית $\tau_C$.

	קבוצות סגורות הן משלימים לקבוצות פתוחות.

	כל סדרה אריתמטית דו־צדדית אינסופית היא גם פתוחה וגם סגורה.
	בפרט חיתוך סופי של סדרות אריתמטיות הוא סגור.
	לכן המשלים שלו הוא פתוח.
\end{example}
\begin{conclusion}[משפט אוקלידס]
	יש אינסוף מספרים ראשוניים.
\end{conclusion}
\begin{proof}
	נניח בשלילה כי יש מספר סופי של ראשוניים, $p_1, \dots, p_k$ עבור $k \in \NN$.
	נבחן את $\bigcup_{i = 1}^k p_i \ZZ$, זוהי קבוצה פתוחה וגם סגורה, לכן
	\[
		\bigcup_{i = 1}^k p_i \ZZ
		= \ZZ \setminus \{ -1, 1 \}
	\]
	ולכן נובע ש־$\{-1, 1\}$ קבוצה פתוחה וזו כמובן סתירה.
\end{proof}
\begin{proposition}[צמצום מרחב טופולוגי]
	נניח ש־$(X, \tau)$ מרחב טופולוגי, לכל $\emptyset \ne Y \subseteq X$ נגדיר $\tau_Y = \{ U \cap Y \mid U \in \tau \}$.
	אז $\tau_Y$ היא טופולוגיה.
	אם $Y \in \tau$ אז $\tau_Y = \{ W \in \tau \mid W \subseteq Y \}$.
\end{proposition}
\begin{proposition}[טופולוגיית מכפלה]
	נניח ש־$(X_1, \tau_1)$ ו־$(X_2, \tau_2)$ מרחבים טופולוגיים, אז נגדיר טופולוגיה על מרחב המכפלה $X_1 \times X_2$ על־ידי
	\[
		\tau_{1, 2}
		= \{ U_1 \times U_2 \mid U_1 \in \tau_1, U_2 \in \tau_2 \}
	\]
	אז $\tau_{1, 2}$ הוא בסיס והטופולוגיה המוגדרת על־ידו נקראת טופולוגיית המכפלה.
\end{proposition}
\begin{example}
	נוכל לבנות כך מכפלה של כמות סופית או אינסופית של מכפלות טופולוגיות.
	עבור אוסף אינסופי (בן־מניה או לא בהכרח) אנו צריכים להיזהר, נניח ש־$(X_\alpha, \tau_\alpha)$ עבור $\alpha \in I$, אז נגדיר
	\[
		\tau_b = \{ \prod_{\alpha \in I} U_\alpha \mid \forall \alpha \in I, U_\alpha \in \tau_\alpha \}
	\]
	זהו בסיס לטופולוגיה שנקרא טופולוגיית הקופסה.
	לעומת זאת נוכל להגדיר גם את
	\[
		\tau_p
		= \{ \prod_{\alpha \in I} U_\alpha \mid U_\alpha = X_\alpha \text{ for almost all } \alpha \in I \}
	\]
	כלומר $\prod_{\alpha \in I} = \{ f : I \to \bigcup_{\alpha \in I} X_\alpha \mid \forall \alpha \in I, f(x) \in X_\alpha \}$.
\end{example}

\section{שיעור 2 --- 25.3.2025}
\subsection{טופולוגיה --- המשך}
בשיעור הקודם דיברנו על מכפלה של טופולוגיות, אמרנו שאם $I$ קבוצת אינדקסים ולכל $\alpha \in I$ גם $(X_\alpha, \tau_\alpha)$ מרחב טופולוגי, אז נתבונן ב־$Z = \prod_{\alpha \in I} X_\alpha$ ונרצה להגדיר טופולוגיה על $Z$.
\begin{remark}
	מגדירים,
	\[
		\prod_{\alpha \in I} X_\alpha = \{ f : I \to \bigcup_{\alpha \in I} X_\alpha, \forall \alpha \in I, f(\alpha) \in X_\alpha \}
	\]
\end{remark}
לאחר מכן נוכל להגדיר טופולוגיית מכפלה,
\begin{definition}[טופולוגיית מכפלה]
	נגדיר את הבסיס,
	\[
		\Bb_{\text{box}} = \{ \prod_{\alpha \in I} U_\alpha \mid \forall \alpha \in I, U_\alpha \subseteq X_\alpha, U_\alpha \in \tau_\alpha \}
	\]
	ואת הבסיס,
	\[
		\Bb_{\text{prod}}
		= \{ \prod_{\alpha \in I} V_\alpha \mid \forall \alpha \in I, V_\alpha \in \tau_\alpha,
			V_\alpha \subseteq X_\alpha, |\{ \beta \in I \mid V_\beta \ne X_\beta \}| < \infty, V_\alpha = X_\alpha \text{ for almost every } \alpha \}
	\]
	אלו הן מכפלות של טופולוגיות המהוות טופולוגיה.
\end{definition}
\begin{definition}[העתקות הטלה]
	אז $Z = \prod_{\alpha \in I} X_\alpha$ אז ישנן הטלות $\forall \alpha \in I, \pi_\alpha : Z \to X_\alpha$ המוגדרת על־ידי $\pi_\alpha(f) = f(\alpha)$.
\end{definition}
אנו רוצים שכל ההטלות $\pi_\alpha$ תהינה רציפות.
כדי שהן אכן יקיימו רציפות צריך שלכל $U_\alpha \in \tau_\alpha$ (קבוצה פתוחה ב־$X_\alpha$) יתקיים $\pi_\alpha^{-1}(U_\alpha) \in \tau$, כלומר המקור יהיה קבוצה פתוחה ב־$Z$.
אבל נבחין כי $\pi_\alpha^{-1}(U_\alpha) = U_\alpha \times \prod_{\beta \ne \alpha} X_\beta$ אבל זהו לא בסיס,
\[
	C = \{ U_\alpha \times \prod_{\beta \ne \alpha} X_\beta \mid \pi_\alpha^{-1}(U_\alpha) \in \tau \}
\]
\begin{definition}[תת־בסיס לטופולוגיה]
	תהי קבוצה $X$, קבוצה $C$ של תת־קבוצות של $X$ כך ש־$\bigcup C = X$. \\
	נגדיר את הבסיס המושרה מתת־בסיס להיות $\Bb_C = \{ \bigcap A \mid A \subseteq C, |A| < \infty \}$,
	כלומר אוסף החיתוכים הסופיים של איברי $C$ (הן קבוצות פתוחות) הוא בסיס.
\end{definition}
\begin{definition}[טופולוגיה חלשה]
	אם $X$ קבוצה ו־$\tau_1, \tau_2$ טופולוגיות על $X$ אומרים ש־$\tau_1$ חלשה יותר מ־$\tau_2$ אם $\tau_1 \subseteq \tau_2$.
\end{definition}
\begin{example}
	יהיו מרחבים מטריים $(X_i, \rho_i)$ לכל $i \in \NN$, ונגדיר $(X_i, \tau_i)$ מרחב טופולוגי מושרה מתאים לכל $i$.
	נרצה להתבונן במכפלתם, $\prod_{i \in \NN} X_i$.
	אז נוכל להתבונן ב־$(\prod X_i, \tau_{\text{prod}})$ שהגדרנו זה עתה.
\end{example}
\begin{definition}[מטריקת מכפלה]
	בהינתן סדרת מרחבים מטריים $(X_i, \rho_i)$ עבור $i \in \NN$ מרצה למצוא מטריקה על $Z = \prod_{i \in \NN} X_i$.
	לכל $x, y \in Z$ כאשר $x = (x_i), y = (y_i)$ אז נגדיר,
	\[
		\rho(x, y)
		= \sum_{i = 1}^{\infty} \frac{1}{2^i} \frac{\rho_i(x_i, y_i)}{1 + \rho_i(x_i, y_i)}
	\]
\end{definition}
ברור שפונקציה זו מוגדרת, וברור אף כי היא מקיימת את התכונה השנייה של מטריקות, אך לא ברור שהיא מקיימת את אי־שוויון המשולש, זהו תרגיל שמושאר לקורא.
\begin{proposition}
	הטופולוגיה המושרית על $Z = \prod_{i = 1}^\infty X_i$ עבור $(X_i, \tau_i)$ מרחבים טופולוגיים יחד עם מטריקת המכפלה שווה ל־$\Bb_{\text{prod}}$.
\end{proposition}
\begin{proof}
	$(Z, \rho)$ מרחב מטרי, ו־$\Bb_\rho = \{ B(x, r) \mid x \in Z, r > 0 \}$ בסיס, אז נוכל להגדיר טופולוגיה $\tau_{\Bb_\rho} = \tau_\rho$.
	כדי להראות ש־$\tau_\rho = \Bb_{\text{prod}}$ מספיק להראות שכל $B \in \Bb_\text{prod}$ שייכת ל־$\tau_\rho$ וכל $C \in \Bb_\rho$ שייכת ל־$\tau_\text{prod}$.
	נוסיף ונבהיר שטופולוגיה נקבעת ביחידות על־ידי בסיס שלה, לכן מספיק להראות את שקילות הבסיסים.

	נתחיל בתנאי הראשון, ונקבע $k \in \NN$ כלשהו. מספיק להראות שקבוצה מהצורה $U_k \times \prod_{i \ne k} X_i$ פתוחה ב־$\tau_\rho$ עבור $U_k \in \tau_k$ היא קבוצה פתוחה ב־$\tau_\rho$,
	זאת שכן נוכל להרחיב הוכחה זו באופן סיסטמתי להיות על כל קבוצה סופית של קבוצות פתוחות.
	יהי $x \in U_k \times \prod_{i \ne k} X_i$ ונסמן את ההטלה על מרחב זה $\pi_j : U_j \times \prod_{i \ne k} X_i \to U_j$, כלומר $\pi_j(x) = x_j$ לכל $j \in \NN$.
	אנו יודעים ש־$x_k \in U_k$ וש־$U_k$ פתוחה ולכן ישנו $r > 0$ כך ש־$B_r(x_k) \subseteq U_k$ כדור פתוח ב־$X_k$. \\
	קיים $s > 0$ כך שאם $t \ge 0$ ומתקיים $\frac{t}{1 + t} < s$ אז $t < r$, ולכן נבחן את הכדור ברדיוס $\frac{s}{2^k}$ סביב $x$ ב־$Z = \prod_{i \in \NN} X_i$ מרחב המכפלה כולו.
	המטרה שלנו היא להראות שהכדור שעתה בחרנו מקיים את התנאי לבסיס.
	נניח ש־$y = {(y_i)}_{i \in \NN} \in B_{\frac{s}{2^k}}(x)$, אז
	\begin{align*}
		& \frac{s}{2^k} > \rho(x, y)
		= \sum_{i = 1}^{\infty} \frac{1}{2^i} \frac{\rho_i(x_i, y_i)}{1 + \rho_i(x_i, y_i)}
		\ge \sum_{i = 1}^{\infty} \frac{1}{2^i} \frac{\rho_i(x_i, y_i)}{1 + \rho_i(x_i, y_i)} \\
		& \implies s > \frac{\rho_i(x_i, y_i)}{1 + \rho_i(x_i, y_i)} \\
		& \implies \rho_k(x_k, y_k) < r \\
		& \implies y_k \in B_r(x_k) \subseteq U_k
	\end{align*}

	נעבור לתנאי השני, נתבונן בכדור הפתוח סביב $x \in Z$, $B_r(x)$, כאשור נחזור ונבהיר כי כדור זה מוגדר להיות,
	\[
		B_r(x)
		= \left\{ y \in Z \mid \sum_{i = 1}^{\infty} \frac{1}{2^i} \frac{\rho_i(x_i, y_i)}{1 + \rho_i(x_i, y_i)} < r \right\}
	\]
	יהי $M \in \NN$ כך ש־$\sum_{i = M + 1}^{\infty} \frac{1}{2^i} \frac{\rho_i(x_i, y_i)}{1 + \rho_i(x_i, y_i)} < \frac{r}{2}$, כלומר נחסום את טור הזנב של המטריקה $\rho$.
	תהי $V \subseteq Z$ המוגדרת על־ידי,
	\[
		V
		= \left\{ (y_1, \dots, y_M) \in \prod_{i = 1}^M \mid \sum_{i = 1}^M \frac{1}{2^i} \frac{\rho_i(x_i, y_i)}{1 + \rho_i(x_i, y_i)} < \frac{r}{2} \right\}
	\]
	ואנו טוענים כי $V \times \prod_{i = M + 1}^\infty X_i \subseteq B_r(x)$.
\end{proof}

\section{שיעור 3 --- 31.3.2025}
\subsection{סגירות}
בדיוק כמו במרחבים מטריים, גם במרחב טופולוגי נרצה לדון במניפולציות על קבוצות במרחב, נתחיל בהגדרת הקונספט של סגור של קבוצה במרחב טופולוגי.
\begin{definition}[סגור של קבוצה במרחב טופולוגי]
	יהי $(X, \tau)$ מרחב טופולוגי, ותהי קבוצה $A \subseteq X$ תת־קבוצה כשלהי.
	נגדיר את הסגור של $A$ כקבוצה הסגורה הקטנה ביותר המכילה את $A$, כלומר,
	\[
		\overline{A}
		= \bigcap_{X \setminus F \in \tau} F
	\]
\end{definition}
בהתאם נקבל מספר תכונות ראשוניות ודומות לתכונות שראינו בעבר,
\begin{lemma}
	התכונות הבאות מתקיימות,
	\begin{enumerate}
		\item $\overline{A \cup B} = \overline{A} \cup \overline{B}$
		\item $\overline{A \cap B} \subseteq \overline{A} \cap \overline{B}$, כאשר במקרה זה אין בהכרח שוויון.
	\end{enumerate}
\end{lemma}
\begin{example}
	נראה דוגמה למקרה בו בהכרח אין שוויון,
	נגדיר $X = \RR$ וכן $A = \QQ, B = \RR \setminus \QQ$, אז מתקיים,
	\[
		\emptyset
		= \overline{\emptyset}
		= \overline{A \cap B}
		\subsetneq \overline{A} \cap \overline{B}
		= \RR \cap \RR
		= \RR
	\]
\end{example}
\begin{proposition}
	אם $(X, \tau)$ מרחב טופולוגי ו־$A \subseteq X$, אז,
	\[
		x \in \overline{A}
		\iff \forall U \in \tau, x \in U \rightarrow U \cap A \ne \emptyset
	\]
	כלומר נקודה נמצאת בסגור של $A$ אם ורק אם כל קבוצה פתוחה סביב הנקודה לא זרה ל־$A$.
\end{proposition}
\begin{proof}
	נראה את שלילת הטענה, כלומר $x \notin \overline{A} \iff \exists U \in \tau, x \in U \land U \cap A = \emptyset$. \\
	נניח ש־$x \notin \overline{A}$ ולכן $x \in X \setminus \overline{A}$ אבל $X \setminus \overline{A}$ פתוחה וזרה מהגדרתה ל־$A$. \\
	בכיוון השני אם יש $U \ni x$ פתוחה כך ש־$U \cap A = \emptyset$ אז $F = X \setminus U$ סגורה ומכילה את $A$, בהתאם $\overline{A} \subseteq F$ ובהכרח $x \notin \overline{A}$.
\end{proof}
\begin{definition}[פנים ושפה]
	נגדיר את הפנים של $A$ להיות, $A^\circ = \bigcup_{U \in \tau, U \subseteq A} U$. \\
	כלומר הפנים הוא איחוד כל הקבוצות הפנימיות הפתוחות של $A$, ובשל הסגירות של הטופולוגיה לאיחוד, נקבל כך את הקבוצה הפתוחה הגדולה ביותר שחלקית ל־$A$.
	נגדיר את השפה של $A$ להיות $\partial A = \overline{A} \setminus A^\circ$.
\end{definition}
נבחין בהגדרה של סביבה ונשתמש בהגדרה זו כדי להגדיר מונח חדש.
\begin{definition}[סביבה של נקודה]
	נאמר ש־$L \subseteq X$ היא סביבה של $x$ אם קיימת קבוצה פתוחה $U \in \tau$ כך ש־$x \in U \subseteq L$.
\end{definition}
\begin{definition}[נקודת הצטברות]
	יהי $(X, \tau)$ מרחב טופולוגי, תהי $A \subseteq X$ תת־קבוצה כלשהי, ו־$x \in A$.
	נאמר ש־$x$ היא נקודת הצטברות של $A$ אם כל סביבה של $x$ מכילה נקודה מ־$A$ שונה מ־$x$, כלומר,
	\[
		\forall U \in \tau, x \in U \implies \exists y \in (U \setminus \{ x \}) \cap A
	\]
	נסמן ב־$A'$ את קבוצת נקודות ההצטברות של $A$.
\end{definition}
נרצה להסתכל על נקודות הצטברות כנקודות שלא משנה כמה קרוב אנחנו מסתכלים אליהן, עדיין נוכל למצוא בסביבתן נקודות נוספות. במובן הזה ברור שהן נמצאות בקרבת נקודות בפנים, אך עלולות להיות גם נקודות לא פנימיות שמקיימות טענה כזו.
\begin{proposition}
	מתקיים $\overline{A} = A \cup A'$.
\end{proposition}
\begin{proof}
	אם $x \in A \cup A'$ אז או $x \in A \subseteq \overline{A}$ או $x \in A'$.
	ובכל סביבה של $x$ יש נקודה מ־$A$ שונה מ־$x$.
	בפרט לאור הטענה ש־$\overline{A}$ היא אוסף כל הנקודות שבכל סביבה שלהן המכילה את $A$ בחיתוך לא ריק נובע ש־$A' \subseteq \overline{A}$, לכן נובע ש־$A \cup A' \subseteq \overline{A}$.

	בכיוון השני נניח ש־$x \in \overline{A}$, אז לכל $U \in \tau$ כך ש־$x \in U$ מתקיים $U \cap A \ne \emptyset$.
	אם $x \in A$ אז בוודאי ש־$x \in A \cup A'$.
	אם $x \notin A$ אז לכל $U \in \tau$ כך ש־$x \in U$ מתקיים $U \cap A \ne \emptyset$.
	מ־$x \notin A$ נובע גם ש־$U \setminus \{ x \} \cap A \ne \emptyset$ ולכן $x \in A'$.
	אז מצאנו ש־$\overline{A} \subseteq A \cup A'$, ונובע משני החלקים ש־$\overline{A} = A \cup A'$.
\end{proof}

\subsection{השלמות לרציפות}
ניזכר בהגדרה\ \ref{function_continuous_definition}, נרצה לדון בקונספט של רציפות באופן רחב יותר.
בהינתן $(Y, \tau_Y)$ מרחב טופולוגי ו־$X$ קבוצה כלשהי, ופונקציה $f : X \to Y$, ניתן להגדיר טופולוגיה על $X$ כך ש־$f$ רציפה. \\
הקבוצה $\{ f^{-1}(U) \mid U \in \tau_Y \}$ היא תת־בסיס, ואפשר להרחיבה לבסיס ולהגדיר עליו טופולוגיה מושרית מהבסיס על $X$.
\begin{proposition}
	$f$ רציפה עבור טופולוגיה זו, וזו הטופולוגיה החלשה ביותר על $X$ עבורה $f$ רציפה.
\end{proposition}
בכיוון השני, בהינתן מרחב טופולוגי $(X, \tau_X)$ וקבוצה כלשהי $Y$ יחד עם פונקציה $f : X \to Y$, נוכל להגדיר את $\{ U \subseteq Y \mid f^{-1}(U) \in \tau_X \}$ להיות תת־בסיס וממנו נוכל שוב לבנות בסיס וטופולוגיה על $Y$.
באופן דומה $f$ רציפה ביחס לטופולוגיה זו וזו הטופולוגיה החזקה ביותר על $Y$ כך ש־$f$ רציפה.
\begin{proposition}[שקילות לרציפות]
	יהיו מרחבים טופולוגיים $(X, \tau_X), (X, \tau_Y)$ ותהי $f : X \to Y$, אז התנאים הבאים שקולים,
	\begin{enumerate}
		\item $f$ רציפה לפי\ \ref{function_continuous_definition}
		\item לכל קבוצה סגורה $F \subseteq Y$, $f^{-1}(F)$ סגורה ב־$X$ \\
			הגדרה זו עוזרת לנו לדון בקבוצות סגורות במקום פתוחות
		\item אם $\Bb$ בסיס לטופולוגיה של $Y$ אז לכל $B \in \Bb$ מתקיים ש־$f^{-1}(B)$ פתוחה ב־$X$ \\
			הגדרה זו מאפשרת לנו לדון בבסיסים ובכך לפשט את העבודה עם טופולוגיות
		\item לכל $x \in X$ ולכל סביבה $W \subseteq Y$ של $f(x)$ מתקיים ש־$f^{-1}(W)$ סביבה של $x$
		\item קיים כיסוי פתוח ${\{U_\alpha\}}_{\alpha \in \Omega}$ של $X$, כלומר $\forall \alpha, U_\alpha \in \tau, X = \bigcup_{\alpha \in \Omega} U_\alpha$,
			כך שלכל $\alpha \in \Omega$ מתקיים $f \mid_{U_\alpha} : U_\alpha \to Y$ רציפה.
		\item קיים כיסוי סגור סופי $X = \bigcup_{i = 1}^n F_i$ עבור $F_i$ סגורות עבור $1 \le i \le n$, כך שכל $f \mid_{F_i} : F_i \to Y$ רציפה.
		\item לכל $A \subseteq X$ מתקיים $f(\overline{A}) \subseteq \overline{f(A)}$
	\end{enumerate}
\end{proposition}
\begin{proof}
	$1 \iff 2$:
	נובע ישירות מהגדרה של משלימים והגדרת הרציפות על קבוצות פתוחות.

	$1 \iff 3$:
	בכיוון הראשון כל איחוד קבוצות מהבסיס הוא קבוצה פתוחה, ונוכל כך להראות את נכונות הטענה.
	לכיוון השני כל קבוצה היא איחוד של קבוצות מהבסיס, $U_\alpha$, ו־$f^{-1}(\bigcup U_\alpha) = \bigcup f^{-1}(U_\alpha)$.

	$1 \implies 4$:
	אם $x \in X$ וכן $f(x) \in W \subseteq Y$ סביבה של $f(x)$ אז קיימת $f(x) \in U \subseteq W$ כך ש־$U$ פתוחה, לכן נובע ש־$x \in f^{-1}(U) \subseteq f^{-1}(W)$ כאשר $f^{-1}(U)$ פתוחה.

	$4 \implies 1$:
	אם $U \subseteq Y$ פתוחה אז צריך להראות ש־$f^{-1}(U)$ פתוחה.
	תהי $x \in f^{-1}(U)$, אז $U$ סביבה ל־$f(x)$ ולכן לפי ההנחה $f^{-1}(U)$ היא סביבה של $x$, כלומר קיימת $x \in V_x \subseteq f^{-1}(U)$ פתוחה, ונסיק ש־$f^{-1}(U) = \bigcup_{x \in f^{-1}(U)} V_x$ פתוחה.

	$1 \implies 5$:
	נוכל לבחור כיסוי טריוויאלי.

	$5 \implies 1$:
	נניח שיש כיסוי פתוח ${\{U_\alpha\}}_{\alpha \in \Omega}$ של $X$ כך ש־$f \mid_{U_\alpha} : U_\alpha \to Y$ רציפה לכל $\alpha \in \Omega$.
	תהי $W \subseteq Y$ פתוחה, אז $f^{-1}(W) \subseteq \bigcup_{\alpha \in \Omega} f^{-1} \mid_{U_\alpha}(W)$.
	מההנחה $U_\alpha \cap f^{-1}(W) = f^{-1} \mid_{U_k}(W)$ פתוחה ב־$U_\alpha$ ומשום ש־$U_\alpha$ פתוחה ב־$X$ נובע ש־$f^{-1} \mid_{U_\alpha}(W)$ פתוחה ב־$X$ ולכן $f^{-1}(W)$ פתוחה ב־$X$.

	$1 \implies 6$:
	נבחר את $X$ ככיסוי סגור של עצמה.

	$6 \implies 1$:
	נניח ש־$X = \bigcup_{i = 1}^n F_i$ כיסוי סגור סופי של $X$, ונניח גם שלכל $i$, $f \mid_{F_i} : F_i \to Y$ רציפה.
	כעת ההוכחה דומה למהלך שעשינו ב־$5 \implies 1$, אבל כעת אפיון רציפות בעזרת $L$, ואיחוד סופי על סגורות הוא סגור.

	$1 \implies 7$:
	נתון כי $f$ רציפה, אנו רוצים להראות ש־$f(\overline{A}) = \overline{f(A)}$.
	יהי $x \in \overline{A}$, נראה כי $f(x) \in \overline{f(A)}$,
	נניח בשלילה ש־$f(x) \notin \overline{f(A)}$, אז יש סביבה פתוחה $f(x) \in U$ כך ש־$U \cap f(A) = \emptyset$.
	$f$ רציפה ולכן $f^{-1}(U)$ פתוחה ב־$X$ וקיים $A \cap f^{-1}(U) = \emptyset$,
	אבל $x \in f^{-1}(U)$ וקיבלנו $x \notin \overline{A}$.

	$7 \implies 2$:
	תהי $F \subseteq Y$ סגורה, אז,
	\[
		f(\overline{f^{-1}(F)})
		\overset{\text{ההנחה}}{\subseteq}
		\subseteq \overline{F}
		\overset{\text{$F$ סגורה}}{=} F
		\implies \overline{f^{-1}(F)}
		\subseteq f^{-1}(F)
	\]
	מהגדרת סגור נוכל להסיק ש־$f^{-1}(F) \subseteq \overline{f^{-1}(F)}$, לכן,
	\[
		\overline{f^{-1}(F)}
		= f^{-1}(F)
	\]
	ובפרט $f^{-1}(F)$ סגורה ב־$X$.
\end{proof}
נבחן תכונה מעניינת שלא תשרת אותנו רבות, אך כן מעלה שאלות,
\begin{definition}[מרחב כוויץ]
	יהי $X$ מרחב טופולוגי, נאמר ש־$X$ כוויץ (Contractible) אם יש פונקציה רציפה $f : I \times X \to X$ עבור $I = [0, 1]$ כך ש־$\forall x \in X, f(0, x) = x$ וקיימת נקודה $x_1 \in X$ כך ש־$\forall x \in X, f(1, x) = x_1$. \\
	נסמן גם $f_t(t, x)$ כאשר $f_t : X \to X$, נקבל $f_0 = Id$ וכן $f_1$ הפונקציה הקבועה $x \mapsto x_1$.
\end{definition}
\begin{example}
	נגדיר $X = I$ ואת $f : I \times I \to I$ המוגדרת על־ידי $f(t, x) = (1 - t)x$. \\
	נגדיר $X = \RR$ ואת $f : I \times \RR \to \RR$ על־ידי $f(t, x) = (1 - t)x$ ונקבל שגם $\RR$ כוויצה בדיוק באותו האופן.
\end{example}
\begin{exercise}
	הראו כי $S^1$ לא כוויץ.
\end{exercise}
נחזור לדבר על פונקציות רציפות.
\begin{exercise}
	נתבונן ב־$f : (\RR, \tau_\RR) \to (\RR^\NN, \tau)$ כך ש־$f(x)(i) = x$ לכל $i \in \NN$. \\
	הראו ש־$f$ רציפה או לא רציפה כהעתקה כאשר $\tau$ טופולוגיית המכפלה, וכאשר $\tau$ טופולוגיית הקופסה.
\end{exercise}
\begin{solution}
	נתבונן ב־$U = \prod_{n = 1}^\infty (-\frac{1}{n}, \frac{1}{n})$, זוהי קבוצה פתוחה, אך $f^{-1}(U) = 0$, וזו כמובן לא קבוצה פתוחה, לכן בטופולוגיית הקופסה היא לא רציפה, לכן בטופולוגיית הקופסה היא לא רציפה. \\
	לעומת זאת בטופולוגיית המכפלה היא אכן רציפה.
\end{solution}
\begin{definition}[הומיאומורפיזם]
	הומיאומורפיזם בין שני מרחבים טופולוגיים $X, Y$ היא העתקה $f : X \to Y$ כך ש־$f$ חד־חד ערכית ועל, $f$ רציפה ו־$f^{-1}$ רציפה אף היא. \\
	$X$ ו־$Y$ יקראו הומיאומורפיות אם יש הומיאומורפיזם $f : X \to Y$ ביניהן.
\end{definition}
אנו נרצה להסתכל על הומיאומורפיזם כאיזומורפיזם של מרחבים טופולוגיים.
\begin{example}
	נגדיר $X = \RR, Y = (0, 1)$, ואת $f : \RR \to (0, 1)$ המוגדרת על־ידי $x \mapsto \frac{e^x}{e^x + 1}$, אז,
	\[
		f'(x)
		= \frac{e^x(e^x + 1) - e^x e^x}{{(e^x + 1)}^2}
		= \frac{e^x}{{(e^x + 1)}^2}
		> 0
	\]
	ולכן $f$ גזירה, ואף חד־חד ערכית, לבסוף $f(x) \xrightarrow{x \to -\infty} 0, f(x) \xrightarrow{x \to \infty} 1$ ולכן היא גם על, ואכן המרחבים הומיאומורפים.
\end{example}
\begin{example}
	נגדיר את $\eta = \{ z = x + iy \in \CC \mid x, y \in \RR, y > 0 \}$ ואת $D = \{ z \in \CC \mid |z| < 1 \}$.
	נגדיר גם $\psi : \eta \to D$ על־ידי $z \mapsto \frac{z - i}{z + i}$. \\
	ההוכחה כי זהו אכן הומיאומורפיזם מושארת לקורא.
\end{example}
נבחין כי הדוגמה האחרונה אינה אלא העתקת מביוס, העתקה קונפורמית ואנליטית.
\begin{example}
	נבחן את $S^1 = \{ z \in \CC \mid |z| = 1 \}$ ואת $J = [0, 2\pi]$, אנו טוענים כי אין הומיאומורפיזם בין שני המרחבים הללו. \\
	נבחן את הפונקציה $t \mapsto e^{it}$ לדוגמה,
	$[0, 2\pi] \to S^1$ לא חד־חד ערכית, מהצד השני $[0, 2\pi) \to S^1$ חד־חד ערכית ועל, אבל 

	נניח שיש העתקה חד־חד ערכית $\alpha : J \to S^1$, ונוציא מ־$J$ נקודה יחידה, אז נקבל איחוד זר של שתי קבוצות זרות, אך מן הצד השני הוצאת נקודה יחידה מהמעגל משאיר אותו כקבוצה קשירה.
	ההוכחה המלאה אומנם סבוכה יותר, אך הצבענו פה על הבדל מהותי בין שני המרחבים.
\end{example}
\begin{exercise}
	הראו כי $\RR$ ו־$\RR^2$ לא הומיאומורפים. \\
	האם גם $\RR^2$ ו־$\RR^3$ הומיאומורפים?
\end{exercise}
\begin{definition}[העתקה פתוחה וסגורה]
	העתקה $f : X \to Y$ תיקרא העתקה פתוחה (סגורה) אם לכל $U \subseteq X$ פתוחה (סגורה) מתקיים $f(U) \subseteq Y$ פתוחה (סגורה) ב־$Y$.
\end{definition}
\begin{example}
	$f : \RR \to \RR$ המוגדרת על־ידי $f(x) = x^2$ היא רציפה, סגורה ולא פתוחה.
\end{example}
\begin{example}
	השיכון $(0, 1) \hookrightarrow \RR$ המוגדר על־ידי $x \mapsto x$ הוא רציף, תפוח אבל לא סגור.
\end{example}
\begin{example}
	$\{ a, b \} \to \{ a, b \}$ המוגדרת טריוויאלית היא פתוחה, סגורה אך לא רציפה.
\end{example}

\section{שיעור 4 --- 7.4.2025}
\subsection{אקסיומות ההפרדה}
\begin{definition}[איברים ניתנים להרחבה]
	נאמר ש־$U$ ו־$V$ מפרידות בין $x, y \in X$ מרחב טופולוגי, אם מתקיים $U \cap V = \emptyset$ וכן $x \in U, v \in V$.
	אם קיימות כאלה, אז נאמר ש־$x, y$ ניתנות להפרדה. \\
	באופן דומה נאמר ש־$A \subseteq X, x \in X$ ניתנות להפרדה אם קיימות $A \subseteq U, x \in V$ כאלה.
\end{definition}
\begin{definition}[מרחב T]
	מרחב טופולוגי $X$ יקרא מרחב $T_i$ אם הוא מקיים את האקסיומה $T_i$ עבור $i \in \{0, 1, 2, 3, 4\}$.

	נגדיר את האקסיומות $T_i$,
	\begin{itemize}
		\item $T_0$, לכל $x, y \in X$ יש קבוצה פתוחה שמכילה את אחת הנקודות אך לא את האחרת
		\item $T_1$, לכל שתי נקודות $x, y \in X$ קיימת פתוחה המכילה את אחת הנקודות ולא את האחרת, וקבוצה פתוחה המכילה את הנקודה השנייה אך לא את הראשונה.
			כלומר אם $x \ne y$ אז קיימת $U \in \tau$ כך ש־$x \in U, y \notin U$
		\item $T_2$ (מרחב האוסדורף), אם לכל זוג נקודות $x \ne y \in X$ יש קבוצות פתוחות זרות $U, V \subseteq X$ כך ש־$x \in U, y \in V$,
			כאשר משמעות היותן זרות היא ש־$U \cap V = \emptyset$
		\item $T_3$, אם המרחב הוא $T_1$ וגם $X$ \textbf{רגולרי},
			כלומר לכל $x \in X$ וקבוצה סגורה $x \notin A \subseteq X$, $x, A$ ניתנות להפרדה
		\item $T_4$, אם המרחב הוא $T_1$ וכן $X$ \textbf{נורמלי},
			כלומר שכל זוג תת־קבוצות סגורות $A, B \subseteq X$ ניתנות להפרדה
	\end{itemize}
\end{definition}
\begin{remark}
	$T_1$ מתקיים אם ורק אם כל $\{x\} \subseteq X$ קבוצה סגורה.
\end{remark}
\begin{proposition}[גרירת מרחבי T]
	נבחין כי $T_4 \implies T_3 \implies T_2 \implies T_1 \implies T_0$, כלומר המספר מייצג סדר גרירה.
\end{proposition}
\begin{proposition}[שקילות למרחב נורמלי]
	מרחב טופולוגי $X$ נורמלי אם ורק אם לכל קבוצה סגורה $A$ וקבוצה פתוחה $A \subseteq U$ קיימת קבוצה פתוחה $V$ כך ש־$A \subseteq V \subseteq \overline{V} \subseteq U$.
\end{proposition}
\begin{proof}
	בכיוון הראשון נניח ש־$X$ נורמלי וכן ש־$A \subseteq U$ קבוצה סגורה המוכלת בקבוצה פתוחה.
	$A, X \setminus U$ סגורות וזרות, ולכן יש קבוצות פתוחות $V, W$ כך ש־$A \subseteq V \subseteq X \setminus W \subseteq U, X \setminus U \subseteq W$ כך ש־$W \cap V = \emptyset$.
	נובע ש־$A \subseteq V \subseteq \overline{V} \subseteq X \setminus W \subseteq U$.

	בכיוון השני, נניח ש־$A, B \subseteq X$ קבוצות סגורות זרות ולכן $A \subseteq X \setminus B$, נסמן $U = X \setminus B$, אז קיימת קבוצה פתוחה $V$ כך שמתקיים,
	\[
		A \subseteq V \subseteq \overline{V} \subseteq X \setminus B
	\]
	ולכן $B \subseteq X \setminus \overline{V}$ ונובע גם ש־$V \cap (X \setminus \overline{V}) = \emptyset$.
\end{proof}
\begin{lemma}[הלמה של אוריסון]\label{orison_lemma}
	יהי $X$ מרחב $T_4$, אז אם לכל זוג קבוצות סגורות זרות $C, D \subseteq X$, אז יש פונקציה רציפה $f : X \to [0, 1]$ כך ש־$f \mid_C = 1, f \mid_D = 0$.
\end{lemma}
\begin{proof}
	
\end{proof}
\begin{proposition}
	אם $X$ מרחב האוסדורף ($T_2$)  אם ורק אם $\Delta_X = \{ (x, y) \mid x, y \in X \} \subseteq X \times X$ תת־קבוצה סגורה בטופולוגיית המכפלה.
\end{proposition}
\begin{proof}
	נניח ש־$X$ מרחב האוסדורף.
	לכל $x \ne y$ יש $x \in U_{x, y}$ ו־$y \in V_{x, y}$ פתוחות זרות, כלומר $(U_{x, y} \cap V_{x, y}) \cap \Delta_X = \emptyset$.
	נבחין כי
	\[
		X \times X \setminus \Delta_X = \bigcup_{x \ne y} (U_{x, y} \times V_{x, y})
	\]
	וזוהי קבוצה פתוחה.

	בכיוון השני נניח ש־$\Delta_X$ סגורה, אז $X \times X \setminus \Delta_X$ פתוחה, אם $x \ne y$ אז $(x, y) \in (X \times X) \setminus \Delta_X$.
	לכן לפי הגדרת טופולוגיית המכפלה יש $U, V$ פתוחות כך ש־$(x, y) \in U \times V \subseteq X^2 \setminus \Delta_X$ ואף ש־$U \cap V = \emptyset$.
\end{proof}
\begin{proposition}
	עבור $i \in \{1, 2, 3\}$ אם $X$ הוא מרחב $T_i$ ו־$Y \subseteq X$, תת־מרחב אז גם $Y$ הוא מרחב $T_i$.
\end{proposition}
\begin{proof}
	\begin{itemize}
		\item עבור $T_1$ ההוכחה היא טריוויאלית.
		\item עבור $T_2$ ההוכחה היא טריוויאלית.
		\item עבור $T_3$, נזכור שמרחב כזה הוא $T_1$ וכן רגולרי, לכן מספיק שנראה שתת־המרחב הוא רגולרי גם כן.
			יהי $y \in Y$ ויהי $A \subseteq Y$ סגורה כך ש־$y \notin A$.
			לכן יש קבוצה סגורה $C \subseteq X$ כך ש־$A = C \cap Y$.
			עוד אנו יודעים ש־$y \notin C$,
			לכן קיימות $U, V$ פתוחות ב־$X$ מפרידות בין $y$ ל־$C$, $y \in U$ ו־$C \subseteq V$, וכן $U \cap V = \emptyset$,
			אז $A \subseteq V \cap Y$ ו־$y \in U \cap Y$.

			בכיוון השני נניח ש־$A, B \subseteq Y$ וש־$C, D \subseteq X$ סגורות,
			אז $A = C \cap Y, B = D \cap Y$, אבל לא ברור מדוע אפשר לדרוש ש־$C, D$ זרות.
	\end{itemize}
\end{proof}
\begin{remark}
	טענה זו לא נכונה עבור $T_4$.
\end{remark}
Counter examples in Topology של J. Arthur Seebach הוא ספר שבו נוכל למצוא דוגמות רבות למרחבים כאלה.
\begin{proposition}
	אם $X, Y$ מרחבים $T_i$ עבור $i \in \{1, 2, 3\}$ אז גם $X \times Y$ הוא מרחב $T_i$.
\end{proposition}
\begin{proof}
	עבור $T_1$ הטענה ברורה, אנו רוצים להראות שלכל $x, y$,
	\[
		\{ (x, y) \}
		= (X \times (Y \setminus \{y\})) \cup ((X \setminus \{x\}) \times Y)
	\]
	סגורה.

	עבור $T_2$ הטענה קלה גם כן.

	נוכיח עבור $T_3$.
	נניח ש־$X, Y$ הם $T_3$, כלומר $T_1$ ורגולריים ועלינו להראות ש־$X \times Y$ רגולרי.
	נניח ש־$A \subseteq X \times Y$ סגורה וכן ש־$(x, y) \notin A$.
	נגדיר למה, ש־$Z$ מרחב רגולרי אם ורק אם לכל $z \in U \subseteq Z$ עבור $U$ פתוחה ויש $z \in V \subseteq \overline{V} \subseteq U$ פתוחה.
	תוך שימוש בלמה, נסמן $W = (X \times Y) \setminus A$ פתוחה, $(x, y) \in W$, אז נובע שקיימות קבוצות פתוחות $x \in U_x \subseteq X$ ו־$y \in U_Y \subseteq Y$ כך ש־$(x, y) \in U_X \times U_Y \subseteq W$.
	מרגולריות נסיק שיש $V_X, V_Y$ פתוחות כך ש־$x \in V_X \subseteq \overline{V}_X \subseteq U_X$ ו־$y \in V_Y \subseteq \overline{V}_Y \subseteq U_Y$ פתוחות.
	אז מתקיים $(x, y) \in V_X \times V_Y \subseteq \overline{V_X \times V_Y} = \overline{V}_X \times \overline{V}_Y \subseteq U_X \times U_Y$.

	נעבור להוכחת הלמה, לכיוון הראשון $z \in U \subseteq Z$, נסמן $C = Z \setminus U$ סגורה, $z \notin C$ ולכן סגורות זרות $V, W$ כך ש־$z \in V, C \subseteq W, Z \setminus W \subseteq U$.
	אז $z \in V \subseteq \overline{V} \subseteq Z \setminus W \subseteq U$.

	בכיוון השני של הלמה נניח ש־$C$ סגורה, $z \in Z, z \notin C$, אז יש $V$ פתוחה כך ש־$z \in V \subseteq \overline{V} \subseteq Z \setminus C$, וכן $C \subseteq U = Z \setminus \overline{V}$ כך ש־$U \cap V = \emptyset$.
\end{proof}
\begin{proposition}
	אם $(X, \rho)$ מרחב מטרי, אז הוא מרחב $T_4$.
\end{proposition}
\begin{proof}
	תת־קבוצה $E$ של $X$ ו־$x \in X$, נגדיר $\rho(x, E) = \inf\{ \rho(x, y) \mid y \in E \}$.
	אם $E$ סגורה ו־$x \notin E$ אז $\rho(x, E) > 0$.
	נניח ש־$A, B \subseteq X$ סגורות זרות, $\forall a \in A,\ \rho(a, B) > 0, \forall b \in B,\ \rho(b, A) > 0$,
	אז $U = \bigcup_{a \in A} B_{\rho(a, B)}(a)$ ו־$V = \bigcup_{b \in B} B_{\rho(b, A)}(b)$ הן פתוחות וזרות.
\end{proof}
נעבור לדוגמות.
\begin{example}
	לא $T_0$, $\{x, y\}$ עם הטופולוגיה $\{X, \emptyset\}$.

	עבור $T_1$ ולא $T_0$, $\{x, y\}$ עם הטופולוגיה $\{\emptyset, \{x\}, X\}$.

	עבור $T_2$ ולא $T_1$ נגדיר $X = \NN$ והטופולוגיה המושרית מהבסיס של כל הקבוצות שמשלימן סופי.

	עבור $T_2$ ולא $T_3$ נסמן $\RR_{\frac{1}{\NN}}$ את המרחב הבא, קבוצת הנקודות היא $\RR$, 
	\[
		\Bb
		= \{ (a, b) \mid a < b, a, b \in \RR \}
		\cup \left\{ (a, b) \setminus \left\{ \frac{1}{n} \mid n \in \NN \right\}, x, y \in \RR \right\}
	\]
	יש לוודא שזה אכן בסיס.
	היא הטופולוגיה המכילה את הטופולוגיה הקגילה של $\RR$, היא עדינה יותר ומכילה האוסדורף ולכן האוסדורף.
	נראה ש־$\RR_{\frac{1}{\NN}}$ לא $T_3$, נבחין כי $\{\frac{1}{n} \mid n \in \NN \}$ סגורה,
	ונראה כי לא ניתן להפריד בינה לבין $0$.
	נניח ש־$0 \in U$ ו־$K = \{ \frac{1}{n} \mid n \in \NN \}$ פתוחות זרות ונקבל סתירה.
	אם $0 \in U$ פתוחה אז $U$ מכילה איבר בסיס,
	לכן $U$ מכילה קבוצה מהצורה $(a_0, b_0) \setminus K$ עבור $a_0 < 0 < b_0$.
	קיים $m \in \NN$ כך ש־$\frac{1}{m} < b_0$, ואז $(a_0, \frac{1}{m}) \setminus K \subseteq U$.
	$\frac{1}{2m} \in K \subseteq V$ ולכן $(a_1, b_1) \subseteq V$ כאשר $a_1 < \frac{1}{2m} < b_1$.
	$U \cap V \supseteq ((a_0, \frac{1}{m}) \setminus K) \cap (a_1, b_1) \ne \emptyset$, וקיבלנו סתירה.

	עבור $T_3$ ולא $T_4$, נסמן את $\RR_L$, הטופולוגיה הנוצרת על $\RR$ עם הבסיס $L = \{ [a, b) \mid a < b, a, b \in \RR \}$.
	אז $\RR_L$ הוא מרחב $T_4$ ולכן בפרט גם $T_3$.
	אז $\RR_L \times \RR_L$ היא בהכרח $T_3$, אבל נרצה להוכיח ש־$\RR_L \times \RR_L$ היא לא $T_4$.
	\begin{exercise}
		מה הטופולוגיה המושרית על $L$ מ־$\RR_L^2$?
	\end{exercise}
	\begin{solution}
		הטופולוגיה הדיסקרטית.
	\end{solution}
	נסיק שכל תת־קבוצה $A \subseteq L$ היא סגורה ב־$\RR_L^2$.
\end{example}

\section{שיעור 5 --- 8.4.2025}

\subsection{אקסיומות ההפרדה --- המשך}
בשיעור הקודם דיברנו על הדוגמה הבאה, נמשיך לדון בה היום,
\begin{example}
	$\RR_L$ היא הקבוצה $\RR$ עם הבסיס לטופולוגיה $L = \{[a, b) \mid a, b \in \RR, a < b\}$, זוהי טופולוגיה עדינה יותר מהטופולוגיה הסטנדרטית על $\RR$.
	$\RR_L$ היא $T_4$, וראינו גם כי $\RR_L^2$ היא $T_3$ אבל לא $T_4$.
\end{example}
בנוסף הגדרנו את הקבוצה $L = \{ (-x, x) \mid x \in \RR \} \subseteq \RR_L^2$, זוהי קבוצה סגורה, וכן הטופולוגיה המושרית מ־$\RR_L^2$ על $L$ היא הטופולוגיה הדיסקרטית על $L$.
הסקנו גם שכל $A \subseteq L$ היא סגורה ב־$L$,
כלומר לכל $A \subseteq L$ יש קבוצה $C_A \subseteq \RR_L^2$ סגורה כך ש־$A = L \cap C_A$.
שתי האחרונות סגורות ב־$\RR_L^2$ ולכן גם $A$ סגורה ב־$\RR_L^2$.
נניח ש־$\RR_L^2$ היא $T_4$, בפרט זהו מרחב נורמלי, ולכן כל שתי קבוצות סגורות זרות ניתנות להפרדה.
בפרט לכל $A \subseteq L$ יש קבוצות פתוחות זרות $U_A, V_A \subseteq \RR_L^2$ כך ש־$A \subseteq U_A, L \setminus A \subseteq V_A$.
נקבע לכל $A \subseteq L$ זוג קבוע כזה (וניצור מיפוי).
נתבונן ב־$D = \{ (r, s) \mid r, s \in \QQ \} \subseteq \RR_L^2$, ונגדיר $\psi(A) = U_A \cap D$, כלומר $\psi : \Pp(L) \to \Pp(D)$.
אם נבחר את $A = \emptyset$ אז גם $U_A = \emptyset$ ובהתאם $\psi(\emptyset) = \emptyset$, ולהפך אם $A = L$ אז $U_A = \RR_L^2, V_A = \emptyset$ ולכן $\psi(A) = \RR_L^2$.
\begin{proposition}
	$\psi$ חד־חד ערכית,
	ולכן מקבלת סתירה.
\end{proposition}
\begin{proof}
	נוכיח חד־חד ערכיות,
	נניח ש־$\emptyset \ne A \subsetneq L$, אז $\psi(A) \ne \emptyset$ כי $D$ צפופה ו־$U_A \ne \emptyset$.
	גם $V_A \ne \emptyset$, שכן $L \setminus A \subseteq V_A$, ולכן נסיק ש־$V_A \cap D \ne \emptyset$.
	עתה נניח ש־$\emptyset \ne A, B \subsetneq L$ כך ש־$A \ne B$, אז בלי הגבלת הכלליות יש $a \in A$ כך ש־$a \notin B$.
	נובע אם כך ש־$a \in L \setminus B \subseteq V_B$ ו־$a \in A \subseteq U_A$ ולכן נובע ש־$a \in U_A \cap V_B$, וזו אף קבוצה פתוחה.
	נסיק ש־$U_A \cap V_B \ne \emptyset$, אז $p \in U_A \cap V_B \cap D$ מקיימת $p \in \psi(A)$ ו־$p \notin \psi(B)$ ובהתאם $\psi(A) \ne \psi(B)$.

	$D$ קבוצה בת־מניה ו־$L$ היא מהעוצמה של $\RR$.
	יש לנו שיכון $\Pp(D) \hookrightarrow \RR$, אבל $|\RR| = |L|$, אז נוכל לבנות $\Pp(L) \hookrightarrow \Pp(D) \hookrightarrow L$ וזה בלתי אפשרי.
\end{proof}
נעבור להוכחת הלמה של אוריסון, למה\ \ref{orison_lemma}.
\begin{proof}[הוכחת הלמה של אוריסון]
	נניח ש־$X$ מרחב $T_4$, ויהיו $C_0 = C$ וכן $V_1 = X \setminus D$, עבור הקבוצות הסגורות הזרות $C, D \subseteq X$.
	נבחין כי $C_0$ סגורה ו־$V_1$ פתוחה, ולכן קיימות קבוצות $C_0 \subseteq V_{\frac{1}{2}} \subseteq C_{\frac{1}{2}} \subseteq V_1$.
	שוב מדובר בקבוצה סגורה ובקבוצה פתוחה.
	נגדיר כך באופן רקורסיבי קבוצות $C_{\frac{k}{2^n}}, V_{\frac{k}{2^n}}$ לכל $n \in \NN$ ו־$0 < k < 2^n$, לכן,
	\[
		C_0
		\subseteq V_{\frac{1}{2^n}}
		\subseteq C_{\frac{1}{2^n}}
		\subseteq V_{\frac{2}{2^n}}
		\subseteq C_{\frac{2}{2^n}}
		\ldots 
	\]
	ונגדיר לכל $x \in X$ את הפונקציה,
	\[
		f(x)
		\begin{cases}
			\inf\{ t \in [0, 1] \mid x \in V_t \} & \exists t, x \in V_t \\
			1 & \text{else}
		\end{cases}
	\]
	אנו טוענים ש־$f$ מקיימת את האמור, כלומר $f(x) = 0$ לכל $x \in C$, וכן $f(x) = 1$ לכל $x \in D$, ו־$f$ רציפה.
	נשים לב ש־$C = C_0 \subseteq V_{\frac{1}{2^n}}$ לכל $n \in \NN$, ולכן נובע ש־$f(x) = 0$.
	נבחין גם שעבור $x \in D$ נובע ש־$x \notin V_t$ לאף $t$ ולכן $f(x) = 1$.
	נותר אם כן להראות רציפות.
	אנו יודעים כי $f : X \to [0, 1]$ ולכן עלינו לבדוק תת־קבוצות של $[0, 1]$, אבל מספיק לבדוק את הרציפות עבור תת־בסיס של הקטע, שכל מקור של קבוצה פתוחה הוא פתוח.
	נבחר את תת־הבסיס $\Bb = \{ [0, b) \mid 0 < b \le 1 \} \cup \{ (b, 1] \mid 0 \le b < 1\}$.
	נתבונן ב־$[0, b)$, כזה, נניח שמתקיים,
	\[
		x \in f^{-1}([0, b))
	\]
	אז נובע ש־$f(x) < b$, לכן קיים $t$ כך ש־$f(x) < t < b$ מספר דיאדי (מהצורה הדרושה).
	לכן $x \notin V_s$ לכל $s < t$, ולכן נקבל ש־$f^{-1}([0, b)) \subseteq \bigcup V_t$
	נניח ש־$x \in \bigcup_t V_t$ אז יש $t_0 < b$ כך ש־$x \in V_t$ ולכן $f(x) < t_0 < b$ ונוע ש־$x \in f^{-1}([0, b))$ כפי שרצינו.
	אז מצאנו ש־$f^{-1}((b, 1])$ פתוחה אם ורק אם $f^{-1}([0, b])$ סגורה, ולכן אנו טוענים שמתקיים $f^{-1}([0, b]) = \bigcap_{b < t} C_t$.
	אם $x \notin f^{-1}([0, b])$ אז $b < f(x) \le 1$, אז מצפיפות קיימים $t_1, t_2$ דיאדיים כך ש־$b < t_1 < t_2 < f(x)$ ולכן $x \notin V_{t_2}$ וכן $C_{t_1} \subseteq V_{t_1}$ ולכן $x \notin \bigcap_{b < t} C_t$.
	אם $x \in f^{-1}([0, b])$ אז $0 \le f(x) \le b < 1$ לכל $b < t$ מתקיים $x \in V_t \subseteq C_t$ ונובע ש־$x \in \bigcap_{b < t} C_t$.
\end{proof}

\section{שיעור 6 --- 21.4.2025}
\subsection{אקסיומות מנייה}
ניזכר בהגדרה\ \ref{topology_basis_definition} וניעזר בו כדי להגדיר את ההגדרה הבאה,
\begin{definition}
	מרחב טופולוגי $X$ יקרא בן־מניה אם הוא מקיים את אקסיומת המנייה הראשונה, כלור אם לכל $X_0 \subseteq X$ יש בסיס בן־מניה משהו יש בסיס בן־מניה.
	 אקסיומת המנייה השנייה, אם יש בסיס בן־מניה לטופולוגיה.
\end{definition}
\begin{definition}
	$X$ יקרא מרחב לינדולף, אם לכל כיסוי פתוח של $X$ יש כיסוי בן־מניה $X = \bigcup_{\beta \in U} V_{\beta}$ עבור $V_{\beta}$ פתוחות וכן $U$ בת־מניה.
\end{definition}
\begin{definition}
	$X$ נקרא ספרבילי אם יש ב־$X$ תת־קבוצה צפופה בת־מניה.
\end{definition}
\begin{proposition}
	מרחב רגולרי המקיים את אקסיומת המנייה השנייה הוא נורמלי.
\end{proposition}
\begin{proof}
	נניח ש־$X$ רגולרי המקיים את אקסיומת המנייה השנייה.
	יהי $\Bb$ בסיס בן־מניה.
	אנו רוצים להראות נורמליות, נניח ש־$A, B \subseteq X$ זרות וסגורות (ולא ריקות).
	ואנו רוצים למצוא להן הפרדה.
	לכל $a \in A$ כך ש־$a \notin \Bb$ יש קבוצה פתוחה $U_a$ כך ש־$a \in U_a \subseteq \overline{U}_a \subseteq X \setminus B$.
	ניתן לבחור את $U_a$ כך ש־$U_a \in \Bb$ ולכן האוסף $\{ U_a \mid a \in A \}$ הוא בן־מניה, ונוכל לכתוב אותו על־ידי $\{ U_{a_n} \mid n \in \NN \}$, כאשר $a_n \in A$ לכל $n$.
	קיבלנו ש־$a \in U_a \subseteq \overline{U}_a \subseteq A \setminus B$
	האוסף $\{ U_a \mid a \in A \}$ מכסה את $A$ אבל גם $A \subseteq \bigcup_{a \in A} U_a = \bigcup_{n = 1}^\infty U_{a_n}$.
	באותו אופן אפשר למצוא קבוצות פתוחות $V_b \in \Bb$ כך ש־$b \in B$ כך ש־$b \in V_b \subseteq \overline{V}_b \subseteq X \setminus A$ וסדרה ${\{ b_n \}}_{n = 1}^\infty \subseteq B$,
	כך ש־$\{ V_b \} = \{ V_b \mid b \in B \}$ ו־$\{ V_{b_n} \}$ הוא כיסוי של $B$.

	לכל $k \in \NN$ נגדיר $S_k = U_{a_k} \setminus \bigcup_{i = 1}^k \overline{V}_{b_i}$ וכן $T_k = V_{b_k} \setminus \bigcup_{i = 1}^k \overline{U}_{a_k}$.
	שתי אלה קבוצות פתוחות לכל $k$, ונגדיר בהתאם $S = \bigcup_{k \in \NN} S_k$ וכן $T = \bigcup_{k \in \NN} T_k$, גם אלה קבוצות פתוחות.
	נבחין כי $A \subseteq S, B \subseteq T$.
	נזכור ש־$\overline{V}_b \subseteq X \setminus A$ ונבדוק ש־$S \cap T = \emptyset$.
	אם החיתוך לא ריק, אז קיים$m, n \in \NN$ כך ש־$S_n \cap T_m \ne \emptyset$, בלי הגבלת הכלליות $n \le m$ ולכן נובע,
	\[
		S_m = U_{b_k} \setminus \bigcup_{i = 1}^k \overline{T}_i \supseteq T_n
	\]
	וזו סתירה.
\end{proof}
\begin{definition}[מרחב מטריזבילי]
	מרחב טופולוגי $X$ נקרא מטריזבילי אם קיימת מטריקה על $X$ שמשרה את הטופולוגיה.
\end{definition}
\begin{remark}
	תת־מרחב של מרחב מטריזבילי הוא מטריזבילי.
\end{remark}
\begin{theorem}[משפט המטריזביליות של אורסון]
	אם $X$ מרחב טופולוגי $T_3$ המקיים את אקסיומת המנייה השנייה,
	אז $X$ מטריזבילי.
\end{theorem}
\begin{proof}
	הרעיון הכללי הוא לשכן במרחב מטרי ב־${[0, 1]}^\NN$ עם טופולוגיית המכפלה עם המטריקה,
	\[
		d(x, y)
		= \sum_{n = 1}^\infty \frac{|x_n - y_n|}{2^n}
	\]
	ונחש העתקה $\psi : X \to {[0, 1]}^\NN$ כך ש־$\psi$ חד־חד ערכית היא משהו ומ־$X$ ל־$\psi(X)$.

	$X \to {[0, 1]}^\NN \xrightarrow{\pi_k} [0, 1]$

	לכל $x, y \in X$ כך ש־$x \ne y$ יש פתוחות זרות $U_{xy}, W_{xy} \subseteq \Bb$ כך ש־$x \in U_{xy}, y \in W_{xy}$.
	ניתן למצוא קבוצות בסיס $x \in V_{xy} \subseteq \overline{V}_{xy} \subseteq U_{xy}$.
	נתבונן באוסף כל הזוגות $\Lambda = \{ (u, u) \in \Bb^2 \mid \emptyset \not\subseteq V \subseteq \overline{V} \subseteq U \}$.
	אז $\Lambda$ בת־מניה.
	מהלמה של אוריסון קיימת פונקציה $f = f_{(u, v)} : X \to [0, 1]$ כך ש־$f \mid_{X \setminus U} = 1$ ו־$f \mid_{\overline{V}} = 0$.
	אנו מקבלים סדרת פונקציות $\{ g_k \mid k \in \NN \} = \{ f_{(u, v)} \mid (u, v) \in \Lambda \}$.
	נגדיר $\psi : X \to {[0, 1]}^\NN$ על־ידי $\psi(x)(k) = g_k(x)$.
	אנו טוענים כי $\psi$ היא רציפה, חד־חד ערכית וכן ש־$\psi : X \to \psi(X)$ היא הומיאומורפיזם.
	רציפות בטופולוגיית המכפלה שקולה לרציפות בכל קורדינטה, לכן מרציפות $g_k$ לכל $k$ נקבל רציפות $\psi$.
	חד־חד ערכיות נובעת מכך שלכל $x, y \in X$ כך ש־$x \ne y$ יש $V, U \in \Bb$ כך ש־$x \in V \subseteq \overline{V}, y \in X \setminus U$.
	יש $g_k = f_{(v, u)}$ ו־$g_k(y) = 1, g_k(x) = 0$.
	נשאר להראות הומיאומורפיזם.
	אנו יודעים ש־$\psi$ חד־חד ערכית, וצריך להראות שלכל ש־$\psi^{-1} : E \to X$ היא רציפה כאשר $E = \psi(X)$, כלומר צריך להראות שלכל $W \subseteq X$ פתוחה, שגם $\psi(W)$ פתוחה ב־$E$.
	לכל $x \in W$ קיימת $U \in \Bb$ כך ש־$x \in U \subseteq W$, וקיימת $V \in \Bb$ כך ש־$x \in V \subseteq \overline{V} \subseteq U$.
	יהי $k(x) \in \NN$ כך ש־$g_{k(x)} = f_{(v, u)}$ ומתקיים, $g_{k(x)}(x) = 0$ וכן $g_{k(x)} \mid_{X \setminus U} = 1$.
	אז $x \in g^{-1}([0, 1)) \subseteq U \subseteq W$ ונובע ש־$\bigcup_{x \in W} g_{k(x)}^{-1}([0, 1)) = W$.
	$g_{k(x)} = \pi_{k(x) \circ \psi}$ ולכן $g_{k(x)}^{-1} = \psi^{-1} \circ \pi_{k(x)}^{-1}$ ולכן,
	\[
		W
		= \bigcup_{x \in W} \psi^{-1}(\pi_{k(x)}^{-1}([0, 1)))
		= \psi^{-1}(\bigcup_{x \in W} \pi_{k(x)}^{-1}([0, 1)))
	\]
	ונובע $\psi(W) = (\bigcup_{x \in W} \pi_{k(x)}^{-1}([0, 1))) \cap E$.
\end{proof}

\subsection{קשירות}
\begin{definition}[קשירות]
	מרחב טופולוגי $X$ יקרא קשיר אם לא ניתן להציג אותו כאיחוד של שתי קבוצות פתוחות זרות לא ריקות.
\end{definition}
\begin{remark}
	באופן שקול גם אם לא ניתן להציג את המרחב כאיחוד זר של קבוצות סגורות.
	זאת שכם אם $X = U \cup V$ אז $U^C \cap V^C = \emptyset$ וכן אם $U \cap V = \emptyset$ אז $U^C \cup V^C = X$ וכמובן $U^C, V^C$ פתוחות.
\end{remark}
\begin{example}
	מהן תתי־הקבוצות הקשירות של $\RR$?
	התשובה היא קטעים, $(a, b), [a, b], (a, b], [a, b)$.
\end{example}
\begin{remark}
	מרחב טופולוגי $X$ הוא קשיר אם ורק אם כל פונקציה רציפה $f : X \to \{0, 1\}$ עם הטופולוגיה הדיסקרטית, היא קבועה.
\end{remark}
\begin{proposition}[תכונות של קשירות]
	התכונות הבאות מתקיימות,
	\begin{enumerate}
		\item אם $f : X \to Y$ רציפה ו־$X$ קשיר אז $f(X)$ קשירה
		\item אם $A \subseteq X$ קשירה אז $\overline{A}$ קשירה
		\item למת כוכב, אם ${\{ A_{\alpha} \}}_{\alpha \in I}$ תתי־קבוצות קשירות וקיים $\beta \in I$ כך ש־$A_{\alpha} \cap A_{\beta}$ קשיר ב־$X$ לכל $\alpha \in I$ אז $\bigcup_{\alpha \in I} A_{\alpha}$ קשירה
		\item אם ${\{ X_{\alpha} \}}_{\alpha \in I}$ קבוצת מרחבים טופולוגיים קשירים אז $Y = \prod_{\alpha \in I} X_{\alpha}$ קשירה
	\end{enumerate}
\end{proposition}
\begin{proof}
	נוכיח את טענה 2.
	נניח ש־$\overline{A}$ לא קשירה, לכן נובע שיש $f : \overline{A} \to \{0, 1\}$ לא קבועה.
	בלי הגבלת הכלליות נניח ש־$f(A) = \{ 0 \}$, אבל $\{ 0 \} \subseteq \{0, 1\}$ סגורה ולכן $A \subseteq f^{-1}(\{0\})$ סגורה ונובע ש־$\overline{A} \subseteq f^{-1}(\{0\})$ וזו סתירה.

	נעבור להוכחת טענה 4.
	נתונים ${\{ X_{\alpha} \}}_{\alpha \in I}$ מרחבים טופולוגיים ונרצה להראות ש־$Y$ קשיר.
	אם $A, B$ מרחבים טופולוגיים קשירים אז $A \times B$ קשיר, כנביעה מטענה 3, שכן,
	\[
		A \times B
		= ( \bigcup_{a \in A} \{ a \} \times B ) \cup ( \bigcup_{b \in B} A \times \{ b \} )
	\]
	נרצה למצוא תת־קבוצה של $Y$ שתהיה צפופה וקשירה.
	נקבע $f \in Y$, כאשר $f : I \to \bigcup X_{\alpha}$.
	יש $f$ כזו מאקסיומת הבחירה.
	נגדיר את $Z = \{ h \in Y \mid |\{ \alpha \in I \mid h(\alpha) \ne f(\alpha) \}| < \infty \} = \bigcup_{F \subseteq I, |F| < \infty} P_F$ כאשר $P_F = \{h \in Y \mid h(\alpha) = f(\alpha) \forall \alpha \notin F\}$.
	אנו טוענים שתי טענות, הראשונה היא שלכל $F$ סופית $P_F$ קשירה, השנייה היא ש־$\bigcup P_F = Z$ קשירה והשלישית היא ש־$Z$ צפופה.
	נבחין כי $P_F \cong \prod_{y \in F} X_y$ מהגדרת טופולוגיית המכפלה.
\end{proof}

\listoftheorems[title=הגדרות ומשפטים,ignoreall,show={theorem,definition},swapnumber,onlynamed={proposition}]

\end{document}
