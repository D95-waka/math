\input{../article_base.tex}
\title{מבוא לטופולוגיה --- סיכום}
\setcounter{secnumdepth}{2}
% chktex-file 9
% chktex-file 17

\usepackage{fancyhdr}
\pagestyle{fancy}
\renewcommand{\headrulewidth}{0pt}

\begin{document}
\maketitle
\maketitleprint{}

\tableofcontents

\section{שיעור 1 --- 24.3.2025}
\subsection{מבוא}
בעבר דיברנו על מרחבים מטריים, באינפי 1 מתבוננים ב־$\RR$ והגדרנו את מושג הגבול של סדרות, ולאחריו את המושג של פונקציה רציפה $f : \RR \to \RR$.
ההגדרה הייתה ש־$f$ תיקרא רציפה אם לכל $x \in \RR$ ולכל $\lim_{n \to \infty} x_n = x$ מתקיים $\lim_{n \to \infty} f(x_n) = f(x)$.
באינפי 3 כבר ראינו את המושג הכללי והרחב יותר של רציפות במרחבים מטריים.
ניזכר בהגדרה של מרחב מטרי.
\begin{definition}[מרחב מטרי]
	מרחב מטרי הוא זוג $(X, d)$ כאשר $X$ קבוצה לא ריקה ו־$d : X \times X \to \RR$ פונקציה (הנקראת מטריקה) המקיימת,
	\begin{enumerate}
		\item $d(x, y) = d(y, x)$ לכל $x, y \in X$
		\item $\forall x, y \in X, d(x, y) \ge 0$ וכן $d(x, y) = 0 \iff x = y$
		\item אי־שוויון המשולש, $\forall x, y, z \in X, d(x, y) \le d(x, y) + d(y, z)$
	\end{enumerate}
\end{definition}
\begin{example}
	נראה דוגמות למרחבים מטריים,
	\begin{enumerate}
		\item $\RR$ יחד עם $d(x, y) = |x - y|$
		\item $(\RR^n, d_2)$ המוגדרת על־ידי $d_2(\bar{x}, \bar{y}) = \sqrt{\sum_{i = 1}^{n} {|x_i - y_i|}^2}$
		\item נוכל עבור $\RR^n$ להגדיר את $d_p(\bar{x}, \bar{y}) = {(\sum_{i = 1}^{n} {|x_i - y_i|}^p)}^\frac{1}{p}$ ואת מטריקת אינסוף, $d_\infty(\bar{x}, \bar{y}) = \max_{1 \le i \le n} |x_i - y_i|$
		\item עבור $C([a, b])$ קבוצת הפונקציות הרציפות $[a, b] \to \RR$ עבור $a < b$, ונגדיר את המטריקה $\rho(f, g) = \sup_{x \in [a, b]} |f(x) - g(x)|$
	\end{enumerate}
\end{example}
נראה את ההגדרה הפורמלית של רציפות,
\begin{definition}[רציפות]
	תהי $f : X \to Y$ עבור $(X, d), (Y, \rho)$ מרחבים מטריים, אז נאמר ש־$f$ רציפה אם ורק אם לכל $\epsilon > 0$ קיים $\delta > 0$ כך שאם $d(x', x) < \delta$ אז $\rho(f(x'), f(x)) < \epsilon$.
\end{definition}
אבל יותר קל לדבר במונחים של קבוצות פתוחות.
\begin{definition}[כדור]
	עבור $(X, d)$ מרחב מטרי,
	נסמן $B(r, x) = B_r(x) = \{ z \in X \mid d(x, z) < r \}$.
\end{definition}
\begin{definition}[קבוצה פתוחה]
	יהי $(X, d)$ מרחב מטרי, תת־קבוצה $U \subseteq X$ תיקרא פתוחה אם לכל $x \in U$ קיים $r > 0$ כך ש־$x \in B(x, r) \subseteq U$.
\end{definition}
\begin{definition}[הגדרה שקולה לרציפות]
	$f : X \to Y$ תיקרא רציפה אם לכל $V \subseteq Y$ קבוצה פתוחה ב־$Y$ מתקיים $f^{-1}(V) = \{ x \in X \mid f(x) \in V \}$ קבוצה פתוחה ב־$X$.
\end{definition}
\begin{definition}[טופולוגיה]
	תהי $X$ קבוצה (לא ריקה), \textbf{טופולוגיה} על $X$ היא אוסף $\tau \subseteq \Pp(X)$, כך שמתקיימים התנאים הבאים,
	\begin{enumerate}
		\item $X, \emptyset \in \tau$
		\item $\tau$ סגור לאיחוד, כלומר אם ${\{U_\alpha\}}_{\alpha \in I}$ לקבוצת אינדקסים $I$, כך ש־$\forall \alpha \in I, U_\alpha \in \tau$ אז $\bigcup_{\alpha \in I} U_\alpha \in \tau$
		\item $\tau$ סגור לחיתוכים סופיים, כלומר לכל $U, V \in \tau$ מתקיים $U \cap V \in \tau$
	\end{enumerate}
\end{definition}
\begin{definition}[מרחב טופולוגי]
	זוג $(X, \tau)$ כאשר $X$ קבוצה לא ריקה ו־$\tau$ טופולוגיה על $X$, יקרא מרחב טופולוגי.
\end{definition}
\begin{remark}
	בעצם הגדרנו כבר מתי פונקציה $f : X \to Y$ עבור מרחבים טופולוגיים $(X, \tau), (Y, \Omega)$ היא רציפה, כאשר $f^{-1}(U) \in \tau$ לכל $U \in \Omega$.
\end{remark}
\begin{notation}
	איברי $\tau$ יקראו קבוצות פתוחות.
\end{notation}
\begin{definition}
	אם $(X, \tau)$ מרחב טופולוגי אז תת־קבוצה $A \subseteq X$ תיקרא סגורה אם $X \setminus A \in \tau$, כלומר המשלים של $A$ היא קבוצה פתוחה.
\end{definition}
\begin{example}
	יהי $(X, d)$ מרחב מטרי, נגדיר $\tau = \{ U \subseteq X \mid \forall x \in U \exists r > 0, B(x, r) \subseteq U \}$, כלומר נגדיר טופולוגיה באופן טריוויאלי כנביעה מהמרחב המטרי.
\end{example}
\begin{exercise}
	הוכיחו כי אכן זהו מרחב טופולוגי.
\end{exercise}
\begin{example}
	יהי $X$ קבוצה כלשהי, אז ניתן להגדיר על $X$ טופולוגיה $\tau_0 = \{ \emptyset, X \}$.
	טופולוגיה זו נקראת טופולוגיה טריוויאלית.
\end{example}
\begin{example}
	נגדיר $\tau_1 = \Pp(X)$ עבור קבוצה $X$, גם קבוצה זו היא טופולוגיה, והיא נקראת הטופולוגיה הדיסקרטית.
\end{example}
\begin{example}
	נניח ש־$(Y, \tau)$ מרחב טופולוגי, ותהי $f : (Y, \tau) \to (X, \tau_0)$, מתי $f$ היא רציפה? התשובה היא שהיא רציפה תמיד.
	מתי $f : (Y, \tau) \to (X, \tau_1)$ רציפה? תלוי בהגדרת הפונקציה, אבל במקרה שבו היא אכן רציפה, אז היא רציפה בכל טופולוגיה שהיא.
	לעומת זאת כל $f : (X, \tau_1) \to (Y, \tau)$ היא רציפה.
\end{example}
\begin{remark}
	לא כל טופולוגיה נובעת ממטריקה.
	לדוגמה הטופולוגיה הטריוויאלית על מרחב עם לפחות 2 נקודות.
\end{remark}
\begin{remark}
	נניח $x, y \in X$ אז נבחר $r = \frac{1}{2} d(x, y)$ ואז $y \notin B(x, r)$ ולכן $\emptyset \ne B(x, r) \ne X$, קל לראות שביחס לטופולוגיה שמושרית מהמטריקה $d$, הקבוצה $B(x, r)$ קבוצה פתוחה.
\end{remark}
\begin{example}
	נגדיר $X = \CC^n$ עבור איזשהו $n \in \NN$ ונגדיר $\Ff = \{ A \subseteq \CC^n \mid \exists {\{f_i\}}_{i \in I} \subseteq \CC[x_1, \dots, x_n], A = \{(p_1, \dots, p_n) \mid \forall i \in I, f_i(p_1, \dots, p_n) = 0 \}\}$.
\end{example}
\begin{definition}[בסיס לטופולוגיה]
	בסיס לטופולוגיה הוא אוסף $\Bb$ של תתי־קבוצות של $X$ כך שמתקיים,
	\begin{enumerate}
		\item לכל $x \in X$ יש $B \in \Bb$ כך ש־$x \in B$
		\item לכל $A, B \in \Bb$ ולכל $x \in A \cap B$ יש $C \in \Bb$ כך ש־$x \in C \subseteq A \cap B$
	\end{enumerate}
\end{definition}
\begin{proposition}
	עבור בסיס $\Bb$ האוסף $\tau_\Bb = \{ U \subseteq X \mid U \text{ is a union of elements of } \Bb \}$ היא טופולוגיה,
	\[
		\forall \alpha \in I, B_\alpha \in \Bb, U = \bigcup_{\alpha \in I} B_\alpha
	\]
\end{proposition}
\begin{proof}
	מכיוון ש־$\tau_\Bb$ סגורה לחיתוך סופי, אז אם $U, V \in \tau_\Bb$ אז $U = \bigcup_{\alpha \in I} B_\alpha \in \Bb$ וכן $V = \bigcup_{\beta \in J} A_\beta, A_\beta \in \Bb$, אז מתקיים,
	\[
		U \cap V
		= (\bigcup_{\alpha \in I} B_\alpha) \cap (\bigcup_{\beta \in J} A_\beta)
		= \bigcup_{\alpha, \beta \in I \times J} B_\alpha \cap A_\beta
		= D
	\]
	לכן לכל $x \in U \cap V$ ישנם $\alpha_0 \in I, \beta_0 \in J$ כך ש־$x \in B_{\alpha_0} \cap A_{\beta_0}$,
	אבל מהגדרת הבסיס קיימת קבוצה $C_{\alpha_0, \beta_0} \in \Bb$ כך ש־$C_{\alpha_0, \beta_0} \subseteq B_{\alpha_0} \cap A_{\beta_0}$.
	לכן $D \subseteq \bigcup_{(x, \alpha, \beta)} C_{x, \alpha, \beta}$.
	בהתאם מצאנו סגירות לחיתוך סופי.
\end{proof}
\begin{remark}
	יהי $(X, d)$ מרחב מטרי, אז $\{ B(x, r) \subseteq X \mid x \in X, r > 0 \}$ הוא טופולוגיה.
	אבל עכשיו נוכל להגדיר גם את $\{ B(x, \frac{1}{n}) \subseteq X \mid x \in X, n \in \NN \}$, זהו בסיס לטופולוגיה לאותה הטופולוגיה שהגדרנו למרחב המטרי.
\end{remark}
\begin{exercise}
	הוכיחו שזהו אכן בסיס עבור המרחב הטופולוגי הנתון.
\end{exercise}
\begin{example}
	נניח ש־$X = \ZZ$, ונגדיר את הבסיס $C$ להיות אוסף הסדרות האריתמטיות הדו־צדדיות, כלומר $C = \{ a + d \ZZ \mid a, d \in \ZZ, d \ne 0 \}$.
	אנו טוענים כי זהו אכן בסיס (לטופולוגיה).
	נתבונן בזוג קבוצות ב־$C$, $a + d \ZZ, b + q \ZZ$, ונניח ש־$p \in (a + d \ZZ) \cap (b + q \ZZ)$ אז $p \in p + dq \ZZ \subseteq (a + d \ZZ) \cap (b + q \ZZ)$.
	נגדיר טופולוגיית $\tau_C$.

	קבוצות סגורות הן משלימים לקבוצות פתוחות.

	כל סדרה אריתמטית דו־צדדית אינסופית היא גם פתוחה וגם סגורה.
	בפרט חיתוך סופי של סדרות אריתמטיות הוא סגור.
	לכן המשלים שלו הוא פתוח.
\end{example}
\begin{conclusion}[משפט אוקלידס]
	יש אינסוף מספרים ראשוניים.
\end{conclusion}
\begin{proof}
	נניח בשלילה כי יש מספר סופי של ראשוניים, $p_1, \dots, p_k$ עבור $k \in \NN$.
	נבחן את $\bigcup_{i = 1}^k p_i \ZZ$, זוהי קבוצה פתוחה וגם סגורה, לכן
	\[
		\bigcup_{i = 1}^k p_i \ZZ
		= \ZZ \setminus \{ -1, 1 \}
	\]
	ולכן נובע ש־$\{-1, 1\}$ קבוצה פתוחה וזו כמובן סתירה.
\end{proof}
\begin{proposition}[צמצום מרחב טופולוגי]
	נניח ש־$(X, \tau)$ מרחב טופולוגי, לכל $\emptyset \ne Y \subseteq X$ נגדיר $\tau_Y = \{ U \cap Y \mid U \in \tau \}$.
	אז $\tau_Y$ היא טופולוגיה.
	אם $Y \in \tau$ אז $\tau_Y = \{ W \in \tau \mid W \subseteq Y \}$.
\end{proposition}
\begin{proposition}[טופולוגיית מכפלה]
	נניח ש־$(X_1, \tau_1)$ ו־$(X_2, \tau_2)$ מרחבים טופולוגיים, אז נגדיר טופולוגיה על מרחב המכפלה $X_1 \times X_2$ על־ידי
	\[
		\tau_{1, 2}
		= \{ U_1 \times U_2 \mid U_1 \in \tau_1, U_2 \in \tau_2 \}
	\]
	אז $\tau_{1, 2}$ הוא בסיס והטופולוגיה המוגדרת על־ידו נקראת טופולוגיית המכפלה.
\end{proposition}
\begin{example}
	נוכל לבנות כך מכפלה של כמות סופית או אינסופית של מכפלות טופולוגיות.
	עבור אוסף אינסופי (בן־מניה או לא בהכרח) אנו צריכים להיזהר, נניח ש־$(X_\alpha, \tau_\alpha)$ עבור $\alpha \in I$, אז נגדיר
	\[
		\tau_b = \{ \prod_{\alpha \in I} U_\alpha \mid \forall \alpha \in I, U_\alpha \in \tau_\alpha \}
	\]
	זהו בסיס לטופולוגיה שנקרא טופולוגיית הקופסה.
	לעומת זאת נוכל להגדיר גם את
	\[
		\tau_p
		= \{ \prod_{\alpha \in I} U_\alpha \mid U_\alpha = X_\alpha \text{ for almost all } \alpha \in I \}
	\]
	כלומר $\prod_{\alpha \in I} = \{ f : I \to \bigcup_{\alpha \in I} X_\alpha \mid \forall \alpha \in I, f(x) \in X_\alpha \}$.
\end{example}

\listoftheorems[title=הגדרות ומשפטים,ignoreall,show={theorem,definition},swapnumber,onlynamed={proposition}]

\end{document}
