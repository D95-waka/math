\input{../article_base.tex}
\title{מבוא לטופולוגיה --- סיכום}
\setcounter{secnumdepth}{2}
% chktex-file 9
% chktex-file 17

\usepackage{fancyhdr}
\pagestyle{fancy}
\renewcommand{\headrulewidth}{0pt}

\begin{document}
\maketitle
\maketitleprint[purple]

\tableofcontents

\section{שיעור 1 --- 24.3.2025}
\subsection{מבוא}
בעבר דיברנו על מרחבים מטריים, באינפי 1 מתבוננים ב־$\RR$ והגדרנו את מושג הגבול של סדרות, ולאחריו את המושג של פונקציה רציפה $f : \RR \to \RR$.
ההגדרה הייתה ש־$f$ תיקרא רציפה אם לכל $x \in \RR$ ולכל $\lim_{n \to \infty} x_n = x$ מתקיים $\lim_{n \to \infty} f(x_n) = f(x)$.
באינפי 3 כבר ראינו את המושג הכללי והרחב יותר של רציפות במרחבים מטריים.
ניזכר בהגדרה של מרחב מטרי.
\begin{definition}[מרחב מטרי]
	מרחב מטרי הוא זוג $(X, d)$ כאשר $X$ קבוצה לא ריקה ו־$d : X \times X \to \RR$ פונקציה (הנקראת מטריקה) המקיימת,
	\begin{enumerate}
		\item $d(x, y) = d(y, x)$ לכל $x, y \in X$
		\item $\forall x, y \in X, d(x, y) \ge 0$ וכן $d(x, y) = 0 \iff x = y$
		\item אי־שוויון המשולש, $\forall x, y, z \in X, d(x, y) \le d(x, y) + d(y, z)$
	\end{enumerate}
\end{definition}
\begin{example}
	נראה דוגמות למרחבים מטריים,
	\begin{enumerate}
		\item $\RR$ יחד עם $d(x, y) = |x - y|$
		\item $(\RR^n, d_2)$ המוגדרת על־ידי $d_2(\bar{x}, \bar{y}) = \sqrt{\sum_{i = 1}^{n} {|x_i - y_i|}^2}$
		\item נוכל עבור $\RR^n$ להגדיר את $d_p(\bar{x}, \bar{y}) = {(\sum_{i = 1}^{n} {|x_i - y_i|}^p)}^\frac{1}{p}$ ואת מטריקת אינסוף, $d_\infty(\bar{x}, \bar{y}) = \max_{1 \le i \le n} |x_i - y_i|$
		\item עבור $C([a, b])$ קבוצת הפונקציות הרציפות $[a, b] \to \RR$ עבור $a < b$, ונגדיר את המטריקה $\rho(f, g) = \sup_{x \in [a, b]} |f(x) - g(x)|$
	\end{enumerate}
\end{example}
נראה את ההגדרה הפורמלית של רציפות,
\begin{definition}[רציפות]\label{function_continuous_definition}
	תהי $f : X \to Y$ עבור $(X, d), (Y, \rho)$ מרחבים מטריים, אז נאמר ש־$f$ רציפה אם ורק אם לכל $\epsilon > 0$ קיים $\delta > 0$ כך שאם $d(x', x) < \delta$ אז $\rho(f(x'), f(x)) < \epsilon$.
\end{definition}
אבל יותר קל לדבר במונחים של קבוצות פתוחות.
\begin{definition}[כדור]
	עבור $(X, d)$ מרחב מטרי,
	נסמן $B(r, x) = B_r(x) = \{ z \in X \mid d(x, z) < r \}$.
\end{definition}
\begin{definition}[קבוצה פתוחה]
	יהי $(X, d)$ מרחב מטרי, תת־קבוצה $U \subseteq X$ תיקרא פתוחה אם לכל $x \in U$ קיים $r > 0$ כך ש־$x \in B(x, r) \subseteq U$.
\end{definition}
\begin{definition}[הגדרה שקולה לרציפות]
	$f : X \to Y$ תיקרא רציפה אם לכל $V \subseteq Y$ קבוצה פתוחה ב־$Y$ מתקיים $f^{-1}(V) = \{ x \in X \mid f(x) \in V \}$ קבוצה פתוחה ב־$X$.
\end{definition}
\begin{definition}[טופולוגיה]
	תהי $X$ קבוצה (לא ריקה), \textbf{טופולוגיה} על $X$ היא אוסף $\tau \subseteq \Pp(X)$, כך שמתקיימים התנאים הבאים,
	\begin{enumerate}
		\item $X, \emptyset \in \tau$
		\item $\tau$ סגור לאיחוד, כלומר אם ${\{U_\alpha\}}_{\alpha \in I}$ לקבוצת אינדקסים $I$, כך ש־$\forall \alpha \in I, U_\alpha \in \tau$ אז $\bigcup_{\alpha \in I} U_\alpha \in \tau$
		\item $\tau$ סגור לחיתוכים סופיים, כלומר לכל $U, V \in \tau$ מתקיים $U \cap V \in \tau$
	\end{enumerate}
\end{definition}
\begin{definition}[מרחב טופולוגי]
	זוג $(X, \tau)$ כאשר $X$ קבוצה לא ריקה ו־$\tau$ טופולוגיה על $X$, יקרא מרחב טופולוגי.
\end{definition}
\begin{remark}
	בעצם הגדרנו כבר מתי פונקציה $f : X \to Y$ עבור מרחבים טופולוגיים $(X, \tau), (Y, \Omega)$ היא רציפה, כאשר $f^{-1}(U) \in \tau$ לכל $U \in \Omega$.
\end{remark}
\begin{notation}
	איברי $\tau$ יקראו קבוצות פתוחות.
\end{notation}
\begin{definition}[קבוצה סגורה]
	אם $(X, \tau)$ מרחב טופולוגי אז תת־קבוצה $A \subseteq X$ תיקרא סגורה אם $X \setminus A \in \tau$, כלומר המשלים של $A$ היא קבוצה פתוחה.
\end{definition}
\begin{example}
	יהי $(X, d)$ מרחב מטרי, נגדיר $\tau = \{ U \subseteq X \mid \forall x \in U \exists r > 0, B(x, r) \subseteq U \}$, כלומר נגדיר טופולוגיה באופן טריוויאלי כנביעה מהמרחב המטרי.
\end{example}
\begin{exercise}
	הוכיחו כי אכן זהו מרחב טופולוגי.
\end{exercise}
\begin{example}
	יהי $X$ קבוצה כלשהי, אז ניתן להגדיר על $X$ טופולוגיה $\tau_0 = \{ \emptyset, X \}$.
	טופולוגיה זו נקראת טופולוגיה טריוויאלית.
\end{example}
\begin{example}
	נגדיר $\tau_1 = \Pp(X)$ עבור קבוצה $X$, גם קבוצה זו היא טופולוגיה, והיא נקראת הטופולוגיה הדיסקרטית.
\end{example}
\begin{example}
	נניח ש־$(Y, \tau)$ מרחב טופולוגי, ותהי $f : (Y, \tau) \to (X, \tau_0)$, מתי $f$ היא רציפה? התשובה היא שהיא רציפה תמיד.
	מתי $f : (Y, \tau) \to (X, \tau_1)$ רציפה? תלוי בהגדרת הפונקציה, אבל במקרה שבו היא אכן רציפה, אז היא רציפה בכל טופולוגיה שהיא.
	לעומת זאת כל $f : (X, \tau_1) \to (Y, \tau)$ היא רציפה.
\end{example}
\begin{remark}
	לא כל טופולוגיה נובעת ממטריקה.
	לדוגמה הטופולוגיה הטריוויאלית על מרחב עם לפחות 2 נקודות.
\end{remark}
\begin{remark}
	נניח $x, y \in X$ אז נבחר $r = \frac{1}{2} d(x, y)$ ואז $y \notin B(x, r)$ ולכן $\emptyset \ne B(x, r) \ne X$, קל לראות שביחס לטופולוגיה שמושרית מהמטריקה $d$, הקבוצה $B(x, r)$ קבוצה פתוחה.
\end{remark}
\begin{example}
	נגדיר $X = \CC^n$ עבור איזשהו $n \in \NN$ ונגדיר $\Ff = \{ A \subseteq \CC^n \mid \exists {\{f_i\}}_{i \in I} \subseteq \CC[x_1, \dots, x_n], A = \{(p_1, \dots, p_n) \mid \forall i \in I, f_i(p_1, \dots, p_n) = 0 \}\}$.
\end{example}
\begin{definition}[בסיס לטופולוגיה]\label{topology_basis_definition}
	בסיס לטופולוגיה הוא אוסף $\Bb$ של תתי־קבוצות של $X$ כך שמתקיים,
	\begin{enumerate}
		\item לכל $x \in X$ יש $B \in \Bb$ כך ש־$x \in B$
		\item לכל $A, B \in \Bb$ ולכל $x \in A \cap B$ יש $C \in \Bb$ כך ש־$x \in C \subseteq A \cap B$
	\end{enumerate}
\end{definition}
\begin{proposition}
	עבור בסיס $\Bb$ האוסף $\tau_\Bb = \{ U \subseteq X \mid U \text{ is a union of elements of } \Bb \}$ היא טופולוגיה,
	\[
		\forall \alpha \in I, B_\alpha \in \Bb, U = \bigcup_{\alpha \in I} B_\alpha
	\]
\end{proposition}
\begin{proof}
	מכיוון ש־$\tau_\Bb$ סגורה לחיתוך סופי, אז אם $U, V \in \tau_\Bb$ אז $U = \bigcup_{\alpha \in I} B_\alpha \in \Bb$ וכן $V = \bigcup_{\beta \in J} A_\beta, A_\beta \in \Bb$, אז מתקיים,
	\[
		U \cap V
		= (\bigcup_{\alpha \in I} B_\alpha) \cap (\bigcup_{\beta \in J} A_\beta)
		= \bigcup_{\alpha, \beta \in I \times J} B_\alpha \cap A_\beta
		= D
	\]
	לכן לכל $x \in U \cap V$ ישנם $\alpha_0 \in I, \beta_0 \in J$ כך ש־$x \in B_{\alpha_0} \cap A_{\beta_0}$,
	אבל מהגדרת הבסיס קיימת קבוצה $C_{\alpha_0, \beta_0} \in \Bb$ כך ש־$C_{\alpha_0, \beta_0} \subseteq B_{\alpha_0} \cap A_{\beta_0}$.
	לכן $D \subseteq \bigcup_{(x, \alpha, \beta)} C_{x, \alpha, \beta}$.
	בהתאם מצאנו סגירות לחיתוך סופי.
\end{proof}
\begin{remark}
	יהי $(X, d)$ מרחב מטרי, אז $\{ B(x, r) \subseteq X \mid x \in X, r > 0 \}$ הוא טופולוגיה.
	אבל עכשיו נוכל להגדיר גם את $\{ B(x, \frac{1}{n}) \subseteq X \mid x \in X, n \in \NN \}$, זהו בסיס לטופולוגיה לאותה הטופולוגיה שהגדרנו למרחב המטרי.
\end{remark}
\begin{exercise}
	הוכיחו שזהו אכן בסיס עבור המרחב הטופולוגי הנתון.
\end{exercise}
\begin{example}
	נניח ש־$X = \ZZ$, ונגדיר את הבסיס $C$ להיות אוסף הסדרות האריתמטיות הדו־צדדיות, כלומר $C = \{ a + d \ZZ \mid a, d \in \ZZ, d \ne 0 \}$.
	אנו טוענים כי זהו אכן בסיס (לטופולוגיה).
	נתבונן בזוג קבוצות ב־$C$, $a + d \ZZ, b + q \ZZ$, ונניח ש־$p \in (a + d \ZZ) \cap (b + q \ZZ)$ אז $p \in p + dq \ZZ \subseteq (a + d \ZZ) \cap (b + q \ZZ)$.
	נגדיר טופולוגיית $\tau_C$.

	קבוצות סגורות הן משלימים לקבוצות פתוחות.

	כל סדרה אריתמטית דו־צדדית אינסופית היא גם פתוחה וגם סגורה.
	בפרט חיתוך סופי של סדרות אריתמטיות הוא סגור.
	לכן המשלים שלו הוא פתוח.
\end{example}
\begin{conclusion}[משפט אוקלידס]
	יש אינסוף מספרים ראשוניים.
\end{conclusion}
\begin{proof}
	נניח בשלילה כי יש מספר סופי של ראשוניים, $p_1, \dots, p_k$ עבור $k \in \NN$.
	נבחן את $\bigcup_{i = 1}^k p_i \ZZ$, זוהי קבוצה פתוחה וגם סגורה, לכן
	\[
		\bigcup_{i = 1}^k p_i \ZZ
		= \ZZ \setminus \{ -1, 1 \}
	\]
	ולכן נובע ש־$\{-1, 1\}$ קבוצה פתוחה וזו כמובן סתירה.
\end{proof}
\begin{proposition}[צמצום מרחב טופולוגי]
	נניח ש־$(X, \tau)$ מרחב טופולוגי, לכל $\emptyset \ne Y \subseteq X$ נגדיר $\tau_Y = \{ U \cap Y \mid U \in \tau \}$.
	אז $\tau_Y$ היא טופולוגיה.
	אם $Y \in \tau$ אז $\tau_Y = \{ W \in \tau \mid W \subseteq Y \}$.
\end{proposition}
\begin{proposition}[טופולוגיית מכפלה]
	נניח ש־$(X_1, \tau_1)$ ו־$(X_2, \tau_2)$ מרחבים טופולוגיים, אז נגדיר טופולוגיה על מרחב המכפלה $X_1 \times X_2$ על־ידי
	\[
		\tau_{1, 2}
		= \{ U_1 \times U_2 \mid U_1 \in \tau_1, U_2 \in \tau_2 \}
	\]
	אז $\tau_{1, 2}$ הוא בסיס והטופולוגיה המוגדרת על־ידו נקראת טופולוגיית המכפלה.
\end{proposition}
\begin{example}
	נוכל לבנות כך מכפלה של כמות סופית או אינסופית של מכפלות טופולוגיות.
	עבור אוסף אינסופי (בן־מניה או לא בהכרח) אנו צריכים להיזהר, נניח ש־$(X_\alpha, \tau_\alpha)$ עבור $\alpha \in I$, אז נגדיר
	\[
		\tau_b = \{ \prod_{\alpha \in I} U_\alpha \mid \forall \alpha \in I, U_\alpha \in \tau_\alpha \}
	\]
	זהו בסיס לטופולוגיה שנקרא טופולוגיית הקופסה.
	לעומת זאת נוכל להגדיר גם את
	\[
		\tau_p
		= \{ \prod_{\alpha \in I} U_\alpha \mid U_\alpha = X_\alpha \text{ for almost all } \alpha \in I \}
	\]
	כלומר $\prod_{\alpha \in I} = \{ f : I \to \bigcup_{\alpha \in I} X_\alpha \mid \forall \alpha \in I, f(x) \in X_\alpha \}$.
\end{example}

\section{שיעור 2 --- 25.3.2025}
\subsection{טופולוגיה --- המשך}
בשיעור הקודם דיברנו על מכפלה של טופולוגיות, אמרנו שאם $I$ קבוצת אינדקסים ולכל $\alpha \in I$ גם $(X_\alpha, \tau_\alpha)$ מרחב טופולוגי, אז נתבונן ב־$Z = \prod_{\alpha \in I} X_\alpha$ ונרצה להגדיר טופולוגיה על $Z$.
\begin{remark}
	מגדירים,
	\[
		\prod_{\alpha \in I} X_\alpha = \{ f : I \to \bigcup_{\alpha \in I} X_\alpha, \forall \alpha \in I, f(\alpha) \in X_\alpha \}
	\]
\end{remark}
לאחר מכן נוכל להגדיר טופולוגיית מכפלה,
\begin{definition}[טופולוגיית מכפלה]
	נגדיר את הבסיס,
	\[
		\Bb_{\text{box}} = \{ \prod_{\alpha \in I} U_\alpha \mid \forall \alpha \in I, U_\alpha \subseteq X_\alpha, U_\alpha \in \tau_\alpha \}
	\]
	ואת הבסיס,
	\[
		\Bb_{\text{prod}}
		= \{ \prod_{\alpha \in I} V_\alpha \mid \forall \alpha \in I, V_\alpha \in \tau_\alpha,
			V_\alpha \subseteq X_\alpha, |\{ \beta \in I \mid V_\beta \ne X_\beta \}| < \infty, V_\alpha = X_\alpha \text{ for almost every } \alpha \}
	\]
	אלו הן מכפלות של טופולוגיות המהוות טופולוגיה.
\end{definition}
\begin{definition}[העתקות הטלה]
	אז $Z = \prod_{\alpha \in I} X_\alpha$ אז ישנן הטלות $\forall \alpha \in I, \pi_\alpha : Z \to X_\alpha$ המוגדרת על־ידי $\pi_\alpha(f) = f(\alpha)$.
\end{definition}
אנו רוצים שכל ההטלות $\pi_\alpha$ תהינה רציפות.
כדי שהן אכן יקיימו רציפות צריך שלכל $U_\alpha \in \tau_\alpha$ (קבוצה פתוחה ב־$X_\alpha$) יתקיים $\pi_\alpha^{-1}(U_\alpha) \in \tau$, כלומר המקור יהיה קבוצה פתוחה ב־$Z$.
אבל נבחין כי $\pi_\alpha^{-1}(U_\alpha) = U_\alpha \times \prod_{\beta \ne \alpha} X_\beta$ אבל זהו לא בסיס,
\[
	C = \{ U_\alpha \times \prod_{\beta \ne \alpha} X_\beta \mid \pi_\alpha^{-1}(U_\alpha) \in \tau \}
\]
\begin{definition}[תת־בסיס לטופולוגיה]
	תהי קבוצה $X$, קבוצה $C$ של תת־קבוצות של $X$ כך ש־$\bigcup C = X$. \\
	נגדיר את הבסיס המושרה מתת־בסיס להיות $\Bb_C = \{ \bigcap A \mid A \subseteq C, |A| < \infty \}$,
	כלומר אוסף החיתוכים הסופיים של איברי $C$ (הן קבוצות פתוחות) הוא בסיס.
\end{definition}
\begin{definition}[טופולוגיה חלשה]
	אם $X$ קבוצה ו־$\tau_1, \tau_2$ טופולוגיות על $X$ אומרים ש־$\tau_1$ חלשה יותר מ־$\tau_2$ אם $\tau_1 \subseteq \tau_2$.
\end{definition}
\begin{example}
	יהיו מרחבים מטריים $(X_i, \rho_i)$ לכל $i \in \NN$, ונגדיר $(X_i, \tau_i)$ מרחב טופולוגי מושרה מתאים לכל $i$.
	נרצה להתבונן במכפלתם, $\prod_{i \in \NN} X_i$.
	אז נוכל להתבונן ב־$(\prod X_i, \tau_{\text{prod}})$ שהגדרנו זה עתה.
\end{example}
\begin{definition}[מטריקת מכפלה]
	בהינתן סדרת מרחבים מטריים $(X_i, \rho_i)$ עבור $i \in \NN$ מרצה למצוא מטריקה על $Z = \prod_{i \in \NN} X_i$.
	לכל $x, y \in Z$ כאשר $x = (x_i), y = (y_i)$ אז נגדיר,
	\[
		\rho(x, y)
		= \sum_{i = 1}^{\infty} \frac{1}{2^i} \frac{\rho_i(x_i, y_i)}{1 + \rho_i(x_i, y_i)}
	\]
\end{definition}
ברור שפונקציה זו מוגדרת, וברור אף כי היא מקיימת את התכונה השנייה של מטריקות, אך לא ברור שהיא מקיימת את אי־שוויון המשולש, זהו תרגיל שמושאר לקורא.
\begin{proposition}
	הטופולוגיה המושרית על $Z = \prod_{i = 1}^\infty X_i$ עבור $(X_i, \tau_i)$ מרחבים טופולוגיים יחד עם מטריקת המכפלה שווה ל־$\Bb_{\text{prod}}$.
\end{proposition}
\begin{proof}
	$(Z, \rho)$ מרחב מטרי, ו־$\Bb_\rho = \{ B(x, r) \mid x \in Z, r > 0 \}$ בסיס, אז נוכל להגדיר טופולוגיה $\tau_{\Bb_\rho} = \tau_\rho$.
	כדי להראות ש־$\tau_\rho = \Bb_{\text{prod}}$ מספיק להראות שכל $B \in \Bb_\text{prod}$ שייכת ל־$\tau_\rho$ וכל $C \in \Bb_\rho$ שייכת ל־$\tau_\text{prod}$.
	נוסיף ונבהיר שטופולוגיה נקבעת ביחידות על־ידי בסיס שלה, לכן מספיק להראות את שקילות הבסיסים.

	נתחיל בתנאי הראשון, ונקבע $k \in \NN$ כלשהו. מספיק להראות שקבוצה מהצורה $U_k \times \prod_{i \ne k} X_i$ פתוחה ב־$\tau_\rho$ עבור $U_k \in \tau_k$ היא קבוצה פתוחה ב־$\tau_\rho$,
	זאת שכן נוכל להרחיב הוכחה זו באופן סיסטמתי להיות על כל קבוצה סופית של קבוצות פתוחות.
	יהי $x \in U_k \times \prod_{i \ne k} X_i$ ונסמן את ההטלה על מרחב זה $\pi_j : U_j \times \prod_{i \ne k} X_i \to U_j$, כלומר $\pi_j(x) = x_j$ לכל $j \in \NN$.
	אנו יודעים ש־$x_k \in U_k$ וש־$U_k$ פתוחה ולכן ישנו $r > 0$ כך ש־$B_r(x_k) \subseteq U_k$ כדור פתוח ב־$X_k$. \\
	קיים $s > 0$ כך שאם $t \ge 0$ ומתקיים $\frac{t}{1 + t} < s$ אז $t < r$, ולכן נבחן את הכדור ברדיוס $\frac{s}{2^k}$ סביב $x$ ב־$Z = \prod_{i \in \NN} X_i$ מרחב המכפלה כולו.
	המטרה שלנו היא להראות שהכדור שעתה בחרנו מקיים את התנאי לבסיס.
	נניח ש־$y = {(y_i)}_{i \in \NN} \in B_{\frac{s}{2^k}}(x)$, אז
	\begin{align*}
		& \frac{s}{2^k} > \rho(x, y)
		= \sum_{i = 1}^{\infty} \frac{1}{2^i} \frac{\rho_i(x_i, y_i)}{1 + \rho_i(x_i, y_i)}
		\ge \sum_{i = 1}^{\infty} \frac{1}{2^i} \frac{\rho_i(x_i, y_i)}{1 + \rho_i(x_i, y_i)} \\
		& \implies s > \frac{\rho_i(x_i, y_i)}{1 + \rho_i(x_i, y_i)} \\
		& \implies \rho_k(x_k, y_k) < r \\
		& \implies y_k \in B_r(x_k) \subseteq U_k
	\end{align*}

	נעבור לתנאי השני, נתבונן בכדור הפתוח סביב $x \in Z$, $B_r(x)$, כאשור נחזור ונבהיר כי כדור זה מוגדר להיות,
	\[
		B_r(x)
		= \left\{ y \in Z \mid \sum_{i = 1}^{\infty} \frac{1}{2^i} \frac{\rho_i(x_i, y_i)}{1 + \rho_i(x_i, y_i)} < r \right\}
	\]
	יהי $M \in \NN$ כך ש־$\sum_{i = M + 1}^{\infty} \frac{1}{2^i} \frac{\rho_i(x_i, y_i)}{1 + \rho_i(x_i, y_i)} < \frac{r}{2}$, כלומר נחסום את טור הזנב של המטריקה $\rho$.
	תהי $V \subseteq Z$ המוגדרת על־ידי,
	\[
		V
		= \left\{ (y_1, \dots, y_M) \in \prod_{i = 1}^M \mid \sum_{i = 1}^M \frac{1}{2^i} \frac{\rho_i(x_i, y_i)}{1 + \rho_i(x_i, y_i)} < \frac{r}{2} \right\}
	\]
	ואנו טוענים כי $V \times \prod_{i = M + 1}^\infty X_i \subseteq B_r(x)$.
\end{proof}

\section{שיעור 3 --- 31.3.2025}
\subsection{סגירות}
בדיוק כמו במרחבים מטריים, גם במרחב טופולוגי נרצה לדון במניפולציות על קבוצות במרחב, נתחיל בהגדרת הקונספט של סגור של קבוצה במרחב טופולוגי.
\begin{definition}[סגור של קבוצה במרחב טופולוגי]
	יהי $(X, \tau)$ מרחב טופולוגי, ותהי קבוצה $A \subseteq X$ תת־קבוצה כשלהי.
	נגדיר את הסגור של $A$ כקבוצה הסגורה הקטנה ביותר המכילה את $A$, כלומר,
	\[
		\overline{A}
		= \bigcap_{X \setminus F \in \tau} F
	\]
\end{definition}
בהתאם נקבל מספר תכונות ראשוניות ודומות לתכונות שראינו בעבר,
\begin{lemma}
	התכונות הבאות מתקיימות,
	\begin{enumerate}
		\item $\overline{A \cup B} = \overline{A} \cup \overline{B}$
		\item $\overline{A \cap B} \subseteq \overline{A} \cap \overline{B}$, כאשר במקרה זה אין בהכרח שוויון.
	\end{enumerate}
\end{lemma}
\begin{example}
	נראה דוגמה למקרה בו בהכרח אין שוויון,
	נגדיר $X = \RR$ וכן $A = \QQ, B = \RR \setminus \QQ$, אז מתקיים,
	\[
		\emptyset
		= \overline{\emptyset}
		= \overline{A \cap B}
		\subsetneq \overline{A} \cap \overline{B}
		= \RR \cap \RR
		= \RR
	\]
\end{example}
\begin{proposition}
	אם $(X, \tau)$ מרחב טופולוגי ו־$A \subseteq X$, אז,
	\[
		x \in \overline{A}
		\iff \forall U \in \tau, x \in U \rightarrow U \cap A \ne \emptyset
	\]
	כלומר נקודה נמצאת בסגור של $A$ אם ורק אם כל קבוצה פתוחה סביב הנקודה לא זרה ל־$A$.
\end{proposition}
\begin{proof}
	נראה את שלילת הטענה, כלומר $x \notin \overline{A} \iff \exists U \in \tau, x \in U \land U \cap A = \emptyset$. \\
	נניח ש־$x \notin \overline{A}$ ולכן $x \in X \setminus \overline{A}$ אבל $X \setminus \overline{A}$ פתוחה וזרה מהגדרתה ל־$A$. \\
	בכיוון השני אם יש $U \ni x$ פתוחה כך ש־$U \cap A = \emptyset$ אז $F = X \setminus U$ סגורה ומכילה את $A$, בהתאם $\overline{A} \subseteq F$ ובהכרח $x \notin \overline{A}$.
\end{proof}
\begin{definition}[פנים ושפה]
	נגדיר את הפנים של $A$ להיות, $A^\circ = \bigcup_{U \in \tau, U \subseteq A} U$. \\
	כלומר הפנים הוא איחוד כל הקבוצות הפנימיות הפתוחות של $A$, ובשל הסגירות של הטופולוגיה לאיחוד, נקבל כך את הקבוצה הפתוחה הגדולה ביותר שחלקית ל־$A$.
	נגדיר את השפה של $A$ להיות $\partial A = \overline{A} \setminus A^\circ$.
\end{definition}
נבחין בהגדרה של סביבה ונשתמש בהגדרה זו כדי להגדיר מונח חדש.
\begin{definition}[סביבה של נקודה]
	נאמר ש־$L \subseteq X$ היא סביבה של $x$ אם קיימת קבוצה פתוחה $U \in \tau$ כך ש־$x \in U \subseteq L$.
\end{definition}
\begin{definition}[נקודת הצטברות]
	יהי $(X, \tau)$ מרחב טופולוגי, תהי $A \subseteq X$ תת־קבוצה כלשהי, ו־$x \in A$.
	נאמר ש־$x$ היא נקודת הצטברות של $A$ אם כל סביבה של $x$ מכילה נקודה מ־$A$ שונה מ־$x$, כלומר,
	\[
		\forall U \in \tau, x \in U \implies \exists y \in (U \setminus \{ x \}) \cap A
	\]
	נסמן ב־$A'$ את קבוצת נקודות ההצטברות של $A$.
\end{definition}
נרצה להסתכל על נקודות הצטברות כנקודות שלא משנה כמה קרוב אנחנו מסתכלים אליהן, עדיין נוכל למצוא בסביבתן נקודות נוספות. במובן הזה ברור שהן נמצאות בקרבת נקודות בפנים, אך עלולות להיות גם נקודות לא פנימיות שמקיימות טענה כזו.
\begin{proposition}
	מתקיים $\overline{A} = A \cup A'$.
\end{proposition}
\begin{proof}
	אם $x \in A \cup A'$ אז או $x \in A \subseteq \overline{A}$ או $x \in A'$.
	ובכל סביבה של $x$ יש נקודה מ־$A$ שונה מ־$x$.
	בפרט לאור הטענה ש־$\overline{A}$ היא אוסף כל הנקודות שבכל סביבה שלהן המכילה את $A$ בחיתוך לא ריק נובע ש־$A' \subseteq \overline{A}$, לכן נובע ש־$A \cup A' \subseteq \overline{A}$.

	בכיוון השני נניח ש־$x \in \overline{A}$, אז לכל $U \in \tau$ כך ש־$x \in U$ מתקיים $U \cap A \ne \emptyset$.
	אם $x \in A$ אז בוודאי ש־$x \in A \cup A'$.
	אם $x \notin A$ אז לכל $U \in \tau$ כך ש־$x \in U$ מתקיים $U \cap A \ne \emptyset$.
	מ־$x \notin A$ נובע גם ש־$U \setminus \{ x \} \cap A \ne \emptyset$ ולכן $x \in A'$.
	אז מצאנו ש־$\overline{A} \subseteq A \cup A'$, ונובע משני החלקים ש־$\overline{A} = A \cup A'$.
\end{proof}

\subsection{השלמות לרציפות}
ניזכר בהגדרה\ \ref{function_continuous_definition}, נרצה לדון בקונספט של רציפות באופן רחב יותר.
בהינתן $(Y, \tau_Y)$ מרחב טופולוגי ו־$X$ קבוצה כלשהי, ופונקציה $f : X \to Y$, ניתן להגדיר טופולוגיה על $X$ כך ש־$f$ רציפה. \\
הקבוצה $\{ f^{-1}(U) \mid U \in \tau_Y \}$ היא תת־בסיס, ואפשר להרחיבה לבסיס ולהגדיר עליו טופולוגיה מושרית מהבסיס על $X$.
\begin{proposition}
	$f$ רציפה עבור טופולוגיה זו, וזו הטופולוגיה החלשה ביותר על $X$ עבורה $f$ רציפה.
\end{proposition}
בכיוון השני, בהינתן מרחב טופולוגי $(X, \tau_X)$ וקבוצה כלשהי $Y$ יחד עם פונקציה $f : X \to Y$, נוכל להגדיר את $\{ U \subseteq Y \mid f^{-1}(U) \in \tau_X \}$ להיות תת־בסיס וממנו נוכל שוב לבנות בסיס וטופולוגיה על $Y$.
באופן דומה $f$ רציפה ביחס לטופולוגיה זו וזו הטופולוגיה החזקה ביותר על $Y$ כך ש־$f$ רציפה.
\begin{proposition}[שקילות לרציפות]
	יהיו מרחבים טופולוגיים $(X, \tau_X), (X, \tau_Y)$ ותהי $f : X \to Y$, אז התנאים הבאים שקולים,
	\begin{enumerate}
		\item $f$ רציפה לפי\ \ref{function_continuous_definition}
		\item לכל קבוצה סגורה $F \subseteq Y$, $f^{-1}(F)$ סגורה ב־$X$ \\
			הגדרה זו עוזרת לנו לדון בקבוצות סגורות במקום פתוחות
		\item אם $\Bb$ בסיס לטופולוגיה של $Y$ אז לכל $B \in \Bb$ מתקיים ש־$f^{-1}(B)$ פתוחה ב־$X$ \\
			הגדרה זו מאפשרת לנו לדון בבסיסים ובכך לפשט את העבודה עם טופולוגיות
		\item לכל $x \in X$ ולכל סביבה $W \subseteq Y$ של $f(x)$ מתקיים ש־$f^{-1}(W)$ סביבה של $x$
		\item קיים כיסוי פתוח ${\{U_\alpha\}}_{\alpha \in \Omega}$ של $X$, כלומר $\forall \alpha, U_\alpha \in \tau, X = \bigcup_{\alpha \in \Omega} U_\alpha$,
			כך שלכל $\alpha \in \Omega$ מתקיים $f \mid_{U_\alpha} : U_\alpha \to Y$ רציפה.
		\item קיים כיסוי סגור סופי $X = \bigcup_{i = 1}^n F_i$ עבור $F_i$ סגורות עבור $1 \le i \le n$, כך שכל $f \mid_{F_i} : F_i \to Y$ רציפה.
		\item לכל $A \subseteq X$ מתקיים $f(\overline{A}) \subseteq \overline{f(A)}$
	\end{enumerate}
\end{proposition}
\begin{proof}
	$1 \iff 2$:
	נובע ישירות מהגדרה של משלימים והגדרת הרציפות על קבוצות פתוחות.

	$1 \iff 3$:
	בכיוון הראשון כל איחוד קבוצות מהבסיס הוא קבוצה פתוחה, ונוכל כך להראות את נכונות הטענה.
	לכיוון השני כל קבוצה היא איחוד של קבוצות מהבסיס, $U_\alpha$, ו־$f^{-1}(\bigcup U_\alpha) = \bigcup f^{-1}(U_\alpha)$.

	$1 \implies 4$:
	אם $x \in X$ וכן $f(x) \in W \subseteq Y$ סביבה של $f(x)$ אז קיימת $f(x) \in U \subseteq W$ כך ש־$U$ פתוחה, לכן נובע ש־$x \in f^{-1}(U) \subseteq f^{-1}(W)$ כאשר $f^{-1}(U)$ פתוחה.

	$4 \implies 1$:
	אם $U \subseteq Y$ פתוחה אז צריך להראות ש־$f^{-1}(U)$ פתוחה.
	תהי $x \in f^{-1}(U)$, אז $U$ סביבה ל־$f(x)$ ולכן לפי ההנחה $f^{-1}(U)$ היא סביבה של $x$, כלומר קיימת $x \in V_x \subseteq f^{-1}(U)$ פתוחה, ונסיק ש־$f^{-1}(U) = \bigcup_{x \in f^{-1}(U)} V_x$ פתוחה.

	$1 \implies 5$:
	נוכל לבחור כיסוי טריוויאלי.

	$5 \implies 1$:
	נניח שיש כיסוי פתוח ${\{U_\alpha\}}_{\alpha \in \Omega}$ של $X$ כך ש־$f \mid_{U_\alpha} : U_\alpha \to Y$ רציפה לכל $\alpha \in \Omega$.
	תהי $W \subseteq Y$ פתוחה, אז $f^{-1}(W) \subseteq \bigcup_{\alpha \in \Omega} f^{-1} \mid_{U_\alpha}(W)$.
	מההנחה $U_\alpha \cap f^{-1}(W) = f^{-1} \mid_{U_k}(W)$ פתוחה ב־$U_\alpha$ ומשום ש־$U_\alpha$ פתוחה ב־$X$ נובע ש־$f^{-1} \mid_{U_\alpha}(W)$ פתוחה ב־$X$ ולכן $f^{-1}(W)$ פתוחה ב־$X$.

	$1 \implies 6$:
	נבחר את $X$ ככיסוי סגור של עצמה.

	$6 \implies 1$:
	נניח ש־$X = \bigcup_{i = 1}^n F_i$ כיסוי סגור סופי של $X$, ונניח גם שלכל $i$, $f \mid_{F_i} : F_i \to Y$ רציפה.
	כעת ההוכחה דומה למהלך שעשינו ב־$5 \implies 1$, אבל כעת אפיון רציפות בעזרת $L$, ואיחוד סופי על סגורות הוא סגור.

	$1 \implies 7$:
	נתון כי $f$ רציפה, אנו רוצים להראות ש־$f(\overline{A}) = \overline{f(A)}$.
	יהי $x \in \overline{A}$, נראה כי $f(x) \in \overline{f(A)}$,
	נניח בשלילה ש־$f(x) \notin \overline{f(A)}$, אז יש סביבה פתוחה $f(x) \in U$ כך ש־$U \cap f(A) = \emptyset$.
	$f$ רציפה ולכן $f^{-1}(U)$ פתוחה ב־$X$ וקיים $A \cap f^{-1}(U) = \emptyset$,
	אבל $x \in f^{-1}(U)$ וקיבלנו $x \notin \overline{A}$.

	$7 \implies 2$:
	תהי $F \subseteq Y$ סגורה, אז,
	\[
		f(\overline{f^{-1}(F)})
		\overset{\text{ההנחה}}{\subseteq}
		\subseteq \overline{F}
		\overset{\text{$F$ סגורה}}{=} F
		\implies \overline{f^{-1}(F)}
		\subseteq f^{-1}(F)
	\]
	מהגדרת סגור נוכל להסיק ש־$f^{-1}(F) \subseteq \overline{f^{-1}(F)}$, לכן,
	\[
		\overline{f^{-1}(F)}
		= f^{-1}(F)
	\]
	ובפרט $f^{-1}(F)$ סגורה ב־$X$.
\end{proof}
נבחן תכונה מעניינת שלא תשרת אותנו רבות, אך כן מעלה שאלות,
\begin{definition}[מרחב כוויץ]\label{definition:contractable_space}
	יהי $X$ מרחב טופולוגי, נאמר ש־$X$ כוויץ (Contractible) אם יש פונקציה רציפה $f : I \times X \to X$ עבור $I = [0, 1]$ כך ש־$\forall x \in X, f(0, x) = x$ וקיימת נקודה $x_1 \in X$ כך ש־$\forall x \in X, f(1, x) = x_1$. \\
	נסמן גם $f_t(t, x)$ כאשר $f_t : X \to X$, נקבל $f_0 = Id$ וכן $f_1$ הפונקציה הקבועה $x \mapsto x_1$.
\end{definition}
\begin{example}
	נגדיר $X = I$ ואת $f : I \times I \to I$ המוגדרת על־ידי $f(t, x) = (1 - t)x$. \\
	נגדיר $X = \RR$ ואת $f : I \times \RR \to \RR$ על־ידי $f(t, x) = (1 - t)x$ ונקבל שגם $\RR$ כוויצה בדיוק באותו האופן.
\end{example}
\begin{exercise}
	הראו כי $S^1$ לא כוויץ.
\end{exercise}
נחזור לדבר על פונקציות רציפות.
\begin{exercise}
	נתבונן ב־$f : (\RR, \tau_\RR) \to (\RR^\NN, \tau)$ כך ש־$f(x)(i) = x$ לכל $i \in \NN$. \\
	הראו ש־$f$ רציפה או לא רציפה כהעתקה כאשר $\tau$ טופולוגיית המכפלה, וכאשר $\tau$ טופולוגיית הקופסה.
\end{exercise}
\begin{solution}
	נתבונן ב־$U = \prod_{n = 1}^\infty (-\frac{1}{n}, \frac{1}{n})$, זוהי קבוצה פתוחה, אך $f^{-1}(U) = 0$, וזו כמובן לא קבוצה פתוחה, לכן בטופולוגיית הקופסה היא לא רציפה, לכן בטופולוגיית הקופסה היא לא רציפה. \\
	לעומת זאת בטופולוגיית המכפלה היא אכן רציפה.
\end{solution}
\begin{definition}[הומיאומורפיזם]
	הומיאומורפיזם בין שני מרחבים טופולוגיים $X, Y$ היא העתקה $f : X \to Y$ כך ש־$f$ חד־חד ערכית ועל, $f$ רציפה ו־$f^{-1}$ רציפה אף היא. \\
	$X$ ו־$Y$ יקראו הומיאומורפיות אם יש הומיאומורפיזם $f : X \to Y$ ביניהן.
\end{definition}
אנו נרצה להסתכל על הומיאומורפיזם כאיזומורפיזם של מרחבים טופולוגיים.
\begin{example}
	נגדיר $X = \RR, Y = (0, 1)$, ואת $f : \RR \to (0, 1)$ המוגדרת על־ידי $x \mapsto \frac{e^x}{e^x + 1}$, אז,
	\[
		f'(x)
		= \frac{e^x(e^x + 1) - e^x e^x}{{(e^x + 1)}^2}
		= \frac{e^x}{{(e^x + 1)}^2}
		> 0
	\]
	ולכן $f$ גזירה, ואף חד־חד ערכית, לבסוף $f(x) \xrightarrow{x \to -\infty} 0, f(x) \xrightarrow{x \to \infty} 1$ ולכן היא גם על, ואכן המרחבים הומיאומורפים.
\end{example}
\begin{example}
	נגדיר את $\eta = \{ z = x + iy \in \CC \mid x, y \in \RR, y > 0 \}$ ואת $D = \{ z \in \CC \mid |z| < 1 \}$.
	נגדיר גם $\psi : \eta \to D$ על־ידי $z \mapsto \frac{z - i}{z + i}$. \\
	ההוכחה כי זהו אכן הומיאומורפיזם מושארת לקורא.
\end{example}
נבחין כי הדוגמה האחרונה אינה אלא העתקת מביוס, העתקה קונפורמית ואנליטית.
\begin{example}
	נבחן את $S^1 = \{ z \in \CC \mid |z| = 1 \}$ ואת $J = [0, 2\pi]$, אנו טוענים כי אין הומיאומורפיזם בין שני המרחבים הללו. \\
	נבחן את הפונקציה $t \mapsto e^{it}$ לדוגמה,
	$[0, 2\pi] \to S^1$ לא חד־חד ערכית, מהצד השני $[0, 2\pi) \to S^1$ חד־חד ערכית ועל, אבל 

	נניח שיש העתקה חד־חד ערכית $\alpha : J \to S^1$, ונוציא מ־$J$ נקודה יחידה, אז נקבל איחוד זר של שתי קבוצות זרות, אך מן הצד השני הוצאת נקודה יחידה מהמעגל משאיר אותו כקבוצה קשירה.
	ההוכחה המלאה אומנם סבוכה יותר, אך הצבענו פה על הבדל מהותי בין שני המרחבים.
\end{example}
\begin{exercise}
	הראו כי $\RR$ ו־$\RR^2$ לא הומיאומורפים. \\
	האם גם $\RR^2$ ו־$\RR^3$ הומיאומורפים?
\end{exercise}
\begin{definition}[העתקה פתוחה וסגורה]
	העתקה $f : X \to Y$ תיקרא העתקה פתוחה (סגורה) אם לכל $U \subseteq X$ פתוחה (סגורה) מתקיים $f(U) \subseteq Y$ פתוחה (סגורה) ב־$Y$.
\end{definition}
\begin{example}
	$f : \RR \to \RR$ המוגדרת על־ידי $f(x) = x^2$ היא רציפה, סגורה ולא פתוחה.
\end{example}
\begin{example}
	השיכון $(0, 1) \hookrightarrow \RR$ המוגדר על־ידי $x \mapsto x$ הוא רציף, תפוח אבל לא סגור.
\end{example}
\begin{example}
	$\{ a, b \} \to \{ a, b \}$ המוגדרת טריוויאלית היא פתוחה, סגורה אך לא רציפה.
\end{example}

\section{שיעור 4 --- 7.4.2025}
\subsection{אקסיומות ההפרדה}
מטרתנו היא לאפיין את הקונספט של הפרדה,
כלומר מתי אנו יכולים לחסום חלקים שונים במרחב הטופולוגי בקבוצות פתוחות.
במקרים המטריים אף ראינו בעבר כמה הפרד היא מועילה, היא פתח לדיון נרחב.
\begin{definition}[איברים ניתנים להפרדה]
	יהי $X$ מרחב טופולוגי ונניח ש־$x, y \in X$.
	נאמר ש־$x, y$ ניתנים להפרדה אם קיימות קבוצות פתוחות $U, V \subseteq X$ כך שהקבוצות האלה זרות, וכן $x \in U, y \in V$. \\
	עבור $x \in X, A \subseteq X$ נאמר שהקבוצה והאיבר ניתנים להפרדה אם $x \in U, A \subseteq V$ כאלה. \\
	לבסוף נאמר ש־$A, B \subseteq X$ כך ש־$A \cap B = \emptyset$ ניתנות להפרדה אם $A \subseteq U, B \subseteq V$ פתוחות וזרות.
\end{definition}
עתה משהגדרנו את הקונספט הכללי של הפרדה, נגדיר באופן בהיר ועקבי סוגים שונים של ''רמת'' ההפרדה שמרחב טופולוגי מקיים.
\begin{definition}[אקסיומות הפרדה]
	מרחב טופולוגי $X$ יקרא מרחב $T_i$ אם הוא מקיים את האקסיומה $T_i$ עבור $i \in \{0, 1, 2, 3, 4\}$, עבור האקסיומות המוגדרות להלן.
	\begin{itemize}
		\item $T_0$, לכל $x, y \in X$ יש קבוצה פתוחה שמכילה את אחת הנקודות אך לא את האחרת
		\item $T_1$, לכל שתי נקודות $x, y \in X$ קיימת פתוחה המכילה את אחת הנקודות ולא את האחרת, וקבוצה פתוחה המכילה את הנקודה השנייה אך לא את הראשונה.
			כלומר אם $x \ne y$ אז קיימת $U \in \tau$ כך ש־$x \in U, y \notin U$
		\item $T_2$, אם לכל זוג נקודות $x \ne y \in X$ יש קבוצות פתוחות זרות $U, V \subseteq X$ כך ש־$x \in U, y \in V$.
			אם $X$ מקיים את $T_2$ אז הוא יקרא מרחב האוסדורף.
			בשפה שהגדרנו קודם, נאמר שבמרחב האוסדורף כל שתי נקודות ניתנות להפרדה
		\item $T_3$, אם המרחב הוא $T_1$ וגם $X$ \textbf{רגולרי},
			כלומר לכל $x \in X$ וקבוצה סגורה $x \notin A \subseteq X$, $x, A$ ניתנות להפרדה
		\item $T_4$, אם המרחב הוא $T_1$ וכן $X$ \textbf{נורמלי},
			כלומר שכל זוג תת־קבוצות סגורות $A, B \subseteq X$ ניתנות להפרדה
	\end{itemize}
\end{definition}
נעבור למספר טענות הנוגעות לסוגי ההפרדה השונים.
\begin{proposition}
	$T_1$ מתקיים אם ורק אם כל $\{x\} \subseteq X$ קבוצה סגורה.
\end{proposition}
\begin{proof}
	נקבע נקודה $x \in X$, אז לכל $X \ni y \ne x$ קיימת קבוצה פתוחה $U_y \subseteq X$ כך ש־$x \notin U_y$.
	מסגירות לאיחוד נקבל שגם $U = \bigcup_{y \in X, y \ne x} U_y$ היא קבוצה פתוחה.
	לכן $U^C$ סגורה.
	אבל מההגדרה שסיפקנו ל־$U$ נקבל ש־$U^C = \{ x \}$.
\end{proof}
\begin{proposition}[גרירת אקסיומות ההפרדה]
	מתקיים $T_4 \implies T_3 \implies T_2 \implies T_1 \implies T_0$,
	כלומר אם מרחב מטרי הוא $T_n$, אז הוא גם $T_i$ לכל $i < n$.
\end{proposition}
בעוד שלא נוכיח טענה זו, נבהיר כי היא נובעת ישירות מהגדרת ההפרדה.
נבחין כי המספור הוא עתה לא ארעי כפי שאולי היינו שוגים לחשוב, אלא האקסיומות מסודרות לפי ''כוחן'' בהפרדת דברים במרחב.
נמשיך ונראה טענה שתיצוק משמעות למרחבים נורמליים.
\begin{proposition}[שקילות למרחב נורמלי]
	מרחב טופולוגי $X$ נורמלי אם ורק אם לכל קבוצה סגורה $A$ וקבוצה פתוחה $A \subseteq U$ קיימת קבוצה פתוחה $V$ כך ש־$A \subseteq V \subseteq \overline{V} \subseteq U$. \\
	כלומר לכל קבוצה סגורה וקבוצה פתוחה שמכילה אותה, יש קבוצה פתוחה ביניהן כך שגם הסגור שלה ביניהן.
\end{proposition}
\begin{proof}
	בכיוון הראשון נניח ש־$X$ נורמלי וכן ש־$A \subseteq U$ קבוצה סגורה המוכלת בקבוצה פתוחה.
	$A, X \setminus U$ סגורות וזרות, ולכן יש קבוצות פתוחות $V, W$ כך ש־$A \subseteq V \subseteq X \setminus W \subseteq U, X \setminus U \subseteq W$ כך ש־$W \cap V = \emptyset$.
	נובע ש־$A \subseteq V \subseteq \overline{V} \subseteq X \setminus W \subseteq U$.

	בכיוון השני, נניח ש־$A, B \subseteq X$ קבוצות סגורות זרות ולכן $A \subseteq X \setminus B$, נסמן $U = X \setminus B$, אז קיימת קבוצה פתוחה $V$ כך שמתקיים,
	\[
		A \subseteq V \subseteq \overline{V} \subseteq X \setminus B
	\]
	ולכן $B \subseteq X \setminus \overline{V}$ ונובע גם ש־$V \cap (X \setminus \overline{V}) = \emptyset$.
\end{proof}
\begin{proposition}[תנאי שקול למרחב האוסדורף]
	$X$ מרחב האוסדורף, כלומר מרחב $T_2$,
	אם ורק אם $\Delta_X = \{ (x, x) \mid x \in X \} \subseteq X \times X$ תת־קבוצה סגורה בטופולוגיית המכפלה.
\end{proposition}
\begin{proof}
	נניח ש־$X$ מרחב האוסדורף.
	לכל $x \ne y$ יש $x \in U_{x, y}$ ו־$y \in V_{x, y}$ פתוחות זרות, כלומר $(U_{x, y} \cap V_{x, y}) \cap \Delta_X = \emptyset$.
	נבחין כי,
	\[
		X \times X \setminus \Delta_X = \bigcup_{x \ne y} (U_{x, y} \times V_{x, y})
	\]
	ובטופולוגיית המכפלה זוהי קבוצה פתוחה.

	בכיוון השני נניח ש־$\Delta_X$ סגורה, אז $X \times X \setminus \Delta_X$ פתוחה, אם $x \ne y$ אז $(x, y) \in (X \times X) \setminus \Delta_X$.
	לכן לפי הגדרת טופולוגיית המכפלה יש $U, V$ פתוחות כך ש־$(x, y) \in U \times V \subseteq X^2 \setminus \Delta_X$ ואף ש־$U \cap V = \emptyset$.
\end{proof}
\begin{proposition}[אקסיומות הפרדה בתתי־מרחבים]
	עבור $i \in \{1, 2, 3\}$ אם $X$ הוא מרחב $T_i$ ו־$Y \subseteq X$, תת־מרחב אז גם $Y$ הוא מרחב $T_i$.
\end{proposition}
\begin{proof}
	עבור $i \in \{1, 2\}$ הטענה נובעת ישירות מהגדרת אקסיומות ההפרדה $T_i$, לכן נעבור להוכחת הטענה עבור $T_3$. \\
	נניח ש־$X$ הוא $T_3$, נזכור שמרחב כזה הוא $T_1$ וכן רגולרי, לכן מספיק שנראה שתת־המרחב הוא רגולרי גם כן.
	יהי $y \in Y$ ויהי $A \subseteq Y$ סגורה כך ש־$y \notin A$.
	לכן יש קבוצה סגורה $C \subseteq X$ כך ש־$A = C \cap Y$.
	עוד אנו יודעים ש־$y \notin C$,
	לכן קיימות $U, V$ פתוחות ב־$X$ מפרידות בין $y$ ל־$C$, $y \in U$ ו־$C \subseteq V$, וכן $U \cap V = \emptyset$,
	אז $A \subseteq V \cap Y$ ו־$y \in U \cap Y$.
\end{proof}
\begin{remark}
	טענה זו לא נכונה עבור $T_4$.
	Counter examples in Topology של J. Arthur Seebach הוא ספר שבו נוכל למצוא דוגמות רבות למרחבים כאלה.
\end{remark}
\begin{proposition}[אקסיומות הפרדה במרחבי מכפלה]
	אם $X, Y$ מרחבים $T_i$ עבור $i \in \{1, 2, 3\}$ אז גם $X \times Y$ הוא מרחב $T_i$.
\end{proposition}
\begin{proof}
	עבור $T_1$ אם $(x, y) \in X \times Y$, אז נוכל להגדיר את הקבוצה,
	\[
		(X \times (Y \setminus \{y\})) \cup ((X \setminus \{x\}) \times Y)
	\]
	זוהי קבוצה סגורה מהגדרת טופולוגיית המכפלה.

	נעבור להוכחת הטענה עבור $T_3$.
	נניח ש־$X, Y$ הם $T_3$, כלומר $T_1$ ורגולריים ועלינו להראות ש־$X \times Y$ רגולרי. \\
	נניח ש־$A \subseteq X \times Y$ סגורה וכן ש־$(x, y) \notin A$.
	נגדיר למה, ש־$Z$ מרחב רגולרי אם ורק אם לכל $z \in U \subseteq Z$ עבור $U$ פתוחה ויש $z \in V \subseteq \overline{V} \subseteq U$ פתוחה.
	תוך שימוש בלמה, נסמן $W = (X \times Y) \setminus A$ פתוחה, $(x, y) \in W$, אז נובע שקיימות קבוצות פתוחות $x \in U_x \subseteq X$ ו־$y \in U_Y \subseteq Y$ כך ש־$(x, y) \in U_X \times U_Y \subseteq W$.
	מרגולריות נסיק שיש $V_X, V_Y$ פתוחות כך ש־$x \in V_X \subseteq \overline{V}_X \subseteq U_X$ ו־$y \in V_Y \subseteq \overline{V}_Y \subseteq U_Y$ פתוחות.
	אז מתקיים $(x, y) \in V_X \times V_Y \subseteq \overline{V_X \times V_Y} = \overline{V}_X \times \overline{V}_Y \subseteq U_X \times U_Y$.

	נעבור להוכחת הלמה, לכיוון הראשון $z \in U \subseteq Z$, נסמן $C = Z \setminus U$ סגורה, $z \notin C$ ולכן סגורות זרות $V, W$ כך ש־$z \in V, C \subseteq W, Z \setminus W \subseteq U$.
	אז $z \in V \subseteq \overline{V} \subseteq Z \setminus W \subseteq U$.

	בכיוון השני של הלמה נניח ש־$C$ סגורה, $z \in Z, z \notin C$, אז יש $V$ פתוחה כך ש־$z \in V \subseteq \overline{V} \subseteq Z \setminus C$, וכן $C \subseteq U = Z \setminus \overline{V}$ כך ש־$U \cap V = \emptyset$.
\end{proof}
האפיון האחרון והחשוב שנראה עתה למרחבים המקיימים אקסיומות הפרדה הוא הקשר למרחבים מטריים.
\begin{proposition}[הפרדה במרחבים מטריים]
	אם $(X, \rho)$ מרחב מטרי, אז הוא מרחב $T_4$.
\end{proposition}
\begin{proof}
	נניח ש־$E \subseteq X$ תת־קבוצה כלשהי ו־$x \in X$.
	נרחיב את הגדרת המטריקה להגדרת הקוטר, כלומר נאמר שמתקיים,
	\[
		\rho(x, E) = \inf\{ \rho(x, y) \mid y \in E \}
	\]
	אם $E$ סגורה ו־$x \notin E$ אז $\rho(x, E) > 0$ כמסקנה מטענה מאינפי 3. \\
	נניח ש־$A, B \subseteq X$ סגורות זרות, $\forall a \in A,\ \rho(a, B) > 0, \forall b \in B,\ \rho(b, A) > 0$,
	אז $U = \bigcup_{a \in A} B_{\rho(a, B)}(a)$ ו־$V = \bigcup_{b \in B} B_{\rho(b, A)}(b)$ הן פתוחות וזרות.
\end{proof}
נעיר שהכיוון ההפוך נקרא מרחב מטריזבילי, ונעסוק בנושא זה בהמשך הקורס.
נעבור לדוגמות.
\begin{example}
	אם $X = \{x, y\}$ עם הטופולוגיה $\{X, \emptyset\}$, אז $X$ הוא לא $T_0$ ולכן לא מקיים אף אקסיומת הפרדה.
\end{example}
\begin{example}
	נגדיר $X = \{x, y\}$ עם הטופולוגיה $\{\emptyset, \{x\}, X\}$.
	במקרה זה $X$ הוא מרחב $T_0$ אבל לא $T_1$.
\end{example}
\begin{example}
	נגדיר $X = \NN$ והטופולוגיה המושרית מהבסיס של כל הקבוצות שמשלימן סופי, כלומר $\Bb = \{ A \subseteq X \mid |X \setminus A| < \omega \}$.
	במקרה זה $X$ הוא מרחב $T_1$ אבל לא $T_2$.
\end{example}
\begin{example}
	נראה מרחב שהוא $T_2$ אבל לא $T_3$. \\
	נגדיר את המרחב הטופולוגי $\RR_{\frac{1}{\NN}}$ כמרחב מעל הקבוצה $\RR$, יחד עם הבסיס,
	\[
		\Bb
		= \{ (a, b) \in \RR^2 \mid a < b \}
		\cup \left\{ (a, b) \setminus \left\{ \textstyle\frac{1}{n} \mid n \in \NN \right\} \mid x, y \in \RR, x < y \right\}
	\]
	ההוכחה ש־$\Bb$ מושארת לקורא. \\
	נבחין כי זוהי טופולוגיה עדינה יותר של $\RR$, וזו האחרונה היא מרחב האוסדורף, לכן נוכל להסיק שגם $\RR_{\frac{1}{\NN}}$ מרחב האוסדורף. \\
	נראה ש־$\RR_{\frac{1}{\NN}}$ לא $T_3$.
	נבחין כי $\{\frac{1}{n} \mid n \in \NN \}$ סגורה,
	ונראה כי לא ניתן להפריד בינה לבין $0$.
	נניח ש־$0 \in U$ ו־$K = \{ \frac{1}{n} \mid n \in \NN \}$ פתוחות זרות ונקבל סתירה.
	אם $0 \in U$ פתוחה אז $U$ מכילה איבר בסיס,
	לכן $U$ מכילה קבוצה מהצורה $(a_0, b_0) \setminus K$ עבור $a_0 < 0 < b_0$.
	קיים $m \in \NN$ כך ש־$\frac{1}{m} < b_0$, ואז $(a_0, \frac{1}{m}) \setminus K \subseteq U$.
	$\frac{1}{2m} \in K \subseteq V$ ולכן $(a_1, b_1) \subseteq V$ כאשר $a_1 < \frac{1}{2m} < b_1$.
	$U \cap V \supseteq ((a_0, \frac{1}{m}) \setminus K) \cap (a_1, b_1) \ne \emptyset$, וקיבלנו סתירה.
\end{example}
\begin{example}
	נראה דוגמה למרחב שהוא $T_3$ אבל לא $T_4$. \\
	נסמן את $\RR_L$, הטופולוגיה הנוצרת על $\RR$ עם הבסיס $L = \{ [a, b) \mid a < b, a, b \in \RR \}$.
	אז $\RR_L$ הוא מרחב $T_4$ ולכן בפרט גם $T_3$.
	אז $\RR_L \times \RR_L$ היא בהכרח $T_3$ מטענה שראינו קודם על מכפלות מרחבי הפרדה. \\
	נרצה להראות ש־$\RR_L^2$ היא לא מרחב $T_4$.
	נבחין כי הטופולוגיה המושרית על $L$ מ־$\RR_L^2$ היא הטופולוגיה הדיסקרטית, ולכן כל תת־קבוצה $A \subseteq L$ היא סגורה ב־$\RR_L^2$, בשיעור הבא נראה את המשך הסתירה ל־$T_4$.
\end{example}

\section{שיעור 5 --- 8.4.2025}
\subsection{אקסיומות ההפרדה --- המשך}
נמשיך בהוכחת הסתירה עבור הדוגמה האחרונה מהשיעור הקודם.
\begin{proof}
	בנוסף הגדרנו את הקבוצה $L = \{ (-x, x) \mid x \in \RR \} \subseteq \RR_L^2$, זוהי קבוצה סגורה, וכן הטופולוגיה המושרית מ־$\RR_L^2$ על $L$ היא הטופולוגיה הדיסקרטית על $L$.
	הסקנו גם שכל $A \subseteq L$ היא סגורה ב־$L$,
	כלומר לכל $A \subseteq L$ יש קבוצה $C_A \subseteq \RR_L^2$ סגורה כך ש־$A = L \cap C_A$.
	שתי האחרונות סגורות ב־$\RR_L^2$ ולכן גם $A$ סגורה ב־$\RR_L^2$.
	נניח ש־$\RR_L^2$ היא $T_4$, בפרט זהו מרחב נורמלי, ולכן כל שתי קבוצות סגורות זרות ניתנות להפרדה.
	בפרט לכל $A \subseteq L$ יש קבוצות פתוחות זרות $U_A, V_A \subseteq \RR_L^2$ כך ש־$A \subseteq U_A, L \setminus A \subseteq V_A$.
	נקבע לכל $A \subseteq L$ זוג קבוע כזה (וניצור מיפוי).
	נתבונן ב־$D = \{ (r, s) \mid r, s \in \QQ \} \subseteq \RR_L^2$, ונגדיר $\psi(A) = U_A \cap D$, כלומר $\psi : \Pp(L) \to \Pp(D)$.
	אם נבחר את $A = \emptyset$ אז גם $U_A = \emptyset$ ובהתאם $\psi(\emptyset) = \emptyset$, ולהפך אם $A = L$ אז $U_A = \RR_L^2, V_A = \emptyset$ ולכן $\psi(A) = \RR_L^2$. \\
	$\psi$ חד־חד ערכית, ולכן מקבלת סתירה, ונותר לנו להוכיח שהיא אכן חד־חד ערכית.

	נניח ש־$\emptyset \ne A \subsetneq L$, אז $\psi(A) \ne \emptyset$ כי $D$ צפופה ו־$U_A \ne \emptyset$.
	גם $V_A \ne \emptyset$, שכן $L \setminus A \subseteq V_A$, ולכן נסיק ש־$V_A \cap D \ne \emptyset$.
	עתה נניח ש־$\emptyset \ne A, B \subsetneq L$ כך ש־$A \ne B$, אז בלי הגבלת הכלליות יש $a \in A$ כך ש־$a \notin B$.
	נובע אם כך ש־$a \in L \setminus B \subseteq V_B$ ו־$a \in A \subseteq U_A$ ולכן נובע ש־$a \in U_A \cap V_B$, וזו אף קבוצה פתוחה.
	נסיק ש־$U_A \cap V_B \ne \emptyset$, אז $p \in U_A \cap V_B \cap D$ מקיימת $p \in \psi(A)$ ו־$p \notin \psi(B)$ ובהתאם $\psi(A) \ne \psi(B)$.

	$D$ קבוצה בת־מניה ו־$L$ היא מהעוצמה של $\RR$.
	יש לנו שיכון $\Pp(D) \hookrightarrow \RR$, אבל $|\RR| = |L|$, אז נוכל לבנות $\Pp(L) \hookrightarrow \Pp(D) \hookrightarrow L$ וזה בלתי אפשרי.
\end{proof}
נסיים עם למה משמעותית במיוחד במרחבי $T_4$.
\begin{lemma}[הלמה של אוריסון]
	אם $X$ מרחב טופולוגי $T_4$,
	אז לכל זוג קבוצות סגורות זרות $C, D \subseteq X$, קיימת פונקציה רציפה $f : X \to [0, 1]$ כך ש־$f \mid_C = 1, f \mid_D = 0$.
\end{lemma}
\begin{proof}
	נניח ש־$X$ מרחב $T_4$, ויהיו $C_0 = C$ וכן $V_1 = X \setminus D$, עבור הקבוצות הסגורות הזרות $C, D \subseteq X$.
	נבחין כי $C_0$ סגורה ו־$V_1$ פתוחה, ולכן קיימות קבוצות $C_0 \subseteq V_{\frac{1}{2}} \subseteq C_{\frac{1}{2}} \subseteq V_1$.
	שוב מדובר בקבוצה סגורה ובקבוצה פתוחה.
	נגדיר כך באופן רקורסיבי קבוצות $C_{\frac{k}{2^n}}, V_{\frac{k}{2^n}}$ לכל $n \in \NN$ ו־$0 < k < 2^n$, לכן,
	\[
		C_0
		\subseteq V_{\frac{1}{2^n}}
		\subseteq C_{\frac{1}{2^n}}
		\subseteq V_{\frac{2}{2^n}}
		\subseteq C_{\frac{2}{2^n}}
		\ldots 
	\]
	ונגדיר לכל $x \in X$ את הפונקציה,
	\[
		f(x)
		\begin{cases}
			\inf\{ t \in [0, 1] \mid x \in V_t \} & \exists t, x \in V_t \\
			1 & \text{else}
		\end{cases}
	\]
	אנו טוענים ש־$f$ מקיימת את האמור, כלומר $f(x) = 0$ לכל $x \in C$, וכן $f(x) = 1$ לכל $x \in D$, ו־$f$ רציפה.
	נשים לב ש־$C = C_0 \subseteq V_{\frac{1}{2^n}}$ לכל $n \in \NN$, ולכן נובע ש־$f(x) = 0$.
	נבחין גם שעבור $x \in D$ נובע ש־$x \notin V_t$ לאף $t$ ולכן $f(x) = 1$.
	נותר אם כן להראות רציפות.
	אנו יודעים כי $f : X \to [0, 1]$ ולכן עלינו לבדוק תת־קבוצות של $[0, 1]$, אבל מספיק לבדוק את הרציפות עבור תת־בסיס של הקטע, שכל מקור של קבוצה פתוחה הוא פתוח.
	נבחר את תת־הבסיס $\Bb = \{ [0, b) \mid 0 < b \le 1 \} \cup \{ (b, 1] \mid 0 \le b < 1\}$.
	נתבונן ב־$[0, b)$, כזה, נניח שמתקיים,
	\[
		x \in f^{-1}([0, b))
	\]
	אז נובע ש־$f(x) < b$, לכן קיים $t$ כך ש־$f(x) < t < b$ מספר דיאדי (מהצורה הדרושה).
	לכן $x \notin V_s$ לכל $s < t$, ולכן נקבל ש־$f^{-1}([0, b)) \subseteq \bigcup V_t$
	נניח ש־$x \in \bigcup_t V_t$ אז יש $t_0 < b$ כך ש־$x \in V_t$ ולכן $f(x) < t_0 < b$ ונוע ש־$x \in f^{-1}([0, b))$ כפי שרצינו.
	אז מצאנו ש־$f^{-1}((b, 1])$ פתוחה אם ורק אם $f^{-1}([0, b])$ סגורה, ולכן אנו טוענים שמתקיים $f^{-1}([0, b]) = \bigcap_{b < t} C_t$.
	אם $x \notin f^{-1}([0, b])$ אז $b < f(x) \le 1$, אז מצפיפות קיימים $t_1, t_2$ דיאדיים כך ש־$b < t_1 < t_2 < f(x)$ ולכן $x \notin V_{t_2}$ וכן $C_{t_1} \subseteq V_{t_1}$ ולכן $x \notin \bigcap_{b < t} C_t$.
	אם $x \in f^{-1}([0, b])$ אז $0 \le f(x) \le b < 1$ לכל $b < t$ מתקיים $x \in V_t \subseteq C_t$ ונובע ש־$x \in \bigcap_{b < t} C_t$.
\end{proof}

\section{שיעור 6 --- 21.4.2025}
\subsection{אקסיומות מנייה}
ראינו עד כה מספר שימושים לבסיסים של טופולוגיה, הגדרה\ \ref{topology_basis_definition}.
עתה נגדיר הגדרה משלימה לבסיס בהקשר מקומי.
\begin{definition}[בסיס לטופולוגיה בנקודה]
	אם $X$ מרחב טופולוגי ו־$x \in X$ נקודה כלשהי,
	אז קבוצה של קבוצות פתוחות ${\{ U_i \}}_{i \in I}$ כך ש־$x \in U_i$ לכל $i \in I$,
	תיקרא בסיס לטופולוגיה או בסיס לקבוצות פתוחות של $x$ אם לכל קבוצה פתוחה $x \in V$ קיים $i$ כך ש־$x \in U_i \subseteq V$.
\end{definition}
בהתאם נגדיר את ההגדרה המהותית הראשונה שעוסקת במנייה.
\begin{definition}[אקסיומת המנייה הראשונה]
	נאמר שמרחב $X$ מקיים את אקסיומת המנייה הראשונה אם לכל $x \in X$ קיים בסיס לפתוחות של $x$ כך שהבסיס בן־מנייה.
\end{definition}
\begin{definition}[אקסיומת המנייה השנייה]
	נאמר שמרחב $X$ מקיים את אקסיומת המנייה השנייה אם קיים בסיס בן־מניה ל־$X$.
\end{definition}
\begin{definition}[מרחב לינדולף]
	$X$ יקרא מרחב לינדולף, אם לכל כיסוי פתוח של $X$ יש כיסוי בן־מניה. \\
	כלומר אם $X = \bigcup_{\alpha \in I} U_{\alpha}$ כיסוי פתוח, אז קיימת $J \subseteq I$ כך ש־$X \subseteq \bigcup_{\alpha \in J} U_{\alpha}$.
\end{definition}
\begin{definition}[מרחב ספרבילי]
	$X$ נקרא ספרבילי אם יש ב־$X$ תת־קבוצה צפופה בת־מניה.
\end{definition}
עתה משהגדרנו שפה לדבר בה על הקונספט של מנייה במרחבים טופולוגיים, נוכל לעבור למספר טענות.
\begin{proposition}
	מרחב רגולרי המקיים את אקסיומת המנייה השנייה הוא נורמלי. \\
	בפרט מרחב $T_3$ המקיים את אקסיומת המנייה השנייה הוא $T_4$.
\end{proposition}
\begin{proof}
	נניח ש־$X$ רגולרי המקיים את אקסיומת המנייה השנייה.
	יהי $\Bb$ בסיס בן־מניה.
	אנו רוצים להראות נורמליות, נניח ש־$A, B \subseteq X$ זרות וסגורות (ולא ריקות).
	ואנו רוצים למצוא להן הפרדה.
	לכל $a \in A$ כך ש־$a \notin \Bb$ יש קבוצה פתוחה $U_a$ כך ש־$a \in U_a \subseteq \overline{U}_a \subseteq X \setminus B$.
	ניתן לבחור את $U_a$ כך ש־$U_a \in \Bb$ ולכן האוסף $\{ U_a \mid a \in A \}$ הוא בן־מניה, ונוכל לכתוב אותו על־ידי $\{ U_{a_n} \mid n \in \NN \}$, כאשר $a_n \in A$ לכל $n$.
	קיבלנו ש־$a \in U_a \subseteq \overline{U}_a \subseteq A \setminus B$
	האוסף $\{ U_a \mid a \in A \}$ מכסה את $A$ אבל גם $A \subseteq \bigcup_{a \in A} U_a = \bigcup_{n = 1}^\infty U_{a_n}$.
	באותו אופן אפשר למצוא קבוצות פתוחות $V_b \in \Bb$ כך ש־$b \in B$ כך ש־$b \in V_b \subseteq \overline{V}_b \subseteq X \setminus A$ וסדרה ${\{ b_n \}}_{n = 1}^\infty \subseteq B$,
	כך ש־$\{ V_b \} = \{ V_b \mid b \in B \}$ ו־$\{ V_{b_n} \}$ הוא כיסוי של $B$.

	לכל $k \in \NN$ נגדיר $S_k = U_{a_k} \setminus \bigcup_{i = 1}^k \overline{V}_{b_i}$ וכן $T_k = V_{b_k} \setminus \bigcup_{i = 1}^k \overline{U}_{a_k}$.
	שתי אלה קבוצות פתוחות לכל $k$, ונגדיר בהתאם $S = \bigcup_{k \in \NN} S_k$ וכן $T = \bigcup_{k \in \NN} T_k$, גם אלה קבוצות פתוחות.
	נבחין כי $A \subseteq S, B \subseteq T$.
	נזכור ש־$\overline{V}_b \subseteq X \setminus A$ ונבדוק ש־$S \cap T = \emptyset$.
	אם החיתוך לא ריק, אז קיים$m, n \in \NN$ כך ש־$S_n \cap T_m \ne \emptyset$, בלי הגבלת הכלליות $n \le m$ ולכן נובע,
	\[
		S_m = U_{b_k} \setminus \bigcup_{i = 1}^k \overline{T}_i \supseteq T_n
	\]
	וזו סתירה.
\end{proof}
נרצה לדון בקשר שבין מרחבים מטריים למרחבים טופולוגיים.
\begin{definition}[מרחב מטריזבילי]
	מרחב טופולוגי $X$ נקרא מטריזבילי אם קיימת מטריקה על $X$ שמשרה את הטופולוגיה.
\end{definition}
כבר ראינו שכל מטריקה משרה טופולוגיה שמקיימת את $T_4$, עתה נרצה להבין מתי בדיוק טופולוגיה אכן מושרית מאיזושהי מטריקה.
\begin{remark}
	תת־מרחב של מרחב מטריזבילי הוא מטריזבילי.
\end{remark}
\begin{theorem}[משפט המטריזביליות של אורסון]
	אם $X$ מרחב טופולוגי $T_3$ המקיים את אקסיומת המנייה השנייה,
	אז $X$ מטריזבילי.
\end{theorem}
\begin{proof}
	הרעיון הכללי הוא לשכן במרחב מטרי ב־${[0, 1]}^\NN$ עם טופולוגיית המכפלה עם המטריקה,
	\[
		d(x, y)
		= \sum_{n = 1}^\infty \frac{|x_n - y_n|}{2^n}
	\]
	ולבנות העתקה $\psi : X \to {[0, 1]}^\NN$ כך ש־$\psi$ חד־חד ערכית והפיכה מ־$X$ ל־$\psi(X)$.

	לכל $x, y \in X$ כך ש־$x \ne y$ יש פתוחות זרות $U_{xy}, W_{xy} \subseteq \Bb$ כך ש־$x \in U_{xy}, y \in W_{xy}$.
	ניתן למצוא קבוצות בסיס $x \in V_{xy} \subseteq \overline{V}_{xy} \subseteq U_{xy}$.
	נתבונן באוסף כל הזוגות $\Lambda = \{ (u, u) \in \Bb^2 \mid \emptyset \not\subseteq V \subseteq \overline{V} \subseteq U \}$.
	אז $\Lambda$ בת־מניה.
	מהלמה של אוריסון קיימת פונקציה $f = f_{(u, v)} : X \to [0, 1]$ כך ש־$f \mid_{X \setminus U} = 1$ ו־$f \mid_{\overline{V}} = 0$.
	אנו מקבלים סדרת פונקציות $\{ g_k \mid k \in \NN \} = \{ f_{(u, v)} \mid (u, v) \in \Lambda \}$.
	נגדיר $\psi : X \to {[0, 1]}^\NN$ על־ידי $\psi(x)(k) = g_k(x)$.
	אנו טוענים כי $\psi$ היא רציפה, חד־חד ערכית וכן ש־$\psi : X \to \psi(X)$ היא הומיאומורפיזם.
	רציפות בטופולוגיית המכפלה שקולה לרציפות בכל קורדינטה, לכן מרציפות $g_k$ לכל $k$ נקבל רציפות $\psi$.
	חד־חד ערכיות נובעת מכך שלכל $x, y \in X$ כך ש־$x \ne y$ יש $V, U \in \Bb$ כך ש־$x \in V \subseteq \overline{V}, y \in X \setminus U$.
	יש $g_k = f_{(v, u)}$ ו־$g_k(y) = 1, g_k(x) = 0$.
	נשאר להראות הומיאומורפיזם.
	אנו יודעים ש־$\psi$ חד־חד ערכית, וצריך להראות שלכל ש־$\psi^{-1} : E \to X$ היא רציפה כאשר $E = \psi(X)$, כלומר צריך להראות שלכל $W \subseteq X$ פתוחה, שגם $\psi(W)$ פתוחה ב־$E$.
	לכל $x \in W$ קיימת $U \in \Bb$ כך ש־$x \in U \subseteq W$, וקיימת $V \in \Bb$ כך ש־$x \in V \subseteq \overline{V} \subseteq U$.
	יהי $k(x) \in \NN$ כך ש־$g_{k(x)} = f_{(v, u)}$ ומתקיים, $g_{k(x)}(x) = 0$ וכן $g_{k(x)} \mid_{X \setminus U} = 1$.
	אז $x \in g^{-1}([0, 1)) \subseteq U \subseteq W$ ונובע ש־$\bigcup_{x \in W} g_{k(x)}^{-1}([0, 1)) = W$.
	$g_{k(x)} = \pi_{k(x) \circ \psi}$ ולכן $g_{k(x)}^{-1} = \psi^{-1} \circ \pi_{k(x)}^{-1}$ ולכן,
	\[
		W
		= \bigcup_{x \in W} \psi^{-1}(\pi_{k(x)}^{-1}([0, 1)))
		= \psi^{-1}(\bigcup_{x \in W} \pi_{k(x)}^{-1}([0, 1)))
	\]
	ונובע $\psi(W) = (\bigcup_{x \in W} \pi_{k(x)}^{-1}([0, 1))) \cap E$.
\end{proof}

\subsection{קשירות}
\begin{definition}[קשירות]
	מרחב טופולוגי $X$ יקרא קשיר אם לא ניתן להציג אותו כאיחוד של שתי קבוצות פתוחות זרות לא ריקות.
\end{definition}
\begin{remark}
	באופן שקול גם אם לא ניתן להציג את המרחב כאיחוד זר של קבוצות סגורות.
	זאת שכם אם $X = U \cup V$ אז $U^C \cap V^C = \emptyset$ וכן אם $U \cap V = \emptyset$ אז $U^C \cup V^C = X$ וכמובן $U^C, V^C$ פתוחות.
\end{remark}
\begin{example}
	מהן תתי־הקבוצות הקשירות של $\RR$?
	התשובה היא קטעים, $(a, b), [a, b], (a, b], [a, b)$.
\end{example}
\begin{remark}
	מרחב טופולוגי $X$ הוא קשיר אם ורק אם כל פונקציה רציפה $f : X \to \{0, 1\}$ עם הטופולוגיה הדיסקרטית, היא קבועה.
\end{remark}
\begin{proposition}[תכונות של קשירות]
	התכונות הבאות מתקיימות,
	\begin{enumerate}
		\item אם $f : X \to Y$ רציפה ו־$X$ קשיר אז $f(X)$ קשירה
		\item אם $A \subseteq X$ קשירה אז $\overline{A}$ קשירה
		\item למת כוכב, אם ${\{ A_{\alpha} \}}_{\alpha \in I}$ תתי־קבוצות קשירות וקיים $\beta \in I$ כך ש־$A_{\alpha} \cap A_{\beta} \ne \emptyset$ ב־$X$ לכל $\alpha \in I$ אז $\bigcup_{\alpha \in I} A_{\alpha}$ קשירה
		\item אם ${\{ X_{\alpha} \}}_{\alpha \in I}$ קבוצת מרחבים טופולוגיים קשירים אז $Y = \prod_{\alpha \in I} X_{\alpha}$ קשירה
	\end{enumerate}
\end{proposition}
\begin{proof}
	נוכיח את טענה 2.
	נניח ש־$\overline{A}$ לא קשירה, לכן נובע שיש $f : \overline{A} \to \{0, 1\}$ לא קבועה.
	בלי הגבלת הכלליות נניח ש־$f(A) = \{ 0 \}$, אבל $\{ 0 \} \subseteq \{0, 1\}$ סגורה ולכן $A \subseteq f^{-1}(\{0\})$ סגורה ונובע ש־$\overline{A} \subseteq f^{-1}(\{0\})$ וזו סתירה.

	נעבור להוכחת טענה 4.
	נתונים ${\{ X_{\alpha} \}}_{\alpha \in I}$ מרחבים טופולוגיים ונרצה להראות ש־$Y$ קשיר.
	אם $A, B$ מרחבים טופולוגיים קשירים אז $A \times B$ קשיר, כנביעה מטענה 3, שכן,
	\[
		A \times B
		= ( \bigcup_{a \in A} \{ a \} \times B ) \cup ( \bigcup_{b \in B} A \times \{ b \} )
	\]
	נרצה למצוא תת־קבוצה של $Y$ שתהיה צפופה וקשירה.
	נקבע $f \in Y$, כאשר $f : I \to \bigcup X_{\alpha}$.
	יש $f$ כזו מאקסיומת הבחירה.
	נגדיר את $Z = \{ h \in Y \mid |\{ \alpha \in I \mid h(\alpha) \ne f(\alpha) \}| < \infty \} = \bigcup_{F \subseteq I, |F| < \infty} P_F$ כאשר $P_F = \{h \in Y \mid h(\alpha) = f(\alpha) \forall \alpha \notin F\}$.
	אנו טוענים שתי טענות, הראשונה היא שלכל $F$ סופית $P_F$ קשירה, השנייה היא ש־$\bigcup P_F = Z$ קשירה והשלישית היא ש־$Z$ צפופה.
	נבחין כי $P_F \cong \prod_{y \in F} X_y$ מהגדרת טופולוגיית המכפלה.

	נבהיר שמטרתנו הייתה למצוא קבוצה צפופה $Z \subseteq Y$ ולהשתמש בטענה על סגור של צפופה.
	עשינו זאת על־ידי הוכחה למקרים סופיים עם למת הכוכב.
	בשלב הבא לכל $F \subseteq I$ סופית המכפלה $\prod_{\alpha \in I} X_{\alpha} = Y_F$ קשירה ונקבע $f \in \prod_{\alpha \in I}  X_{\alpha}$.
	נגדיר $Z_F = \{ h \in \prod_{\alpha \in I} X_{\alpha} = Y \mid \forall \beta \notin F, h(\beta) = f(\beta) \}$,
	אז $Z_F$ בעצם סוג של שווה ל־$Y_F \times \{ f_F \}$ עבור $f_F : I \setminus F \to \bigcup_{\alpha \in I \setminus F} X_{\alpha}$, ו־$f_F(\alpha) = f(\alpha)$.
	אם נגדיר $Z = \bigcup_{F \subseteq I, |F| < \omega} Z_F$ נקבל קבוצה קשירה, זאת שכן $f \in Z_F$ לכל $F$ שכן מתקיימים התנאים של למת כוכב.
	$Z$ צפופה ולכן מספיק להתבונן בבסיס $\Bb$ של הטופולוגיה ולהראות שלכל $\emptyset \ne B \in \Bb$ מתקיים $Z \cap B \ne \emptyset$.
	נתבונן בבסיס שהגדרנו בעזרתו את טופולוגיית המכפלה, כל קבוצה בבסיס הזה היא מהצורה $\prod_{\alpha \in F} U_{\alpha} \times \prod_{\beta \notin F} X_{\beta}$,
	כאשר $F \subseteq I$ סופית ו־$\emptyset \ne U_{\alpha} \subseteq X_{\alpha}$ לכל $\alpha \in I$.
	קיימת $g \in B$ כך ש־$g(\beta) = f(\beta)$ לכל $\beta \notin F$.
	מ־$\Bb \ne \emptyset$ נובע שקיימת איזושהי $h \in B$, אז נגדיר,
	\[
		B
		\ni g(\alpha)
		= \begin{cases}
			h(\alpha) & \alpha \in F \\
			f(\alpha) & \alpha \notin F
		\end{cases}
	\] 
	נטען כי $g \in Z$, זאת שכן $g \in Z_F \subseteq Z$.
\end{proof}

\section{שיעור 7 --- 22.5.2025}
\subsection{קשירות --- המשך}
\begin{definition}[קשירות מקומית]
	נאמר שהמרחב הטופולוגי $X$ הוא קשיר מקומית בנקודה $x \in X$ אם לכל סביבה $W$ של $x$ יש קבוצה פתוחה וקשירה $x \in U \subseteq W$.
	נאמר ש־$X$ קשיר מקומית אם $x$ קשיר מקומית לכל $x \in X$.
\end{definition}
\begin{definition}[רכיב קשירות]
	רכיב הקשירות של $x$ במרחב הטופולוגי $X$ הוא תת־הקבוצה הקשירה המקסימלית אשר מכילה את $x$.
\end{definition}
\begin{remark}
	אכן קיימת קבוצה כזו בשל הסגירות לאיחוד של הטופולוגיה, נבחר את $\bigcup_{x \in Z \subseteq X} Z$.
\end{remark}
\begin{example}
	מה הוא רכיב הקשירות של $\frac{1}{3}$ ב־$\QQ$?
	התשובה היא ש־$\{\frac{1}{3}\}$.
\end{example}
\begin{definition}[מסילה]
	מסילה ב־$X$ היא פונקציה רציפה $\alpha : [a, b] \to X$ כך ש־$a, b \in \RR, a < b$.
	נאמר שזוהי מסילה בין $\alpha(a)$ ל־$\alpha(b)$, וכן שהמסילה $\alpha$ מחברת את שתי הנקודות הללו.
\end{definition}
\begin{definition}[קשירות מסילתית]
	נאמר שהמרחב הטופולוגי $X$ הוא קשיר מסילתית אם לכל $x, y \in X$ קיימת מסילה $\alpha : [0, 1] \to X$ כך ש־$\alpha(0) = x, \alpha(1) = y$.
\end{definition}
\begin{definition}[קשירות מסילתית מקומית]
	המרחב $X$ קשיר מסילתית מקומית ב־$x$ אם לכל סביבה $W$ של $x$ יש קבוצה פתוחה $x \in U \subseteq W$ כך ש־$U$ קשירה מסילתית. \\
	בהתאם $X$ קשיר מסילתית מקומית אם $x$ קשיר מסילתית מקומית לכל $x \in X$.
\end{definition}
נתעניין להבין מה הקשר בין ארבעת מושגי הקשירות שראינו זה עתה.
נתחיל בתכונה חשובה של קשירות מסילתית.
\begin{proposition}
	אם $X$ קשיר מסילתית ו־$f : X \to Y$ רציפה אז $f(X)$ קשירה מסילתית.
\end{proposition}
\begin{proof}
	יהיו $p, q \in f(X)$, אז קיימות נקודות $p', q' \in X$ כך ש־$f(p') = p, f(q') = q$ וכן יש מסילה $\alpha : [0, 1] \to X$ כך ש־$\alpha(0) = p'$ ו־$\alpha(1) = q'$.
	הרכבת פונקציות רציפות היא רציפה ולכן $f \circ \alpha$ מסילה המקשרת את $p$ ל־$q$.
\end{proof}
עתה נראה את הקשר בין קשירות וקשירות מסילתית.
\begin{proposition}
	אם $X$ קשיר מסילתית אז $X$ קשיר.
\end{proposition}
\begin{proof}
	אם $X$ לא קשיר אז יש פונקציה רציפה $f : X \to \{0, 1\}$ עם הטופולוגיה הדיסקרטית כך ש־$f(X) = \{0, 1\}$ אבל $\{0, 1\}$ לא קשיר מסילתית כי $[0, 1]$ קשיר.
\end{proof}
נבחין כי קשירות לא גוררת קשירות מסילתית, נראה דוגמה מתאימה.
\begin{example}
	נתבונן בגרף של $\sin \frac{1}{x}$ עבור $0 < x \le 1$, $G$.
	זוהי תת־קבוצה של $\RR^2$, ונניח ש־$X$ הסגור של גרף זה ב־$\RR^2$.
	נבחין כי $X = \{0\} \times [-1, 1] \cup G$.
	סגור של קבוצה קשירה הוא קשיר ולכן סגור זה אכן קשיר.
	מהצד השני הוא לא קשיר מסילתית, לא קיימת מסילה $\alpha : [0, 1] \to X$ כך ש־$\alpha(0) = (0, 0), \alpha(1) = (1, \sin 1)$.
\end{example}

\section{שיעור 8 --- 28.4.2025}

\subsection{קשירות --- סגירת פינות}
\begin{example}
	נראה מרחב קשיר אך איננו קשיר מקומית.
	זהו מרחב המסרק,
	\[
		( \{0\} \times [0, 1]) \cup \{ [0, 1] \times \{ 0 \} \} \bigcup_{n \in \NN} \{ \frac{1}{n} \} \times [0, 1]
	\]
\end{example}
מן הצד השני ראינו גם כי קשירות לא גוררת קשירות מסילתית.
\begin{example}
	הצמצום של $\RR^2$ לגרף של $\sin \frac{1}{x}$ ב־$(0, 1]$,
	\[
		Y = (\{ 0 \} \times [0, 1]) \cup \{ (x, \sin \frac{1}{x}) \mid 0 < x \le 1 \}
	\]
	מרחב זה הוא קשיר שכן הוא צמצום של מרחב קשיר והגרף רציף כתמונה של פונקציה רציפה ממרחב קשיר (קטע).

	נניח בשלילה ש־$Y$ קשיר מסילתית ולכן יש בפרט מסילה $\alpha : [0, 1] \to Y$ כך שמתקיים,
	\[
		\alpha(0) = (0, 0),
		\qquad \alpha(1) = (1, \sin 1)
	\]
	נגדיר גם $\delta(t) = (\alpha_{1}(t), \alpha_{2}(t))$ ולכן $\alpha_{1}(0) = 0, \alpha_{1}(1) = 1$.
	ממשפט ערך הביניים קיים $0 < t_1 < 1$ כך ש־$\alpha_{1}(t_1) = \frac{1}{\frac{\pi}{2}}$.
	נמצא $0 < t_2 < t_1$ כך שמתקיים,
	\[
		\alpha(t_2) = (?, -1)
	\]
	ואז נוכל למצוא $0 < t_3 < t_2$ כך ש־$\alpha(t_3) = (?, 1)$.
	נוכל לבנות ככה סדרה של נקודות כאלה ונקבל שלנקודות אלה יש גבול $(0, 0)$ ולכן מאפיון היינה לגבולות נקבל,
	\[
		\alpha(0)
		= \lim_{n \to \infty} t_n
		= \lim_{n \to \infty} {(-1)}^n
	\]
	אבל גבול זה לא קיים.
\end{example}
\begin{proposition}
	אם $X$ קשיר וקשיר מסילתית מקומית אז $X$ קשיר מסילתית.
\end{proposition}
\begin{proof}
	יהי $x_0 \in X$ ונתבונן במחלקת הקשירות המסילתית של $x_0$ ונסמנו $A$.
	אנו יודעים ש־$x_0 \in A$ ולכן $A \ne \emptyset$ ואנו יודעים גם כי $A$ פתוחה,
	זאת שכן לכל $a \in A$ עבור $a \in X$ אנו יודעים כי $A$ קשיר מסילתית ולכן בפרט ישנה סביבה של $a$ ב־$X$. \\
	נטען גם כי $A$ סגורה,
	הראינו שבמרחב קשיר מסילתית מקומית כל רכיב קשירות מסילתית הוא קבוצה פתוחה, אבל זה גורר שכל רכיב קשירות מסילתית הוא גם קבוצה סגורה כמשלים של איחוד קבוצות פתוחות, הן רכיבי הקשירות המסילתית האחרים. \\
	אז $A \in \{X, \emptyset\}$ אבל $x_0 \in A$ ונסיק ש־$A = X$.
\end{proof}

\subsection{קומפקטיות}
\begin{definition}[קומפקטיות]
	מרחב טופולוגי $X$ יקרא קומפקטי אם לכל כיסוי פתוח של $X$ יש תת־כיסוי סופי. \\
	המשמעות היא שלכל אוסף קבוצות פתוחות ${\{ U_{\alpha} \}}_{\alpha \in I}$ כך ש־$X = \bigcup_{\alpha \in I} U_{\alpha}$ אז קיים $I_{0} \subseteq I$ סופי כך ש־$X = \bigcup_{\alpha \in I_{0}} U_{\alpha}$. \\
	תת־קבוצה $K \subseteq X$ תיקרא קומפקטית אם היא מרחב קומפקטי כתת־מרחב של $X$, זה נכון באופן דומה עבור כיסוי פתוח המכיל את $K$.
\end{definition}
נראה הגדרה שקולה בניסוח של קבוצות סגורות,
\begin{definition}[שקילות לקומפקטיות]
	$X$ מרחב טופולוגי קומפקטי אם ורק אם לכל אוסף ${\{ F_{\alpha} \}}_{\alpha \in I}$ של תתי־קבוצות סגורות ב־$X$ כך שיש להן את תכונת החיתוך הסופי,
	כלומר ש־$\bigcap_{\alpha \in I_{0}} F_{\alpha} \ne \emptyset$ לכל $I_{0} \subseteq I$ סופית,
	אם $F_{0}$ סגורה לכל $\alpha$ ו־$\emptyset = \bigcap_{\alpha \in I} F_{\alpha}$ אז יש $I_{0} \subseteq I$ סופית כך שמתקיים,
	\[
		\bigcap_{\alpha \in I_{0}} F_{\alpha} = \emptyset
	\]
\end{definition}
\begin{remark}
	ראינו בקורסים קודמים שתת־קבוצה $A \subseteq \RR^n$ היא קומפקטית אם ורק אם $A$ סגורה וחסומה.

	עבור המקרה של $A \subseteq \RR$ נניח שמתקיים,
	\[
		A \subseteq \bigcup_{n \in \NN} (-n, n) = \RR
	\]
	ולכן קיים $I \subseteq \NN$ סופי כך ש־$A \subseteq \bigcup_{n \in I} (-n, n)$.
	לכל $a, b \in A$ כך ש־$a \ne b$ אנו יודעים כי $\RR$ האוסדורף ולכן קיימות $a \in U_a$ ו־$b \in V_a$ וכן $U_a \cap V_a = \emptyset$.
	אז $A \subseteq \bigcup_{a \in A} U_a$ ומקומפקטיות $A$ יש תת־כיסוי סופי כזה, כלומר $A \subseteq \bigcup_{i = 1}^N U_{a_n}$ וכן $b \in \bigcap_{i = 1}^N V_{a_i}$ ולכן,
	\[
		V \cap (\bigcup_{i = 1}^N U_{a_n}) = \emptyset
	\]
	ונובע ש־$V \cap A = \emptyset$ וכן $V \subseteq \RR \setminus A$ ולכן $\RR \setminus A$ פתוחה.
	לכיוון ההפוך נראה טענה יותר כללית בהמשך.

	הוכחנו כרגע טענה חזקה יותר, כל תת־קבוצה קומפקטית $A$ במרחב טופולוגי האוסדורף $X$ היא סגורה.
\end{remark}
\begin{remark}
	קיימים מרחבים טופולוגיים עם תת־קבוצה קומפקטית שאינה סגורה.
	לדוגמה $X = \{ a, b \}$ עם הטופולוגיה הטריוויאלית,
	אז $A = \{ a \}$ היא קומפקטית אבל לא סגורה.
\end{remark}
\begin{proposition}
	אם $X$ קומפקטית ו־$A \subseteq X$ סגורה אז $A$ קומפקטית.
\end{proposition}
\begin{proof}
	נניח כי ${\{ U_{\alpha} \}}_{\alpha \in I}$ אוסף של פתוחות כך שהן מכסות את $A$, אז
	\[
		X = (X \setminus A) \cup \bigcup_{\alpha \in I} U_{\alpha}
	\]
	וקיבלנו כי יש למרחב תת־סיכוי סופי.
	כלומר יש $I_{0} \subseteq I$ סופית כך שמתקיים,
	\[
		X = (X \setminus A) \cup \bigcup_{\alpha \in I_{0}} U_{\alpha}
	\]
	ולכן $A \subseteq \bigcup_{\alpha \in I_{0}} U_{\alpha}$.
\end{proof}
\begin{proposition}
	תמונה רציפה של מרחב קומפקטי היא קומפקטית,
	כלומר אם $X$ מרחב קומפקטי ו־$f : X \to Y$ פונקציה רציפה מ־$X$ למרחב טופולוגי $Y$ אז $f(X) \subseteq Y$ היא קומפקטית.
\end{proposition}
\begin{proof}
	נניח ש־$f(X) \subseteq \bigcup_{\alpha \in I} U_{\alpha}$ אז $X \subseteq f^{-1}( \bigcup_{\alpha \in I} U_{\alpha}) = \bigcup_{\alpha \in I} f^{-1}(U_{\alpha})$,
	ולכן יש תת־כיסוי סופי כזה, כלומר קיימת $I_{0} \subseteq I$ סופית כך ש־$X \subseteq \bigcup_{\alpha \in I_{0}} f^{-1}(U_{\alpha})$ ולכן $f(X) \subseteq \bigcup_{\alpha \in I_{0}} U_{\alpha}$.
\end{proof}
\begin{proposition}
	אם $X$ מרחב האוסדורף קומפקטי אז $X$ מרחב רגולרי.
\end{proposition}
\begin{proof}
	רגולריות מתקיימת אם ורק אם $T_0$ וגן אפשר להפריד בין כל קבוצה סגורה $A$ ונקודה $b \notin A$.

	אז אם $A \subseteq X$ סגורה עבור $X$ קומפקטי אז נובע ש־$A$ קומפקטית, כל $a \in A$ כך ש־$a \ne b$ נובע שיש פתוחות $a \in U_a, b \in V_a$ עבור $U_a, V_a$ זרות.
	אנו יודעים כי $A \subseteq \bigcup_{a \in A} U_a$ ולכן קיימות נקודות $a_1, \ldots, a_N \in A$ כך ש־$A  \subseteq \bigcup_{i = 1}^N U_{a_i} = U$,
	וכן $b \in V = \bigcap_{i = 1}^N V_{a_i}$.
	$U, V$ פתוחות זרות כך ש־$A \subseteq U$ ו־$b \in V$.
\end{proof}
\begin{conclusion}
	אם $X$ מרחב טופולוגי קומפקטי ו־$Y$ מרחב טופולוגי האוסדורף, $f : X \to Y$ רציפה חד־חד ערכית ועל, אז $f$ היא הומיאומורפיזם.
\end{conclusion}
\begin{proof}
	עלינו להראות רק ש־$f$ מקיימת ש־$f^{-1}$ רציפה, ונקבל שכלל התנאים להומיאומורפיזם חלים.
	לכל תת־קבוצה סגורה $C \subseteq X$ עלינו להראות ש־${(f^{-1})}^{-1}(C) \subseteq Y$ סגורה.
	$X$ קומפקטית ו־$C$ סגורה ולכן היא קומפקטית ולכן נובע ש־$f(C)$ קומפקטית אבל $Y$ האוסדורף ולכן $f(C)$ סגורה.
\end{proof}
\begin{proposition}
	אם $X$ מרחב האוסדורף קומפקטי אז $X$ מרחב נורמלי.
\end{proposition}
\begin{proof}
	נניח ש־$A, B \in X$ קתי קבוצות סגורות וזרות, אז לכל $b \in B$ מתקיים $b \notin A$ ולכן יש $b \in V_b$ ו־$A \subseteq U_b$ פתוחות זרות,
	ו־$B \subseteq \bigcup_{b \in B} U_{b}$ קומפקטית כי היא סגורה במרחב קומפקטי ולכן $B \subseteq \bigcup_{i = 1}^n V_{b_i}$ כיסוי פתוח סופי, וכן $A \subseteq \bigcap_{i = 1}^n U_{b_i}$,
	ושתי הקבוצות הללו מפרידות בין $A$ ל־$B$ ופתוחות.
\end{proof}
\begin{proposition}
	$X$ מרחב טופולוגי קומפקטי ו־$f : X \to \RR$ רציפה, אז,
	\begin{enumerate}
		\item $f(X)$ חסומה (וסגורה)
		\item יש ל־$f$ מקסימום ומינימום
		\item נניח $X$ מטריזבילי ותהי $\rho$ המטריקה אז $f$ רציפה במידה שווה.
	\end{enumerate}
\end{proposition}
\begin{proof}
	נוכיח את הטענות,
	\begin{enumerate}
		\item ראינו ש־$f(X) \subseteq \RR$ קומפקטית ותת־קבוצה קומפקטית של $\RR$ היא סגורה וחסומה.
		\item נניח ש־$M = \sup_{x \in X} f(x)$ מתקבל וסופי, נסמן גם $A = f(X)$, אז מההגדרה $M$ הוא הסופרימום של $A$ ולכן כל $x \in X$ מקיים ש־$f(x) \le M$ וגם לכל $\epsilon > 0$,
			יש $a \in A$ כך ש־$M - \epsilon \le a = f(x)$.
			נובע אם כך ש־$A \cap [M - \epsilon, M] \ne \emptyset$.
			יש אוסף של תתי־קבוצות סגורות $F_{\epsilon_i} = A \cap [M - \epsilon_i, M]$ לכל $\epsilon_n > 0$ כך שלכל $n \in \NN$,
			\[
				\bigcup_{i = 1}^n F_{\epsilon_i}
				= A \cap [M - \delta, M]
			\]
			עבור $\delta = \min\{ \epsilon_1, \ldots, \epsilon_n \}$.
			נובע אם כך,
			\[
				A \cap \{ M \}
				= \bigcap_{\epsilon > 0} (A \cap [M - \epsilon, M])
				= \bigcap_{\epsilon > 0} F_{\epsilon} \ne \emptyset
			\]
			ולכן נסיק ש־$M \in A = f(X)$.
		\item מושאר כתרגיל, אבל רמז הוא מספר לבג לכיסוי.
	\end{enumerate}
\end{proof}

\subsection{קומפקטיות במרחבים מטריים}
לא נגדיר אך ניזכר במספר הגדרות חשובות מעולם המרחבים המטריים, הן סדרות קושי, שלמות, חסימות לחלוטין.
בהינתן שאנו מכירים את המונחים הללו, נעבור למשפט, אך לפני זה נגדיר מונח חדש שיעזור לנו בהוכחת משפט זה.
\begin{definition}[התכנסות סדרה במרחב טופולוגי]
	סדרה ${\{ x_n \}}_{n = 1}^\infty \subseteq X$ עבור מרחב טופולוגי $X$ מתכנסת ל־$x$ אם לכל סביבה פתוחה $x \in U$ מתקיים $x_n \in U$ לכמעט כל $n$, כלומר קיים $N \in \NN$ כך שלכל $n \ge N$ אז $x_n \in U$.
\end{definition}
\begin{definition}[מספר לבג]
	יהי $X$ מרחב טופולוגי קומפקטי מטרי, ויהי ${\{ U_{\alpha} \}}_{\alpha \in I}$ כיסוי פתוח של $X$.
	אז $\lambda > 0$ נקרא מספר לבג של הכיסוי אם לכל $x \in X$ קיים $\alpha \in I$ כך ש־$B_{\lambda}(x) \subseteq U_{\alpha}$.
\end{definition}
\begin{remark}
	במקרה של מרחבים מטריים קומפקטיים, תמיד יש מספר לבג.
	כדי לראות זאת, לכל $n \in \NN$ יש $x \in X$ כך ש־$U_{\alpha} \not\supseteq B_{\frac{1}{n}}(x)$ לכל $\alpha \in I$.
	נגדיר $x_{n_k} \to y \in X$ מקומפקטיות סדרתית ונקבל סתירה.
\end{remark}
\begin{remark}
	באופן כללי קומפקטיות לא גוררת קומפקטיות סדרתית וגם לא להיפך.
\end{remark}
\begin{example}
	נראה דוגמה שמצביה שקומפקטיות סדרתית לא גוררת קומפקטיות.
	נגדיר $I = [0, 1]$ וכן $X = {\{0, 1\}}^I$ עם טופולוגיית המכפלה.
	$X$ קומפקטי כמקרה פרטי של משפט טיכונוף שנוכיח בהמשך.
	נגדיר $Y = \{ x = {(x_i)}_{i \in I} \in X \mid |\{ \alpha \in I \mid x = 1\}| \le \aleph_0 \}$ כתת־מרחב של $X$ עם הטופולוגיה המושרית ממנו.
	אנו טוענים כי $Y$ קומפקטי סדרתית אבל לא קומפקטי.

	נראה ש־$Y$ לא קומפקטי, לכל $\alpha \in I$ נסמן $U_{\alpha} = \{ x \in X \mid x_{\alpha} = 0 \}$ פתוחה, וכן $Y \subseteq \bigcup_{\alpha \in I} U_{\alpha}$.
	מצד שני, לכל $\alpha_1, \ldots, \alpha_n \in I$,
	\[
		Y \not\subseteq \bigcup_{i = 1}^n U_{\alpha_i}
	\]

	כדי להראות קומפקטיות סדרתית, לכל $y_n \in Y$, $J_n \subseteq [0, 1]$ בת־מניה מ־$y_n(\alpha) = 0$ לכל $\alpha \notin J$ עבור $J = \bigcup_{n = 1}^\infty J_n$ עבור $y_n \in {\{0, 1\}}^J$.
\end{example}
\begin{theorem}[שקילות לקומפקטיות במרחבים מטריים]
	יהי $X$ מרחב מטרי, אז התנאים הבאים שקולים,
	\begin{enumerate}
		\item $X$ קומפקטי
		\item $X$ קומפקטי סדרתית
		\item $X$ שלם וחסום לחלוטין
	\end{enumerate}
\end{theorem}
הסדר בו נוכיח את המשפט יהיה $1 \implies 2 \implies 3 \implies 2 \implies 1$.

\section{שיעור 9 --- 29.4.2025}
\subsection{קומפקטיות --- תכונות}
נמשיך במתן דוגמות,
\begin{example}
	נראה דוגמה למרחב קומפקטי סדרתית שאינו קומפקטי.
	נגדיר $I = [0, 1]$ וכן $X{\{0, 1\}}^I$, $X$ קומפקטי ממשפט טיכונוף שנוכיח בהמשך.
	נגדיר גם $Y = \{ x = {(x_i)}_{i \in I} \in X \mid |\{ \alpha \in I \mid x = 1\}| \le \aleph_0 \}$.
	אנו טוענים כי $Y$ לא קומפקטית אבל כן קומפקטית סדרתית.
	לכל $\alpha \in I$ נגדיר $U_{\alpha} = \{ x \in X \mid x_{\alpha} = 0 \}$ קבוצה פתוחה, וכן זהו כיסוי של $Y$, כלומר $Y \subseteq \bigcup_{\alpha \in I} U_{\alpha}$.
	אבל אין תת־כיסוי סופי של $Y$, על־ידי קבוצות מהצורה $U_{\alpha}$, זאת שכן אם $\alpha_1, \ldots, \alpha_n$ נקודות, אז,
	\[
		\bigcup_{i = 1}^n U_{\alpha_i}
		\subseteq \{ x \in X \mid \exists 1 \le i \le n, x_{\alpha} = 0 \}
	\]
	ובמקרה זה נבחר $Z = Z_{\alpha}$ עבור,
	\[
		Z_{\alpha} = \begin{cases}
			1 & \alpha = \alpha_i, 1 \le i \le n \\
			0 & \text{else}
		\end{cases}
	\]
	נוכיח עתה כי $Y$ קומפקטית סדרתית.
	תהי ${\{ y^n \}}_{n = 1}^\infty \subseteq Y$ כאשר $y^n = {( y_{\alpha}^n )}_{\alpha \in I}$ לכל $n$.
	נגדיר,
	\[
		J_n = \{ \alpha \in I \mid y_{\alpha}^n = 1 \}
	\]
	ונבחין כי $|J_n| \le \aleph_0$, נגדיר גם $J = \bigcup_{n \in \NN} J_n$.
	נתבונן במרחב הטופולוגי ${\{0, 1\}}^J$, זהו מרחב קומפקטי מכיוון ש־$J$ בתמונה ו־$\{0, 1\}$ מרחב מטרי.
	רעינו שיש מטריקה על ${\{0, 1\}}^J$ שמתאימה לטופולוגיית המכפלה.
	נגדיר את ההטלות $Z_n = \pi(y_n)$ כאשר $\pi : {\{0, 1\}}^I \to {\{0, 1\}}^J$.
	מרחב מטרי קומפקטי הוא קומפקטי סדרתית ולכן יש תת־סדרה $n_k$ כך ש־$Z^{n_k}$ סדרה מתכנסת,
	נשאר לנו לבדוק שנובע שגם $y^{n_k}$ מתכנסת.
\end{example}
\begin{example}
	נראה דוגמה למרחב קומפקטי שאינו קומפקטי סדרתית. \\
	נגדיר $X = I^I$, כלומר $f : I \to I$ מקיימת $f \in X$.
	מטיכונוף שוב $X$ קומפקטי.
	נגדיר סדרת איברים ${\{ f_n \}}_{i = 1}^\infty \subseteq X$ כאשר $f_n : [0, 1] \to [0, 1]$.
	כל $t \in [0, 1]$ ניתן לכתוב כפיתוח בינארי, $t = 0. t_1 t_2 t_3 \ldots$ עבור $t_i \in \{0, 1\}$, ומתקיים, $t = \sum_{i = 1}^\infty \frac{t^i}{2^i}$.
	נוכל למשל לבחור את הפיתוח האינסופי $0.111\ldots$.
	נגדיר עתה $f_n(t) = t_n$, כלומר סדרת הפונקציות שמחלצות את הספרה ה־$n$ מהמספר שהן מקבלות.
	נניח של־$\{ f_n \}$ יש תת־סדרה מתכנסת $\{ {f_{n_k} \}}_{k = 1}^\infty \subseteq \{ f_n \}$.
	נגדיר $s = 0. s_1 s_2 \ldots$ כאשר,
	\[
		s_m
		= \begin{cases}
			1 & m = n_{2k} \\
			0 & \text{else}
		\end{cases}
	\]
	ונחשב,
	\[
		f_{n_k}(s)
		= \begin{cases}
			1 & k \in 2\NN \\
			0 & k \in 2\NN + 1
		\end{cases}
	\]
	ולכן $f_{n_k}$ לא מתכנסת.
\end{example}
מצאנו שתי דוגמות שאכן מעידות על זה שקומפקטיות וקומפקטיות סדרתית לא גוררות אחת את השנייה במרחבים כלליים.
\begin{theorem}[משפט טיכונוף]
	מכפלה של מרחבים טופולוגיים קומפקטיים היא קומפקטית,
	כלומר אם $X_{\alpha}$ מרחב קומפקטי לכל $\alpha \in I$, אז $\prod_{\alpha \in I} X_{\alpha}$ עם טופולוגיית המכפלה הוא קומפקטי.
\end{theorem}
\begin{proof}
	יהיו $X_1, X_2$ מרחבים טופולוגיים קומפקטיים, ונוכיח ש־$X_1 \times X_2$ קומפקטי.
	נניח בשלילה שאכן $X_1, X_2$ קומפקטיים אבל ש־$Y = X_1 \times X_2$ לא קומפקטי.
	לכן יש ${\{ W_{\lambda}\}}_{\lambda \in \Omega}$ כיסוי פתוח של $Y$ כך שאין לו תת־כיסוי סופי.
	נראה שיש נקודה $y = (a, b) \in Y$ כך שאין קבוצת בסיס פתוחה שמכילה את $y$ אשר ניתנת לכיסוי על־ידי מספר סופי של קבוצות מהאוסף ${\{ W_{\omega} \}}_{\omega \in \Omega}$,
	וזה בלתי אפשרי כי $\bigcup_{\omega \in \Omega} W_{\omega} = Y \ni y$ ולכן יש $\beta \in \Omega$ כך ש־$y \in W_{\omega}$ ו־$W_{\beta}$ פתוחה ולכן מכילה קבוצת בסיס שמכילה את $y$.

	נטען כי יש $a \in X_1$ כך שלא קיימת קבוצה פתוחה $a \in U \subseteq X_1$ כך ש־$U \times X_2$ מוכלת באיחוד סופי של קבוצות מ־${\{ W_{\omega} \}}_{\omega \in \Omega}$.
	נניח בשלילה ונקבל שלכל $a \in X_1$ יש קבוצה פתוחה $a \in U_a \subseteq X_1$ כך שקיים תת־כיסוי סופי של $U_a \times X_2$ על־ידי קבוצות מהכיסוי הנתון.
	נבחן את $X = \bigcup_{a \in X_1} U_a$, כיסוי פתוח, אבל $X_1$ קומפקטית ולכן קיימות $a_1, \ldots, a_n \in X_1$ כך ש־$X = \bigcup_{i = 1}^n U_{a_i}$.
	כל $U_{a_i} \times X_2$ היא פתוחה, ולכן מצאנו כיסוי סופי ל־$Y$.
	זאת כמובן סתירה בשל ההנחה כי אין תת־כיסוי סופי.

	עתה נטען כי יש $b \in X_2$ כך שלכל קבוצה פתוחה $a \in U \subseteq X_1$ ולכל פתוחה $b \in V \subseteq X_2$,
	הקבוצה $U \times V$ לא ניתנת לכיסוי סופי על־ידי קבוצות מהכיסוי ${\{ W_{\omega} \}}_{\omega \in \Omega}$.
	נניח בשלילה שלכל $b$ יש $V_b$ ו־$U_b$ כך ש־$U_b \times V_b$ ניתנת לכיסוי סופי כזה.
	לכן $X_2 = \bigcup_{b \in X_2} V_b$ ו־$X_2$ קומפקטית ולכן ישנם $b_1, \ldots, b_k \in X_2$ כך ש־$X_2 = \bigcup_{i = 1}^k V_{b_i}$.
	נגדיר גם $U = \bigcap_{i = 1}^k U_{b_i} \subseteq X_1$, מתקיים $U \times X_2 = U \times \bigcup_{i = 1}^k V_{b_i} \subseteq \bigcup_{i = 1}^k U_{b_i} \times  V_{b_i}$,
	אך זו האחרונה ניתנת לכיסוי סופי, ולכן קיבלנו ש־$U \times X_2$ ניתנת לכיסוי סופי, ו־$a \in U$, וזו סתירה לבחירת $a$ מהטענה הקודמת.
\end{proof}

\section{שיעור 10 --- 5.5.2025}
\subsection{קומפקטיות --- משפט טיכונוף}
ניזכר בכמה הגדרות שמגיעות אליהו מתורת הקבוצות.
\begin{definition}[קבוצה סדורה]
	סדר על קבוצה, או קבוצה סדורה, הוא הזוג הסדור $(X, \le)$, כאשר $X$ קבוצה ו־$\le \subseteq X^2$ יחס דו־מקומי רפלקסיבי, אנטי־סימטרי וטרנזיטיבי.
\end{definition}
\begin{definition}[סדר טוב]
	סדר טוב הוא סדר קווי, כלומר יש יחס לפחות לאחד הכיוונים בין כל שני איברים בקבוצה,
	וכן שלכל תת־קבוצה של $X$ יש מינימלי ביחס הסדר.
\end{definition}
עיקרון הסדר הטוב מעיד שלכל קבוצה יש סדר טוב כלשהו שמוגדר עליה, והוא שקול לאקסיומת הבחירה.

בשיעור הקודם הוכחנו את משפט טיכונוף למקרה הסופי, עתה נראה את ההוכחה עבור המקרה הכללי.
נבחין כי משפט טיכונוף שקול לאקסיומת הבחירה (ולעיקרון הסדר הטוב), ולכן במהלך ההוכחה נהיה מחויבים להשתמש באקסיומה.
\begin{proof}
	נניח בשלילה ש־$Y = \prod_{\alpha \in I} X_{\alpha}$ אינה קומפקטית,
	כלומר יש כיסוי פתוח שאין לו תת־כיסוי סופי, נסמן את הכיסוי הזה $\Ff$.
	נבנה באינדוקציה לכל $\gamma \in I$ איזשהו $a_{\gamma} \in X_{\gamma}$ כך שאם $U$ בסיס טופולוגי ל־$Y$,
	המכילה תת־הקבוצה,
	\[
		\prod_{\alpha \le \gamma} \{ X_{\alpha} \} \times \left( \prod_{\gamma < \alpha} X_{\alpha}\right)
		\tag{1}
	\]
	או את,
	\[
		\prod_{\alpha < \gamma} \{ a_{\alpha} \} \times \prod_{\gamma \le \alpha} X_{\alpha}
		\tag{2}
	\]
	אז $U$ אינה ניתנת לכיסוי על־ידי אוסף סופי של $\Ff$.
	נבנה את $a_{\gamma}$ באינדוקציה טרנספיניטית (אינדוקציה על סודרים).
	נניח שהגדרנו את $a_{\alpha}$ לכל $\alpha < \gamma$,
	אז מתקיים שלכל בסיס שמכילה את $\prod_{\alpha < \gamma} \{ X_{\alpha} \} \times \prod_{\gamma \le \alpha} X_{\alpha}$ אינה ניתנת לכיסוי על־ידי תת־אוסף סופי מ־$\Ff$ (ונבהיר, זו הנחת האינדוקציה).
	זה הזמן להעיר שבעולם של סודרים, יהיו סודרים עוקבים, אלו שמתקבלים מהוספת $1$ לאיבר קיים כלשהו, ויש איברים גבוליים, עליהם נסתכל כאיברים אינסופיים, גבול בראי החיבור של איברים אחרים.
	כדי להתמודד עם הקושי הזה ולהשתמש באינדוקציה טרנספיניטית, מסתכלים על איברים גבוליים אלה או כאיברים מינימליים בקבוצה המתאימה להם, או כסופרימום של קבוצת האיברים הקטנים מהם, כך נוכל לאפיין את המספרים הללו משני הכיוונים.

	אנחנו רוצים לבחור $a_{\gamma} \in X_{\gamma}$ כך שיתקיים שכל קבוצת בסיס המקיימת את $(1)$,
	ניתנת לכיסוי על־ידי תת־אוסף סופי של $\Ff$ ואז מצאנו $a_{\gamma}$ כנדרש.
	או שלכל $a_{\gamma} \in X_{\gamma}$ יש קבוצת בסיס $W_{a_{\gamma}} \supseteq \prod_{\alpha \le \gamma} \{ a_{\gamma} \} \times \prod_{\alpha > \gamma} X_{\alpha}$ ויש ל־$W_{a_{\gamma}}$ תת־כיסוי סופי על ידי איברי $\Ff$.
	כלומר או שיש קבוצה כפי שרצינו או שמתקיימת שלילת הטענה.
	נבחין כי,
	\[
		a_{\gamma} \in \pi{\gamma}(W_{a_{\gamma}})
	\]
	קבוצה פתוחה, אז מתקיים,
	\[
		X_{\gamma} = \bigcup_{\alpha_{\gamma} \in X_{\gamma}} \pi_{\gamma}(W_{a_{\gamma}})
	\]
	אז יש תת־כיסוי סופי,
	\[
		X_{\gamma} = \bigcup_{i = 1}^k \pi_{\gamma}(W_{a_{\gamma}^i})
	\]
	לכן לקבוצה $\bigcup_{i = 1}^k W_{a_{\gamma}^i}$ יש תת־כיסוי סופי על־ידי איברי $\Ff$.
	נגדיר,
	\[
		V_i
		= \left( \prod_{j = 1}^k \pi_{\gamma^<}(W_{a_{\gamma}^i}) \right) \times \pi_{\gamma}(W_{a_{\gamma}^i}) \times \prod_{\alpha > \gamma} X_{\alpha}
	\]
	כאשר $\pi_{\gamma^<} : Y \to \prod_{\alpha < \gamma} X_{\alpha}$.
	אז,
	\[
		\bigcup_{i = 1}^k V_i
		= \left( \bigcap_{j = 1}^k \pi_{\gamma^<}(W_{a_{\gamma}^j}) \right) \times \left( \bigcup \pi_{\gamma}(W_{a_{\gamma}^i}) \right) \times \prod_{\alpha > \gamma} X_{\gamma}
	\]
	ולכן,
	\[
		\bigcup_{i = 1}^k V_i
		= \left( \bigcap_{j = 1}^k \pi_{\gamma^<}(W_{a_{\gamma}^i}) \right) \times \left(\prod_{\alpha \ge \gamma} X_{\alpha} \right)
	\]
	וקיבלנו סתירה כי הנחנו שהקבוצה הזו לא ניתנת לכיסוי סופי בעזרת איברי $\Ff$, ובכל זאת מצאנו כיסוי סופי כזה.

	לכן באינדוקציה טרנספיניטית מקבלים $a_{\gamma} \in X_{\gamma}$ לכל $\gamma \in I$ כך ש־$(1)$ מתקיים,
	\[
		Y = \prod_{\alpha \in I} X_{\alpha} \ni f = {( a_{\gamma} )}_{\gamma \in I}
	\]
	ולכן יש איבר בסיס $f \in W \subseteq L$ כאשר $W = \prod_{\alpha \in I} S_{\alpha}$, ולכמעט כל $\alpha$, $S_{\alpha} = X_{\alpha}$ וכל $S_{\alpha}$ פתוחה.
	יש $\gamma_0 \in I$ כך שלכל $\alpha > \gamma_0$ מתקיים $S_{\alpha} = X_{\alpha}$ ולכן קיבלנו איבר בסיס,
	\[
		\prod_{\alpha \le \gamma_0} \{ a_{\alpha} \} \times \prod_{\alpha > \gamma_0} X_{\alpha} \subseteq L
	\]
	וסתירה.
\end{proof}
אנו כבר יודעים כי אנו יכולים לראות קומפקטיות גם כך שאם $Z$ קומפקטי אז לכל $L$ אוסף סופי של קבוצות סגורות ב־$Z$ עם תכונת החיתוך הסופי, יש חיתוך לא טריוויאלי.
\begin{definition}[תכונת החיתוך הסופי]
	נאמר שלאוסף $L$ של תתי־קבוצות של קבוצה $Z$ יש את תכונת החיתוך הסופי, אם לכל תת־קבוצה סופית של $L$ יש חיתוך לא טריוויאלי.
\end{definition}
יהיה נוח להסתכל על אפיון אחר,
\begin{proposition}[שקילות לקומפקטיות]
	מרחב טופולוגי $Z$ הוא קומפקטי אם לכל אוסף $L$ של תתי־קבוצות $Z$ עם תכונת החיתוך הסופי, מתקיים ש־$\bigcap_{A \in L} \overline{A} \ne \emptyset$.
\end{proposition}
נעבור למספר טענות לקראת משפט שנראה בהמשך.
\begin{proposition}
	אם לאוסף קבוצות $L \subseteq \prod_{\alpha \in I} X_{\alpha}$ יש את תכונת החיתוך הסופי, אז גם ל־$L_{\beta} = \{ \pi_{\beta}(A) \mid A \in L \}$ יש את תכונת החיתוך הסופי ביחס ל־$X_{\beta}$.
\end{proposition}
אומנם לא נוכיח טענה זו, אבל נשים לב שהיא נובעת באופן ישיר מהאפיון הנוסף לקומפקטיות ושימוש בקבוצות הסגורות המושרות מהסגור שהגדרנו על $L$.
\begin{proposition}
	אם $L$ אוסף תתי־קבוצות של $Y$ המקיים את תכונת החיתוך הסופי, אז $L$ מוכל באוסף תתי־הקבוצות של $Y$ עם תכונת החיתוך הסופי, כך שהאוסף מקסימלי.
\end{proposition}
\begin{proof}
	נסתכל באוסף $\Omega$ של כל תתי־הקבוצות $L \subseteq C \subseteq \Pp(Y)$,
	$\Omega = \{ C_{\alpha} \}$, המקיימות את תכונת החיתוך הסופי,
	זו קבוצה לא ריקה סדורה חלקית על־ידי הכלה, ולכן מהלמה של צורן נובע שאכן יש איבר מקסימלי כזה.
\end{proof}
נראה טענה כללית נוספת ובעלת חשיבות.
\begin{proposition}\label{maximal_collection_of_closed_cap_proposition}
	אם $M$ אוסף מקסימלי של תתי־קבוצות של איזושהי קבוצה $R$ שיש לו את תכונת החיתוך הסופי,
	\begin{enumerate}
		\item לכל $m \in \NN$ ולכל $A_1, \ldots, A_m \in M$, גם $\bigcap_{i = 1}^n A_i \in M$
		\item אם $B \subseteq R$ ולכל $A \in M$, אם $A \cap B \ne \emptyset$ אז $B \in M$
	\end{enumerate}
\end{proposition}
גם כאן, ההוכחה היא ברורה ונובעת מהמקסימליות, ומושארת כתרגיל לקורא. \\
נעבור להוכחה נוספת למשפט טיכונוף, תוך שימוש בטענות שראינו זה עתה.
\begin{proof}
	$Y = \prod_{\alpha \in I} X_{\alpha}$ כאשר $X_{\alpha}$ קומפקטי לכל $\alpha \in I$, ונניח ש־$L \subseteq \Pp(Y)$ עם תכונת החיתוך הסופי.
	יש $L \subseteq M \subseteq \Pp(Y)$ מקסימלי עם תכונת החיתוך הסופי.

	לכל $\alpha$ נגדיר $M_{\alpha} = \{ \pi_{\alpha}(A) \mid A \in M \}$. \\
	ל־$M_{\alpha} \subseteq \Pp(X_{\alpha})$ יש את תכונת החיתוך הסופי.
	נובע ש־$X_{\alpha}$ קומפקטי ו־$\bigcap_{A \in M_{\alpha}} \overline{A} \ne \emptyset$.
	נבחר לכל $\alpha$ את $y_{\alpha} \in \bigcap_{A \in M_{\alpha}} \overline{A}$.

	אנו נוכיח כי הנקודה $y = {( y_{\alpha} )}_{\alpha \in I} \in \prod_{\alpha \in I} X_{\alpha} = Y$ מקיימת,
	\[
		y \in \bigcap_{B \in M} \overline{B}
		\subseteq \bigcap_{A \in L} \overline{A}
	\]

	תהי $B \in M$ ונראה ש־$y \in \overline{B}$.
	כלומר, כל פתוחה שמכילה את $y$ חותכת את $B$.
	מספיק להראות שכל קבוצת בסיס $y \in W \subseteq Y$ שמקיימת $W \cap B \ne \emptyset$. 
	כל קבוצת בסיס $W$ היא חיתוך של מספר סופי של קבוצות $V_{\beta} = Z_{\beta} \times \prod_{\alpha \ne \beta} X_{\alpha}$ עבור $Z_{\beta} \subseteq X_{\beta}$ פתוחה.
	מטענה\ \ref{maximal_collection_of_closed_cap_proposition} נובע שמספיק להוכיח שכל $V_{\beta}$ כזו כך ש־$y \in V_{\beta}$ ($y_{\beta} \in Z_{\beta}$) חותכת כל קבוצה $D$ באוסף $M$.
	אבל $M$ מקסימלי ולכן אם $V_{\beta} \cap D \ne \emptyset$ לכל $D \in M$ גם $V_{\beta} \in M$.
	נובע ש־$W \in M$ כי היא חיתוך של מספר סופי של $Z_{\beta}$, אך אלה ב־$M$.
	נסיק ש־$W$ חותך כל איבר ב־$M$.

	$y_{\beta} \in Z_{\beta}$ עבור $Z_{\beta}$ פתוחה, ולכן $y_{\beta} \in \bigcap_{A \in M_{\beta}} \overline{A}$ וכן $A = \pi_{\beta}(D), D \in M$.
	נובע שלכל $D \in M$, גם $y_{\beta} \in \pi_{\beta}(D)$, אז גם $Z_{\beta} \cap \pi_{\beta}(D) \ni y_{\beta}$, בפרט חיתוך זה לא ריק.
	לכן גם $y \in \pi_{\beta}^{-1}(Z_{\beta}) = V_{\beta} \cap D$, וחיתוך זה לא ריק, כפי שרצינו להראות.
\end{proof}

\section{שיעור 11 --- 6.5.2025}

בהינתן מרחב טופולוגי $X$ האם יש מרחב קומפקטי שמכיל את $X$?
נענה על שאלה זו בהרצאה הקרובה.
נתחיל בהגדרת הרעיון באופן פורמלי.
\begin{definition}[קומפקטיזציה]
	קומפקטיזציה $Y$ של $X$ היא מרחב קומפקטי $Y$ כך ש־$X \subseteq Y$ וגם $Y = \overline{X}$.
\end{definition}
ועתה משיש לנו טרמינולוגיה מתאימה, נוסיף הגדרה שתעזור לנו.
\begin{definition}[מרחב טופולוגי קומפקטי מקומית]
	מרחב טופולוגי $X$ נקרא קומפקטי מקומית אם לכל נקודה $x \in X$ יש סביבה קומפקטית,
	כלומר קיימת $x \in C \subseteq X$ קומפקטית וקיימת $x \in W \subseteq C$ פתוחה ב־$X$.
\end{definition}
\begin{example}
	נגדיר את $X = (0, 1)$, ונרצה למצוא קומפקטיזציה של $X$.
	יש שני מרחבים המהווים קומפקטיזציה ל־$X$, הם $S^1$ ו־$[0, 1]$.
\end{example}
\begin{theorem}[תנאי מרחב קומפקטי מקומית לקומפקטיות]
	אם $X$ מרחב טופולוגי קומפקטי מקומית והאוסדורף,
	אז המרחב $\hat{X} = Y = X \cup \{ \infty \}$ (עבור $\infty \notin X$ נקודה חדשה כלשהי),
	עם הטופולוגיה,
	\[
		\hat{\tau}
		= \tau \cup \{ Y \setminus K \mid K \subseteq X, K \text{ is compact} \}
	\]
	הוא מרחב קומפקטי והאוסדורף.
\end{theorem}
\begin{proof}
	נראה תחילה ש־$\hat{\tau}$ טופולוגיה, כלומר סגורה לאיחודים וסגורה לחיתוך סופי.
	נניח ש־${\{ V_{\alpha} \}}_{\alpha \in I} \subseteq \hat{\tau}$, אז קבוצה זו שקולה ל־$\{ V_{\alpha} \mid V_{\alpha} = T \setminus K_{\alpha} \} \cup \{ V_{\alpha} \mid V_{\alpha} \in \tau \}$,
	כאשר $K_{\alpha} \subseteq X$ קומפקטית. נסמן את זו הראשונה $\Omega$ ואת זו השנייה $\Lambda$.
	נבחין כי,
	\[
		\bigcup_{\alpha \in I} V_{\alpha}
		= \bigcup_{V \in \Lambda} V \cup \bigcup_{V \in \Omega} V
		= U \cup \bigcup_{U \in \Omega} U
	\]
	אבל מההגדרה קיימת $J \subseteq I$ כך שמתקיים,
	\[
		\bigcup_{\alpha \in J} (Y \setminus K_{\alpha})
		= Y \setminus \bigcap_{\alpha \in J} K_{\alpha}
	\]
	זאת שכן $X$ האוסדורף וכל $K_{\alpha}$ היא סגורה, לכן גם $\bigcap K_{\alpha}$ סגורה ולכן קומפקטית כמוכלת ב־$K_{\alpha_0}$.
	נובע ש־$V \cup \bigcup_{U \in \Omega} U = V \cup (Y \setminus K)$ עבור $K$ קומפקטית. \\
	$\hat{\tau}$ סגורה לחיתוכים סופיים, כנביעה מהשלמה לאיחודים.

	$\hat{\tau}$ משרה את $\tau$ על $X$,
	תת־קבוצה $A \subseteq X$ היא פתוחה בטופולוגיה המושרית על $X$ מ־$\hat{\tau}$ אם יש פתוחה $V \in \hat{\tau}$ כך ש־$A = X \cap V$.
	אם $V \in \tau$ אז בוודאי $A X \cap V = V \in \tau$.
	אם $V = Y \setminus K$ אז,
	\[
		A
		= X \cap V
		= X \cap (Y \setminus K)
		= X \setminus K
		\in \tau
	\]
	כי $K$ סגורה, זאת שכן $K$ קומפקטית ו־$X$ האוסדורף.

	$(Y, \hat{\tau})$ מרחב האוסדורף כי אם $y, y' \in Y$ ו־$y, y' \ne \infty$, כלומר $y, y' \in X$,
	אז קיימות פתוחות $U, W \in \tau \subseteq \hat{\tau}$ המפרידות את $y, y'$, כלומר $y \in U, y' \in W, U \cap W = \emptyset$.
	אם $y' = \infty$ ו־$y \in X$, $X$ קומפקטית מקומית ולכן יש $y \in W \subseteq C$ קומפקטית ב־$X$.
	לכן $y \in W, Y \setminus C$ והן פתוחות ב־$\hat{\tau}$.

	נראה ש־$(Y, \hat{\tau})$ קומפקטית.
	נניח ש־$\{ V_{\alpha} \} = L$ כיסוי פתוח של $Y$.
	יש $\infty \in V_{\alpha_0} = Y \setminus K$, ולכן $\{ V_{\alpha} \cap X \mid V_{\alpha} \in L \}$ כיסוי פתוח של $K \subseteq X$.
	$\{ V_{\alpha} \cap X \mid V_{\alpha} \in L \}$ כיסוי פתוח של $K$ שכן $K$ קומפקטית,
	לכן יש $\alpha_1, \ldots, \alpha_N$ כך שמתקיים,
	\[
		K \subseteq \bigcup_{i = 1}^N (V_{\alpha_i} \cap X)
	\]
	ונסיק ש־$Y = \bigcup_{i = 1}^N V_{\alpha_i}$.
\end{proof}
מצאנו קומפקטיזציה על־ידי הוספת נקודה יחידה.
\begin{remark}
	אם $X$ אינו קומפקטי אז $Y = \overline{X}$,
	אחרת $\overline{X} = X$ בלבד, ו־$\infty$ נקודה מבודדת.
\end{remark}
המטרה שלנו עתה היא להראות שאם $X$ מרחב האוסדורף קומפקטי מקומי אז יש מרחב קומפקטי, נסמן $\check{X}$, ב־$\check{X}$ כך ש־$z : X \hookrightarrow \check{X}$,
ו־$z : X \to z(X)$ הומיאומורפיזם, ו־$\overline{z(X)} = \check{X}$,
וכן שכל פונקציה רציפה וחסומה של $X$ ניתנת להרחבה לפונקציה רציפה של $\check{X}$.
נגדיר $F = C(X, [0, 1])$, אוסף כל הפונקציות הרציפות מ־$X$ ל־$[0, 1]$.
נגדיר $z : X \to {[0, 1]}^F$, עם טופולוגיית המכפלה, אז נקבל שלכל $x \in X$ ולכל $f \in F$, מתקיים $z(x)(f) = f(x)$.
הסגור של התמונה $z(X)$ תסומן ב־$\check{X}$ וזוהי קומפקטיזציה של $X$.

\section{שיעור 12 --- 12.5.2025}
\subsection{קומפקטיזציה}
נמשיך עם המשפט שדנו בו בשיעור הקודם.
\begin{theorem}[סטון־צ'ק]
	אם $X$ מרחב טופולוגי האוסדורף קומפקטי מקומית אז קיים מרחב טופולוגי קומפקטי האוסדורף $Y$ כך שקיים שיכון $f : X \hookrightarrow Y$ כך ש־$\overline{f(X)} = Y$,
	וכל פונקציה רציפה וחסומה על $X$ ניתנת להרחבה לפונקציה רציפה על $Y$ כך שההרחבה יחידה.
\end{theorem}
\begin{proof}
	נבחן את $F = C(X, [0, 1])$ אוסף הפונקציות הרציפות $X \to [0, 1]$.
	נתבונן במרחב המכפלה ${[0, 1]}^F$.
	ממשפט טיכונוף זהו מרחב קומפקטי וכמו־כן הוא האוסדורף.
	נגדיר העתקה $\iota : X \to {[0, 1]}^F$ על־ידי $\iota(x)(f) = f(x)$ לכל $f \in F, x \in X$.
	נגדיר גם $Y = \overline{\iota(X)}$. \\
	$Y$ קומפקטית כי היא תת־קבוצה סגורה של מרחב קומפקטי, וכן $Y$ היא האוסדורף כתת־מרחב של מרחב האוסדורף. \\
	$X \hookrightarrow Y$ שיכון אם ורק אם היא העתקה חד־חד ערכית כך ש־$X \to \iota(X)$ הומיאומורפיזם, אז נבדוק. \\
	עבור חד־חד ערכיות תהינה $x_1, x_2 \in X$ כך ש־$x_1 \ne x_2$.
	אנו טוענים כי קיימת $f \in F$ כך ש־$f(x_1) \ne f(x_2)$.
	בהינתן טענה זו נסיק ש־$\iota(x_1)(f) \ne \iota(x_2)(f)$ ולכן $\iota(x_1) \ne \iota(x_2)$.

	יש $U_1, U_2$ פתוחות ב־$X$ כך ש־$x_1 \in U_1, x_2 \in U_2$ וגם $U_1 \cap U_2 = \emptyset$ מהאוסדורף.
	ניזכר בלמה של אוריסון, עבור מרחב קומפקטי והאוסדורף.
	נבנה פונקציה רציפה על $C_1 \cup C_2$ קבוצות סגורות קומפקטיות סביב $U_1, U_2$, כך שמהלמה של אוריסון יתקיים $f(C_1) = 0, f(C_2) = 1$.

	נותר להראות ש־$\iota : X \to \iota(X)$ היא הומיאומורפיזם.
	כלומר צריך להראות שכל קבוצה פתוחה $W \subseteq X$ מקיימת ש־$\iota(W) \subseteq \iota(X)$ היא פתוחה, וגם להראות ש־$\iota$ רציפה. \\
	נניח ש־$W \subseteq X$ פתוחה ולא ריקה, אנו רוצים להראות ש־$\iota(W)$ פתוחה.
	תהי $x \in W$, אז $\iota(x) \in \iota(W)$.
	נובע מ־$x \in W$ שיש $f \in F$ כך ש־$f(x) = 0$ וכן ש־$f \restriction X \setminus U = 1$ עבור $U \supseteq W$.

	נמשיך ונטען כי $V = \pi_f^{-1}([0, 1])$ היא פתוחה כך ש־$\iota(x) \in V \cap \iota(X) \subseteq \iota(W)$.
	נבהיר כי $\pi_f : {[0, 1]}^F \to [0, 1]$.

	תהי $g : X \to \RR$ רציפה ונרצה להרחיבה ל־$Y$.
	אנו יודעים כי $X$ קומפקטית ולכן $g$ חסומה ונסמן $M > 0$ כך ש־$|g(x)| \le M$ לכל $x \in X$.
	נגדיר גם $f(x) = \frac{1}{M} g(x) + \frac{1}{2}$.
	נתבונן בפונקציה $s : {[0, 1]}^F \to \RR$ המוגדרת על־ידי $s(h) = M(h(f) - \frac{1}{2})$, אז קל לראות לכל $x \in X$ מתקיים $s(\iota(x)) = g(x)$ ולכן $s \restriction Y$ הרחבה רציפה של $g$.

	אם $\tilde{F} = \{ f : X \to \RR \mid f \text{ is bounded and continuous}$ אז נבחן את $\prod_{f \in \tilde{F}} [a_f, b_f]$ עבור $a_f = \inf\{ f(x) \mid x \in X \}, b_f = \sup\{ f(x) \mid x \in X \}$.
\end{proof}
\begin{notation}
	נסמן את המרחב $Y$ שבנינו ב־$\beta(X)$.
\end{notation}
\begin{theorem}[הרחבה רציפה לפונקציות במרחבים קומפקטיים מקומית]
	יהי $X$ מרחב קומפקטי מקומית האוסדורף, $C$ קומפקטי והאוסדורף.
	אז כל פונקציה רציפה $\varphi : X \to C$ ניתנת להרחבה רציפה $\hat{\varphi} : \beta(X) \to C$.
\end{theorem}
\begin{proof}
	קיימת קבוצת אינדקסים $J$ כך שיש שיכון $X \to C \hookrightarrow {[0, 1]}^J$.
	לכל $j \in J$ יש פונקציה $g_j = \pi_j \circ \varphi : X \to [0, 1]$.
	אז ניתן להרחיב את $g_j$ ל־$\hat{g}_j : \beta(X) \to [0, 1]$ באופן רציף.
	נסמן $\tilde{g} : \beta(X) \to {[0, 1]}^J$ הפונקציה הרציפה כך ש־$\tilde{g} \restriction X = g$.
	אז $\tilde{g}(\beta(X)) \subseteq C$ שכן $\beta(X) = \overline{X}$ כאשר בוחנים את $X$ כתת־קבוצה של $\beta(X)$.
	אנו מסיקים ש־$\tilde{g}(\overline{X}) \subseteq \overline{\tilde{g}(X)} \subseteq \overline{C} = C$.
\end{proof}
\begin{proposition}
	נניח ש־$X$ מרחב האוסדורף קומפקטי מקומית ו־$Y_1, Y_2$ קומפקטיות האוסדורף עם שיכונים $X \hookrightarrow Y_i$ צפופים כך שכל פונקציה רציפה וחסומה מ־$X$ ל־$\RR$ ניתנת להרחבה רציפה של $Y_i$,
	אז $Y_1, Y_2$ הומיאומורפים.
\end{proposition}

\begin{definition}[קבוצה דלילה]
	יהי $X$ מרחב טופולוגי.
	קבוצה $A \subseteq X$ תיקרא \textbf{דלילה} אם ${(\overline{A})}^\circ = \emptyset$, כלומר לסגור שלה יש פנים ריק.
\end{definition}
\begin{example}
	$\ZZ \subseteq \RR$ (קבוצות קנטור הסטנדרטיות ב־$\RR$ הן דלילות). \\
	מהצד השני $\QQ \subseteq \RR$ לא דלילה.
\end{example}
\begin{definition}[קטגוריה ראשונה ושנייה]
	קבוצה תיקרא מהקטגוריה הראשונה אם היא איחוד בן־מניה של קבוצות דלילות, \\
	אחרת נאמר שהיא מהקטגוריה השנייה.
\end{definition}
\begin{theorem}[בייר]
	יהי $X$ מרחב קומפקטי האוסדורף או מרחב מטרי שלם, \\
	אז לכל אוסף בן־מניה ${\{ A_n \}}_{n = 1}^\infty$ של קבוצות דלילות מתקיים שלאיחוד $\bigcup_{n = 1}^\infty A_n$ יש פנים ריק.
\end{theorem}
\begin{remark}
	המשפט שקול לטענה שאם ${\{ U_n \}}_{n = 1}^\infty$ הן קבוצות פתוחות וצפופות אז $\bigcap_{n = 1}^\infty U_n$ צפופה.
\end{remark}
\begin{proof}
	המשפט הוא למעשה שני משפטים על שני תנאים שונים, אנו נוכיח את המקרה של מרחב קומפקטי האוסדורף, והמקרה השני מושאר כתרגיל ומשתמש בעקרונות דומים. \\
	נניח ש־$X$ מרחב קומפקטי האוסדורף ונניח ש־${\{ A_n \}}_{n = 1}^\infty \subseteq X$ קבוצות דלילות.
	בלי הגבלת הכלליות נניח ש־$A_n$ סגורות, אחרת נבחר את $\overline{A}_n$ לכל $n \in \NN$.
	נוכיח ש־$\bigcup_{n = 1}^\infty A_n$ בעלת פנים ריק.
	תהי $U \subseteq X$ קבוצה פתוחה לא ריקה ונראה ש־$U \not\subseteq \bigcup_{n = 1}^\infty A_n$.
	נבנה סדרת קבוצות פתוחות ${\{ V_n \}}_{n = 1}^\infty \subseteq X$ באופן הבא:
	$X$ קומפקטי והאוסדורף ולכן נורמלי, $A_1$ דלילה וסגורה ונגדיר $U_1 = U \cap (X \setminus A_1)$ קבוצה פתוחה.
	נובע ש־$a_1 \in V_1 \subseteq \overline{V}_1 \subseteq U_1$ עבור $a_1 \in X$ ו־$V_1$ פתוחה.
	אז $\overline{V}_1 \cap A_1 = \emptyset$ ואנו יודעים כי $A_2$ דלילה, אז $U_2 = V_1 \cap (X \setminus A_2) \ne \emptyset$ פתוחה.
	אז קיימת $a_2 \in V_2$ נקודה וקבוצה פתוחה כך ש־$a_2 \in V_2 \subseteq \overline{V}_2 \subseteq V_1$.
	נבחין כי $\overline{V}_2 \cap A_2 = \emptyset$.
	נמשיך כך ונבנה סדרת קבוצות פתוחות,
	\[
		a_n \in V_n \subseteq \overline{V}_n \subseteq U_{n - 1}
	\]
	ו־$\overline{V}_n \cap A_n = \emptyset$.
	האוסף $\{ \overline{V}_n \}$ הוא אוסף קבוצות סגורות המקיימות שכל מספר סופי מביניהן לא ריק ולכן,
	\[
		\bigcap_{n = 1}^\infty \overline{V}_n \ne \emptyset
	\]
	ויהי $b \in \bigcap \overline{V}_n$.
	נסיק ש־$b \notin \bigcup_{n = 1}^\infty A_n$, אבל $b \in U$ ונקבל שאכן $U \not\subseteq \bigcup_{n = 1}^\infty A_n$.
\end{proof}
\begin{definition}[מרחב מושלם]
	מרחב $X$ נקרא מושלם אם כל נקודה $x \in X$ היא נקודת הצטברות $X \setminus \{ x \}$.
\end{definition}
\begin{conclusion}
	נניח ש־$X$ מרחב קומפקטי האוסדורף מושלם, אז $X$ לא בן־מניה.
\end{conclusion}
\begin{definition}[תכונת בייר]
	נאמר שמרחב $X$ הוא מרחב בייר אם מתקיים שלאיחוד בן־מניה של קבוצות דלילות אין פנים.
\end{definition}
\begin{exercise}
	נניח ש־$X$ מרחב בייר ו־$Y$ מרחב מטרי, ונניח ש־$f_n : X \to Y$ היא סדרת פונקציות רציפות על $X$ כך שלכל $x_0 \in X$ מתקיים ${\{ f_n(x_0) \}}_{n = 1}^\infty$ מתכנסת ל־$f(x_0)$,
	אז $f$ רציפה בקבוצה צפופה של נקודות.
\end{exercise}
נראה רמז לתרגיל;
יהי $\epsilon > 0$, ונגדיר $B_n(\epsilon) = \{ x \in X \mid \forall m, n \in \NN,\ d(f_n(x), f_m(x)) \le \epsilon \}$
אז לכל $\epsilon > 0$ מתקיים $X = \bigcup_{N = 1}^\infty B_N(\epsilon)$, זוהי קבוצה סגורה עם פנים, ולכן לאיזושהי קבוצה באיחוד אמור להיות פנים.

\section{שיעור 13 --- 13.5.2025}

\subsection{השלמות לקומפקטיזציה}
$X$ מרחב בייר (לאיחוד בן־מניה של דלילות יש פנים ריק).
נניח ש־$Y$ מרחב מטרי וגם ש־$f_n : X \to Y$ רציפות ויש $f : X \to Y$ כלשהי, ונניח שלכל $x \in X$ גם $\lim_{n \to \infty} f_n(x) = f(x)$.
אז $\{ x \in X \mid f \text{ is continuous at } x \}$ צפופה ב־$X$.
\begin{proof}
	תת־קבוצה פתוחה של מרחב בייר היא מרחב בייר (ביחס לטופולוגיה המושרית עליה),
	נגדיר גם,
	\[
		\forall \epsilon > 0, N \in \NN,\ 
		B_N(\epsilon) = \{ x \in X \mid \forall n, m \ge N,\ |f_n(x) - f_m(x)| \le \epsilon \}
	\]
	אז $\bigcup_{N = 1}^\infty B_N(\epsilon) = X$ וכן נובע ש־$U(\epsilon) = \bigcup_{N = 1}^\infty B_N^\circ$ פתוחה וצפופה.
	$f$ רציפה ב־$\bigcap_{k = 1}^\infty U(\frac{1}{k})$ צפופה כי $X$ מרחב בייר. \\
	סוף ההוכחה מושאר כתרגיל.
\end{proof}

\subsection{משחק מזור}
עתה נדון במשחק מזור (Mazur).
\begin{definition}[משחק מזור]
	אנו מניחים כי יש לנו שני שחקנים, א' וב'.
	נניח גם כי קיימת $A \subseteq [0, 1] = I_0$.
	כל שחקן בתורו בוחר קטע סגור $I_n \subseteq I_{n - 1}$.
	שחקן א' יבחר את $I_1 \subseteq I_0$ וב' יבחר $I_1 \subseteq I_2$ וכן הלאה.
	נגדיר ששחקן א' מנצח אם ורק אם $A \cap \bigcap_{n = 1}^\infty I_n \ne \emptyset$.
\end{definition}
\begin{exercise}
	האם יש אסטרטגיית ניצחון?
	אם יש, מה התנאים שלה ולמי?
\end{exercise}

\subsection{מבוא לטופולוגיה אלגברית}
נחזור עתה למרחבי מנייה.
נניח ש־$X$ מרחב טופולוגי ו־$R \subseteq X \times X$ יחס שקילות על $X$.
\begin{notation}
	נסמן מחלקות שקילות של $R$ ב־$X$ על־ידי,
	\[
		[x] = {[x]}_R = \{ y \in X \mid (x, y) \in R \}
	\]
	וכן נסמן,
	\[
		X / R
		= \{ [x] \mid x \in X \}
	\]
	וכן $\pi : X \to X / R$ על־ידי $\pi(x) = [x]$.
\end{notation}
אנו רוצים למצאו טופולוגיה על $X / R$ החזקה ביותר כך ש־$\pi$ היא רציפה.
נגדיר $L \subseteq X / R$ להיות פתוחה אם ורק אם $\pi^{-1}(L) \subseteq X$ פתוחה.
באופן דומה נוכל להגדיר בצורה כזו טופולוגיה בהינתן פונקציה $f : X \to Y$ שהיא על $Y$.
\begin{example}
	בהינתן $X = [0, 1]$ נוכל להגדיר $R = \{ (0, 1) \}$ ונקבל ש־$[0] = \{0, 1\}$ ובהתאם $X / R$ יתנהג למעשה כמו מעגל.
\end{example}
\begin{example}
	עבור $X = \RR$ נוכל להגדיר $x \sim x + n$ לכל $n \in \ZZ$, ובמקרה זה נקבל שקילות למעגל שוב.
	נהוג לסמן גם $\RR / \ZZ$ עבור $\RR / \sim$.
\end{example}
\begin{definition}[אוקלידיות מקומית]
	מרחב טופולוגי $X$ יקרא אוקלידי מקומית (ממימד $n$) אם לכל $x \in X$ יש סביבה פתוחה $x \in U \subseteq X$ כך שמתקיים,
	\begin{itemize}
		\item $U$ הומיאומורפית לכדור היחידה הפתוח ב־$\RR^n$
		\item $U$ הומיאומורפית ל־$\RR^n$
		\item $U$ הומיאומורפית לקבוצה פתוחה ב־$\RR^n$
	\end{itemize}
	כאשר התנאים הללו שקולים.
\end{definition}
\begin{definition}[יריעה טופולוגית]
	מרחב טופולוגי $X$ יקרא יריעה טופולוגית (ממימד $n$) אם מתקיימות התכונות הבאות,
	\begin{enumerate}
		\item $X$ אוקלידי מקומית ממימד $n$
		\item $X$ האוסדורף
		\item $X$ מרחב מנייה שנייה
	\end{enumerate}
\end{definition}
נראה מספר דוגמות ליריעות.
\begin{example}
	כל תת־קבוצה פתוחה של $\RR^n$ היא יריעה טופולוגית ממימד $n$.
\end{example}
\begin{example}
	נבחין כי $\TT^2 = \RR^2 / \ZZ^2$, הוא למעשה שפת הריבוע, היא לא יריעה.
\end{example}
\begin{example}
	בקבוק קליין הוא יריעה.
\end{example}
\begin{example}
	גרף של פונקציה רציפה $f : U \to \RR$ עבור $U \subseteq \RR^n$, הוא יריעה, כלומר,
	\[
		\{ (x, f(x)) \in \RR^{n + 1} \mid x \in U \}
	\]
	היא יריעה טופולוגית.
\end{example}
נבחין כי עבור $n = 1$ יש רק סוג אחד של יריעה קומפקטית, המעגל.
עבור $n = 2$ יש לנו את הספירה, את הטורוס, מתומן הקסם ואת בקבוק קליין.
בהרצאות הבאות ניכנס לתחום הטופולוגיה האלגברית, ונפתח כלים לאפיון של מרחבים כאלה.

\section{שיעור 14 --- 19.5.2025}
\subsection{מבוא לטופולוגיה אלגברית --- החבורה היסודית}
המטרה שלנו היא להיות מסוגלים לענות על השאלה הבאה,
\begin{exercise}
	איך מבדילים בין מרחבים טופולוגיים?
	כלומר, נניח שנתונים $X, Y$ מרחבים טופולוגיים, ואנו רוצים לענות על השאלה האם הם הומיאומורפיים.
\end{exercise}
בעולם של אלגברה לינארית לדוגמה אפיינו בצורה מדויקת שקילות של מרחבים לינאריים, פה המצב מורכב ומסועף יותר, ונצטרך להבין לעומק האובייקטים שאנו דנים בהם כדי שנוכל לאפיין אותם.
\begin{exercise}
	האם $S^2$ הספירה הדו־מימדית ו־$T^2$ הטורוס הדו־מימדי הם הומיאומורפיים?
\end{exercise}
\begin{solution}
	כל מסילה סגורה ב־$S^2$ ניתן לכווץ לנקודה, אבל לא כל מסילה סגורה בטורוס ניתן לכווץ באותו האופן.
\end{solution}
נתחיל בהגדרות.
\begin{definition}[הומוטופיה]
	יהיו $X, Y$ מרחבים טופולוגיים ו־$f_0, f_1 : X \to Y$ העתקות רציפות.
	אז הומוטופיה מ־$f_0$ ל־$f_1$ היא העתקה רציפה $H : [0, 1] \times X \to Y$ כך שמתקיים,
	\[
		\forall x \in X,\ 
		H(0, x) = f_0(x),
		H(1, x) = f_1(x)
	\]
	לפעמים נכתוב גם $H_s(x) = H(s, x)$.
\end{definition}
\begin{example}
	בהרצאות קודמות הגדרנו שמרחב כוויץ אם יש הומוטופיה מהעתקת הזהות $f_0(x) = x$ להעתקה קבועה $f_1(x) = x_0$ עבור $x_0 \in X$ כלשהו.
\end{example}
\begin{notation}
	הגדרנו מסילה על־ידי $\gamma : [0, 1] \to X$, ובהינתן $p, q \in X$ אז נסמן,
	\[
		\Omega(X, p, q) = \{ \gamma : [0, 1] \to X \mid \gamma(0) = p, \gamma(1) = q, \gamma \text{ is continuous path} \}
	\]
	מרחב כל המסילות הרציפות מ־$p$ ל־$q$.
\end{notation}
\begin{definition}[מסילות הומוטופיות]
	תהינה שתי מסילות $\gamma_0, \gamma_1 \in \Omega(X, p, q)$ הן הומוטופיות אם יש הומוטופיה ביניהן,
	כלומר אם קיימת $H : [0, 1] \times [0, 1] \to X$ כך ש־$H$ רציפה ולכל $t \in [0, 1]$,
	\[
		H(0, t) = \gamma_0(t),
		\quad
		H(1, t) = \gamma_1(t),
		\quad
		\forall s \in [0, 1],\ H(s, 0) = p = \gamma_0(0) = \gamma_1(0), H(s, 1) = q = \gamma_0(1) = \gamma_1(1)
	\]
\end{definition}
הרעיון הוא שיש לנו דרך ''להעביר'' כל מסילה בין הנקודות באופן רציף מאחת לשנייה.
הרעיון לא זר למי שלמד אנליזה על יריעות, שם השתמשנו בכלי דומה לזה כדי לאפיין קשר בין מסילות, ראינו שאם כל שתי מסילות הומוטופיות בשדה משמר מקומית, אז הוא משמר.
\begin{proposition}
	היחס $\sim$ על $\Omega(X, p, q)$, המוגדר על־ידי $\gamma_0 \sim \gamma_1$ אם ורק אם קיימת הומוטופיה ביניהן,
	הוא יחס שקילות.
\end{proposition}
\begin{proof}
	\textbf{רפלקסיביות},
	בהינתן $\gamma \in \Omega(X, p, q)$ נבחר $H(s, t) = \gamma(t)$.

	\textbf{סימטריה},
	נניח ש־$\gamma_0 \sim \gamma_1$, ותהי $H$ הומוטופיה המעידה על כך.
	נגדיר $G(s, t) = H(1 - s, t)$, אז זו הומוטופיה המעידה על $\gamma_1 \sim \gamma_0$.

	\textbf{טרנזיטיביות},
	נניח ש־$\gamma_0 \sim \gamma_1, \gamma_1 \sim \gamma_2$, ונניח ש־$H, G$ מעידות על כך בהתאמה.
	נרצה להגדיר הומוטופיה מ־$\gamma_0$ ל־$\gamma_2$, נגדיר על־ידי,
	\[
		F(s, t)
		= \begin{cases}
			H(2s, t) & s \in [0, \frac{1}{2}] \\
			G(2s - 1, t) & \text{otherwise}
		\end{cases}
	\]
	עלינו לבדוק שאכן $F$ מוגדרת היטב, כלומר לבדוק את המקרה $s = \frac{1}{2}$, ולהראות ש־$F$ רציפה.
	נובע ש־$F$ הומוטופיה מלמת ההדבקה, אותה נגדיר ונוכיח עתה.
\end{proof}
\begin{lemma}[למת הדבקה]
	נניח ש־$Y$ מרחב טופולוגי ונניח ש־$A \cup B = Y$ עבור קבוצות סגורות.
	תהי $f : Y \to Z$ פונקציה כך ש־$f \restriction A$ רציפה וכן $f \restriction B$ רציפה.
	אז נובע ש־$f$ רציפה.
\end{lemma}
\begin{proof}
	צריך לבדוק שלכל סגורה $C \subseteq Z$ מתקיים $f^{-1}(C)$ סגורה גם כן.
	אבל,
	\[
		f^{-1}(C)
		(f^{-1}(C) \cap A) \cup (f^{-1}(C) \cap B)
		= {(f \restriction A)}^{-1}(C) \cup {(f \restriction B)}^{-1}(C)
	\]
	ולכן הטענה נובעת ישירות.
\end{proof}
\begin{definition}[החבורה היסודית של מרחב טופולוגי]
	באנגלית Fundamental group,
	נסמן $\pi_1(X, p, q) = \Omega(X, p, q) / \sim$. \\
	אם $p = q$ אז נסמן גם $\Omega(X, p) = \Omega(x, p, p)$ ובהתאם גם $\pi_1(X, p) = \Omega(x, p) / \sim$. \\
	נגדיר $\pi_1(X, p)$ החבורה היסודית של המרחב המנוקב $(X, p)$.
\end{definition}
נשים לב כי זוהי הגדרה אפריורית, כלומר לא הראינו בשום צורה שזוהי אכן חבורה, וכרגע זהו רק שם.
אנו רוצים עתה להראות שזו אכן חבורה ושהגדרה זו תלויה בטופולוגיה שלנו בלבד.
\begin{example}
	נתבונן במרחב $\RR^2$ וב־$p, q \in \RR^2$.
	כל זוג מסילות מ־$p$ ל־$q$ הן הומוטופיות, זאת שכן לכל $\gamma_0, \gamma_1$ נוכל להגדיר,
	\[
		H(s, t)
		= \gamma_1(t) \cdot s + \gamma_0(t) \cdot (1 - s)
	\]
	נקבל ש־$H : [0, 1] \times [0, 1] \to \RR^2$,
	זוהי אכן הומוטופיה כהרכבת פונקציות רציפות, ולכן $p \sim q$.
	נסיק גם ש־$\pi_1(\RR^2, p) = \RR^2$.
\end{example}
נראה דוגמה למרחב בו לא כל המסילות הומוטופיות.
\begin{example}
	נבחן הפעם את $X = \CC \setminus \{ 0 \}$.
	נגדיר $\gamma_0(t) = e^{\pi i t}$ ו־$\gamma_1(t) = 1 + e^{\pi i - \pi i t}$,
	ולמרות שאין לנו עדיין את היכולת להוכיח זאת, אלו הן מסילות לא הומוטופיות.
	מי שלמד את הקורס פונקציות מרוכבות כבר יודע שמהגרסה המורחבת למשפט האינטגרל של קושי נובע שהאינטגרל המסילתי של שתי המסילות שונה, ובהמשך נראה טיעון שדומה לטיעון זה עבור הוכחת אי־השקילות.
\end{example}
ניזכר בהגדרת החבורה,
\begin{definition}[חבורה]
	חבורה $(G, \cdot)$ היא זוג הכולל קבוצה ופעולה $\cdot : G^2 \to G$ כך שמתקיים,
	\begin{enumerate}
		\item \textbf{אסוציאטיביות}, לכל $a, b, c \in G$ מתקיים $a \cdot (b \cdot c) = (a \cdot b) \cdot c$
		\item \textbf{קיום איבר ניטרלי}, קיים $e \in G$ כך ש־$e \cdot g = g \cdot e = g$ לכל $g \in G$
		\item \textbf{קיום הופכי}, לכל $g \in G$ קיים $h \in G$ כך ש־$g \cdot h = h \cdot g = e$ עבור $e$ האיבר הנייטרלי
	\end{enumerate}
\end{definition}
\begin{remark}
	האיבר ההופכי של $g \in G$ הוא יחיד.
\end{remark}
\begin{definition}[שרשור של מסילות]
	נניח ש־$X$ מרחב טופולוגי ו־$a, b, c \in X$.
	נניח גם ש־$\alpha \in \Omega(X, a, b), \beta \in \Omega(X, b, c)$.
	אז נגדיר מסילה המסומנת $\alpha * \beta$ כך ש־$\alpha * \beta \in \Omega(X, a, c)$ כך ש־$\alpha * \beta : [0, 1] \to X$ מוגדרת על־ידי,
	\[
		(\alpha * \beta)(t)
		= \begin{cases}
			\alpha(2t) & 0 \le t \le \frac{1}{2} \\
			\beta(st - 1) & \text{otherwise}
		\end{cases}
	\]
	נבחין כי $\alpha * \beta$ מוגדרת היטב מלמת ההדבקה.
\end{definition}
\begin{remark}
	נניח ש־$x_0, x_1, x_2, x_3 \in X$ ותהינה $\alpha \in \Omega(X, x_0, x_1), \beta \in \Omega(X, x_1, x_2), \gamma \in \Omega(X, x_2, x_3)$.
	אז $\alpha * (\beta * \gamma), (\alpha * \beta) * \gamma \in \Omega(X, x_0, x_3)$ מסילות לאו דווקה שוות.
\end{remark}
\begin{proposition}
	נניח ש־$\alpha \sim \alpha' \in \Omega(X, a, b)$ וכן ש־$\beta \sim \beta' \in \Omega(X, b, c)$.
	אז $\alpha * \beta \sim \alpha' * \beta'$.
\end{proposition}
את ההוכחה לא נראה, אבל היא נובעת ישירות מהגדרת מחלקות השקילות.
\begin{conclusion}
	אפשר להגדיר את פעולת השרשור על מחלקות הומוטופיה, כלומר הפעולה מוגדרת היטב על מחלקות שקילות. \\
	נסמן במקרה זה $[\alpha] * [\beta] = [\alpha * \beta]$ עבור נציגים כלשהם.
\end{conclusion}
\begin{proposition}
	לכל $[\alpha] \in \pi_1(X, x_0, x_1)$, $[\beta] \in \pi(X, x_1, x_2)$ ו־$\gamma \in \pi_1(X, x_2, x_3)$ מתקיים,
	\[
		([\alpha] * [\beta]) * [\gamma]
		= [\alpha] * ([\beta] * [\gamma])
	\]
\end{proposition}
\begin{definition}[רפרמטריזציה של מסילה]
	תהי $\alpha : [0, 1] \to X$ מסילה.
	מסילה $\beta : [0, 1] \to X$ תיקרא רפרמטריזציה של $\alpha$ אם קיימת פונקציה רציפה $\psi : [0, 1] \to [0, 1]$ כך ש־$\psi(0) = 0, \psi(1) = 1$ ו־$\beta = \alpha \circ \psi$.
\end{definition}
\begin{proposition}
	אם $\beta$ רפרמטריזציה של $\alpha$ אז $\alpha \sim \beta$ (שקול ל־$[\alpha] = [\beta]$).
\end{proposition}
\begin{proof}
	$\psi$ היא רפרמטריזציה על המכפלה $\iota : [0, 1] \to [0, 1]$, כאשר $\iota(t) = t$, ומתקיים $\psi = \iota \circ \psi$.
	אז $\psi \in \Omega([0, 1], 0, 1)$.
	כל שתי מסילות עם אותן נקודות קצה בקבוצה קמורה הן הומוטופיה.
	אם כך $\alpha, \beta : [0, 1] \to A$ עבור $A$ קבוצה קמורה, ונגדיר,
	\[
		H(s, t)
		= s \beta(t) + (1 - s) \alpha(t) \in A
	\]
	אז מבדיקה ישירה נסיק שאכן $H$ היא הומוטופיה ולכן מעידה על $\alpha \sim \beta$.
\end{proof}
\begin{conclusion}
	$\pi_1(X, x_0)$ היא חבורה יחד עם פעולת השרשור.
\end{conclusion}
\begin{proof}
	הפעולה $*$ שהגדרנו על $\pi_1(X, x_0)$ מקיימת שלכל $u, v, w \in \pi_1(X, x_0)$ מתקיים $u (v w) = (u v) w$, כלומר אסוציאטיביות. \\
	המסילה הקבועה $c_{x_0}(t) = x_0$ לכל $t \in [0, 1]$ היא איבר ניטרלי ביחס לפעולה, כלומר נגדיר $e = [c_{x_0}]$ ונבחין כי לכל $\gamma \in \Omega(X, x_0)$ מתקיים $e * \gamma = \gamma * e = \gamma$. \\
	בהינתן מסילה $\alpha \in \Omega(X, x_0, x_1)$ נגדיר מסילה $\overline{\alpha} \in \Omega(\Omega, x_1, x_0)$ על־ידי $\overline{\alpha}(t) = \alpha(1 - t)$, ולכן $\alpha * \overline{\alpha} = c_{x_0}$.
	כלומר לכל $u \in \pi_1(X, x_0)$ קיים $v \in \pi_1(X, x_0)$ כך ש־$u * v = v * u = e$.
\end{proof}
נסיים בטענה המושארת כתרגיל לקורא.
\begin{proposition}
	אם $X$ מרחב כוויץ אז $\pi_1(X, x_0)$ היא החבורה הטריוויאלית.
\end{proposition}

\section{שיעור 15 --- 20.5.2025}
\subsection{החבורה היסודית}
\begin{proposition}
	נניח ש־$X$ מרחב טופולוגי, ו־$x_0, x_1 \in X$ נקודות כך שיש מסילה $\alpha \in \Omega(X, x_0, x_1)$,
	אז החבורות $\pi_1(X, x_0)$ ו־$\pi_1(X, x_1)$ איזומורפיות.
\end{proposition}
\begin{proof}
	נגדיר העתקה $f_{\alpha} : \Omega(X, x_0) \to \Omega(X, x_1)$ על־ידי,
	\[
		f_{\alpha}(\gamma)
		= \overline{\alpha} * \gamma * \alpha 
	\]
	לכל $\gamma \in \Omega(X, x_0)$ וכאשר $\overline{\alpha}$ המסילה ההפוכה ל־$\gamma$.
	נראה שהעתקה זו משרה העתקה בין החבורות, ואז נראה שההעתקה הזו היא הומומורפיזם, ולבסוף נראה שאף איזומורפיזם. \\
	כדי להראות ש־$f_{\alpha}$ משרה העתקה $\hat{f}_{\alpha} : \pi_1(X, x_0) \to \pi_1(X, x_1)$, מספיק להראות שאם $\gamma, \gamma' \in \Omega(X, x_0)$ מסילות הומוטופיות אז גם $f_{\alpha}(\gamma) \sim f_{\alpha}(\gamma')$.
	למעשה אנו כבר יודעים זאת ישירות מהעובדה ש־$\pi_1(X, x_0)$ חבורה, ולכן נוכל להגדיר $\hat{f}_{\alpha}([\gamma]) = [\overline{\alpha} * \gamma * \alpha]$.
	אם $[\gamma_1], [\gamma_2] \in \pi_1(X, x_0)$ אז,
	\[
		\hat{f}_{\alpha}([\gamma_1] [\gamma_2])
		= \hat{f}_{\alpha}([\gamma_1 * \gamma_2])
		= [\overline{\alpha} * \gamma_1 * \gamma_2 * \alpha]
	\]
	ומהצד השני,
	\[
		\hat{f}_{\alpha}([\gamma_1]) \hat{f}_{\alpha}([\gamma_2])
		= [\overline{\alpha} * \gamma_1 * \alpha] \cdot [ \overline{\alpha} * \gamma_2 * \alpha]
		= [\overline{\alpha} * \gamma_1 * \alpha * \overline{\alpha} * \gamma_2 * \alpha]
		= [\overline{\alpha} * \gamma_1 * \gamma_2 * \alpha] 
	\]
	ונסיק כי זהו הומומורפיזם. \\
	נעבור לבדיקת איזומורפיזם.
	נניח ש־$e = \hat{f}_{\alpha}([\gamma])$ איבר היחידה ב־$\pi_1(X, x_0)$.
	נגדיר $\hat{g}_{\alpha} : \pi_1(X, x_1) \to \pi_1(X, x_1)$ ונראה ש־$\hat{g}_{\alpha} \circ \hat{f}_{\alpha} = \id_{\pi_1(X, x_0)}$,
	וגם ש־$\hat{f}_{\alpha} \circ \hat{g}_{\alpha} = \id_{\pi_1(X, x_1)}$.
	נניח ש־$[\beta] \in \pi_1(X, x_1)$ וכן $\hat{g}_{\alpha}(\beta) = [\alpha * \beta * \overline{\alpha}]$.
	לכל $[\gamma] \in \pi_1(X, x_0)$ נובע,
	\[
		(\hat{g}_{\alpha} \circ \hat{f}_{\alpha})(\gamma)
		= \hat{g}_{\alpha}(\hat{f}_{\alpha})(\gamma)
		= \hat{g}_{\alpha}([\overline{\alpha} * \gamma * \alpha])
		= [\alpha * \overline{\alpha} * \gamma * \alpha * \overline{\alpha}]
		= [\gamma]
	\]
	ולכן נסיק שאכן $\hat{g}_{\alpha} \circ \hat{f}_{\alpha} = \id_{\pi_1(X, x_0)}$ ובאופן דומה נסיק שאכן $\hat{f}_{\alpha}$ איזומורפיזם.
\end{proof}
ניזכר בהגדרה\ \ref{definition:contractable_space}, המדברת על כוויצות.
\begin{remark}
	אם $X$ מרחב כוויץ אז $\pi_1(X, x_0)$ טריוויאלית.
\end{remark}
\begin{definition}[מרחב פשוט קשר]
	נאמר ש־$X$ פשוט קשר אם $X$ קשיר מסילתית ו־$\pi_1(X, x_0) = \{ e \}$ ל־$x_0 \in X$.
\end{definition}
\begin{definition}[נסג עיוות]
	יהי $X$ מרחב טופולוגי ו־$Y \subseteq X$ תת־מרחב.
	נאמר ש־$Y$ הוא נסג עיוות (Deformation retract) של $X$ אם יש $H : [0, 1] \times X \to X$ כך ש־$H(0, x) = x$ לכל $x \in X$ וכן,
	$H(s, y) = y$ לכל $y \in Y, s \in [0, 1]$ וש־$H(1, x) \in Y$ לכל $x \in X$.
\end{definition}
הגדרה זו היא בעצם הרעיון שאנו יכולים לצמצם באופן רציף את המרחב שלנו.
\begin{example}
	אם $Y = \{ y_0 \}$ ו־$Y$ נסג עיוות של $X$ אז $X$ כוויץ.
\end{example}
\begin{exercise}
	האם $X$ כוויץ אז יש נסג עיוות מ־$X$ ל־$x_0 \in X$?
\end{exercise}
\begin{solution}
	תהי $H : [0, 1] \times X \to X$ ההעתקה של הכיווץ.
	נניח ש־$\alpha : [0, 1] \to X$ על־ידי $\alpha(t) = H(t, x_0)$.
	תהי $\gamma \in \Omega(X, x_0)$ ונרצה לבנות הומוטופיה עם $\gamma$ ל־$\alpha * \overline{\alpha}$.
	נגדיר את ההעתקה,
	\[
		G(s, t)
		= \begin{cases}
			\alpha(2t) & 0 \le t \le \frac{s}{2} \\
			H(s, \gamma(\frac{t - \frac{s}{2}}{1 - s})) & \frac{s}{2} \le t \le 1 - \frac{s}{2}, s < 1 \\
			\alpha(2 - 2t) & \frac{s}{2} \le t \le 1
		\end{cases}
	\]
	אנו טוענים כי $G$ היא העתקה רציפה, ובמקרה זה $G$ מגדירה הומוטופיה בין $\gamma$ ל־$\alpha * \overline{\alpha} \sim c_{x_0}$.
	כדי להראות זאת נשתמש בעובדה ש־$[0, 1]$ קומפקטי והתמונה רציפה ולכן קומפקטית גם כן.
\end{solution}

\section{שיעור 16 --- 26.5.2025}
\subsection{חבורה יסודית וכוויצות}
נמשיך ונדון בבעיה שהצגנו בפעם הקודמת.
$X$ כוויץ אם יש נקודה יחידה כך שיש הומוטופיה מכל המרחב לנקודה הזו.
מהצד השני מרחב הוא נסג עיוות אם מתקיים מצב דומה עם תת־מרחב.
הפעם נאמר שכל מרחב שהוא נסג עיוות לנקודה גורר שהוא כוויץ לנקודה, אבל גם נראה דוגמה נגדית למצב ההפוך.
\begin{theorem}
	אם $X$ כוויץ אז $X$ פשוט קשר.
\end{theorem}
\begin{proof}
	תהי $F : I \times X \to X$ של הכיווץ. \\
	נראה ש־$X$ קשיר מסילתית.
	לכל זוג נקודות $x, y \in X$, נתבונן בשרשור $\alpha_x * \beta_y$ שתי מסילות המקיימות $\alpha_x(t) = F(t, x)$ ו־$\beta_y(s) = F(1 - s, y)$.
	השרשור שלהן כמובן מעיד על קשירות מסילתית של $x, y$. \\
	נמשיך ונצרה להראות ש־$1 = |\pi_1(X, x_0)|$.
	תהי $\gamma \in \Omega(X, x_0)$ ונבחן את $\alpha = \alpha_{x_0}$ המסילה שהגדרנו קודם לכן.
	נגדיר,
	\[
		G(s, t)
		= \begin{cases}
			\alpha(t) & 0 \le t \le \frac{s}{2} \\
			F(s, \gamma(\frac{t - \frac{s}{2}}{1 - s})) & \frac{s}{2} \le t \le 1 - \frac{s}{2}, s < 1 \\
			\alpha(2 - 2t) & 1 - \frac{s}{2} \le t \le 1
		\end{cases}
	\]
	העתקה זו מעבירה את המסילה ל־$\alpha$ ולכן מוכיחה שיש רכיב יחיד בחבורה היסודית של המרחב, אבל עלינו להראות שהיא בכלל רציפה.
	בבירור היא כבר רציפה בשלושת תחומיה בנפרד, זאת כהרכבת העתקות רציפות.
	נותר לנו לבדוק את שתי הנקודות שמחברות את הקטעים הללו.
	אם $s < 1$ וגם $t = \frac{s}{2}$ או $t = 1 - \frac{s}{2}$ אז הרציפות נובעת מלמת ההדבקה.
	נותר עלינו לבדוק את $s = 1$.
	נקבע קבוצה פתוחה כלשהי $G(1, t) \in U \subseteq X$ ונראה שיש קבוצה פתוחה $(1, t) \in L \subseteq I \times I$ ומתקיים $\forall (p, q) \in L,\ G(p, q) \in U$.
	אם $t < \frac{1}{2}$ אז יש סביבה פתוחה $(1, t) \in L' \subseteq I \times I$ שבה $G(p, q) = \alpha(2q)$ ו־$\alpha$ רציפה לכן יש סביבה פתוחה דומה.
	אם $t > \frac{1}{2}$ אז נוכל לפעול באופן דומה עם $\alpha(2 - 2q)$. \\
	אם $t = \frac{1}{2}$ ו־$G(1, \frac{1}{2}) = z$ אז המסילה $\gamma$ רציפה והקטע $[0, 1]$ קומפקטי ולכן $\gamma([0, 1])$ קבוצה קומפקטית ב־$X$.
	$F : I \times X \to X$ רציפה ולכן הצמצום $F \mid_{I \times \gamma([0, 1])}$ היא בעצמה פונקציה רציפה.
	$F(1, x) = z$ לכל $x$.
	לכל $x \in \gamma(I)$ יש סביבה פתוחה $(1, x) \in V_x \subseteq [0, 1] \times X$ כך שלכל $(p, y) \in V_x$ מקיימת,
	\[
		F(1, y) \in U
	\]
	בלי הגבלת הכלליות $V_x = (r_x, 1] \times W_x$ כאשר $r_x < 1$ ו־$x \in W_x \subseteq X$ פתוחה.
	אז ${\{ W_x \}}_{x \in \gamma(I)}$ כיסוי פתוח של $\gamma(I)$ קומפקטית ולכן יש תת־כיסוי סופי ${\{ W_{x_i} \}}_{i = 1}^n$.
	קיים $0 \le r < 1$ כך שלכל $x \in \gamma(I)$ ולכל $r < p \le 1$,
	\[
		F(p, x) \in U
	\]
	ונובע שלכל $(p, q) \in I \times I$ כך ש־$r < p < 1$ ו־$\frac{p}{2} < q < 1 - \frac{p}{2}$ גם $G(p, q) \in U$.
\end{proof}

\subsection{מרחבי כיסוי והעתקות כיסוי}
\begin{definition}[העתקת כיסוי]
	יהיו $B, E$ מרחבים טופולוגיים,
	העתקה $p : E \to B$ תיקרא העתקת כיסוי אם היא רציפה,
	ולכל נקודה $b \in B$ יש סביבה פתוחה $b \in U \subseteq B$ כך ש־$p^{-1}(U)$ ניתן להצגה כאיחוד זר של קבוצות פתוחות $V_{\alpha} \subseteq E$ עבור $\alpha \in \Omega$,
	\[
		p^{-1}(U)
		= \biguplus_{\alpha \in \Omega} V_{\alpha}
	\]
	כך שלכל $\alpha \in \Omega$, $p \mid_{V_{\alpha}} : V_{\alpha} \to U$ היא הומיאומורפיזם. \\
	נאמר ש־$U$ מכוסה בצורה אחידה על־ידי $p$. \\
	למרחב $E$ נקרא מרחב כיסוי של $B$.
\end{definition}
\begin{proposition}
	כל העתקת כיסוי היא העתקה פתוחה.
\end{proposition}
\begin{proof}
	תהי קבוצה פתוחה $W \subseteq E$, נרצה להראות שגם $p(W) \subseteq B$ פתוחה.
	תהי $x \in p(W)$ ותהי $y \in W$ כך ש־$p(y) = x$.
	$y \in V_{\alpha_0}$ כלשהו וכן $p |_{V_{\alpha_0}} : V_{\alpha_0} \to U$ הומיאומורפיזם.
	אז $p |_{V_{\alpha_0}}(W \cap V_{\alpha_0})$ פתוחה.
\end{proof}
נעבור לדוגמות.
\begin{example}
	$\id_B : B \to B$ העתקת הזהות, היא העתקת כיסוי.
\end{example}
\begin{example}
	$p : B \times \{ 1, 2, \ldots, n \} \to B$ העתקת הצמצום, גם היא העתקת כיסוי.
\end{example}
\begin{example}
	העתקת הישר למעגל, $t \mapsto e^{2 \pi i t}$ (או לחלופין $t \mapsto (\cos(2\pi t), \sin(2\pi t))$), אף היא העתקת כיסוי.
\end{example}
\begin{example}
	נבחן את $p : \CC \to \CC \setminus \{ 0 \}$ המוגדרת על־ידי $z \mapsto e^z$, וגם זו העתקת כיסוי.
\end{example}

\section{שיעור 17 --- 27.5.2025}
\subsection{מרחבי כיסוי}
נעסוק היום בהרמות במרחבי כיסוי.
\begin{theorem}[הרמה של מסילות]
	נניח ש־$p : E \to B$ העתקת כיסוי ו־$\gamma : I \to B$ מסילה רציפה. \\
	נסמן $\gamma(0) = b_0$, אז לכל $e_0 \in p^{-1}(b_0)$ יש מסילה רציפה ויחידה $\tilde{\gamma} : I \to E$ כך ש־$p \circ \tilde{\gamma} = \gamma$ ו־$\tilde{\gamma}(0) = e_0$.
\end{theorem}
\begin{proof}
	לכל נקודה $b \in B$ יש סביבה פתוחה $U_b$ אשר מכוסה באופן אחיד על־ידי $p$,
	\[
		p^{-1}(U_b)
		= \biguplus V_{\alpha}^b
	\]
	נניח ש־$\gamma : I \to B$, ו־$\{ \gamma^{-1}(U_b) \mid b \in B \}$ כיסוי פתוח של $I$.
	יהי $\lambda$ מספר לבג של הכיסוי, כלומר לכל $t \in I$ הקטע $(t - \lambda, t + \lambda) \subseteq p^{-1}(U_b)$ עבור איזשהו $b \in B$.
	אנחנו נבנה את ההרמה $\tilde{\gamma}$ באופן אינדוקטיבי.
	נניח שהגדרנו כבר פונקציה רציפה $\tilde{\gamma} : [0, t_j] \to E$ כך שמתקיים $\tilde{\gamma}(0) = e_0$ ו־$(p \circ \tilde{\gamma})(t) = \gamma(t)$.
	אם $j = n$ אז סיימנו, אחרת נשים לב שמכיוון ש־$t_{j + 1} - t_j > \lambda$ אז נובע שיש $b \in B$ כך שמתקיים,
	\[
		\gamma([t_j, t_{j + 1}]) \subseteq U_b
	\]
	ולכן $p^{-1}(U_b) = \biguplus V_{\alpha}^b$ כך ש־$p |_{V_{\alpha}^b} : V_{\alpha}^b \to U_b$ הומיאומורפיזם.
	קיים $\alpha_0$ יחיד כך ש־$\tilde{\gamma}(t_j) \in V_{\alpha_0}^b$ כי $\gamma(t_j) \in U_b$ ולכן נגדיר,
	\[
		\tilde{\gamma}(t)
		= {(p |_{V_{\alpha}^b})}^{-1}(\gamma(t))
	\]
	לכל $t_j \le t \le t_{j + 1}$. \\
	קל לראות שהתהליך מסתיים לאחר מספר סופי של חזרות, כאשר נגיע ל־$j = n$,
	מלמת ההדבקה אם $\tilde{\gamma}$ היא מסילה רציפה ו־$p \circ \tilde{\gamma} = \gamma$, וזוהי מסילה יחידה.
\end{proof}
\begin{theorem}
	תהי $p : E \to B$ העתקת כיסוי ונניח ש־$p(e_0) = b_0$.
	נניח ש־$F : I \times I \to B$ רציפה, אז יש הרמה יחידה $\tilde{F} : I \times I \to E$ כך ש־$\tilde{F}(0, 0) = e_0$.
	אם $F$ היא הומוטופיה של מסילות אז גם $\tilde{F}$ היא הומוטופיה של מסילות.
\end{theorem}

\section{שיעור 18 --- 3.6.2025}

\subsection{הרמות}
דיברנו עד כה על החבורה היסודית של מרחב טופולוגי קשיר מסילתית.
לאחר מכן דיברנו על מרחבי כיסוי.
ראינו כי אם $p : E \to B$ העתקת כיסוי, וכן $f : [0, 1] \to B$ מסילה כך ש־$b_0 \in B$ ו־$f(0) = b_0$, ו־$e_0 \in p^{-1}(b_0)$,
אז קיימת הרמה יחידה $\tilde{f} : [0, 1] \to E$ כך ש־$p \circ \tilde{f} = f$.
\begin{theorem}
	תהי $p : E \to B$ העתקת כיסוי ותהי $F : I \times I \to B$, נסמן $F(0, 0) = b_0$.
	תהי $e_0 \in p^{-1}(b_0)$, אז קיימת הרמה יחידה $\tilde{F} : I \times I \to E$ המקיימת $\tilde{F}(0, 0) = e_0$. \\
	בנוסף, אם $F$ הומוטופיה בין מסילות בין מסילות אז $\tilde{F}$ גם היא הומוטופיה בין מסילות.
\end{theorem}
\begin{proof}
	נתבונן בתמונה $B \supseteq F(I \times I)$. $I \times I$ קומפקטית ו־$F$ רציפה ולכן גם $F(I \times I)$ קומפקטית.
	לכל נקודה $b \in F(I \times I)$ קיימת סביבה $b \in U_b$ המכוסה באחידות.
	נבחן את הסביבות $F^{-1}(U_b)$, זהו כיסוי פתוח של $I \times I$.
	לכיסוי הפתוח $p^{-1}(U_b)$ יש מספר לבג $\delta > 0$.
	מהמשפט הקודם קיימת הרמה לצמצומים $F |_{\{ 0 \} \times I}$ ו־$F |_{I \times \{ 0 \}}$ כהרמות של מסילות.
	אם $J = [0, \frac{1}{k}]$ אז $\tilde{F}(I \times I) = p^{-1} \circ F |_{J \times J}$ על־ידי שימוש ברציפות ובחירת $k$.
	קיבלנו העתקה רציפה עבור ריבוע אחד בתוך $I \times I$, נוכל להמשיך כך ולבנות את ההעתקה לכלל הריבועים.
	קיבלנו $\tilde{F} : I \times I \to E$ המקיימת $p \circ \tilde{F} = F$, לפי למת ההדבקה $\tilde{F}$ רציפה.
	ההרמה היא יחידה לפי אותו טיעון ששימש אותנו ליחידות של הרמת מסילות. \\
	הרמה של מסילה קבועה היא מסילה קבועה ישירות מהגדרתה, ולכן אם $F$ הומוטופיה עם נקודות קצה אז זו הומוטופיה של מסילות עם נקודות קצה.
\end{proof}
\begin{conclusion}
	תהינה $f, g$ מסילות הומוטופיות ב־$B$ המתחילות ב־$b_0$, ותהי $e_0 \in p^{-1}(b_0)$, תהינה גם $\tilde{f}, \tilde{g}$ הרמות של $f, g$ המתחילות ב־$e_0$, אז $\tilde{f}(1) = \tilde{g}(1)$.
\end{conclusion}
\begin{proof}
	$f, g$ הומוטופיות ולכן קיימת הומוטופיה $F : I \times I \to B$ כך ש־$F |_{I \times \{ 0 \}} = f$ ו־$F |_{I \times \{ 1 \}} = g$.
	להומוטופיה $F$ קיימת הרמה יחידה $\tilde{F}$ המקיימת $\tilde{F}(0, 0) = e_0$,
	\[
		\tilde{F} |_{I \times \{ 0 \}} = \tilde{f},
		\quad
		\tilde{F} |_{I \times \{ 1 \}} = \tilde{g}
	\]
	ו־$\tilde{F}$ היא הומוטופיה בין $\tilde{f}$ ל־$\tilde{g}$ ולכן בפרט $\tilde{f}(1) = \tilde{g}(1)$.
\end{proof}

\section{שיעור 19 --- 9.6.2025}

\subsection{בין מרחבי כיסוי להומוטופיה}
נמשיך ישירות מתוצאות השיעור הקודם ונבחין במספר מסקנות.
\begin{conclusion}
	אם $p : E \to B$ העתקת כיסוי,
	אז לכל זוג מסילות $\alpha, \beta \in \Omega(B, b, c)$, אם הן הומוטופיות אז,
	לכל זוג הרמות $\tilde{\alpha}, \tilde{\beta}$ ל־$E$ כך ש־$\tilde{\alpha}(0) = \tilde{\beta}(0)$ מתקיים גם $\tilde{\alpha}(1) = \tilde{\beta}(1)$. \\
	בפרט מתקבלת ההעתקה $\Phi : \pi_1(B, b_0) \to p^{-1}(\{ b_0 \}) = p^{-1}(b_0)$.
\end{conclusion}
בתרגול ראינו כי אם $E$ פשוט־קשר אז,
\[
	\Phi : \pi_1(B, b_0) \to p^{-1}(b_0)
\]
היא חד־חד ערכית ועל.
השתמשנו בזה כדי להוכיח ש־$\pi_1(\RR P^n) \simeq \ZZ / n\ZZ$ עבור $n \ge 2$.
עתה נסתכל על $\RR P^3$.
נגדיר את $SO(3) = \{ A \in M_3(\RR) \mid \det(A) = 1, A^T A = I \}$.
חבורת הסיבובים של $\RR^3$.
אנו טוענים כי $\RR P^3 \simeq SO(3)$.
\begin{theorem}
	נניח ש־$p : (E, e_0) \to (B, b_0)$ העתקת כיסוי, נניח ש־$E$ קשיר מסילתית,
	ונסמן,
	\[
		G = \pi_1(B, b_0),
		\quad
		H = p_*(\pi_1(E, e_0))
	\]
	כאשר $H \le G$ ו־$p_*([\gamma]) = [ p \circ \gamma ]$.
	אז במקרה זה מתקיים,
	\begin{enumerate}
		\item $p_* : \pi_1(E, e_0) \to p_1(BB, b_0)$ היא חד־חד ערכית
		\item יש התאמה חד־חד ערכית ועל בין $p^{-1}(b_0)$ לבין $H \backslash G = \{ H g \mid g \in G \}$.
		\item למסילה סגורה $f \in \Omega(B, b_0)$ מגדירה איבר $[f] \in H$ אם ורק אם ההרמה $\tilde{f} : [0, 1] \to E$ כך ש־$\tilde{f}(0) = e_0$ מסתיימת ב־$e_0$.
	\end{enumerate}
\end{theorem}
\begin{proof}
	\textbf{1.}
	$p_*$ הוא הומומורפיזם של חבורות ולכן כדי להוכיח ש־$p_*$ חד־חד ערכית די להראות ש־$\ker p_*$ טריוויאלי.
	נניח ש־$\tilde{\gamma} \in \Omega(E, e_0)$ כך שמתקיים,
	\[
		p_*([\tilde{\gamma}])
		= 1 \in \pi_1(B, b_0)
	\]
	לכן $[p \circ \tilde{\gamma}] = 1$.
	המסילה $\gamma = p \circ \tilde{\gamma}$ הומוטופית למסילה הקבועה $t \mapsto b_0$.
	נרים את ההומוטופיה להומוטופיה ב־$E$ מהמסילה $t \mapsto e_0$ למסילה $\tilde{\gamma}$ וסיימנו.

	נתבונן בשני איברים,
	\[
		[f], [g] \in \pi_1(B, b_0)
	\]
	אנו רוצים להראות כי,
	\[
		H[f] = H[g]
		\iff \Phi([f]) = \Phi([g])
	\]
	נסמן את ההרמות $\tilde{f}, \tilde{g}$ כך ש־$\tilde{f}(0) = \tilde{g}(0) = e_0$ ולכן,
	\[
		H[f] = H[g]
		\iff \tilde{f}(1) = \tilde{g}(1)
	\]
	נניח ש־$[f] \in H [g]$, אז,
	\[
		f \sim h * g
	\]
	עבור $[h] \in H$,
	כלומר קיימת מסילה סגורה $\tilde{h} \in \Omega(E, e_0)$ כך ש־$h = p \circ \tilde{h}$.
	נובע ש־$\tilde{f}(1) = \tilde{g}(1)$. \\
	בכיוון השני נניח ש־$\tilde{f}(1) = \tilde{g}(1)$.
	נסמן $\tilde{h} = \tilde{f} * \tilde{g}$ עבור $p \circ \tilde{h} = h$ כאשר $[h] \in H$.
	אז,
	\[
		\tilde{h} * \tilde{g}
		= \tilde{f} * \overline{\tilde{g}} * \tilde{g} \sim \tilde{f}
		\implies [h] [g] = [f],
		[f] \in H [g]
	\]
	אז $[\tilde{f}] \in H[\tilde{g}]$.
\end{proof}
\begin{theorem}
	$\pi_1(S^1, 1) \simeq \ZZ$.
\end{theorem}
\begin{proof}
	ראינו כבר שמ־$p : \RR \to S^1$ המוגדרת על־ידי $p(t) = e^{2 \pi i t}$ היא העתקת כיסוי.
	בנוסף $\RR$ פשוט קשר, ולכן נובע שיש העתקה חד־חד ערכית ועל בין $\pi_1(S^1, 1)$ ל־$p^{-1}(1) = \ZZ$. \\
	נניח ש־$[f], [g] \in \pi_1(S^1, 1)$.
	אז,
	\[
		\Phi([f]) = n,
		\quad
		\Phi([g]) = m
	.\]
	כאשר $\tilde{f} : [0, 1] \to \RR$ הרמה של $f$ כך ש־$\tilde{f}(0) = 0$ ובאופן דומה $\tilde{g}$ הרמה של $g$.
	אז $\tilde{f}(1) = n, \tilde{g}(1) = m$.
	נתבונן ב־$[f * g] = [f] * [g]$.
	אם $l = f * g$ אז נבחן את $\tilde{l}$ כך ש־$\tilde{l}(0) = 0$ ונחשב את $\tilde{l}(1)$. \\
	נגדיר,
	\[
		T_n(\tilde{g}) : [0, 1] \to \RR,
		\qquad
		T_n(\tilde{g})(t) = \tilde{g}(t) + n
	\]
	אז $T_n(\tilde{g})$ הרמה של $g$ ל־$\RR$ כך ש־$T_n(\tilde{g})(0) = n$ ו־$\tilde{g}(0) + n = 0 + n = n$.
	נוכל לבנות את השרשור $\tilde{f} * T_n(\tilde{g})$ וקל לראות שזו הרמה ל־$\RR$ המתיחה ב־$0$ על המסילה $l$.
	לכן,
	\[
		\Phi([f] [g])
		= \Phi([f * g])
		= (\tilde{f} * T_n(\tilde{g}))(1)
		= m + n
	\]
\end{proof}
\begin{definition}[רטרקציה]
	יהי $X$ מרחב טופולוגי ו־$Y \subseteq X$ תת־מרחב.
	\textbf{רטרקציה} מ־$X$ ל־$Y$ היא העתקה רציפה $\rho : X \to Y$ כך ש־$\rho |_Y = \id_Y$.
\end{definition}
\begin{proposition}
	אם יש רטרקציה מ־$X$ ל־$Y$ אז העתקת השיכון $\iota_Y : Y \to X$ משרה הומומורפיזם חד־חד ערכי,
	\[
		\iota_{Y_*} : \pi_1(Y, y_0) \to \pi_1(X, y_0)
	\]
\end{proposition}
\begin{proof}
	נניח ש־$U, V, W$ מרחבים טופולוגיים וכן ש־$f, g$ העתקות רציפות $U \to V \to W$.
	\[
		f_* : \pi_1(U, u_0) \to \pi_1(V, v_0),
		\qquad
		g_* : \pi_1(V, v_0) \to \pi_1(W, w_0)
	\]
	אז $g \circ f : U \to W$ ומתקיים,
	\[
		{(g \circ f)}_* : \pi(U, u_0) \to \pi_1(W, w_0)
	\]
	כך ש־${(g \circ f)}_* = g_* \circ f_*$.

	יש $\rho : X \to Y$ ו־$\iota_Y : Y \hookrightarrow X$ אז $Y \to X \to Y$ ונקבל,
	\[
		\id_{\pi_1(Y, y_0)}
		= {(\rho \circ \iota_Y)}_*
		= {(\rho \circ \iota_Y)}_*
		= \rho_* \circ {(\iota_Y)}_*
	\]
\end{proof}
\begin{conclusion}[משפט נקודת השבת של בראואר]
	לכל העתקה רציפה $f : D \to D$ עבור $D = \overline{B}(0, 1) \subseteq \RR^2$ יש נקודת שבת.
\end{conclusion}
\begin{proof}
	$S^1 \hookrightarrow D$ וכן $\ZZ = \pi_1(S^1, 1)$, אבל גם $\pi_1(D, 1) = 1$ טריוויאלית.
	אין שיכון של $\ZZ$ בחבורה הטריוויאלית ולכן אין רטרקציה מעיגול הסגור לשפה שלו.

	נניח ש־$f : D \to D$ העתקה שאין לה נקודות שבת ונבנה רטרקציה.
	לכל $x \in D$ נגדיר $\rho(x)$ להיות נקודת החיתוך של קרן מ־$f(x)$ ל־$x$, כלומר ש־$f(x) + t(x - f(x)) = 0$ עבור $t > 0$.
\end{proof}

\section{שיעור 20 --- 10.6.2025}
ננסח שוב את המשפט שראינו אתמול,
\begin{theorem*}
	נניח ש־$p : (E, e_0) \to (B, b_0)$ העתקת כיסוי,
	אז,
	\begin{enumerate}
		\item $p_* : \pi_1(E, e_0) \to \pi_1(B, b_0)$ שיכון
		\item ההעתקה $\Phi : \pi_1(B, b_0) \to p^{-1}(b_0)$ משרה העתקה חד־חד ערכית,
			\[
				\overline{\Phi} : p_*(\pi_1(E, e_0)) \backslash \pi_1(B, b_0) \to p^{-1}(b_0) 
			\]
		\item אם $E$ קשיר מסילתית אז $\Phi$ היא על
	\end{enumerate}
\end{theorem*}
\begin{remark}
	נניח ש־$X$ מרחב טופולוגי ו־$Y \subseteq X$ תת־מרחב.
	נסג עיוות מ־$X$ ל־$Y$ הוא העתקה $D : I \times X \to Y$ כך שמתקיים,
	\[
		\forall x \in X,\ 
		D(1, x) \in Y,
		\quad
		D(0, x) = x,
		\quad
		\forall y \in Y,\ D(s, y) = y
	\]
	בפרט $\rho : X \to Y$ המוגדרת על־ידי $\rho(x) = D(1, x)$ היא רטרקציה.
	לכן העתקת השיכון $\iota_Y : Y \hookrightarrow X$ משרה הומומורפיזם חד־חד ערכי,
	\[
		\iota_{Y *} : \pi_1(Y, y_0) \to \pi_1(X, x_0)
	\]
\end{remark}
\begin{lemma}
	אם $f, g : (X, x_0) \to (Y, y_0)$ העתקות רציפות,
	ויש הומוטופיה ביניהן,
	\[
		H : I \times X \to Y
	\]
	רציפה,
	אז מתקיים $f_* = g_*$, כאשר,
	\[
		f_*, g_* : \pi_1(X, x_0) \to \pi_1(Y, y_0)
	\]
\end{lemma}
\begin{theorem}
	אם $Y \subseteq X$ נסג עיוות,
	אז,
	\[
		{(\iota_Y)}_* : \pi_1(Y, y_0) \to \pi_1(X, y_0)
	\]
	איזומורפיזם.
\end{theorem}
\begin{proof}
	אנו כבר יודעים כי ${(\iota_Y)}_*$ הומומורפיזם של חבורות והיא חד־חד ערכית כי יש רטרקציה מ־$X$ על $Y$.
	נותר להראות שהיא על.
	תהי $[\gamma] \in \pi_1(X, y_0)$.
	נניח גם ש־$D : I \times X \to Y$ נסג עיוות ונגדיר,
	\[
		L : I \times I \to X,
		\quad
		L(s, t) = D(s, \gamma(t))
	\]
	ולכן,
	\[
		L(0, t) = D(0, \gamma(t)) = \gamma(t)
	\]
	ונסיק ש־$L(1, t)$ מסילה ב־$Y$ וש־$L$ הומוטופיה.
\end{proof}
\begin{proposition}
	$\pi_1(\RR^2 \setminus \{ (0, 0) \}) = \ZZ$.
\end{proposition}
\begin{proof}
	קיים $S^1 \hookrightarrow \RR^2 \setminus \{(0, 0)\}$,
	ונגדיר,
	\[
		D(t, x)
		= x + t(\frac{x}{\lVert x \rVert} - x)
	\]
	וקיבלנו את הטענה.
\end{proof}
\begin{definition}[שקילות הומוטופית]
	נאמר ש־$X, Y$ שקולים הומוטופית אם יש העתקות $f : X \to Y$ ו־$g : Y \to X$ רציפות כך ש־$g \circ f \sim \id_X$ ו־$f \circ g \sim \id_Y$,
	כאשר נאמר ש־$f \sim g$ הומוטופיות, אם יש הומוטופיה ביניהן.
	במקרה זה נאמר גם ש־$f$ שקילות הומוטופית וש־$g^{-1}$ הופכית הומוטופית של $f$.
\end{definition}
\begin{exercise}
	הוכיחו כי זהו אכן יחס שקילות.
\end{exercise}
\begin{theorem}
	אם $X, Y$ שקולים הומוטופית ו־$f : X \to Y$ העתקת שקילות הומוטופית,
	אז $f_* : \pi_1(X, x_0) \to \pi_1(Y, y_0)$ כאשר $y_0 = f(x_0)$ היא איזומורפיזם.
\end{theorem}

\section{שיעור 21 --- 16.6.2025}

\subsection{משפט ואן־קמפן}
היום נעסוק בהוכחת משפט ואן־קמפן.
משפט זה יאפשר לנו לחשב את החבורה היסודית של מרחבים שונים.
לדוגמה של שמונה ושל גרפים קשירים.
נתחיל בניסוח הכי פשוט של המשפט, ומשם אולי נמשיך לגרסות המורכבות יותר.
\begin{example}
	חבורה חופשית עם יוצר אחד היא $\ZZ$ וכאן נחשוב עליה כחבורה כפלית.
	כלומר אם $a$ איבר, אז נעסוק בחבורה,
	\[
		\langle a \rangle
		= \{ e = a^0, a, a^2, \ldots \}
	\]
	נסמן ב־$W(\{ a \})$ את כל המילים הסופיות עם $a, a^{-1}$.
\end{example}
\begin{example}
	המילה $a a a^{-1} a a \in W(\{ a \})$, לכן נסמן $F(\{ a \}) \subseteq W(\{ a \})$ קבוצת המילים המצומצמות.
\end{example}
הרעיון הוא להסתכל על חבורות חופשיות כחבורות שמורכבות כחבורות המורכבות ממכפלת אותיות, כלומר אוסף מחרוזות.
פעולת הכפל על $F(\{ a \})$ היא שרשור של מילים ומעבר למילה מצומצמת.
ברור כי $\ZZ \simeq F(\{ a \})$.
\begin{notation}
	נסמן ב־$F(\{ a \})$ את $\langle a \rangle$.
\end{notation}
\begin{definition}[חבורה חופשית על שני איברים]
	נניח ש־$S = \{ a, b \}$, אז החבורה החופשית הנוצרת על־ידי $S$ היא אוסף כל המילים המצומצמות,
	כלומר אוסף המכפלות הסופיות של איברי $S$ והאיברים ההופכיים להם.
	\[
		W(S)
		= \{ s_0^{\varepsilon_0} s_1^{\varepsilon_1} \cdots s_n^{\varepsilon_n} \mid s_i \in S, \varepsilon_i \in \{ \pm \} \}
	\]
	ונקרא למילה מצומצמת אם היא לא מכילה את הצירוף $a^{-1} a, a a^{-1}, b b^{-1}, b^{-1} b$.
	מכל מילה ב־$W(S)$ ניתן לעבור בכמות צעדים סופית ל־$F(S)$.
\end{definition}
\begin{exercise}
	הוכיחו כי כל מילה ב־$W(S)$ שקולה למילה מצומצמת יחידה, כאשר שתי מילים $u, w \in W(S)$ הן שקולות אם ניתן לעבור מאחת לשנייה על־ידי סדרה סופית של צעדים מאלה שציינו בהגדרת הצמצום.
\end{exercise}
נבחין כי $F(S)$ היא חבורה כחבורת מנה או כאוסף מצומצם (כאשר זה האחרון יותר נוח) ויחד עם פעולת השרשור היא מהווה חבורה.
עלינו להוכיח כי מתקיימת אסוציאטיביות במקרה זה, ההוכחה היא ישירה ולא מאוד מעניינת.
בשלב הבא נבנה גרף המוגדר על־ידי איברי $F(S)$, כלומר יהיה הצומת $e$ המחובר לארבעת הצמתים $a, b, a^{-1}, b^{-1}$ וכן הלאה.
הגרף המתקבל הוא 4־רגולרי ואם $w = w' a$ אז יש קשר בין $w$ ל־$w'$.
הגרף הזה הוא קשיר וחסר מעגלים ולכן גם עץ.
לעץ זה נקרא $T$ ונגדיר העתקות,
\[
	L_a : T \to T,
	L_a(w)
	= \begin{cases}
		aw & w \ne a^{-1} w' \\
		w' & w = a^{-1} w'
	\end{cases},
	\quad
	L_b : T \to T
	b_a(w)
	= \begin{cases}
		bw & w \ne b^{-1} w' \\
		w' & w = b^{-1} w'
	\end{cases}
\]
אלו הן תמורות על קודקודי $T$ אשר שומרת על יחס החבורה הנוצרת על־ידי $L_a, L_b$.
היא החבורה האיזומורפית ל־$F(S)$ שהגדרנו.
במקרה זה $F(S) = \langle S \rangle$ היא החבורה הנוצרת על־ידי $S$/
במקרה זה $F(S) = \langle S \rangle$ היא החבורה הנוצרת על־ידי $S$. \\
אנו רוצים מושג כללי יותר של מכפלה חופשית על חבורות.
לצורך כך נאפיין את התכונה של חבורה חופשית.
\begin{definition}[תכונת האוניברסליות של החבורה החופשית]
	לכל חבורה $G = \langle S \rangle$ ולכל העתקה $f : S \to G$ יש הומומורפיזם יחיד $\varphi : \langle S \rangle \to G$ כך ש־$\varphi |_S = f$.
	במקרה זה נאמר ש־$G$ חופשית.
\end{definition}
החבורה חופשית במובן שהגדרת פונקציה נקבעת באופן חופשי על־ידי האיברים המגדירים אותה.
עתה משהגדרנו תכונה כללית, נוכל לדון בחבורה חופשית כללית על קבוצה כלשהי $S$.
החבורה החופשית על $S$ מאופיינת כך שהיא חבורה $H$ עם העתקה $f : S \to H$ כך שלכל חבורה $G$ והעתקה $f : S \to G$ יש הומומורפיזם יחיד $\varphi$ כמתואר. \\
תהינה $A, B$ חבורות זרות (למעט אולי איבר היחידה), נרצה להגדיר מכפלה חופשית עליהן.
נרצה חבורה $H$ עם זוג הומומורפיזמים $\iota_A : A \to H, \iota_B : B \to H$ כך שלכל חבורה $G$ והומומורפיזמים $\varphi_A : A \to G, \varphi_B : B \to G$ קיים הומומורפיזם יחיד $\varphi : H \to G$ כך שיתקיים,
\[
	\varphi_A = \varphi \circ \iota_A,
	\quad
	\varphi_B = \varphi \circ \iota_B
\]
\begin{proposition}
	יש חבורה עם תכונה כזו והיא יחידה עד כדי איזומורפיזם, נסמן אותה ב־$A * B$ המכפלה החופשית של $A, B$.
\end{proposition}
\begin{definition}[מכפלה חופשית]
	המכפלה החופשית $A * B$ תהיה חבורה שאיבריה הם כל המילים הסופיות מעל $A \cup B$,
	כך שאם $g_1 g_2 \cdots g_m \in A * B$ מילה, אז $g_i$ מחבורה שונה מאשר מ־$g_{i + 1}$.
	מילים אלה תקראנה מילים מצומצמות, והשרשור יוגדר מעליהן,
	נגדיר את,
	\[
		(g_1 \cdots g_m)(h_1 \cdots h_k)
	\]
	כך שאם $h_1, g_m$ שייכות לחבורות שונות אז מחברים ומקבלים מילה מצומצמת.
	אם $h_1, g_m$ שייכות לאותה חבורה והמכפלה בחבורה הזו $g_m h_1 \ne e$ אז נגדיר את המילה $g_1 \cdots g_{m - 1} (g_m h_1) h_2 \cdots h_k$.
	אם $h_1 g_m = e$ אז נגדיר את המכפלה על־ידי $(g_1 \cdots g_{m - 1})(h_2 \cdots h_k)$ וכך נקבל הגדרה רקורסיבית הולמת.
\end{definition}
\begin{exercise}
	מקבלים חבורה, ובהתאם ההגדרה היא טובה.
	בנוסף $A, B$ משוכנות ב־ $A * B$.
	לבסוף גם $A * B$ מקיימת את התכונה האוניברסלית.
\end{exercise}
\begin{remark}
	$\ZZ * \ZZ$ היא חבורה חופשית על שני יוצרים.
	באופן דומה $(\ZZ * \ZZ) * \ZZ$ תהיה חבורה חופשית עם שלושה יוצרים.
\end{remark}
\begin{theorem}[ואן־קמפן, ניסוח ראשון]
	נניח ש־$X$ מרחב טופולוגי קשיר מסילתית ונניח שיש שתי תתי־קבוצות פתוחות $U, V \subseteq X$ קשירות מסילתית כך ש־$X = U \cup V$ ו־$x_0 \in U \cap V$ לאיזשהו $x_0 \in X$ ו־$U \cap V$ פשוט קשר (בפרט קשיר מסילתית).
	אז,
	\[
		\pi_1(X, x_0)
		\simeq \pi_1(U, x_0) * \pi_1(V, x_0)
	\]
\end{theorem}
\begin{conclusion}
	אם $X$ היא המרחב הטופולוגי בצורת 8, אז מתקיים $\pi_1(X) \simeq F_2$ עבור $F_2$ חבורה חופשית עם שני יוצרים.
	נבחר $U$ כחלק השמאלי של ה־8 ללא נקודות הקצה, ו־$V$ באופן דומה החלק הימני ללא נקודות קצה.
\end{conclusion}
\begin{exercise}
	מה קורה במקרים של גרפים סופיים כלליים?
\end{exercise}
ראשית נתבונן במספר סופי של מעגלים עם נקודה יחידה משותפת (מעין פרח), נקבל חבורה חופשית הנוצרת על־ידי מספר הלולאות הסגורות יוצרים.
נעבור להוכחת המשפט.
\begin{proof}
	אנו טוענים כי יש איזומורפיזם,
	\[
		\varphi : \pi_1(U, x_0) * \pi_1(V, x_0) \to \pi(X, x_0)
	\]
	נשים לב כי יש לנו הומומורפיזם $\alpha : \pi_1(U, x_0) \to \pi_1(X, x_0)$ ו־$\beta : \pi_1(V, x_0) \to \pi_1(X, x_0)$, כאשר הם מושרים מהשיכונים,
	\[
		\iota_U : (U, x_0) \hookrightarrow (X, x_0),
		\quad
		\iota_V : (V, x_0) \hookrightarrow (X, x_0)
	\]
	מהתכונה האוניברסלית יש הומומורפיזם יחיד $\pi_1(U, x_0) * \pi_1(V, x_0) \to \pi_1(X, x_0)$ המקיים את ההרמה שציינו לעיל. \\
	אנו טוענים כי $\varphi$ הוא על.
	נתבונן במסילה $\gamma \in \Omega(X, x_0)$ המייצגת איבר $[\gamma] \in \pi(X, x_0)$ ורוצים להראות יש איבר במכפלה החופשית אשר מועתק ל־$[\gamma]$ על־ידי $\varphi$.
	אנו יודעים כי $\gamma : [0, 1] \to X$ כך ש־$\gamma(0) = \gamma(1) = x_0$ העתקה רציפה.
	$X$ מכוסה על־ידי שתי קבוצות פתוחות, בעזרת שימוש במספר לבג ניתן למצוא $n \in \NN$ וחלוקות,
	\[
		0 = t_0 < t_1 < \cdots < t_n = 1
	\]
	כך שהצמצום של $\gamma$ לכל קטע $[t_i, t_{i + 1}]$ מוכל ב־$U$ או מוכל ב־$V$.
	נסמן $W_i$ קבוצות פתוחות ב־$U$ או $V$ כך ש־$\gamma([t_i, t_{i + 1}]) \subseteq W_i$, מובטח כי יש כאלה מפתיחות $U, V$.
	לכל $t_i$ נבחר מסילה $\tau_i : [0, 1] \to W_{i - 1} \cap W_i$ מ־$x_0$ ל־$\gamma(t_i)$.
	נסמן גם $\gamma_i = \gamma |_{[t_i, t_{i + 1}]}$ ולכן,
	\[
		\gamma
	\sim (\gamma_0 * \tau_1^{-1}) * (\tau_1 * \gamma_2 * \tau_2^{-1}) * \cdots (\tau_{n - 1} * \gamma_n * \tau_n)
	\]
	ונקבל,
	\[
		[\gamma] = [\gamma_0 * \tau_i^{-1}] * \cdots * [\tau_{n - 1} * \gamma_n]
	\]
	ומכאן נובע שההעתקה היא על.
\end{proof}

\section{שיעור 22 --- 17.6.2025}

\subsection{משפט ואן־קמפן --- המשך}
נמשיך בהוכחת המשפט בניסוחו שראינו.
\begin{proof}
	ראינו כי $\varphi$ היא על.
	עתה עלינו להראות גם ש־$\varphi$ חד־חד ערכית.
	כלומר נראה שעבור,
	\[
		\gamma_1 \in \Omega(U, x_0),
		\gamma_2 \in \Omega(V, x_0),
		\ldots,
		\gamma_n \in \Omega(V \lor U, x_0)
	\]
	מתקיים,
	\[
		g = [\gamma_1] [\gamma_2] \cdots [\gamma_n] \in \pi_1(U, x_0) * \pi_1(V, x_0)
	\]
	מקיים $\varphi(g) = e$ ב־$\pi_1(X, x_0)$.
	צריך להוכיח שבעצם האיבר $g$ הוא האיבר הטריוויאלי במכפלה החופשית.
	נבחין כי יתכן ש־$[\gamma_i] = r$ בחבורה $\pi_1(U, x_0)$ או ב־$\pi_1(V, x_0)$.
	נשים לב כי,
	\[
		\varphi(g)
		= [\gamma_1 * \cdots * \gamma_n]
	\]
	ונסמן $\gamma = \gamma_1 * \cdots * \gamma_n$.
	נניח ש־$\varphi(g) = e$, כלומר שהמסילה $\gamma$ הומוטופית למסילה הקבועה.
	ל־$\varphi$ יש חלוקה של $I \times I$ למלבנים קטנים כך שכל אחד מהם מועתק לתוך $U$ או לתוך $V$.
	בנוסף חלוקה זו היא חלוקה מלאה, קרי חלוקה מהתצורה של חלוקות של אינפי 3.
	לכל קודקוד של מלבן נסתכל במלבנים שמכילים קודקוד זה, יש בין אחד לארבעה כאלה,
	המלבנים הללו מועתקים כל אחד ל־$U$ או ל־$V$, אם כולם מועתקים ל־$U$ אז קודקוד זה הוא ב־$U$ ונבחר מסיחה סגורה המוכלת ב־$U$.
	נבחר באופן דומה מסילה ב־$V$ במקרה הדומה, ואם חלק מהמלבנים ב־$U$ וחלק ב־$V$, אז הקודקוד ב־$U \cap V$, 
	ונבחר מסילה ב־$U \cap V$. \\
	נתבונן בהומוטופיה $H$ מצומצמת לצלע התחתונה $I \times I$, כלומר ל־$\{ 0 \} \times I$, ואז אפשר לבחון את המילה לאורכה.
	נעבור מהמילה הזאת למילה ארוכה יותר על־ידי בחינת החלוקה העדינה יותר של הצלע התחתונה.
	נחליף את $\gamma_1$ במילה מהצורה,
	\[
		(\alpha_1^1 \tau_1 * \tau_1^{-1} \tau_2 * \cdots)
	\]
	לכל $\gamma_i$, עוברים כך מ־$[\gamma_1] \cdots [\gamma_n]$ למילה,
	\[
		[\alpha_1^1 \tau_1] [\tau_1^1 * \alpha_2^1 * \tau_2] \cdots [\alpha_n^1 * \tau_n]
	\]
	ונקבל שנוכל להרכיב מ־$H$ העתקה רציפה ונקבל חד־חד ערכיות.
\end{proof}

\listoftheorems[title=הגדרות ומשפטים,ignoreall,show={theorem,definition},swapnumber,onlynamed={proposition}]

\end{document}
