\input{../article_base.tex}
\title{פתרון מטלה 03 --- מבוא לטופולוגיה, 80516}
% chktex-file 9
% chktex-file 17

\begin{document}
\maketitle
\maketitleprint{}

\question[2]
בכל אחד מן הסעיפים הבאים נגדיר מרחב טופולוגי $(X, \tau)$ ותת־קבוצה $A \subseteq X$ ונמצא את $A^\circ, \partial A$.

\subquestion{}
נגדיר $X = \RR, \Bb_{\tau} = \{ [a, b) \mid a, b \in \RR \}$, כלומר הישר של זורגנפריי,  יחד עם $A = (0, 1]$.
\begin{solution}
	נבחין כי $\{ [a, b) \mid 0 < a < b < 1 \} \subseteq \Bb_{\tau}$ היא קבוצת איברי הבסיס החלקיים ל־$A$, ולכן גם כל קבוצה $U \subseteq A$ פתוחה היא איחוד מהקבוצה שהצגנו,
	נובע אם כך מהגדרה שמתקיים,
	\[
		A^\circ
		= \bigcup \{ [a, b) \mid 0 < a < b \le 1 \}
		= (0, 1)
	\]
	נבחן עתה את ${(A^C)}^\circ$.
	אנו יודעים ש־$A^C = (-\infty, 0] \cup (1, \infty)$, ונוכל מהליך זהה לזה שעשינו עבור הקבוצה המקורית לקבוע כי ${(A^C)}^\circ = (-\infty, 0) \cup (1, \infty)$, ולכן $\overline{A} = [0, 1]$.
	נסיק ש־$\partial A = \{0, 1\}$.
\end{solution}

\subquestion{}
נגדיר $X = \RR$ עם הטופולוגיה הקו־סופית, והקבוצה $A = [0, 1)$.
\begin{solution}
	נתחיל ונבחין ש־$|A| = |\RR|$, לכן $A \in \tau$, כלומר $A = A^\circ$.
	בטופולוגיה הקו־סופית כל קבוצה סגורה היא מגודל סופי או $X$, לכן לא קיימת קבוצה סגורה מלבד $X$ כך שהיא מכילה את $A$, נסיק אם כן ש־$\overline{A} = \RR$.
	לבסוף נובע ש־$\partial A = A^C$.
\end{solution}

\subquestion{}
נגדיר $X = C[0, 1]$ יחד עם מטריקת סופרימום, ונגדיר את $\tau$ להיות הטופולוגיה המושרית ממטריקה זו. \\
נגדיר גם $A = \{ f \in C[0, 1] \mid \exists x \in [0, 1], f(x) \in [0, 1] \}$.
\begin{solution}
	תהי $f \in A$, נרצה להבין אם $f \in A^\circ$.
	אנו רוצים לבחון אם קיים $\epsilon > 0$ כך ש־$B(f, \rho) \subseteq A$.
	אם $g \in B(f, \epsilon)$ אז לכל $x \in [0, 1]$ מתקיים $|f(x) - g(x)| < \epsilon$, ולכן אם $f(x) \in (0, 1)$ עבור איזשהו $x$, אז בהכרח קיים $\epsilon$ כזה.
	אילו $f(x) \in \{0, 1\}$ עבור איזשהו $x$ בלבד ו־$f(x) \notin (0, 1)$, אז תמיד נוכל לבחור $g(x) = f(x) + \epsilon$ ונקבל ש־$g \notin A$.
	נסיק שמתקיים,
	\[
		A^\circ
		= \{ f \in C[0, 1] \mid \exists x \in [0, 1], f(x) \in (0, 1) \}
	\]
	נטען גם ש־$A$ קבוצה סגורה, זאת שכן $A^C = \{ f \in C[0, 1] \mid \forall x, f(x) \notin [0, 1] \}$, ומשיקולים דומים קיים כדור מוכל סביב כל פונקציה בקבוצה זו.
	לכן נסיק,
	\[
		\partial A
		= \{ f \in X \mid \exists x \in [0, 1], f(x) \in \{0, 1\}, \forall y, f(y) \notin (0, 1) \}
	\]
\end{solution}

\question{}
יהיו $(X, \tau), (Y, \sigma)$ מרחבים טופולוגיים ותהי $f : X \to Y$ העתקת מנה.
נניח ש־$\sim_f$ יחס השקילות על $X$ המוגדר על־ידי $x \sim_f Y \iff f(x) = f(y)$.
נסמן ב־$p$ את ההטלה של $X$ ל־$X / \sim_f$.
נוכיח כי קיים ויחיד הומיאומורפיזם $i : (Y, \sigma) \to (X / \sim_f, p_* \tau)$ כך ש־$i \circ f = p$.
\begin{proof}
	נגדיר $i$ כזה.
	לכל $y \in Y$ יהי $x \in X$ כך ש־$f(x) = y$, ידוע כי $f$ על ולכן יש כזה.
	נגדיר אם כך $i(y) = {[x]}_{\sim_f}$.
	נבחין כי הגדרה זו לא תלויה בבחירת $x$, שכן אם $f(x') = y$ גם כן, אז $[x] = [x']$.
	זוהי אם כן גם העתקה חד־חד ערכית, אחרת נקבל ש־$f$ לא מקיימת את תנאי הפונקציה, וכן $i$ על ישירות מהגדרת יחס השקילות $\sim_f$.
	לבסוף ישירות מהגדרת העתקת מנה נסיק ש־$i$ היא העתקה פתוחה ורציפה, ולכן גם הומיאומורפיזם.

	עתה נרצה להוכיח ש־$i$ מקיימת את הטענה האמורה, וכי אם $j$ הומיאומורפיזם המקיים את הדרישה אף הוא, אז $i = j$.
	יהי $x \in X$, אז $i(f(x)) = {[x]}_{\sim_f} = p(x)$ ישירות מהגדרה.
	לכל $y \in Y$ קיים $x \in X$ כך ש־$i(y) = [x]$, לכן גם $j(f(x)) = j(y) = [x]$, וקיבלנו $i(x) = j(x)$, ונסיק $i = j$.
	לכן $i$ הומיאומורפיזם יחיד.
\end{proof}

\question{}
בשאלה זו נעסוק ב־$\RR P^n$.

\subquestion{}
נוכיח כי $\RR P^1$ הומיאומורפי ל־$S^1$.
\begin{proof}
	מטעמי נוחות נבחן את שני המרחבים מעל המרוכבים, נניח ש־$z \in \CC \setminus \{0\}$, אז $[z] = \{ t z \mid t \in \RR \setminus \{0\}\}$.
	נגדיר $f([z]) = z^2$ עבור הנציג המקיים $|z| = 1$. נבחין כי זוהי הגדרה עקבית, זאת שכן אם $z \sim w$ אבל $z \ne w$, אז $|z| = |w| = 1$ וכן $-z = w$, אז $z^2 = w^2$.
	העתקה זו כמובן גם חד־חד ערכית כפולינום מרוכב ועל מאותה הסיבה.
	זוהי גם העתקה רציפה ל־$\CC$ ולכן גם לצמצום של המרחב.

	נרצה אם כך להראות שהיא פתוחה.
	נניח ש־$U^* \subseteq \RR P^1$ קבוצה פתוחה, אז נבחר את $U \subseteq \CC$ קבוצת הנציגים של $U^*$ כך ש־$\forall x \in U, |x| = 1$.
	נגדיר גם ש־$x, -x \in U$, ונקבל קבוצה פתוחה ב־$\CC$, ולכן תמונתה פתוחה אף היא ב־$\CC$ וכן צמצומה פתוח.
	נסיק ש־$f$ היא הומיאומורפיזם.
\end{proof}

\subquestion{}
יהי $n \in \NN$, נראה ש־$\RR P^n$ הוא מנה של $S^n$.
\begin{proof}
	נגדיר את ההעתקה $f : S^n \to \RR P^n$ על־ידי $f([x]) = \{ t x \mid t \in \RR \setminus \{0\}\}$.
	זוהי העתקה מוגדרת היטב מטעמי כיסוי כל הנציגים בישר, והיא על ישירות מבחירת נציג בטווח.

	אנו רוצים להראות כי $U \subseteq \RR P^n$ היא פתוחה אם ורק אם $f^{-1}(U)$ פתוחה.
	הטופולוגיות על שני המרחבים בנויים על מנה של $\RR^{n + 1}$, ולכן אם $U$ קבוצה פתוחה אז בהכרח $\bigcup [U]$ פתוחה ב־$\RR^{n + 1}$, אבל במקרה זה נובע ישירות ש־$\bigcup [U] / \sim$ עבור היחס המשרה את $S^n$ פתוחה אף היא.
	נסיק כי אכן מתקיימת הדרישה, ולכן $\RR P^n$ הוא מנה של $S^n$.
\end{proof}

\end{document}
