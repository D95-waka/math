\input{../article_base.tex}
\title{פתרון מטלה 04 --- מבוא לטופולוגיה, 80516}
% chktex-file 9
% chktex-file 17

\begin{document}
\maketitle
\maketitleprint[purple]

\question{}
יהי $X$ מרחב טופולוגי ותהי $A \subseteq X$ קשירה.
נראה שכל $B$ המקיימת $A \subseteq B \subseteq \overline{A}$ היא קשירה.
\begin{proof}
	נניח בשלילה ש־$B$ לא קשירה, ויהיו $B \subseteq B_0 \uplus B_1$ קבוצות פתוחות זרות המעידות על כך.
	בלי הגבלת הכלליות נניח ש־$A \subseteq B_0$ (לא יתכן שגם $A \cap B_1 \ne \emptyset$).
	$B_1$ פתוחה ולכן $B_1^C$ סגורה ובפרט $\overline{A} \setminus B_1$ קבוצה סגורה המכילה את $A$, אבל $\overline{A}$ מינימלית בהכלתה את $A$,
	לכן נובע $\overline{A} = \overline{A} \setminus B_1$, כלומר $B_1 \cap \overline{A} = \emptyset$.
	נסיק אם כן ש־$B_1 = \emptyset$ בלבד, אבל $B_0, B_1$ לא ריקות מהגדרת הקשירות, ולכן קיבלנו סתירה.
\end{proof}

\question[3]
ניזכר ונגדיר שמרחב מטרי חסום אם הוא מוכל בכדור.

\subquestion{}
יהי $(X, \rho)$ מרחב מטרי קומפקטי,
נראה שהוא חסום.
\begin{proof}
	יהי $\epsilon > 0$ ונגדיר $U_x = B(x, \epsilon)$ לכל $x \in X$.
	אז $\bigcup_{x \in X} U_x = X$ הוא כיסוי פתוח, ולכן מקומפקטיות יש תת־כיסוי סופי $X = \bigcup_{i = 1}^n U_x$ עבור $x_1, \ldots, x_n \in X$ ו־$n \in \NN$.
	נגדיר $d = \sup_{i \le n} \rho(x_1, x_i)$ ונבחן את $B(x_1, r + \epsilon)$.
	כל נקודה $x \in X$ מקיימת $x \in B(x_i, \epsilon)$ עבור איזשהו $i$, וכן $\rho(x_1, x) \le \rho(x_1, x_i) + \rho(x_i, x) \le r + \epsilon$ ולכן $x \in B(x_1, r + \epsilon)$, כלומר כדור זה חוסם את $X$.
\end{proof}

\subquestion{}
נמצא דוגמה למרחב מטרי חסום שאינו קומפקטי.
\begin{solution}
	נתחיל בהבחנה שאנו יודעים, במרחבים מטריים קומפקטיות שקולה לקומפקטיות סדרתית, לכן מספיק שנמצא מרחב מטרי חסום ולא קומפקטי סדרתי.
	נגדיר $X = (0, 1)$ יחד עם המטריקה הסטנדרטית $\rho(x, y) = |x - y|$, ונבחין כי $X = B(\frac{1}{2}, \frac{1}{2})$.
	נגדיר ${\{ x_n \}}_{n = 1}^\infty \subseteq X$ על־ידי $x_n = \frac{1}{n}$,
	ב־$\RR$ נקבל ש־$\lim_{n \to \infty} x_n = 0$, לכן $\{ x_n \}$ לא מתכנסת ב־$X$, ואילו יש לה תת־סדרה מתכנסת, אז היא מתכנסת ל־$0$ ולכן לא מתכנסת ב־$X$ גם, כלומר הסדרה מעידה על אי־קומפקטיות סדרתית.
	נסיק אם כך ש־$(X, \rho)$ הוא מרחב מטרי חסום אך לא קומפקטי.
\end{solution}

\question{}
נראה שאם $X$ אינסופית אז טופולוגיית המשלים בן־המנייה,
כלומר $A \subseteq X$ פתוחה אם ורק אם $|X \setminus A| \le \omega$,
היא לא קומפקטית.
\begin{proof}
	נסמן $|X| = \kappa$ ונגדיר $f : \kappa \to X$ כדי להימנע משימוש בבחירה.
	וידוע ש־$\kappa \ge \omega$ מהגדרת $\omega$ כסודר אינסופי מינימלי.
	לכן נוכל להסיק גם ש־$\kappa = \kappa + \omega$, כלומר קיימת תת־קבוצה $Y \subseteq X$ כך ש־$|X \setminus Y| = \omega$ וכן $|Y| = \kappa$.
	נגדיר $f : \omega \to X \setminus Y$ העתקה חד־חד ערכית ועל,
	ונגדיר $U_n = Y \cup \{ f(n) \}$ לכל $n < \omega$.
	נבחין כי $|X \setminus U_n| = |(X \setminus Y) \setminus \{ f(n) \}| = |X \setminus Y| = \omega$ ולכן $U_n$ קבוצה פתוחה לכל $n < \omega$.
	נובע אם כך,
	\[
		\bigcup_{n < \omega} U_n
		= \bigcup_{n < \omega} Y \cup \{ f(n) \}
		= Y \cup \bigcup_{n < \omega} \{ f(n) \}
		= Y \cup f(\NN)
		= X
	\]
	כלומר זהו כיסוי פתוח של $X$.
	לכל תת־קבוצה סופית $I \subseteq \omega$ נבחר $m = S(\sup I)$ ונקבל $f(m) \notin \bigcup_{n \in I} U_n$, כלומר מצאנו איבר שלא נמצא בתת־כיסוי סופי המושרה מ־$I$ לכל $I$ שנבחר.
	נסיק ש־$X$ לא קומפקטית.
\end{proof}

\question{}
יהי $X$ מרחב טופולוגי קומפקטי ותהי ${\{ x_n \}}_{n = 0}^\infty \subseteq X$ סדרה,
נסמן $A_N = \{ x_n \mid N \le n \in \NN \}$.
נראה ש־$\bigcap_{N = 0}^\infty \overline{A_N} \ne \emptyset$.
\begin{proof}
	מהתנאי השקול לקומפקטיות, לכל אוסף קבוצות סגורות של $X$ אשר מקיים את תכונת החיתוך הסופי, יש חיתוך לא ריק.
	נרצה אם כך להראות שתכונת החיתוך הסופי חלה.
	נניח ש־$I \subseteq \NN$ כך ש־$|I| < \omega$.
	נגדיר $M = \sup I$, מסופיות נובע ש־$M$ מתקבל, וכן מהגדרת $A_N$ נקבל שלכל $N \in I$ גם $N \le M$ ולכן $x_M \in A_N$.
	נובע ש־$x_M \in \bigcap_{i \in I} \overline{A_i}$, כלומר החיתוך שלהם לא ריק, ובהתאם תכונת החיתוך הסופי אכן נשמרת.
	נסיק אם כך שגם $\bigcap_{N = 0}^\infty \overline{A_N} \ne \emptyset$.
\end{proof}

\question{}
תהי קבוצה $S$ לא ריקה, ו־$D \subseteq \Pp(S)$ משפחת קבוצות המקיימת את תכונת החיתוך הסופי.
נוכיח שקיים מסנן $\langle D \rangle$ כך ש־$D \subseteq \langle D \rangle$ וכך שאם $F$ מסנן המקיים $D \subseteq F$, אז גם $\langle D \rangle \subseteq F$.
\begin{proof}
	נתחיל בהוכחה עבור המקרה $D = \emptyset$.
	במקרה זה נגדיר $\langle D \rangle = \{ S \}$, ונבחין כי אוסף זה מקיים $S \in \langle D \rangle, \emptyset \notin \langle D \rangle$, וכן מקיים באופן ריק סגירות לחיתוך ולקבוצות מכילות.
	כל מסנן $D \subseteq F \subseteq \Pp(S)$ מקיים $S \in F$ ולכן $\langle D \rangle \subseteq F$ ולכן סיימנו.

	נניח ש־$D \ne \emptyset$.
	נגדיר את הקבוצה $\langle D \rangle$ כסגירות לחיתוך ולהוספת איברים מעל $D$.
	כלומר נגדיר $D \subseteq \langle D \rangle$ וכן לכל $A, B \in \langle D \rangle$ גם $A \cap B \in \langle D \rangle$ וכן לכל $\langle D \rangle \ni A \subseteq B \subseteq S$ גם $B \in \langle D \rangle$.
	נראה ש־$\langle D \rangle$ מסנן.
	ידוע ש־$D$ לא ריקה ולכן קיים $A \in D$, אבל $A \subseteq S$ ולכן $S \in \langle D \rangle$.
	אם $\emptyset \in D$ אז $D$ לא מקיימת את תכונת החיתוך הסופי, לכן נסיק ש־$\emptyset \notin D$, וכן לכל $A, B \in D$ ידוע ש־$A \cap B \ne \emptyset$, ואף לכל $A \in \langle D \rangle$ לא מתקיים $A \subsetneq \emptyset$,
	לכן נוכל להסיק ש־$\emptyset \notin \langle D \rangle$.
	הקבוצה $\langle D \rangle$ סגורה לחיתוך סופי מהגדרתה ככזו, וכן סגורה ללקיחת קבוצות מכילות גם מהגדרה, לכן נסיק ש־$\langle D \rangle$ אכן מסנן. \\
	נניח ש־$D \subseteq F$ מסנן כלשהו מעל $S$.
	לכל $A, B \in D$ גם $A, B \in F$ ולכן $A \cap B \in F$, וכן לכל $A \in D$ ו־$A \subseteq B$ גם $B \in F$, ולכן לכל $A \in \langle D \rangle$ נסיק ש־$A \in F$, כלומר $\langle D \rangle \subseteq F$.
\end{proof}

\question{}
תהי קבוצה $S$ ויהי $F \subseteq \Pp(S)$ מסנן על $S$.
יהיו $R$ קבוצה נוספת ו־$f : S \to R$ פונקציה כלשהי.
נגדיר $f_* F = \{ A \subseteq R \mid f^{-1}(A) \in F \}$, ונוכיח כי זהו מסנן מעל $R$.
\begin{proof}
	$f^{-1}(R) = S \in F$ מהגדרת $f$, לכן נובע ש־$R \in f_* F$.
	נבחין גם כי $\emptyset \notin F$ ולכן $f^{-1}(\emptyset) = \emptyset \notin f_* F$. \\
	נניח ש־$A, B \in f_* F$, אז $f^{-1}(A) \cap f^{-1}(B) \in F$ מהגדרת $F$.
	יהי $s \in f^{-1}(A) \cap f^{-1}(B)$, אז $f(s) \in A, B$ ונובע שבפרט גם $f(s) \in A \cap B$.
	נקבל אם כך ש־$s \in f^{-1}(f(s)) \subseteq f^{-1}(A \cap B)$, ולכן $f^{-1}(A) \cap f^{-1}(B) \subseteq f^{-1}(A \cap B)$ ולכן $f^{-1}(A \cap B) \in F$ ו־$A \cap B \in f_* F$. \\
	נניח ש־$A \in f_* F$, בהתאם $f^{-1}(A) \in F$, ונניח ש־$A \subseteq B$, אז גם $f^{-1}(A) \subseteq f^{-1}(B) \in F$ ולכן $B \in f_* F$. \\
	מצאנו אם כן ש־$f_* F$ מסנן מעל $R$.
\end{proof}

\end{document}
