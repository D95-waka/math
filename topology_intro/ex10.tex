\input{../article_base.tex}
\title{פתרון מטלה 10 --- מבוא לטופולוגיה, 80516}
% chktex-file 9
% chktex-file 17

\usepackage{pgfplots}
\pgfplotsset{width=7cm,compat=1.18}

\begin{document}
\maketitle
\maketitleprint[purple]

\question{}
נניח ש־$0 < n \in \NN$,
נראה כי התנאים הבאים שקולים,
\begin{enumerate}
	\item לכל $f : S^n \to \RR^n$ רציפה קיים $x \in S^n$ כך ש־$f(x) = f(-x)$
	\item לא קיימת פונקציה רציפה אי־זוגית $S^n \to S^{n - 1}$
\end{enumerate}
\begin{proof}
	$1 \implies 2$,
	נניח בשלילה שקיימת $g : S^n \to S^{n - 1}$ אי־זוגית, קרי $f(-x) = -f(x)$ לכל $x \in S^n$.
	תהי פונקציה המהווה הרחבה רציפה של $g$, $f : S^n \to \RR^n$ (קיימת מהלמה של אוריסון), אז קיימת $x \in S^n$ כך ש־$f(x) = f(-x)$, אבל $f \restriction S^n = g$ ולכן $g(x) \in S^{n - 1}$ ובפרט $g(x) = g(-x) = -g(x)$.
	נסיק אם כך ש־$g(-x) = g(x) = 0$ בלבד, אבל $0 \notin S^{n - 1}$ בסתירה לקיום $g$.

	$2 \implies 1$,
	תהי $f : S^n \to \RR^n$.
	נגדיר,
	\[
		g(x)
		= \frac{f(x) - f(-x)}{2}
	\]
	נבחין כי הפונקציה $g$ היא אי־זוגית.
	אילו $g(x) = 0$ ל־$x$ כלשהו, אז $f(x) = f(-x)$ וסיימנו, לכן נניח אחרת.
	לכן בפרט נוכל להגדיר,
	\[
		h(x)
		= \frac{g(x)}{\lVert g(x) \rVert}
	\]
	ונקבל פונקציה $h : S^n \to S^{n - 1}$ רציפה, ובהכרח גם אי־זוגית, בסתירה ישירה להנחה שלנו, ולכן $h$ לא קיימת, ובהתאם $g$ בעלת שורש.
\end{proof}

\question{}
נניח את משפט בורסוק־אולם ונוכיח שאין שיכון $S^n \hookrightarrow \RR^n$.
\begin{proof}
	נניח בשלילה ש־$f : S^n \hookrightarrow \RR^n$ שיכון.
	אז קיים $x \in S^n$ כך ש־$f(x) = f(-x)$, אבל מחד־חד ערכיות השיכון $x \ne -x \implies f(x) \ne f(-x)$ ולכן $x = 0$ בלבד, אבל $0 \notin S^n$, וזו סתירה להנחה שלנו.
\end{proof}

\question{}
נגדיר את $S^1 = \{ z \in \CC \mid |z| = 1 \}$ ויהי פולינום $p(z) = z^n + a_{n - 1} z^{n - 1} + \cdots + a_0$.

\subquestion{}
אם $q : \CC \to \CC$ רציפה כך ש־$|z| = r \implies q(z) \ne 0$ עבור $r \ge 0$ כלשהו.
אז הפונקציה,
\[
	f_r^q : I \to S^1,
	\quad
	f_r^q(s)
	= \frac{q(r e^{2 \pi i s}) / q(r)}{|q(r e^{2 \pi i s}) / q(r)|}
\]
היא פונקציה רציפה.

\subquestion{}
יהי $R > |a_{n - 1}| + \cdots + |a_0| + 1$,
לכל $0 \le t \le 1$ נגדיר פולינום,
\[
	p_t(z)
	= z^n + t(a_{n - 1} z^{n - 1} + \cdots + a_0)
\]
ונגדיר את הפונקציה $h : I^2 \to S^1$ המוגדרת על־ידי $h(t, s) = f_R^{p_t}(s)$.
נראה כי $h$ היא הומוטופיה מ־$s \mapsto e^{2 \pi i n s}$ ל־$f_R^p$.
\begin{proof}
	ברור כי $h$ רציפה, ולכן עלינו רק לבדוק את התנאים להומוטופיה.
	\[
		h(0, s)
		= f_R^{z^n}(s)
		= \frac{R^n e^{2 \pi i n s} / R^n}{|R^n e^{2 \pi i n s} / R^n|}
		= \frac{e^{2 \pi i n s}}{|e^{2 \pi i n s}|}
		= e^{2 \pi i n s}
	\]
	מהצד השני,
	\[
		h(1, s)
		= f_R^p
	\]
	ישירות מאיך שהגדרנו את $p_t$ וההתלכדות $p_1 = p$.
	לבסוף גם,
	\[
		h(t, 0)
		= f_R^{p_t}(0)
		= \frac{p_t(R) / p_t(R)}{|p_t(R) / p(R)|}
		= 1
	\]
	ובאופן דומה נקבל גם $h(t, 1) = 1$.
\end{proof}

\subquestion{}
נסיק שאם $n \ge 1$ אז $f_R^p$ לא הומוטופית ללולאה קבועה.
\begin{proof}
	הומוטופיה היא יחס טרנזיטיבי, לכן מספיק שנראה שאין הומוטופיה בין הלולאה $t \mapsto e^{2 \pi i n t}$ ללולאה הקבועה $0$.
	למעשה זו טענה שהוכחה בכיתה על־ידי שימוש בכיסוי של המעגל על־ידי הישר הממשי.
\end{proof}

\subquestion{}
נסיק את המשפט היסודי של האלגברה.
\begin{proof}
	נניח בשלילה שהוא לא מתקיים, כלומר קיים פולינום ממעלה חיובית אשר אין לו שורש.
	נסמן פולינום זה כ־$p$, ונבחין כי $f_R^p$ מוגדרת לכל $s$, ונגדיר,
	\[
		h(t, s)
		= f_{\gamma(s)}^p(t)
	\]
	עבור מסילה $\gamma = [0, R]$.
	זוהי כמובן פונקציה רציפה ישירות מהגדרה ולכן יש הומוטופיה בין $f_0^p$ לבין $f_R^p$.
	אבל זו סתירה לסעיף הקודם, ולכן נסיק שלא קיים פולינום כזה.
\end{proof}

\end{document}
