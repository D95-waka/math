\input{../article_base.tex}
\title{פתרון מטלה 01 --- אנליזה פונקציונלית, 80417}

\begin{document}
\maketitle
\maketitleprint{}

\question{}
תהי $A \subseteq \RR$.
נראה כי $A$ חסומה לחלוטין אם ורק אם היא חסומה.
בפרט, אם $I \subseteq \RR$ קטע סגור וחסום אז לכל $\epsilon > 0$  קיימים קטעים סגורים $I_1, \dots, I_n \subseteq I$ כך ש־$I = \bigcup_{i = 1}^n I_i$ ואורכו של כל $I_i$ הוא לכל היותר $\epsilon$.
\begin{proof}
	נניח ש־$A$ חסומה לחלוטין, ונניח ש־${\{ B_\epsilon(x_i) \}}_{i = 1}^n \subseteq \RR$ קבוצת כדורים המעידים על חסימות בהחלט זו עבור $\epsilon > 0$ קבוע כלשהו.
	נבחר את $A \subseteq \bigcup_{i = 1}^n B_\epsilon(x_i)$, זהו איחוד של קבוצות פתוחות ולכן פתוח, וזהו איחוד סופי של קבוצות חסומות ולכן חסומה, וקבוצה זו מכילה ומעידה על חסימות של $A$ מחסימות בהחלט, ולכן קיבלנו ש־$A$ חסומה.

	נניח עתה ש־$A$ חסומה ויהי $\epsilon > 0$.
	קיים $[-r, r] \supseteq A$ עבור $r > 0$ כלשהו כהיסק ישיר מהחסימות של $A$, ונגדיר ${\{ x_i \}}_{i = 0}^n$ כך ש־$x_i = -r + i \frac{\epsilon}{2}$ עבור $n \in \NN$ סופי כך ש־$n > \frac{4r}{\epsilon}$.
	עתה נגדיר $I = \bigcup_{i = 0}^n B_\epsilon(x_i)$, מבדיקה ישירה נקבל $A \subseteq [-r, r] \subseteq I$, כלומר $I$ היא $\epsilon$־כיסוי של $A$.
	נסיק אם כך ש־$A$ חסומה לחלוטין.

	נניח עתה ש־$I$ קטע חסום, לכן קיימים לכל $\epsilon_0 = \frac{\epsilon}{2}$ קבוצת קטעים $I_1, \dots, I_n$ כך ש־$I = \bigcup I_i$ כך ש־$l(I_i) = l(B_{\epsilon_0}(x_i)) = \epsilon$, ובהתאם נוכל להסיק את התנאי הנוסף.
\end{proof}

\question{}
מרחב מטרי $X$ נקרא ספרבילי אם קיימת קבוצה בת־מניה $A \subseteq X$ שצפופה ב־$X$. \\
נוכיח שאם מרחב מטרי חסום לחלוטין אז הוא ספרבילי.
\begin{proof}
	ידוע כי $X$ חסומה לחלוטין ולכן לכל $\epsilon > 0$ קיימת קבוצה נקודות $D_\epsilon$ כך ש־$X \subseteq \bigcup_{x \in D_\epsilon} B_\epsilon(x)$.
	בפרט לכל $n \in \NN$ קיימת $D_{\frac{1}{n}}$ סופית כזו, ונגדיר $D = \bigcup_{n \in \NN} D_{\frac{1}{n}}$.
	מתקיים $D \subseteq X$, ואנו נראה ש־$D$ אף צפופה ב־$X$.
	תהי $x \in X$ כלשהי, אם $x \in D$ אז סיימנו, ולכן נניח ש־$x \notin D$.
	לכל $n \in \NN$ קיים $y \in D_{\frac{1}{n}}$ כך ש־$x \in B_{\frac{1}{n}}(y)$ מהחסימות בהחלט, ולכן נגדיר $x_n = y$ לכל $n$.
	לכן ${\{ x_n \}}_{n = 1}^\infty \subseteq D$, וכן ישירות מהגדרת הגבול מתקיים $\lim_{n \to \infty} x_n = x$.
	לכן $x \in \overline{D}$, כלומר $\overline{D} = X$, לכן $D$ צפופה ב־$X$.
\end{proof}

\question{}
יהי $(X, \rho)$ מרחב מטרי.
נראה שקבוצה $A \subseteq X$ היא חסומה לחלוטין אם ורק אם לכל סדרה ${\{ x_n \}}_{n = 1}^\infty \subseteq A$ יש תת־סדרה ${\{ x_{n_k} \}}_{k = 1}^\infty \subseteq \{ x_n \}$,
כך שלכל $k \in \NN$ מתקיים $\rho(x_{n_k}, x_{n_{k + 1}}) < 2^{-k}$.
\begin{proof}
	נניח ש־$A$ חסומה לחלוטין ותהי סדרה ${\{ x_n \}}_{n = 1}^\infty \subseteq A$.
	עבור $m = 1$ קיים כיסוי של $m$־כדורים ל־$A$, משובך היונים קיים כדור כך שיש אינסוף איברי $x_n$ בכדור, נגדיר $y_1$ מרכז כדור זה.
	תת־קבוצה של קבוצה חסומה לחלוטין היא חסומה לחלוטין אף היא, זאת שכן כיסוי של הקבוצה הגדולה יותר הוא בפרט כיסוי של תת־הקבוצה,
	לכן עבור $m = 2$ קיים $\frac{1}{2}$־כיסוי של $B_1(y_1)$, ובאחד הכדורים בכיסוי יש אינסוף נקודות של $\{x_n\}$, נסמן ב־$y_2$ את מרכז כדור זה.
	נמשיך ונגדיר כך סדרה ${\{ y_m \}}_{m = 1}^\infty \subseteq A$ כך שלכל $m \in \NN$ יש אינסוף נקודות של $\{ x_n \}$ ב־$B_{2^{-m}}(y_m)$.
	לבסוף נגדיר תת־סדרה של $\{x_n\}$ כך ש־$x_{n_k} \in B_{2^{-k}}(y_k)$ לכל $k \in \NN$.
	זוהי תת־סדרה מוגדרת היטב מקיומם של אינסוף איברי $\{x_n\}$ בכל כדור כזה, לכן בפרט קיימת סדרת אינדקסים מונוטונית.
	מהגדרת תת־הסדרה גם נובע שלכל $k$, $\rho(x_{n_k}, x_{n_{k + 1}}) < 2^{-k}$, זאת שכן שתי הנקודות מוכלות בכדור $B_{2^{-k}}(y_k)$.

	נניח עתה את הכיוון ההפוך.
	נניח בשלילה שקיים $\epsilon > 0$ עבורו אין $\epsilon$־כיסוי בכדורים של $A$, אז נבחר סדרה ${\{ x_n \}}_{n = 1}^\infty \subseteq A$ המעידה על כך, כלומר מתקיים $\rho(x_n, x_m) \ge \epsilon$ לכל $n, m \in \NN, n \ne m$.
	לסדרה זו יש תת־סדרה כך ש־$\rho(x_{n_k}, x_{n_{k + 1}}) < 2^{-k}$ לכל $k \in \NN$, בפרט לכל $k > - \log \epsilon$, אבל במקרה זה מתקיים, $2^{-k} < 2^{\log \epsilon} = \epsilon$.
	לכן יש אינסוף נקודות של $\{ x_n \}$ כך שמרחקן קטן מ־$\epsilon$, וזו סתירה להגדרת הסדרה, לכן אין $\epsilon > 0$ כזה, ובהתאם $A$ חסומה לחלוטין.
\end{proof}

\question{}
נתבונן במרחב $(C([-\pi, \pi]), \lVert \cdot \rVert_2)$, מרחב הפונקציות הרציפות $[-\pi, \pi] \to \RR$ עם הנורמה,
\[
	\lVert f \rVert_2
	= \sqrt{\int_{-\pi}^{\pi} f^2(x)\ dx}
\]
נוכיח כי במרחב זה הסדרה ${\{ \sin(nt) \}}_{n = 1}^\infty$ היא חסומה אבל לא חסומה לחלוטין.
\begin{proof}
	תחילה נוכיח את חסימות סדרת הפונקציות הנתונה,
	\[
		\forall n \in \NN,\ 
		\lVert f \rVert_2
		= \sqrt{\int_{-\pi}^{\pi} \sin^2(x)\ dx}
		\overset{(1)}{\le} \sqrt{\int_{-\pi}^{\pi} 1\ dx}
		= \sqrt{2\pi}
	\]
	כאשר $(1)$ נובע מתכונות אינטגרלים.

	נעבור להוכחה שלסדרה זו אין תת־סדרת קושי.
	נניח ש־$\{ \sin(n_k t) \}$ תת־סדרת קושי, אז סדרת הנורמות מתכנסת לאפס, לכן לכל $\epsilon > 0$ קיים $N \in \NN$ עבורו
	\[
		\forall n, m \in \{ n_k \mid k > N \},\ \int_{-\pi}^{\pi} {(\sin(nt) - \sin(mt))}^2\ dt < \epsilon^2
		\implies \int_{-\pi}^{\pi} \sin^2(nt) + \sin^2(mt) - 2\sin(nt)\sin(mt)\ dt < \epsilon^2
	\]
	אבל $\int_{-\pi}^{\pi} 2\sin(nt)\sin(mt)\ dt = \im 4 \int_0^{n\pi} \exp(it + i \frac{m}{n} t)\ dt \to 0$ ונסיק שעבור בחירת $k$ מספיק גדולה, מחובר זה זניח.
	נחשב את האינטגרל $\int_{-\pi}^{\pi} \sin^2(nt)\ dt$,
	\[
		\int_{-\pi}^{\pi} \sin^2(nt)\ dt
		= \frac{1}{n} \int_{-n\pi}^{n\pi} \sin^2(t)\ dt
		= \frac{1}{2n} \int_{-n\pi}^{n\pi} 1 - \cos(2t)\ dt
		= \frac{1}{2n} \cdot {\left[ t - \frac{1}{2} \sin(2t) \right]}_{t = -n\pi}^{t = n\pi}
		= \frac{2n\pi}{2n}
		= \pi
	\]
	לכן נסיק,
	\[
		\int_{-\pi}^{\pi} \sin^2(nt) + \sin^2(mt) - 2\sin(nt)\sin(mt)\ dt
		= 2\pi - 2\int_{-\pi}^{\pi} \sin(nt)\sin(mt)\ dt
		\to 2\pi
		< \epsilon^2
	\]
	ולכן עבור בחירה $\epsilon = 1$ נקבל סתירה.
	בהתאם נוכל להסיק שאין תת־סדרת קושי לסדרה שבחרנו, ובפרט ממשפט השקילות לחסימות בהחלט (או לחלופין וריאציה של שאלה 3) נובע שהסדרה לא חסומה לחלוטין.
\end{proof}

\end{document}
