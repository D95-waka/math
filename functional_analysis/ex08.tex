\input{../article_base.tex}
\title{פתרון מטלה 08 --- אנליזה פונקציונלית, 80417}

\DeclareMathOperator\sgn{sgn}

\begin{document}
\maketitle
\maketitleprint[teal]

\question[2]
נחשב את מקדמי טור פורייה של הפונקציה האינטגרבילית,
\[
	f(x)
	= \sgn(x) \cdot \frac{\pi - |x|}{2}
\]
עבור,
\[
	\sgn(x)
	= \begin{cases}
		-1 & x < 0 \\
		0 & x = 0 \\
		1 & x > 0
	\end{cases}
\]
\begin{solution}
	נבחין כי הפונקציה $f$ היא אי־זוגית, לכן $a_n = 0$ לכל $n \in \NN \cup \{ 0 \}$.
	מספיק שנחשב את $b_n$ אם כך,
	\begin{align*}
		b_n
		& = \frac{1}{\pi} \int_{-\pi}^{\pi} f(x) \cdot \sin(nx)\ dx \\
		& = \frac{2}{\pi} \int_{0}^{\pi} f(x) \cdot \sin(nx)\ dx \\
		& = \frac{1}{\pi} \int_{0}^{\pi} \sgn(x) (\pi - |x|) \cdot \sin(nx)\ dx \\
		& = \frac{1}{\pi} \int_{0}^{\pi} (\pi - x) \sin(nx)\ dx \\
		& = \frac{1}{\pi} \left( \pi \int_{0}^{\pi} \sin(nx)\ dx - \int_{0}^{\pi} x \sin(nx)\ dx \right) \\
		& = {\left[ -\frac{1}{n} \cos(nx) \right]}_{x = 0}^{x = \pi} - \frac{1}{\pi} {\left[ -x \cdot \frac{1}{n} \cos(nx) \right]}_{x = 0}^{x = \pi} - \frac{1}{n \pi} \int_{0}^{\pi} \cos(nx)\ dx  \\
		& = \mp \frac{1}{n} + \frac{1}{n} \pm \frac{1}{n} - \frac{1}{n^2 \pi} {\left[ \sin(nx) \right]}_{x = 0}^{x = \pi} \\
		& = \frac{1}{n}
	\end{align*}
	ולכן,
	\[
		\sum_{n = 1}^\infty \frac{1}{n} \sin(nx)
	\]
	טור פורייה של $f$.
\end{solution}

\question{}
נגדיר טור ונחשב את סכומו על־ידי שימוש בזהות פרסבל.

\subquestion{}
\[
	\sum_{n = 1}^\infty \frac{1}{n^2}
\]
ונסיק את ערך הטור $\sum_{n = 1}^\infty \frac{1}{{(2n - 1)}^2}$.
\begin{solution}
	נגדיר את הפונקציה $f(x) = x$,
	אז מאי־זוגיות $a_n = 0$ לכל $n$, ונחשב את $b_n$,
	\[
		b_n
		= \frac{1}{\pi} \int_{-\pi}^{\pi} x \sin(nx)\ dx
		= \frac{2}{\pi} \int_{0}^{\pi} x \sin(nx)\ dx
		= \mp \frac{2}{\pi n}
	\]
	ולכן משוויון פרסבל,
	\[
		\frac{\pi}{4} \cdot \frac{1}{\pi} \int_{-\pi}^{\pi} f^2(x)\ dx
		= 0 + \sum_{n = 1}^\infty 0 + \frac{1}{n^2}
		= \sum_{n = 1}^\infty \frac{1}{n^2}
	\]
	אבל,
	\[
		\frac{\pi}{4} \int_{-\pi}^{\pi} f^2(x)\ dx
		= \frac{\pi}{4} \int_{-\pi}^{\pi} x^2\ dx
		= \frac{2 \pi}{12} \pi^3
		= \frac{\pi^2}{6}
	\]

	נבחין כי,
	\[
		\frac{\pi^2}{24}
		= \sum_{n = 1}^\infty \frac{1}{4n^2}
		= \sum_{n = 1}^\infty \frac{1}{{(2n)}^2}
	\]
	ולכן,
	\[
		\sum_{n = 1}^\infty \frac{1}{{(2n - 1)}^2}
		= \sum_{n = 1}^\infty \frac{1}{n^2} - \sum_{n = 1}^\infty \frac{1}{{(2n)}^2}
		= \frac{\pi^2}{6} - \frac{\pi^2}{24}
		= \frac{\pi^2}{8}
	\]
\end{solution}

\subquestion{}
נחשב את הטור,
\[
	\sum_{n = 1}^\infty \frac{1}{{(2n - 1)}^4}
\]
\begin{solution}
	נגדיר $f(x) = |x|$.
	זוהי פונקציה זוגית ולכן $b_n = 0$ לכל $n$.
	נחשב את $a_n$,
	\[
		a_n
		= \frac{1}{\pi} \int_{-\pi}^{\pi} |x| \cos(nx)\ dx
		= \frac{2}{\pi} \int_{0}^{\pi} x \cos(nx)\ dx
		= \frac{2}{\pi n^2} ({(-1)}^n - 1)
	\]
	וכן,
	\[
		a_0
		= \frac{1}{\pi} \int_{-\pi}^{\pi} f(x)\ dx
		= \frac{1}{\pi} \int_{0}^{\pi} 2x\ dx
		= \pi
	\]
	ומשוויון פרסבל,
	\[
		\int_{-\pi}^{\pi} f^2(x)\ dx
		= \frac{\pi^3}{2} + \sum_{n = 1}^\infty \frac{4}{\pi n^4} |{(-1)}^n - 1|
	\]
	כלומר,
	\[
		\sum_{n = 1}^\infty \frac{1}{{(2n - 1)}^4}
		= \sum_{n = 1}^\infty \frac{1}{n^4} |{(-1)}^n - 1|
		= \frac{\pi}{4} \int_{-\pi}^{\pi} f^2(x)\ dx - \frac{\pi^4}{8}
		= \frac{\pi^4}{6} - \frac{\pi^4}{8}
		= \frac{\pi^4}{24}
	\]
\end{solution}

\question{}
נוכיח את מבחן $M$ של ויירשטראס.
נניח ש־$I \subseteq \RR$ קטע, ${\{ f_n \}}_{n = 1}^\infty \subseteq C(I)$ ו־${\{ M_n \}}_{n = 1}^\infty \subseteq \RR$ סדרת חסמים כך שלכל $n \in \NN$ מתקיים $\lVert f_n \rVert_\infty < M_n$.
נראה שהתכנסות הטור $\sum_{n = 1}^\infty M_n$ גוררת שהפונקציה $f$ המוגדרת על־ידי,
\[
	f(x)
	= \sum_{n = 1}^\infty f_n(x)
\]
מוגדרת היטב ב־$I$, רציפה והטור מתכנס אליה במידה שווה.
\begin{proof}
	לכל $N \in \NN$ מתקיים,
	\[
		\sum_{n = 1}^N f_n(x)
		< \sum_{n = 1}^N M_n
	\]
	ישירות מהנתונים ולכל $x \in I$, לכן ממבחן הסנדוויץ' לגבולות ממשיים נקבל ש־$f$ מוגדרת בכל $I$ ושההתכנסות היא במידה שווה (שכן לא הייתה תלות ב־$x$). \\
	נותר להראות ש־$f$ רציפה.
	יהי $\varepsilon > 0$, אז,
	\[
		|f(x) - f(y)|
		\le \left\lvert f(x) - \sum_{n = 1}^N f_n(x) \right\rvert + \left\lvert \sum_{n = 1}^N f_n(x) - \sum_{n = 1}^N f_n(y) \right\rvert + \left\lvert \sum_{n = 1}^N f_n(y) - f(y) \right\rvert
	\]
	מהתכנסות במידה שווה קיים $N \in \NN$ (נבחר מקסימלי) כך שמתקיים,
	\[
		\left\lvert f(x) - \sum_{n = 1}^N f_n(x) \right\rvert
		\le \frac{\varepsilon}{3},
		\qquad
		\left\lvert f(y) - \sum_{n = 1}^N f_n(y) \right\rvert
		\le \frac{\varepsilon}{3}
	\]
	וכן מרציפות $f_n$ לכל $n \le N$ קיים $\delta > 0$ כך שלכל $|x - y| < \delta$ מתקיים,
	\[
		\left\lvert \sum_{n = 1}^N f_n(x) - \sum_{n = 1}^N f_n(y) \right\rvert
		\le \frac{\varepsilon}{3}
	\]
	ולכן $|f(x) - f(y)| \le \varepsilon$ ומצאנו כי $f$ רציפה במידה שווה ובפרט רציפה.
\end{proof}

\question{}
תהי $f \in \tilde{C}[-\pi, \pi]$ כך ש־$f' \in \tilde{C}[-\pi, \pi]$.
נסמן ב־$a_n^f, b_n^f$ את מקדמי טור פורייה של $f$ וב־$a_n', b_n'$ את מקדמי טור פורייה של $f'$.

\subquestion{}
נראה שלכל $n \in \NN$ מתקיים,
\[
	a_n^f
	= - \frac{1}{n} b_n',
	\qquad
	b_n^f
	= - \frac{1}{n} a_n'
\]
\begin{proof}
	על־ידי אינטגרציה בחלקים,
	\[
		a_n^f
		= \frac{1}{\pi} \int_{-\pi}^{\pi} f(x) \cos(nx)\ dx
		= \frac{1}{n \pi} {\left[ f(x) \sin(nx) \right]}_{x = -\pi}^{x = \pi} - \frac{1}{n \pi} \int_{-\pi}^{\pi} f'(x) \sin(nx)\ dx
	\]
	אבל $f'$ שווה בקצוות ו־$\sin$ אי־זוגית ולכן,
	\[
		a_n^f
		= - \frac{1}{n \pi} \int_{-\pi}^{\pi} f'(x) \sin(nx)\ dx
		= - \frac{1}{n} b_n'
	\]
	ההוכחה עבור $a_n^f$ זהה.
\end{proof}

\subquestion{}
ידוע כי ${( a_n' )}_{n = 1}^\infty, {( b_n )}_{n = 1}^\infty \in l^2$ ונסיק ש־${( a_n^f )}_{n = 1}^\infty, {( b_n^f )}_{n = 1}^\infty \in l^1$ ובפרט שטור פורייה של $f$ מתכנס במידה שווה.
\begin{proof}
	אנו יודעים כי הטור,
	\[
		\sum_{n = 1}^\infty {(a_n')}^2
	\]
	מתכנס, אבל מסעיף א',
	\[
		\sum_{n = 1}^\infty {(a_n')}^2
		= \sum_{n = 1}^\infty {(n b_n^f)}^2
	\]
	ממבחן ההשוואה,
	\[
		\frac{|b_n^f|}{{(n b_n^f)}^2}
		\xrightarrow{n \to \infty} 0
	\]
	ולכן גם,
	\[
		\sum_{n = 1}^\infty | b_n^f |
	\]
	הוא טור מתכנס, כלומר הטור $\sum_{n = 1}^\infty b_n^f$ מתכנס בהחלט.
	נוכל להראות באופן זהה שגם $\sum_{n = 1}^\infty a_n^f$ מתכנס בהחלט.
	נשים לב שישירות ממבחן $M$ של ויירשטראס יחד עם התכנסות טורי המקדמים נובע שטור פורייה של $f$ מתכנס במידה שווה.
\end{proof}

\subquestion{}
נסיק שהטור של $f$ מתכנס ל־$f$ במידה שווה.
\begin{proof}
	ראינו כבר כי הטור מתכנס במידה שווה, אבל עלינו להראות שהוא מתכנס ל־$f$.
	ב־$\lVert \cdot \lVert_2$ אנו כבר יודעים כי הטור מתכנס ל־$f$, כלומר,
	\[
		\left\lVert f(x) - \sum_{n = 1}^N a_n \cos(nx) + b_n \sin(nx) \right\rVert_2
		= \int_{-\pi}^{\pi} {\left\lvert f(x) - \sum_{n = 1}^N a_n \cos(nx) + b_n \sin(nx) \right\rvert}^2\ dx
		\xrightarrow{N \to \infty} 0
	\]
	אבל מתקיים,
	\begin{align*}
		& \int_{-\pi}^{\pi} {\left\lvert f(x) - \sum_{n = 1}^N a_n \cos(nx) + b_n \sin(nx) \right\rvert}^2\ dx \\
		= & \int_{-\pi}^{\pi} {\left\lvert f(x) + \sum_{n = 1}^N a_n \cos(nx) + b_n \sin(nx) \right\rvert} \cdot {\left\lvert f(x) - \sum_{n = 1}^N a_n \cos(nx) + b_n \sin(nx) \right\rvert}\ dx \\
		\le & \int_{-\pi}^{\pi} {\left\lvert \max\{ f(x) \} + \sum_{n = 1}^N a_n + b_n \right\rvert} \cdot {\left\lvert f(x) - \sum_{n = 1}^N a_n \cos(nx) + b_n \sin(nx) \right\rvert}\ dx \\
		= & {\left\lvert \max\{ f(x) \} + \sum_{n = 1}^N a_n + b_n \right\rvert} \cdot \int_{-\pi}^{\pi} {\left\lvert f(x) - \sum_{n = 1}^N a_n \cos(nx) + b_n \sin(nx) \right\rvert}\ dx \\
	\end{align*}
	ומהתכנסות במידה שווה של הטור נובע ישירות,
	\[
		\sup \left\lvert f(x) - \sum_{n = 1}^\infty a_n \cos(nx) + b_n \sin(nx) \right\rvert
		\xrightarrow{N \to \infty} 0
	\]
	כפי שרצינו.
\end{proof}

\end{document}
