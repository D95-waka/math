\input{../article_base.tex}
\title{פתרון מטלה 05 --- אנליזה פונקציונלית, 80417}

\begin{document}
\maketitle
\maketitleprint[teal]

\question{}
יהי $T : C[0, 1] \to \RR$ פונקציונל מתמטי (כלומר העתקה לינארית) רציף ונניח שלכל $k \in \NN \cup \{ 0 \}$ מתקיים,
\[
	T(x^k)
	= \frac{1}{k + 1}
\]
נראה שלכל $f \in C[0, 1]$ מתקיים,
\[
	T(f)
	= \int_{0}^{1} f(x)\ dx
\]
\begin{proof}
	ממשפט הקירוב של ויירשטראס אנחנו יודעים שקיימים ${\{ p_n \}}_{n = 1}^\infty \subseteq C[0, 1]$ סדרת פולינומים המתכנסים במידה שווה ל־$f$. \\
	אנו גם יודעים שאופרטורים לינאריים משמרים התכנסות במידה שווה, ולכן $T(p_n) \rightrightarrows T(f)$.
	אילו נניח ש־$p_n(x) = \sum_{i = 0}^N \alpha_n^i x^i$ אז נובע,
	\[
		T(p_n)
		= \sum_{i = 0}^N \alpha_n^i \cdot \frac{1}{k + 1}
		= \int_{0}^{1} \alpha_n^i \cdot x^i
	\]
	ולכן נובע ש־$T(p_n) = \int_{0}^{1} p_n(x)\ dx$ לכל $n \in \NN$.
	אבל אינטגרל משמר התכנסות במידה שווה ונובע,
	\[
		T(f)
		= \lim_{n \to \infty} T(p_n)
		= \lim_{n \to \infty} \int_{0}^{1} p_n(x)\ dx
		= \int_{0}^{1} f(x)\ dx
	\]
	ונסיק שהטענה אכן מתקיימת.
\end{proof}

\question{}
עבור $k \in \NN$,
נראה שאוסף הפולינומים צפוף במרחב $C^k[0, 1]$ בנורמה על $C^k$ המוגדרת על־ידי,
\[
	\lVert f \rVert_k
	= \sum_{j = 0}^k \lVert f^{(j)} - p_n^{(j)} \rVert_\infty
\]
\begin{proof}
	נוכיח את הטענה באינדוקציה על $k$.
	עבור $k = 0$ הטענה מתלכדת עם משפט הקירוב של ויירשטראס ולכן סיימנו.

	נניח שהטענה נכונה עבור $k - 1$ ונבדוק את הטענה עבור $k$. \\
	נגדיר ${\{ p_n \}}_{n = 1}^\infty \subseteq C^k[0, 1]$ סדרת פולינומים כך ש־$p_n \rightrightarrows f^{(k)}$.
	נגדיר את הסדרה ${\{ p_n^1 \}}_{n = 1}^\infty \subseteq C^k[0, 1]$ על־ידי,
	\[
		p_n^1(x) = f^{(k - 1)}(0) + \int_{0}^{x} p(x)\ dx
	\]
	אנו יודעים כי לכל פולינום אינסוף קדומות וכי אף הן פולינומים, וכן ש־$\frac{d}{dx} p_n^1 = p_n$ לכל $n \in \NN$.
	מטענה מתרגול 4 נובע,
	\[
		p_n^1
		\rightrightarrows f^{(k - 1)}(0) + \int_0^x f^{(k)}(x)\ dx
		= f^{(k - 1)}(x) + f^{k - 1}(0) - f^{k - 1}(0)
		= f^{(k - 1)}(x)
	\]
	אבל מהנחת האינדוקציה נובע ש־$p_n^i \rightrightarrows f^{(i)}$ לכל $0 \le i < k$ וסיימנו.
\end{proof}

\question{}
במרחב $C[0, 1]$ עם נורמת $\lVert \cdot \rVert_\infty$ נגדיר $W = \Sp\{ x, x^2, \ldots \}$ ו־$D = \{ f \in C[0, 1] \mid f(0) = 0 \}$. \\
נראה ש־$\overline{W} = D$.
\begin{proof}
	תהי סדרת פולינומים מתכנסת במידה שווה ${\{ p_n \}}_{n = 1}^\infty \subseteq W$, ונסמן $p_n \rightrightarrows f$, אז $\lim_{n \to \infty} p_n(0) = f(0)$.
	אבל אנו יודעים כי $x \mid p_n$ לכל $n \in \NN$ מהגדרת $W$ ולכן $p_n(0) = 0$ בלבד, ונסיק $f(0) = 0$, כלומר $f \in D$, ונקבל $\overline{W} \subseteq D$.

	ממשפט הקירוב של ויירשטראס אנו יודעים כבר כי קבוצת הפולינומים צפופה ב־$C[0, 1]$.
	תהי $f \in D$ ותהי ${\{ p_n \}}_{n = 1}^\infty \subseteq C[0, 1]$ סדרת פולינומים כך ש־$p_n \rightrightarrows f$ ממשפט הקירוב.
	נגדיר גם את סדרת הפונקציות ${\{ g_n \}}_{n = 1}^\infty \subseteq C[0, 1]$ על־ידי $g_n(x) = p_n(0)$.
	אנו יודעים ש־$p_n(0) \to f(0) = 0$ ולכן הסדרה $\{ g_n \}$ מתכנסת במידה שווה, ובהתאם גם $\{ p_n - g_n \}$ היא סדרת פולינומים מתכנסת במידה שווה.
	נסיק ש־$p_n - g_n \rightrightarrows f$, ונשאר להראות ש־$\{ p_n - g_n \} \subseteq D$.
	יהי $n \in \NN$ ונניח ש־$p_n(x) = \sum_{n = 0}^N \alpha_i^n x^n$, אז $g_n(x) = \alpha_0^n$ ובהתאם $(p_n - g_n)(x) = \sum_{i = 1}^N \alpha_i^n x^n$ ולכן $p_n \in W$.
	כלומר $f \in \overline{W}$.

	מצאנו אם כך ש־$\overline{W} = D$.
\end{proof}

\question{}
יהי $(K, \rho)$ מרחב מטרי קומפקטי.
נראה ש־$\operatorname{Lip}(K)$ צפופה ב־$C(K)$.
\begin{proof}
	נזכיר כי,
	\[
		\operatorname{Lip}(K)
		= \{ f \in C(K, \RR) \mid \exists L > 0, \forall x, y \in X, |f(x) - f(y)| \le L \cdot \rho(x, y) \}
	\]
	וראינו בתרגול כי זוהי אלגברה של פונקציות רציפות, אנו נראה כי היא אף מפרידה נקודות ולא מתאפסת ב־$K$.

	נניח ש־$x, y \in K$ וכן ש־$x \ne y$, ונרצה להראות שקיימת $f \in \operatorname{Lip}(K)$ כך ש־$f(x) \ne f(y)$.
	נגדיר $f(t) = \rho(t, x)$ לכל $t \in K$.
	באינפי 3 ראינו כי פונקציה זו רציפה, וכמובן מהגדרתה היא $1$־ליפשיצית, כלומר $f \in \operatorname{Lip}(K)$.
	אבל מהגדרת המטריקה $f(x) = \rho(x, x) = 0$ אבל $x \ne y \iff \rho(x, y) \ne 0 \iff f(y) \ne 0$, ולכן $f$ מפרידה את $x, y$.

	נראה כי $\operatorname{Lip}(K)$ לא מתאפסת ב־$K$.
	נגדיר את $f(x) = 1$, זוהי פונקציה קבועה ולכן $0$־ליפשיצית, אבל לכל $x \in K$ גם $f(x) \ne 0$, ולכן מעידה ש־$\operatorname{Lip}(K)$ לא מתאפסת ב־$K$.

	נסיק ממשפט סטון־ויירשטראס כי $\operatorname{Lip}(K)$ צפופה ב־$C(K)$.
\end{proof}

\end{document}
