\input{../article_base.tex}
\title{פתרון מטלה 02 --- אנליזה פונקציונלית, 80417}

\begin{document}
\maketitle
\maketitleprint{}

\question{}
תהי ${\{ f_n \}}_{n = 1}^\infty$ משפחה חסומה במידה אחידה של פונקציות אינטגרביליות רימן בקטע $[a, b]$, ונגדיר $F_n(x) = \int_{a}^{x} f_n(t)\ dt$. \\
נראה שקיימת תת־סדרה ${\{ F_{n_k} \}}_{k = 1}^\infty$ שמתכנסת במידה שווה על $[a, b]$.
\begin{proof}
	נתונה חסימות במידה אחידה, כלומר קיים $M$ אשר חוסם את כל הפונקציות $f_n$.
	מלינאריות האינטגרל נובע $\int_a^x f_n(t)\ dt \le \int_a^b M\ dt$, כלומר $F_n$ חסומה על־ידי $M' = (b - a)M$ לכל $n \in \NN$.
	נסיק אם כך ש־$\{ F_n \}$ חסומה במידה אחידה.
	נבחין עתה ש־$F_n$ וכן $F_n'$ חסומות במידה אחידה ולכן תנאי משפט ארצלה־אקסולי חלים ונובע ש־$\{ F_n \}$ רציפה במידה אחידה ולכן בפרט יש לה תת־סדרה מתכנסת במידה שווה.
\end{proof}

\question{}
נקבע $0 < K < \infty$ ו־$0 < d$, ונגדיר,
\[
	\operatorname{Lip}_{K, d}
	= \{ f \in C[0, 1] \mid \forall x, y \in [0, 1],\ |f(x) - f(y)| \le K {|x - y|}^d \}  
\] 

\subquestion{}
נראה שאם $d \le 1$, אז $\{ f \in \operatorname{Lip}_{K, d} \mid f(0) = 0 \}$ היא תת־קבוצה קומפקטית של $C[0, 1]$.
\begin{proof}
	נתון שלכל פונקציה בקבוצה, $f(0) = 0$, ולכן גם $|f(x) - f(0)| = |f(x)| \le K x^d \le K$, כלומר הקבוצה חסומה במידה אחידה.
	נבחין גם כי לכל $\epsilon > 0$ נוכל לבחור $\delta^d < \frac{\epsilon}{K}$ ולקבל שאם $|x - y| < \delta$ אז $|f(x) - f(y)| \le K{|x - y|}^d < \epsilon$, לכן הקבוצה גם רציפה במידה אחידה.
	לבסוף מרציפות וחסימות במידה אחידה ומשפט ארצלה נובע שיש תת־סדרה מתכנסת לכל סדרה בקבוצה, אבל משלמות $C[0, 1]$ נובע שיש גבול לתת־סדרה זו, ובהכרח לפונקציית הגבול יתקיים $f(0) = 0$, לכן היא שייכת לקבוצה.
	נקבל אם כך שמטענה מהתרגול הקבוצה אכן קומפקטית.
\end{proof}

\subquestion{}
נראה שאם $d > 1$ ו־$f \in \operatorname{Lid}_{K, d}$ אז $f$ קבועה.
\begin{proof}
	נבחין כי $|x - y| \le 1$ לכל $x, y \in [0, 1]$ ולכן גם מתקיים ${|x - y|}^d \le |x - y|$, כלומר מתקיים $|f(x) - f(y)| < K|x - y|$.
	זוהי למעשה ההגדרה ל־$K$־ליפשיציות, וראינו באינפי 2 שכל פונקציה כזו היא גזירה ושנגזרתה חסומה אף היא.
	נחשב את ערך הנגזרת בנקודה $x \in [0, 1]$,
	\[
		|f'(x)|
		= \lim_{h \to 0} \frac{|f(x + h) - f(x)|}{h}
		\le \lim_{h \to 0} \frac{{|x + h - x|}^d}{h}
		\le \lim_{h \to 0} h^{d - 1}
		= 0
	\]
	ולכן $f' = 0$ ובהתאם $f$ פונקציה קבועה.
\end{proof}

\question{}
נמצא סדרה ${\{ f_n \}}_{n = 1}^\infty$ של פונקציות רציפות על הקטע $[0, 1]$ שמתכנסות נקודתית לפונקציה לא רציפה $f$ על הקטע $[0, 1]$.
\begin{solution}
	נגדיר $f_n(x) = x^n$.
	לכל $x < 1$ הסדרה $x^n$ היא סדרה אפסה, אך עבור $x = 1$ נקבל $f_n(x) = 1$ לכל $n \in \NN$, לכן $f_n \to f$ עבור הפונקציה,
	\[
		f(x)
		= \begin{cases}
			0 & 0 \le x < 1 \\
			1 & x = 1
		\end{cases}
	\]
	לכל $n$ הפונקציה $f_n$ היא פולינום ובפרט רציפה, אבל $f$ לא רציפה ב־$x = 1$.
\end{solution}

\subquestion{}
נראה ישירות מהגדרה שההתכנסות בסעיף א' איננה במידה שווה.
\begin{proof}
	נוכיח את שלילת התכנסות במידה שווה, \\
	כלומר נוכיח שקיים $\epsilon > 0$ כך שלכל $N$ קיים $n \ge N$ עבורו קיים $x \in [0, 1]$ כך ש־$|f(x) - f_n(x)| > \epsilon$. \\
	נקבע $\epsilon = \frac{1}{2}$, ויהי $N \in \NN$, ונבחר $x < 1$ וכן $n = N$, לכן $f(x) = 0$, לכן $|f(x) - f_n(x)| = |0 - x^n| = x^n$, לכן נבחר $\frac{1}{\sqrt[n]{2}} < x < 1$.
	במקרה זה מתקיים $x^n > \epsilon$, ולכן אכן מתקיימת השלילה להתכנסות במידה שווה, כלומר $f_n \not\rightrightarrows f$.
\end{proof}

\subquestion{}
נראה ישירות מהגדרה כי ${\{ f_n \}}_{n = 1}^\infty$ אינה רציפה במידה אחידה.
\begin{proof}
	גם הפעם נרצה להבין את שלילת ההגדרה ולעבוד על־פיה,
	אז נאמר שקבוצת הפונקציות לא רציפה במידה אחידה אם קיים $\epsilon > 0$ עבורו לכל $\delta > 0$ קיימת $f \in \{ f_n \}$ ו־$x, y \in [0, 1]$ עבורן $|x - y| < \delta, |f(x) - f(y)| > \epsilon$.
	גם הפעם נקבע $\epsilon = \frac{1}{2}$, נבחין כי $|f(x) - f(y)| = |x^n - y^n|$, נקבע $y = 1$ ולכן $1 - x^n > \frac{1}{2} \iff \frac{1}{\sqrt[n]{2}} > x$ כדי לקיים את התנאי.
	יהי $\delta > 0$, אז נבחר $1 - \delta < x < \frac{1}{\sqrt[n]{2}}$ עבור $n$ מספיק גדול כלשהו, ולכן נקבל את שני אי־השוויונות המבוקשים.
\end{proof}

\question{}
בכל אחד מן הסעיפים הבאים נגדיר קבוצת פונקציות $[0, 1] \to \RR$, ונקבע האם לכל סדרה בקבוצה יש תת־סדרה מתכנסת במידה שווה (לא בהכרח לתוך הקבוצה). \\
נעיר שממשפט ארצלה־אסכולי לכל סדרה בקבוצה יש תת־סדרה מתכנסת אם הסדרה וסדרת נגזרותיה מתכנסות במידה אחידה $(1)$.

\subquestion{}
תהי הקבוצה $\{ f_n \mid n \in \NN \}$ עבור $f_n(x) = x^n$.
\begin{solution}
	בשאלה 3 ראינו כי הסדרה ${\{ f_n \}}_{n = 1}^\infty$ מתכנסת נקודתית ל־$f$ שהוגדרה בסעיף א' של השאלה, לכן בפרט לא לכל סדרה יש תת־סדרה מתכנסת במידה שווה.
\end{solution}

\subquestion{}
תהי הקבוצה $\{ f_n \mid n \in \NN \}$ עבור $f_n(x) = \sin(nx)$.
\begin{solution}
	נבחן את הסדרה ${\{ f_n \}}_{n = 1}^\infty$, זוהי סדרת פונקציות רציפות וגזירות ולכן אם היא מתכנסת במידה שווה אז היא מתכנסת לפונקציה רציפה וגזירה.
	נבחין גם כי $f_n'(0) = n$ לכל $n \in \NN$, ולכן $f_n'(0) \to \infty$, בסתירה לטענה.
	לכן לא כל סדרה מכילה תת־סדרה מתכנסת במידה שווה.
\end{solution}

\subquestion{}
תהי הקבוצה $\{ f_d \mid d \in \RR \}$ עבור $f_d(x) = \sin(dx)$.
\begin{solution}
	נובע מהסעיף הקודם שאין, על־ידי בחירת אותה סדרה.
\end{solution}

\subquestion{}
תהי הקבוצה $\{ f_d \mid d \in \RR \}$ עבור $f_d(x) = \sin(x + d)$.
\begin{solution}
	נבחין כי $|f_d(x)| \le 1$ לכל $x, d \in \RR$, וכן $|f_d'(x)| \le 1$ לכל $x, d \in \RR$, לכן מ־$(1)$ שלכל סדרה יש תת־סדרה מתכנסת במידה שווה.
\end{solution}

\subquestion{}
תהי הקבוצה $\{ f_d \mid d \in \RR \}$ עבור $f_d(x) = \arctan(dx)$.
\begin{solution}
	נגדיר את הסדרה ${\{ f_{\frac{1}{n}} \}}_{n = 1}^\infty$, ונקבל שהיא מתכנסת לפונקציה,
	\[
		f(x)
		= \begin{cases}
			-\frac{\pi}{2} & x < 0 \\
			0 & x = 0 \\
			\frac{\pi}{2} & x > 0
		\end{cases}
	\]
	לכן בדומה לסעיף הקודם לא כל סדרה מתכנסת לתת־סדרה במידה שווה.
\end{solution}

\end{document}
