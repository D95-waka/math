\input{../article_base.tex}
\title{פתרון מטלה 09 --- אנליזה פונקציונלית, 80417}

\DeclareMathOperator\sgn{sgn}

\begin{document}
\maketitle
\maketitleprint[teal]

\question{}
תהינה $f, g : \RR \to \RR$ אינטגרביליות רימן ומחזוריות, ונניח ש־$x_0 \in [-\pi, \pi]$.
נניח גם ש־$f \equiv g$ בסביבה של $x_0$.
נסמן $S_N^f, S_N^g$ סכום פורייה מסדר $N$ של $f, g$. \\
נראה ש־$\lim_{N \to \infty} S_N^f(x_0) = L$ אם ורק אם $\lim_{N \to \infty} S_N^g(x_0) = L$.
\begin{proof}
	נניח ש־$\lim_{N \to \infty} S_N^f(x_0) = L$.
	נבחין כי $h = f - g$ מקיימת $h \equiv 0$ בסביבת $x_0$.
	בפרט הגבול שלה ב־$x_0$ הוא $0$ בלבד.
	בנוסף,
	\[
		|h(x_0 + u) - h(x_0)|
		= 0 \le u
	\]
	ונקבל שתנאי ליפשיץ חל, ובהתאם,
	\[
		\lim_{N \to \infty} S_N^{f - g}(x_0)
		= h(x_0)
		= 0
	\]
	אבל אנו יודעים כי $S_N^{f - g} = S_N^f - S_N^g$ מתכונות ולכן למעשה קיבלנו,
	\[
		\lim_{N \to \infty} S_N^f(x_0) - S_N^g(x_0)
		= 0
	\]
	ולכן מההנחה נובע ש־$\lim_{N \to \infty} S_N^g(x_0) = L$ גם כן. \\
	מטעמי סימטריה ההוכחה עבור הכיוון השני זהה.
\end{proof}

\question{}
נוכיח שקיים קבוע $C > 0$ כך שלכל $N > 1$ מתקיים,
\[
	\int_{-\pi}^{\pi} | D_N(u) |\ du
	\ge C \ln(N - 1)
\]
\begin{proof}
	\begin{align*}
		\int_{-\pi}^{\pi} | D_N(u) |\ du
		& = \int_{-\pi}^{\pi} \left\lvert \frac{\sin((N + \frac{1}{2}) u)}{2 \sin \frac{u}{2}} \right\rvert\ du \\
		& = \int_{0}^{\pi} \frac{|\sin((N + \frac{1}{2}) u)|}{2 \sin \frac{u}{2}}\ du - \int_{-\pi}^{0} \frac{|\sin((N + \frac{1}{2}) u)|}{2 \sin \frac{u}{2}}\ du \\
		& = \int_{0}^{\pi} \frac{|\sin((N + \frac{1}{2}) u)|}{\sin \frac{u}{2}}\ du \\
		& \ge \int_{0}^{\pi} \frac{|\sin((N + \frac{1}{2}) u)|}{\frac{u}{2}}\ du \\
		& \ge \int_{0}^{\frac{\pi}{N + \frac{1}{2}}} \frac{1}{\frac{u}{2}}\ du \\
		& \ge 2 \ln(N + 1)
	\end{align*}
\end{proof}

\question{}
\subquestion{}
נראה ש־$A = \Sp {\{ e^{in x} \}}_{n \in \ZZ}$ היא אלגברה ב־$C([-\pi, \pi], \CC)$ שלא מתאפסת באף נקודה ומפרידה נקודות, ואף סגורה להצמדה. \\
נסיק ש־$A$ צפופה ב־$\tilde{C}([-\pi, \pi], \CC)$ בנורמה $\lVert \cdot \rVert_\infty$.
\begin{proof}
	תחילה נראה ש־$A$ אלגברה.
	עבור חיבור הטענה טריוויאלית, עבור כפל,
	\[
		e^{i n x} \cdot e^{i m x}
		= e^{i (n + m) x}
	\]
	והטענה נובעת גם, ועבור סגירות לכפל בסקלר נובע מהגדרת $\Sp$.

	נראה ש־$A$ לא מתאפסת.
	נבחר $f(x) = 1$ על־ידי שימוש ב־$n = 0$.

	נראה ש־$A$ מפרידה נקודות.
	מספיק שנבחר $f(x) = e^{i x}$ ונקבל שמלבד $\pm \pi$ הפונקציה היא חד־חד ערכית ועל.

	מתקיים גם,
	\[
		\overline{e^{i n x}}
		= \cos(n x) - i \sin(n x)
		= \cos(- n x) + i \sin (- n x)
		= e^{-i n x}
	\]
	ולכן $A$ סגורה גם להצמדה.

	כל תנאי משפט סטון־ויירשטראס המרוכב חלים ולכן $A$ צפופה.
\end{proof}

\subquestion{}
נסיק שהקבוצה $A$ היא מערכת אורתוגונלית שלמה ב־$\tilde{C}([-\pi, \pi], \CC)$ יחד עם המכפלה הפנימית,
\[
	\langle f, g \rangle
	= \int_{-\pi}^{\pi} \overline{f}(x) g(x)\ dx 
\]
\begin{proof}
	לכל $n \ne m$,
	\[
		\langle e^{i n x}, e^{i m x} \rangle
		= \int_{-\pi}^{\pi} e^{-i n x} e^{i m x}\ dx
		= \int_{-\pi}^{\pi} e^{i (m - n) x}
		= 0
	\]
	ממשפט קושי למסילות סגורות ורציפות במישור המרוכב נסיק ישירות את הערך הסופי $0$.
	אם כך הסדרה היוצרת את $A$ היא מערכת אורתוגונלית ומאופטימליות טורי פורייה ותנאים שקולים למערכות שלמות נקבל ש־$A$ מערכת שלמה.
\end{proof}

\subquestion{}
עבור $f \in C([-\pi, \pi], \CC)$ נגדיר לכל $n \in \ZZ$ את מקדם פורייה ביחס למערכת המגדירה את $A$ על־ידי,
\[
	c_n
	= \frac{1}{2 \pi} \int_{-\pi}^{\pi} f(x) e^{-i n x}\ dx
\]
נסיק שאם $f$ ממשית אז מתקיים,
\[
	\frac{a_0}{2} = c_0,
	\quad
	\forall n > 0,\ a_n = c_n + c_{-n},
	\quad
	\forall n > 0,\ b_n = i(c_n - c_{-n})
\]
\begin{proof}
	ישירות מהגדרה נקבל,
	\begin{align*}
		a_n
		& = \frac{1}{2 \pi} \int_{-\pi}^{\pi} 2 f(x) \cos(nx)\ dx \\
		& = \frac{1}{2 \pi} \int_{-\pi}^{\pi} f(x) \cos(nx) + i f(x) \sin(nx) - i f(x) \sin(nx) + f(x) \cos(nx)\ dx \\
		& = \frac{1}{2 \pi} \int_{-\pi}^{\pi} f(x) \cos(nx) + i f(x) \sin(nx) + i f(x) \sin(-nx) + f(x) \cos(-nx)\ dx \\
		& = c_n + c_{-n}
	\end{align*}
	ישירות מהגדרה.
	נבחין כי נובע ישירות גם $\frac{a_0}{2} = c_0$ משוויון זה.
	עבור המקרה של $b_n$,
	\begin{align*}
		b_n
		& = \frac{1}{2 \pi} \int_{-\pi}^{\pi} 2 f(x) \sin(nx)\ dx \\
		& = \frac{1}{2 \pi} i \int_{-\pi}^{\pi} i 2 f(x) \sin(-nx) + f(x) \cos(nx) - f(x) \cos(nx)\ dx \\
		& = \frac{1}{2 \pi} i \int_{-\pi}^{\pi} i f(x) \sin(nx) + f(x) \cos(nx) - i f(x) \sin(-nx) - f(x) \cos(-nx)\ dx \\
		& = i(c_n - c_{-n})
	\end{align*}
	וקיבלנו כי השוויון נכון.
\end{proof}

\end{document}
