\input{../article_base.tex}
\title{פתרון מטלה 06 --- אנליזה פונקציונלית, 80417}

\begin{document}
\maketitle
\maketitleprint[teal]

\question{}
יהי $1 \le p < \infty$ ונגדיר את המרחב $l^p$ להיות מרחב הסדרות הממשיות ${\{ x_n \}}_{n = 1}^\infty \subseteq \RR$ כך שמתקיים $\sum_{n = 1}^\infty {| x_n |}^p < \infty$.
נגדיר גם $\lVert x \rVert_p = {\left(\sum_{n = 1}^\infty {| x_n |}^p \right)}^{1/p}$, ראינו בקורסים קודמים כי זוהי נורמה על $l^p$.

\subquestion{}
נראה שאם $p \ne 2$ אז $\lVert \cdot \rVert_p$ אינה מושרית ממכפלה פנימית על $l^p$.
\begin{proof}
	נניח שקיימת מכפלה פנימית כך ש־$\sqrt{\langle x, x \rangle} = \lVert x \rVert_p$ לכל $x \in l^p$, אז נובע משאלה 4,
	\[
		2(\lVert x \rVert_p^2 + \lVert y \rVert_p^2)
		= \lVert x + y \rVert_p^2 + \lVert x - y \rVert_p^2
	\]
	ואילו נבחר $x_n = q^n$ אז,
	\[
		\lVert x \rVert_p^2
		= {\left(\sum_{n = 1}^\infty q^{np} \right)}^{2/p}
		= {\left( \frac{q^p}{1 - q^p} \right)}^{2/p}
		= \frac{q^2}{{(1 - q^p)}^{2/p}}
	\]
	בפרט עבור $q = \frac{1}{2}, \frac{1}{3}$,
	\[
		\lVert x + y \rVert_p^2
		= \frac{{(\frac{2}{3})}^2}{{(1 - {(\frac{2}{3})}^p)}^{2/p}},
		\qquad
		\lVert x - y \rVert_p^2
		= \frac{{(\frac{1}{6})}^2}{{(1 - {(\frac{1}{6})}^p)}^{2/p}}
	.\]
	ולכן,
	\begin{align*}
		& 2 \cdot \left(\frac{{(\frac{1}{3})}^2}{{(1 - {(\frac{1}{3})}^p)}^{2/p}} + \frac{{(\frac{1}{2})}^2}{{(1 - {(\frac{1}{2})}^p)}^{2/p}}\right)
		= \frac{{(\frac{2}{3})}^2}{{(1 - {(\frac{2}{3})}^p)}^{2/p}} + \frac{{(\frac{1}{6})}^2}{{(1 - {(\frac{1}{6})}^p)}^{2/p}} \\
		& \frac{9}{{(3^p - 1)}^{2/p}} + \frac{4}{{(2^p - 1)}^{2/p}} = \frac{18}{{(3^p - 2^p)}^{2/p}} + \frac{18}{{(6^p - 1)}^{2 / p}}
	\end{align*}
	ומבדיקה ישירה נקבל שרק $p = 2$ מקיים את הטענה, לכל ערך אחר מתקבלים באופן ישיר מספרים לא רציונליים ובפרט לא מתקבל פתרון של המשוואה.
	אבל הנחנו ש־$p \ne 2$ ולכן נסיק שאכן לא מתקיים חוק המקבילית, ואין מכפלה פנימית המשרה את הנורמה.

	נוכל לפתור בדרך נוספת.
	נגדיר $x = (1, 0, \ldots)$ וכן $y = (0, 1, 0, \ldots)$.
	במקרה זה $\lVert x \rVert_p^2 = \lVert y \rVert_p^2 = 1$ וכן $\lVert x + y \rVert_p^2 = \lVert x - y \rVert_p^2 = 2^{\frac{2}{p}}$.
	מכלל המקבילית נובע,
	\[
		2 (\lVert x \rVert_p^2 + \lVert y \rVert_p^2)
		= \lVert x + y \rVert_p^2 + \lVert x - y \rVert_p^2
		\iff 2 = 2^{\frac{2}{p}}
		\iff p = 2
	\]
	ונסיק שכלל המקבילית מתקיים אם ורק אם $p = 2$, לכן עבור $p \ne 2$ בהכרח הנורמה לא מושרית ממכפלה פנימית.
\end{proof}

\subquestion{}
נוכיח שנורמת $\infty$ על המרחב $C[a, b]$ לא מושרית ממכפלה פנימית.
\begin{proof}
	ללא הגבלת הכלליות נוכל לבחון את $C[0, 1]$, זאת שכן עבור ההוכחה שנציג אפשר להגדיר מתיחות של הפונקציות.
	נגדיר $f(x) = x, g(x) = 1 - x$ ונקבל $\lVert f \rVert_\infty^2 = \lVert g \rVert_\infty^2 = 1$.
	בנוסף גם $f + g = 1, f - g = 2x - 1$ ומתקיים $\lVert f + g \rVert_\infty^2 = \lVert f - g \rVert_\infty^2 = 1$.
	נסיק ישירות סתירה לכלל המקבילית.
\end{proof}

\question{}
יהי $(H, \langle \cdot, \cdot \rangle)$ מרחב מכפלה פנימית ויהיו $x \in H$ וכן ${\{ x_n \}}_{n = 1}^\infty \subseteq H$.

\subquestion{}
נראה כי אם $\lVert x_k \rVert \to \lVert x \rVert$ וגם $\langle x, x_k \rangle \to \langle x, x \rangle$ אז $x_k \to x$.
\begin{proof}
	מתקיים,
	\begin{align*}
		\lVert x - x_k \rVert
		& = \langle x - x_k, x - x_k \rangle \\
		& = \langle x - x_k, x \rangle + \langle x - x_k, -x_k \rangle \\
		& = \overline{\langle x, x - x_k \rangle} + \overline{\langle -x_k, x - x_k \rangle} \\
		& = \overline{\langle x, x \rangle} + \overline{\langle x, -x_k \rangle} - \overline{\langle x_k, x \rangle} - \overline{\langle x_k, -x_k \rangle} \\
		& = \lVert x \rVert^2 - \langle x_k, x \rangle - \overline{\langle x_k, x \rangle} + \lVert x_k \rVert^2 \\
		& \xrightarrow{k \to \infty} \lVert x \rVert^2 - \langle x, x \rangle - \langle x, x \rangle + \lVert x \rVert^2 \\
		& = 0
	\end{align*}
	ולכן נסיק ש־$x_k \to x$.
\end{proof}

\subquestion{}
תהי ${\{ e_k \}}_{k = 1}^\infty \subseteq H$ מערכת אורתונורמלית,
נראה שהסדרה $y_k = \langle e_k, x \rangle \cdot e_k$ מתכנסת.
\begin{proof}
	מאי־שוויון בסל נובע,
	\[
		\lVert x \rVert
		\ge \sum_{k = 1}^\infty {|\langle e_k, x \rangle|}^2
	\]
	ולכן מהתנאי ההכרחי להתכנסות בהחלט של טורים ממשיים נובע,
	\[
		{|\langle e_k, x \rangle|}^2
		\to 0
		\implies \langle e_k, x \rangle \to 0
	\]
	נסיק אם כן ש־$\lVert y_k \rVert = |\langle e_k, x \rangle| \to 0$.
	נבחין גם כי $\langle y_k, 0 \rangle = 0$ לכל $k \in \NN$ ישירות מהומוגניות מכפלה פנימית, ולכן תנאי סעיף א' חלים ו־$y_k \to 0$.
\end{proof}

\question{}
נוכיח כי הבסיס הסטנדרטי הוא מערכת אורתונורמלית שלמה ב־$l^2$.
\begin{proof}
	נסמן ${\{ e^k \}} \subseteq l^2$ הבסיס הסטנדרטי, כלומר $e_n^k = 1 \iff n = k$ ו־$e_n^k = 0$ אחרת. \\
	לכל $k \ne m$ מתקיים,
	\[
		\langle e^k, e^m \rangle
		= \sum_{n = 1}^\infty \overline{e_n^k} e_n^m
		= 0
	\]
	בנוסף מאותה סיבה בדיוק $\langle e^k, e^k \rangle = 1$ לכל $k \in \NN$.
	ונסיק כי זה אכן בסיס אורתונורמלי.

	נראה ש־$\overline{\Sp\{ e^k \mid k \in \NN \}} = l^2$ ונסיק משקילות שהבסיס שלם.
	נניח ש־$x \in l^2$ סדרה כלשהי, ונרצה לבנות סדרה ${\{ a^k \}} \subseteq \Sp\{ e^k \}$ כך ש־$a^k \to x$.
	נגדיר,
	\[
		a_n^k
		= \begin{cases}
			x_n & n \le k \\
			0 & n > k
		\end{cases}
	\]
	כלומר $a^k = \sum_{n = 1}^n x_n e^n$, לכן $a^k \in \Sp\{ e^k \}$ לכל $k \in \NN$.
	כמובן גם $a_n^k \to x$ שכן $\lVert x - a^k \rVert^2 = \sum_{n = k}^\infty x_n^k$.
	נסיק שאכן הבסיס אורתונורמלי ושלם.
\end{proof}

\question{}
יהי $(H, \langle \cdot, \cdot \rangle)$ מרחב מכפלה פנימית.

\subquestion{}
נראה כי נורמה המושרית ממכפלה פנימית מקיימת את חוק המקבילית, כלומר שלכל $x, y \in H$,
\[
	2( \lVert x \rVert^2 + \lVert y \rVert^2) = {\lVert x + y \rVert}^2 + {\lVert x - y \rVert}^2
\]
\begin{proof}
	יהיו $x, y \in H$, אז,
	\[
		2( \lVert x \rVert^2 + \lVert y \rVert^2) 
		= 2 \langle x, x \rangle + 2 \langle y, y \rangle
	\]
	וכן,
	\begin{align*}
		{\lVert x + y \rVert}^2 + {\lVert x - y \rVert}^2
		& = \langle x + y, x + y \rangle - \langle x - y, x - y \rangle \\
		& = \langle x + y, x \rangle + \langle x + y, y \rangle + \langle x - y, x \rangle - \langle x - y, y \rangle \\
		& = \overline{\langle x, x + y \rangle} + \overline{\langle y, x + y \rangle} + \overline{\langle x, x - y \rangle} - \overline{\langle y, x - y \rangle} \\
		& = \overline{\langle x, x \rangle} + \overline{\langle y, x \rangle} + \overline{\langle x, x \rangle} - \overline{\langle y, x \rangle}
		+ \overline{\langle x, y \rangle} + \overline{\langle y, y \rangle} - \overline{\langle x, y \rangle} + \overline{\langle y, y \rangle} \\
		& = 2 \langle x, x \rangle + 2 \langle y, y \rangle
	\end{align*}
	כפי שרצינו להראות.
\end{proof}

\subquestion{}
נניח ש־$H$ מרחב מכפלה פנימית ממשי, אנו נראה כי לכל $x, y \in H$,
\[
	\langle x, y \rangle
	= \frac{\lVert x + y \rVert^2 - \lVert x - y \rVert^2}{4}
\]
\begin{proof}
	זהו מרחב ממשי ולכן $\langle x, y \rangle = \langle y, x \rangle$, לכן,
	\[
		\lVert x + y \rVert^2 - \lVert x - y \rVert^2
		= {\lVert x \rVert}^2 + {\lVert y \rVert}^2 + 2 \langle x, y \rangle
		- ({\lVert x \rVert}^2 + {\lVert y \rVert}^2 - 2 \langle x, y \rangle)
		= 4 \cdot \langle x, y \rangle
	\]
	והנוסחה נובעת ישירות.
\end{proof}

\end{document}
