\input{../article_base.tex}
\title{פתרון מטלה 03 --- אנליזה פונקציונלית, 80417}

\begin{document}
\maketitle
\maketitleprint{}

\question{}
בשאלה זו נוכיח את משפט הקיום של פאנו.
יהיו $a < b, c < d$ ממשיים ותהי $F : [a, b] \times [c, d] \to \RR$ פונקציה $K$־ליפשיצית.
נניח ש־$(x_{0}, y_{0}) \in (a, b) \times (c, d)$, ונבקש להוכיח שקיים $h > 0$ ופונקציה גזירה $f : [x_0 - h, x_0 + h] \to [c, d]$ כך ש־$f(x_0) = y_0$ ומתקיים,
\[
	\forall x \in [x_0 - h, x_0 + h], f'(x) = F(x, f(x))
\]

\subquestion{}
נגדיר סדרת פונקציות על־ידי $f_0(x) = y_0$ וכן,
\[
	f_{n + 1}(x) = y_0 + \int_{x_0}^{x} F(t, f_n(t))\ dt
\]
נראה שקיים $h > 0$ כך ש־$f_n : [x_0 - h, x_0 + h] \to [c, d]$ מוגדרת היטב ורציפה לכל $n$.
\begin{proof}
	יהי $h > 0$ כלשהו כך ש־$f_1$ מוגדרת ורציפה, כלומר שמתקיים,
	\[
		\forall x \in [x_0 - h, x_0 + h],\ 
		f_1(x) \in [c, d]
	\]
	בהכרח יש כזה, שכן $f_1$ פונקציה רציפה.
	ידוע כי $F$ היא $K$־ליפשיצית, לכן עבור $0 < h < \frac{1}{K}$ גם $\lVert F \rVert_\infty < 1$, כלומר הפונקציה חסומה על־ידי $1$ ובתחום, נצמצם את $h$ שבחרנו לסביבה זו.
	נשתמש בהגדרה זו כבסיס אינדוקציה עבור הטענה שכל פונקציה $f_n$ מוגדרת ורציפה, נרצה להוכיח באינדוקציה שמתקיים $|f_n(x) - f_n(x_0)| \le |x - x_0|$, כאשר טענה זו אף היא נובעת מהמהלך שראינו זה עתה.

	נניח ש־$f_n : [x_0 - h, x_0 + h] \to [c, d]$ מוגדרת ורציפה ומקיימת את אי־השוויון,
	ונגדיר את $f_{n + 1}$ כמוגדר לעיל, אנו יודעים כי זוהי פונקציה רציפה ישירות מהגדרתה כאינטגרל לפונקציה רציפה.
	נבדוק אם היא מקיימת את אי־השוויון,
	\begin{align*}
		|f_{n + 1}(x) - f_{n + 1}(x_0)|
		& = \left\lvert \int_{x_0}^{x} F(t, f_n(t))\ dt \right\rvert \\
		& \le \int_{x_0}^{x} |F(t, f_n(t))|\ dt  \\
		& \le \int_{x_0}^{x} 1\ dt \\
		& = |x - x_0|
	\end{align*}
	כלומר היא אכן מקיימת אותו, וכן נובע שהיא מוגדרת בתחום שרצינו מההגדרה המקורית של $h$ כך ש־$B_h(y_0) \subseteq [c, d]$.
\end{proof}

\subquestion{}
נוכיח ש־${\{ f_n \}}_{n = 1}^\infty$ חסומה במידה אחידה ורציפה במידה אחידה. \\
נסיק ממשפט ארצלה שיש לה תת־סדרה מתכנסת במידה שווה.
\begin{proof}
	למעשה מצאנו שכל פונקציה בסדרה חסומה על־ידי $h$ בתהליך הוכחת סעיף א', ולכן עלינו להראות רציפות במידה אחידה בלבד.
	לכל $x, y \in [x_0 - h, x_0 + h]$ ולכל $n \in \NN$,
	\begin{align*}
		|f_n(x) - f_n(y)|
		& = \left\lvert \int_{x_0}^x F(t, f_{n - 1}(t))\ dt - \int_{x_0}^y F(t, f_{n - 1}(t))\ dt \right\rvert \\
		& = \left\lvert \int_x^y F(t, f_{n - 1}(t))\ dt \right\rvert \\
		& \le \int_x^y |F(t, f_{n - 1}(t))|\ dt \\
		& \le |x - y|
	\end{align*}
	ולכן אם $\epsilon > 0$ אז נוכל להגדיר $\delta = \epsilon$ ונקבל $|f_n(x) - f_n(y)| < \epsilon$ לכל $|x - y| < \delta$, כלומר מצאנו רציפות במידה אחידה.

	ממשפט ארצלה נובע שלכל סדרה בקבוצה $\{ f_n \}$ יש תת־סדרת קושי, בפרט קיימת ${\{ f_{n_k} \}}_{k = 1}^\infty \subseteq \{ f_n \}$ סדרת קושי, בפרט סדרה זו מתכנסת במידה שווה.
\end{proof}

\subquestion{}
נוכיח שהסדרה ${\{ F(x, f_{n_k}(x)) \}}_{k = 1}^\infty$ מתכנסת במידה שווה.
\begin{proof}
	נניח ש־$f_{n_k} \rightrightarrows f$ עבור $f : [a, b] \to [c, d]$ רציפה.
	נראה שהסדרה היא סדרת קושי.
	\[
		|F(x, f_{n_k}(x)) - F(y, f_{n_k}(y))|
		\le K \sqrt{{(x - y)}^2 + {(f_{n_k}(x) - f_{n_k}(y))}^2}
		\le K |x - y| (1 + 1)
		= 2K |x - y|
	\]
	ולכן לכל $\epsilon > 0$ נוכל לבחור $\delta = \frac{\epsilon}{2K}$ ונקבל שאכן הסדרה היא סדרת קושי.
	נבחין כי מצאנו ש־$f_{n_k} \rightrightarrows f$ וכן נובע ש־$f_{n_k}' \rightrightarrows f'$, כלומר הפונקציה $f$ היא רציפה וגזירה, מקיימת $f(x_0) = y_0$ מהתכנסות נקודתית של הסדרה הקבועה $f_n(x_0) = y_0$,
	וכן מרציפות $F$ נובע $f_{n_k} \rightrightarrows F(x, f(x))$, ולכן $f$ מקיימת את כל התנאים למשפט הקיום של פאנו.
\end{proof}

\question{}
תהי ${\{ f_n \}}_{n = 1}^\infty \subseteq C^1[0, 1]$ ונניח שקיים קטע $[a, b] \subseteq [0, 1]$ ושקיימת סדרה ${\{ M_n \}}_{n = 1}^\infty \subseteq \RR$ שואפת לאינסוף כך שלכל $n$ ולכל $x \in [a, b]$ מתקיים,
\[
	f_n'(x) \ge M_n
\]
נראה שהסדרה $\{ f_n \}$ לא רציפה במידה אחידה.
\begin{proof}
	נראה את שלילת הגדרת רציפות במידה אחידה.
	נקבע $\epsilon > 0$ ותהי $\delta > 0$ כלשהי, יהי גם $x = a + \delta$, אז $|f_n(a) - f_n(x_0)| \ge M_n |a - x_0|$, מהנתון ש־$M_n \to \infty$ נסיק שנוכל לבחור $N$ כך שלכל $n > N$ מתקיים $M_n |a - x_0| > \epsilon$.
	נסיק אם כך שהסדרה $\{ f_n \}$ לא רציפה במידה אחידה.
\end{proof}
טענה זו לכאורה מהווה סתירה לדוגמה 1 מתרגול 3, אך נבחין הבחנה חשובה.
בטענה זו נתון תחום קבוע בו הנגזרת שואפת לאינסוף, זאת בעוד בדוגמה 1 התחום בו הנגזרת שאפה לאינסוף הלך וקטן, כלומר סדרת פונקציות זו לא מקיימת את תנאי הטענה שהוכחנו זה עתה.

\question{}
תהי הסדרה ${\{ f_n \}}_{n = 1}^\infty \subseteq C(\RR)$ המוגדרת על־ידי,
\[
	f_n(x)
	= \frac{1}{{(x - n)}^2 + 1}
\]

\subquestion{}
נראה שהסדרה חסומה במידה אחידה, רציפה במידה אחידה ומתכנסת נקודתית לפונקציית האפס.
\begin{proof}
	נבחין כי ${(x - n)}^2 + 1 \ge 1$ לכל $x \in \RR$ ולכל $n \in \NN$ ולכן $f_n(x) \le 1$, כלומר הסדרה $\{ f_n \}$ חסומה במידה אחידה.
	נבחין כי גם אם $|x - y| < \delta$,
	\[
		|f_n(x) - f_n(y)|
		= \left\lvert \frac{-2n (x + y) + x^2 - y^2}{({(x - n)}^2 + 1)({(y - n)}^2 + 1)} \right\rvert
		\le |x - y| \cdot 1
	\]
	כלומר אם $\epsilon > 0$ אז נוכל להגדיר $\delta = \epsilon$ ונקבל שהסדרה $\{ f_n \}$ גם רציפה במידה שווה.
	לבסוף נבחין כי אם $x \in \RR$ קבוע, אז ${(x - n)}^2 \xrightarrow{n \to \infty} \infty$ ולכן $f_n(x) \to 0$, כלומר $f_n \to 0$ בהתכנסות נקודתית.
\end{proof}

\subquestion{}
נראה כי הסדרה לא מתכנסת במידה שווה לפונקציית האפס.
\begin{proof}
	נקבע $\epsilon = \frac{1}{2}$ ויהי $N \in \NN$ כלשהו, אז לכל $n > N$ ולכל $x = n$ מתקיים,
	\[
		|f_n(x) - 0|
		= \frac{1}{1 + 0}
		= 1
		\ge \epsilon
	\]
	דהינו מצאנו כי מתקיימת השלילה של התכנסות במידה שווה.
\end{proof}

\question{}
יהי $P \subseteq C[0, 1]$ מרחב הפולינומים המוגדרים על הקטע $[0, 1]$ עם נורמת סופרימום.
ידוע כי $P$ הוא מרחב נורמי, נראה כי מרחב זה אינו שלם (אינו בנך).
\begin{proof}
	נראה שקיימת סדרה של פולינומים ${\{ p_n \}}_{n = 1}^\infty \subseteq P$ כך שהיא מתכנסת במידה שווה לפונקציה $f \in C[0, 1]$ כך ש־$f \notin P$.
	נגדיר,
	\[
		p_n(x) = \sum_{i = 0}^n \frac{x^i}{i!} 
	\]
	אנו יודעים כי זהו פיתוח טיילור של $\exp$ וכן ש־$p_n \to \exp$ בהתכנסות נקודתית.
	מטענה מתרגול 2 נובע גם כי $p_n \rightrightarrows \exp$.
	לבסוף נטען כי $\exp \notin P$, זאת מהטענה הידועה שלכל $p \in P$ מתקיים $\lim_{x \to \infty} \frac{e^x}{p(x)} = \infty$.
\end{proof}

\end{document}
