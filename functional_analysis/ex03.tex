\input{../article_base.tex}
\title{פתרון מטלה 03 --- אנליזה פונקציונלית, 80417}

\begin{document}
\maketitle
\maketitleprint{}

\question{}
בשאלה זו נוכיח את משפט הקיום של פאנו.
יהיו $a < b, c < d$ ממשיים ותהי $F : [a, b] \times [c, d] \to \RR$ פונקציה $K$־ליפשיצית.
נניח ש־$(x_{0}, y_{0}) \in (a, b) \times (c, d)$, ונבקש להוכיח שקיים $h > 0$ ופונקציה גזירה $f : [x_0 - h, x_0 + h] \to [c, d]$ כך ש־$f(x_0) = y_0$ ומתקיים,
\[
	\forall x \in [x_0 - h, x_0 + h], f'(x) = F(x, f(x))
\]

\subquestion{}
נגדיר סדרת פונקציות על־ידי $f_0(x) = y_0$ וכן,
\[
	f_{n + 1}(x) = y_0 + \int_{x_0}^{x} F(t, f_n(t))\ dt
\]
נראה שקיים $h > 0$ כך ש־$f_n : [x_0 - h, x_0 + h] \to [c, d]$ מוגדרת היטב ורציפה לכל $n$.
\begin{proof}
	יהי $h > 0$ כלשהו כך ש־$f_1$ מוגדרת ורציפה, כלומר שמתקיים,
	\[
		\forall x \in [x_0 - h, x_0 + h],\ 
		f_1(x) \in [c, d]
	\]
	בהכרח יש כזה, שכן $f_1$ פונקציה רציפה.
	נשתמש בהגדרה זו כבסיס אינדוקציה עבור הטענה שכל פונקציה $f_n$ מוגדרת ורציפה.

	נניח ש־$f_n : [x_0 - h, x_0 + h] \to [c, d]$ מוגדרת ורציפה, ונגדיר את $f_{n + 1}$ כמוגדר לעיל, אנו יודעים כי זוהי פונקציה רציפה ישירות מהגדרתה כאינטגרל לפונקציה רציפה, אך עלינו לבחון את טווחה,
	\begin{align*}
		|f_{n + 1}(x) - f_{n + 1}(y)|
		& = \left\lvert \int_{x_0}^{x} F(t, f_n(t))\ dt - \int_{x_0}^{y} F(t, f_n(t))\ dt \right\rvert \\
		& = \left\lvert \int_{x}^{y} F(t, f_n(t))\ dt \right\rvert \\
		& \le K d(x - y, f_n(x) - f_n(y)) \\
		& \le K 2h |x - y|
	\end{align*}

	TODO
\end{proof}

\question{}
תהי ${\{ f_n \}}_{n = 1}^\infty \subseteq C^1[0, 1]$ ונניח שקיים קטע $[a, b] \subseteq [0, 1]$ ושקיימת סדרה ${\{ M_n \}}_{n = 1}^\infty \subseteq \RR$ שואפת לאינסוף כך שלכל $n$ ולכל $x \in [a, b]$ מתקיים,
\[
	f_n'(x) \ge M_n
\]
נראה שהסדרה $\{ f_n \}$ לא רציפה במידה אחידה.
\begin{proof}
	נראה את שלילת הגדרת רציפות במידה אחידה.
	נקבע $\epsilon > 0$ ותהי $\delta > 0$ כלשהי, יהי גם $x = a + \delta$, אז $|f_n(a) - f_n(x_0)| \ge M_n |a - x_0|$, מהנתון ש־$M_n \to \infty$ נסיק שנוכל לבחור $N$ כך שלכל $n > N$ מתקיים $M_n |a - x_0| > \epsilon$.
	נסיק אם כך שהסדרה $\{ f_n \}$ לא רציפה במידה אחידה.
\end{proof}
טענה זו לכאורה מהווה סתירה לדוגמה 1 מתרגול 3, אך נבחין הבחנה חשובה.
בטענה זו נתון תחום קבוע בו הנגזרת שואפת לאינסוף, זאת בעוד בדוגמה 1 התחום בו הנגזרת שאפה לאינסוף הלך וקטן, כלומר סדרת פונקציות זו לא מקיימת את תנאי הטענה שהוכחנו זה עתה.

\question{}
תהי הסדרה ${\{ f_n \}}_{n = 1}^\infty \subseteq C(\RR)$ המוגדרת על־ידי,
\[
	f_n(x)
	= \frac{1}{{(x - n)}^2 + 1}
\]

\subquestion{}
נראה שהסדרה חסומה במידה אחידה, רציפה במידה אחידה ומתכנסת נקודתית לפונקציית האפס.
\begin{proof}
	נבחין כי ${(x - n)}^2 + 1 \ge 1$ לכל $x \in \RR$ ולכל $n \in \NN$ ולכן $f_n(x) \le 1$, כלומר הסדרה $\{ f_n \}$ חסומה במידה אחידה.
	נבחין כי גם אם $|x - y| < \delta$,
	\[
		|f_n(x) - f_n(y)|
		= \left\lvert \frac{-2n (x + y) + x^2 - y^2}{({(x - n)}^2 + 1)({(y - n)}^2 + 1)} \right\rvert
		\le |x - y| \cdot 1
	\]
	כלומר אם $\epsilon > 0$ אז נוכל להגדיר $\delta = \epsilon$ ונקבל שהסדרה $\{ f_n \}$ גם רציפה במידה שווה.
	לבסוף נבחין כי אם $x \in \RR$ קבוע, אז ${(x - n)}^2 \xrightarrow{n \to \infty} \infty$ ולכן $f_n(x) \to 0$, כלומר $f_n \to 0$ בהתכנסות נקודתית.
\end{proof}

\subquestion{}
נראה כי הסדרה לא מתכנסת במידה שווה לפונקציית האפס.
\begin{proof}
	נקבע $\epsilon = \frac{1}{2}$ ויהי $N \in \NN$ כלשהו, אז לכל $n > N$ ולכל $x = n$ מתקיים,
	\[
		|f_n(x) - 0|
		= \frac{1}{1 + 0}
		= 1
		\ge \epsilon
	\]
	דהינו מצאנו כי מתקיימת השלילה של התכנסות במידה שווה.
\end{proof}

\question{}
יהי $P \subseteq C[0, 1]$ מרחב הפולינומים המוגדרים על הקטע $[0, 1]$ עם נורמת סופרימום.
ידוע כי $P$ הוא מרחב נורמי, נראה כי מרחב זה אינו שלם (אינו בנך).
\begin{proof}
	נראה שקיימת סדרה של פולינומים ${\{ p_n \}}_{n = 1}^\infty \subseteq P$ כך שהיא מתכנסת במידה שווה לפונקציה $f \in C[0, 1]$ כך ש־$f \notin P$.
	נגדיר,
	\[
		p_n(x) = \sum_{i = 0}^n \frac{x^i}{i!} 
	\]
	אנו יודעים כי זהו פיתוח טיילור של $\exp$ וכן ש־$p_n \to \exp$ בהתכנסות נקודתית.
	מטענה מתרגול 2 נובע גם כי $p_n \rightrightarrows \exp$.
	לבסוף נטען כי $\exp \notin P$, זאת מהטענה הידועה שלכל $p \in P$ מתקיים $\lim_{x \to \infty} \frac{e^x}{p(x)} = \infty$.
\end{proof}

\end{document}
