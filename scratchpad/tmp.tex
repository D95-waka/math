\input{../article_base.tex}

\usepackage{emoji}

\begin{document}

\begin{proposition}
	יהי $b \in \NN \setminus \{ 0 \}$.
	לכל $x \in \NN \setminus \{ 0 \}$ קיים $N \in \NN$ יחיד וסדרה יחידה ${\{ x_n \}}_{n = 0}^N \subseteq \{0, \ldots, b - 1\}$,
	כך שמתקיים,
	\[
		x
		= \sum_{n = 0}^N b^n \cdot x_n
	\]
	ו־$b_N \ne 0$.
\end{proposition}
\begin{proof}
	נוכיח את הטענה באינדוקציה על $x$.

	נניח ש־$x = 1$.
	אילו $N > 0$ אז $x_N \ge 1$ ומתקיים,
	\[
		\sum_{n = 0}^N b^n \cdot x_n
		\ge b^N \cdot x_n
		\ge b
	\]
	ולכן בהכרח $N = 0$ בלבד.
	נובע אם כך ש־$x = x_0$ עבור $0 \le x_0 < N$, ולכן $x_0 = 1$ בלבד.

	נניח שהטענה נכונה עבור $1 \le y < x$ לכל $y$ כזה ונוכיח את הטענה על $x$.
	נגדיר,
	\[
		N
		= \min\{ n \in \NN \mid b^{n + 1} > x \}
	\]
	קיים כזה מהסדר הטוב על הטבעיים (ולכן הטענה לא נכונה ב־$\operatorname{PA}$).
	נגדיר גם,
	\[
		x_N
		= \max\{ n \in [b] \mid x \le b^N \cdot n \}
	\]
	נסמן $y = x - b^N \cdot x_N$, אז $y < x$ ולכן מהנחת האינדוקציה קיים $M \le N$ ו־${\{ y_n \}}_{n = 0}^M$ כך שמתקיים,
	\[
		y
		= \sum_{n = 0}^M b^n \cdot y_n
	\]
	אז נגדיר $x_n = y_n$ לכל $0 \le n \le M$ ו־$x_n = 0$ עבור $M < n < N$.
	נקבל,
	\[
		\sum_{n = 0}^N b^n \cdot x_n
		= b^N \cdot x_N + \sum_{n = 0}^M b^n \cdot x_n
		= b^N \cdot x_N + y
		= x
	\]
	לכן מצאנו $N$ וסדרה המקיימות את הטענה, עלינו להראות יחידות. \\
	נניח ש־$N'$ ו־${\{ x_n' \}}_{n = 0}^{N'}$ מקיימים את הטענה.
	אז מהחישוב שראינו של $N$ בהכרח נובע $N = N'$.
	אם $N = 0$ אז $b^0 \cdot x_0 = x = b^0 \cdot x_0'$ ונובע $x_0 = x_0'$ בלבד.
	עתה נניח שהטענה נכונה עבור רישא של הסדרה, $x_n = x_n'$ לכל $n < m \le N$, ונבחן את $m$.
	נניח בלי הגבלת הכלליות ש־$x_n = 0$ לכל $n < m$ ולכן,
	\[
		x
		= \sum_{n = m}^N b^n \cdot x_n
		= b^m \sum_{n = m}^N b^{n - m} \cdot x_n
	\]
	ונוכל באופן דומה לקבל ש־$x_m = x_m'$ עבור $y = \sum_{n = m}^N b^{n - m} \cdot x_n$. \\
	נסיק שאכן $x_n = x_n'$ לכל $n \le N$, וסיימנו להוכיח יחידות.
\end{proof}

\end{document}
