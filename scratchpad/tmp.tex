\input{../article_base.tex}

\begin{document}

\begin{proposition}
	קיימת הרחבת שדות $K / \QQ$ כך שהיא הרחבת גלואה ו־$\gal(K / \QQ) \simeq \ZZ / 5\ZZ$.
\end{proposition}
\begin{proof}
	נתחיל בהגדרה $\zeta = \xi_{11}$ שורש היחידה הפרימיטיבי מסדר $11$ ב־$\QQ$.
	נגדיר גם $L = \QQ(\zeta)$ ונקבל ש־$L / \QQ$ היא הרחבה ציקלוטומית וגלואה, ומתקיים,
	\[
		[L : \QQ]
		= | {(\ZZ / 11\ZZ)}^\times |
		= 10
	\]
	וכן אנו יודעים כי הרחבות ציקלוטומיות הן בעלות חבורת גלואה ציקלית ולכן,
	\[
		\gal(L / \QQ)
		\simeq \ZZ / 10\ZZ
	\]
	עשינו את כל זה כי אנחנו רוצים למצוא את החבורה $\ZZ / 5\ZZ$, אז מספיק למצוא חבורה ככה שהיא תת־חבורה שלה, ואכן,
	\[
		\ZZ / 5\ZZ
		\le \gal(L / \QQ)
	\]
	אבל אנחנו צריכים למצוא את המבנה של $H$, לא רק להגיד שקיימת כזאת, אז נגדיר,
	\[
		H
		= \langle \zeta \mapsto \zeta^2 \rangle
		= \{ \zeta \mapsto \zeta^{2 n} \mid 1 \le n \le 5 \}
	\]
	זוהי אכן חבורת אוטומורפיזמים כפי שרצינו ומתקיים $|H| = 5$.
	אבל ממשפט התאמת גלואה מתקיים $L / L^H$ היא הרחבת גלואה כך ש־$[L : L^H] = 5$.
	בשלב הזה בגדול סיימנו את השאלה כי מצאנו הרחבת שדות כמו שביקשו, אבל בתרגול ראינו דרך לבנות ממש את השדה הזה.

	נגדיר $\alpha = \sum_{n = 1}^5 \zeta^{2 n}$, הדבר הזה מקיים,
	\[
		\forall \sigma \in H,\ 
		\sigma(\alpha)
		= \sum_{n = 1}^5 \sigma(\zeta^{2 n})
		= \alpha
	\]
	מטעמי סימטריה, אז $\QQ(\alpha) \subseteq L^H$ ממש מהגדרת $L^H$ כשדה שבט של $H$, ונסיק,
	\[
		\QQ(\alpha) = L^H
	\]
	ומצאנו ש־$\QQ(\zeta) / \QQ(\alpha)$ הרחבת גלואה כך ש־$\gal(\QQ(\zeta) / \QQ(\alpha)) \simeq \ZZ / 5\ZZ$.
\end{proof}

\end{document}
