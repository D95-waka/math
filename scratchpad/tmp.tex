\input{../article_base.tex}

\usepackage{emoji}

\begin{document}

\begin{definition}[רציפות במידה שווה]
	תהי $f$ רציפה במידה שווה, כלומר לכל $\varepsilon > 0$ קיים $\delta > 0$ כך שאם $|x - y| < \delta$ אז $|f(x) - f(y)| < \varepsilon$.
\end{definition}
\begin{definition}[רציפות חסומה במידה שווה]
	תהי $f$ רציפה במידה שווה, כלומר לכל $\varepsilon > 0$ קיים $1 > \delta > 0$ כך שאם $|x - y| < \delta$ אז $|f(x) - f(y)| < \varepsilon$.
\end{definition}
\begin{proposition}
	פונקציה $f : I \to \RR$ רציפה במידה שווה אם ורק אם היא רציפה חסומה במידה שווה.
\end{proposition}
\begin{proof}
	נניח ש־$f$ רציפה חסומה במידה שווה ויהי $\varepsilon > 0$.
	אז קיימת $1 > \delta > 0$ המקיימת את טענת הרציפות.
	בפרט $\delta > 0$ ומעידה על רציפות במידה שווה.

	נניח ש־$f$ רציפה במידה שווה.
	יהי $\varepsilon > 0$ ותהי $\delta > 0$ המקיימת את טענת הרציפות.
	אם $\delta < 1$ אז סיימנו, ולכן נניח ש־$\delta > 1$.
	נגדיר $\delta_1 = \frac{1}{2}$, אז מתקיים,
	\[
		\forall x, y \in I,\ 
		|x - y| < \delta_1
		\implies |x - y| < \delta
		\implies |f(x) - f(y)| < \varepsilon
	\]
	כלומר קיבלנו ש־$\delta_1$ מעיד על נכונות הרציפות במידה שווה, וזהו ערך חסום.
\end{proof}

\begin{remark}
	נבחר $\delta_1 = \min\{ \delta, \frac{1}{2} \}$, אז אם $|x - y| < \delta_1$ בהכרח גם $|x - y| < \frac{1}{2}$ וגם $|x - y| < \delta$, אבל אז גם $|f(x) - f(y)| < \varepsilon$ בהתאם להגדרה.
	נוכל אם כל להניח בלי הגבלת הכלליות ש־$\delta \le \frac{1}{2}$ ולכן בהכרח,
	\[
		1 - \delta^2
		< 1
	\]
\end{remark}

\end{document}
