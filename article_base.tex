\documentclass{article}

% packages
\usepackage{amsmath, amsthm, thmtools, amsfonts, amssymb, luacode, catchfile, hyperref, ifthen, graphicx}
\ifcsname c@kobocompile\endcsname
	\usepackage[a5paper, total={1072pt, 1448pt}, margin=10pt, includeheadfoot]{geometry} % set page margins
\else
	\usepackage[a4paper, margin=50pt, includeheadfoot]{geometry}
\fi
\usepackage[shortlabels]{enumitem}
\usepackage[skip=3pt, indent=0pt]{parskip}

% language
\usepackage[bidi=basic, layout=tabular, provide=*]{babel}
\ifcsname c@english\endcsname
	\babelprovide[main, import]{english}
\else
	\babelprovide[main, import]{hebrew}
	\babelprovide{rl}
\fi
\babelfont{rm}[Renderer=Harfbuzz]{Libertinus Serif}
\babelfont{sf}{Libertinus Sans}
\babelfont{tt}{Libertinus Mono}

% style
\AddToHook{cmd/section/before}{\clearpage}	% Add line break before section
\linespread{1.3}
\setcounter{secnumdepth}{0}		% Remove default number tags from sections, this won't do well with theorems
\AtBeginDocument{\setlength{\belowdisplayskip}{3pt}}
\AtBeginDocument{\setlength{\abovedisplayskip}{3pt}}
\graphicspath{ {../images/} }

% operators
\DeclareMathOperator\cis{cis}
\DeclareMathOperator\Sp{Sp}
\DeclareMathOperator\tr{tr}
\DeclareMathOperator\im{Im}
\DeclareMathOperator\re{Re}
\DeclareMathOperator\diag{diag}
\DeclareMathOperator*\lowlim{\underline{lim}}
\DeclareMathOperator*\uplim{\overline{lim}}
\DeclareMathOperator\rng{rng}
\DeclareMathOperator\Sym{Sym}
\DeclareMathOperator\Arg{Arg}
\DeclareMathOperator\Log{Log}
\DeclareMathOperator\dom{dom}
\DeclareMathOperator\supp{Supp}
\DeclareMathOperator\var{Var}
\DeclareMathOperator\cov{Cov}
\DeclareMathOperator\aut{Aut}
\DeclareMathOperator\id{id}
\DeclareMathOperator\cl{cl}

% commands
\newcommand{\cupdot}{\mathbin{\mathaccent\cdot\cup}}
\newcommand{\bigcupdot}{\mathbin{\mathaccent\cdot\bigcup}}
\newcommand{\Aa}[0]{\mathcal{A}}
\newcommand{\Bb}[0]{\mathcal{B}}
\newcommand{\CC}[0]{\mathbb{C}}
\newcommand{\Cc}[0]{\mathcal{C}}
\newcommand{\EE}[0]{\mathbb{E}}
\newcommand{\FF}[0]{\mathbb{F}}
\newcommand{\Ff}[0]{\mathcal{F}}
\newcommand{\Ii}[0]{\mathcal{I}}
\newcommand{\Gg}[0]{\mathcal{G}}
\newcommand{\HH}[0]{\mathbb{H}}
\newcommand{\Hh}[0]{\mathcal{H}}
\newcommand{\Kk}[0]{\mathcal{K}}
\newcommand{\Ll}[0]{\mathcal{L}}
\newcommand{\Mm}[0]{\mathcal{M}}
\newcommand{\NN}[0]{\mathbb{N}}
\newcommand{\Nn}[0]{\mathcal{N}}
\newcommand{\PP}[0]{\mathbb{P}}
\newcommand{\Pp}[0]{\mathcal{P}}
\newcommand{\QQ}[0]{\mathbb{Q}}
\newcommand{\RR}[0]{\mathbb{R}}
\newcommand{\Rr}[0]{\mathcal{R}}
\newcommand{\Ss}[0]{\mathcal{S}}
\newcommand{\TT}[0]{\mathbb{T}}
\newcommand{\Uu}[0]{\mathcal{U}}
\newcommand{\Vv}[0]{\mathcal{V}}
\newcommand{\Ww}[0]{\mathcal{W}}
\newcommand{\ZZ}[0]{\mathbb{Z}}
\newcommand{\acts}[0]{\circlearrowright}
\newcommand{\explain}[2] {
	\begin{flalign*}
		 && \text{#2} && \text{#1}
	\end{flalign*}
}
\newcommand{\maketitleprint}[1][green]{
	\begin{center}
		\includegraphics[width=6cm]{dragon_#1}
	\end{center}
}

% theorem commands
\newtheoremstyle{c_remark}
	{}	% Space above
	{}	% Space below
	{}% Body font
	{}	% Indent amount
	{\bfseries}	% Theorem head font
	{}	% Punctuation after theorem head
	{.5em}	% Space after theorem head
	{\thmname{#1}\thmnumber{ #2}\thmnote{ \normalfont{\text{(#3)}}}}	% head content
\newtheoremstyle{c_definition}
	{3pt}	% Space above
	{3pt}	% Space below
	{}% Body font
	{}	% Indent amount
	{\bfseries}	% Theorem head font
	{}	% Punctuation after theorem head
	{.5em}	% Space after theorem head
	{\thmname{#1}\thmnumber{ #2}\thmnote{ \normalfont{\text{(#3)}}}}	% head content
\newtheoremstyle{c_plain}
	{3pt}	% Space above
	{3pt}	% Space below
	{\itshape}% Body font
	{}	% Indent amount
	{\bfseries}	% Theorem head font
	{}	% Punctuation after theorem head
	{.5em}	% Space after theorem head
	{\thmname{#1}\thmnumber{ #2}\thmnote{ \text{(#3)}}}	% head content

\ifcsname c@english\endcsname
	\theoremstyle{plain}
	\newtheorem{theorem}{Theorem}[section]
	\newtheorem{lemma}[theorem]{Lemma}
	\newtheorem{proposition}[theorem]{Proposition}
	\newtheorem*{proposition*}{Proposition}
	%\newtheorem{corollary}[theorem]{אין חלופה עברית}

	\theoremstyle{definition}
	\newtheorem{definition}[theorem]{Definition}
	\newtheorem*{definition*}{Definition}
	\newtheorem{example}{Example}[section]
	\newtheorem{exercise}{Exercise}[section]

	\theoremstyle{remark}
	\newtheorem*{remark}{Remark}
	\newtheorem*{solution}{Solution}
	\newtheorem{conclusion}[theorem]{Conclusion}
	\newtheorem{notation}[theorem]{Notation}
\else
	\theoremstyle{c_plain}
	\newtheorem{theorem}{משפט}[section]
	\newtheorem{lemma}[theorem]{למה}
	\newtheorem{proposition}[theorem]{טענה}
	\newtheorem*{proposition*}{טענה}

	\theoremstyle{c_definition}
	\newtheorem{definition}[theorem]{הגדרה}
	\newtheorem*{definition*}{הגדרה}
	\newtheorem{example}{דוגמה}[section]
	\newtheorem{exercise}{תרגיל}[section]

	\theoremstyle{c_remark}
	\newtheorem*{remark}{הערה}
	\newtheorem*{solution}{פתרון}
	\newtheorem{conclusion}[theorem]{מסקנה}
	\newtheorem{notation}[theorem]{סימון}
\fi

% Questions related commands
\newcounter{question}
\setcounter{question}{1}
\newcounter{sub_question}
\setcounter{sub_question}{1}

\ifcsname c@english\endcsname
	\newcommand{\question}[1][0]{
		\ifthenelse{#1 = 0}{}{\setcounter{question}{#1}}
		\section{Question \arabic{question}}
		\addtocounter{question}{1}
		\setcounter{sub_question}{1}
	}

	\newcommand{\subquestion}[1][0]{
		\ifthenelse{#1 = 0}{}{\setcounter{sub_question}{#1}}
		\subsection{Part \alph{sub_question}}
		\addtocounter{sub_question}{1}
	}
\else
	\newcommand{\question}[1][0]{
		\ifthenelse{#1 = 0}{}{\setcounter{question}{#1}}
		\section{שאלה \arabic{question}}
		\addtocounter{question}{1}
		\setcounter{sub_question}{1}
	}

	\newcommand{\subquestion}[1][0]{
		\ifthenelse{#1 = 0}{}{\setcounter{sub_question}{#1}}
		\subsection{סעיף \localecounter{letters.gershayim}{sub_question}}
		\addtocounter{sub_question}{1}
	}
\fi

% import lua and start of document
\directlua{common = require ('../common')}

\GetEnv{AUTHOR}

% headers
\author{\AUTHOR}
\date\today
