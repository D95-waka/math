\input{../article_base.tex}
\title{פתרון מטלה 1 --- גאומטריה דיפרנציאלית אלמנטרית, 80560}

\begin{document}
\maketitle
\maketitleprint[purple]

\section{שאלה 0}
יהי $(\Pp, \Ll)$ מישור אפיני בגישה סינתטית.

\subquestion{}
נוכיח כי במישור האפיני יש לפחות שלושה ישרים שונים.
\begin{proof}
	נסמן ב־$P, Q, R, S$ את 4 הנקודות הלא קולינאריות שנתון ושמצאנו שקיימות.
	מצאנו במהלך הוכחה ש־$l = \langle P, Q \rangle$ ו־$m = \langle P, R \rangle$ יחד עם $l' \parallel l \ni R$ וכן $m \parallel m' \ni S$ הם ישרים שונים.
\end{proof}

\subquestion{}
נוכיח שלא קיים ישר ללא נקודות.
\begin{proof}
	נניח בשלילה שקיים ישר כזה $l \in \Ll$.
	נניח גם ש־$P, Q, R \in \Pp$ נקודות לא קולינאריות, וכן נסמן $m = \langle P, Q \rangle, n = \langle P, R \rangle$, אז ידוע ש־$m \not\parallel n$.
	נגדיר את המשיקים $l \parallel l' \ni Q$ וכן את $l \parallel l'' \ni R$, אז בהכרח $l' \parallel l''$ וכן $Q \in l' \cap m$ וגם $R \in l'' \cap R$.
	אילו $l \parallel m$ וגם $l \parallel n$ אז נקבל ש־$l' = l''$ ולכן $P, Q, R$ קולינאריות בסתירה, ולכן $l \lnot\parallel m$ וקיימת נקודת חיתוך.
\end{proof}

\subquestion{}
נוכיח כי לכל ישר לפחות שתי נקודות שונות.
\begin{proof}
	יהי ישר $l \in \Ll$, ותהי $P \in l$ נקודה כלשהי שידוע שקיימת מהסעיף הקודם.
	נניח שוב ש־$P, Q, R \in \Pp$ נקודות לא קולינאריות ונגדיר את $m = \langle P, Q \rangle$. \\
	נסמן גם את $n = \langle Q, R \rangle$, אם $n \lnot\parallel l$ אז סיימנו, שכן $Q \in l$ או $R \in l$, לכן נניח ש־$n \parallel l$.
	נגדיר את המשיק $R \in m' \parallel m$, אז $l \not\parallel m'$ בהכרח ולכן יש להם נקודת חיתוך.
\end{proof}

\subquestion{}
נראה כי לכל שני ישרים כמות זהה של נקודות.
\begin{proof}
	נניח ש־$l_0, l_1 \in \Ll$ ישרים מקבילים, ונניח ש־$P_0 \in l_0, P_1 \in l_1$. \\
	אם $l_0 = l_1$ אז סיימנו, לכן נניח ש־$l_0 \ne l_1$, כלומר הנקודות המרכיבות אותם שונות. \\
	נגדיר את הישר $m = \langle P_0, P_1 \rangle$, בהכרח $m \not\parallel l_0, l_1$.
	תהי נקודה $Q \in l_0$, אז נגדיר את $Q \in n \parallel m$, מתקיים $n \not\parallel l_1$ ולכן יש להם נקודת חיתוך $Q' \in l_1$. 

	נניח עתה ש־$Q_0, Q_1 \in l_0$ וכן ש־$Q_0', Q_1' \in l_1$ מתאימות אליה, נראה ש־$Q_0 \ne Q_1 \implies Q_0' \ne Q_1'$.
	נניח ש־$Q_0' = Q_1'$, וכן נניח ש־$Q_0 \ne Q_1$, אז $o = \langle Q_0, Q_0' \rangle, p = \langle Q_1, Q_0' \rangle$ הם ישרים לא מקבילים.
	אבל $o \parallel m \parallel p \implies o \parallel p$ בסתירה.
\end{proof}

\subquestion{}
נראה שלכל שני ישרים נחתכים יש אותה כמות של נקודות.
\begin{proof}
	נפעל באופן זהה לסעיף הקודם.
	אם $l_0, l_1 \in \Ll$ כך ש־$l_0 \cap l_1 = \{ O \}$ אז קיימות נקודות שונות $P_0, P_1 \in \Pp$ כך ש־$P_0 \in l_0, P_1 \in l_1$, מכאן ההוכחה זהה תוך שימוש ב־$m = \langle P_0, P_1 \rangle$.
\end{proof}

\subquestion{}
נסיק שלכל הישרים במישור האפיני יש אותה כמות של נקודות.
\begin{proof}
	נסמן שתי נקודות כלשהן $P, Q \in \Pp$ ואת $l = \langle P, Q \rangle$.
	יהי $m \in \Ll$ ישר כלשהו.
	אם $l \parallel m$ אז יש להם אותו מספר נקודות בהתאם לסעיף ד'.
	אם $l \not\parallel m$ אז מסעיף ה' יש לישרים אותו מספר נקודות.
\end{proof}

\section{שאלות מוספות}
\subquestion[1]
נגדיר,
\[
	\Pp = \{ P, Q, R, S \},
	\quad
	\Ll = \{ \{P, Q\}, \{Q, R\}, \{R, S\}, \{P, S\}, \{P, R\}, \{Q, S\}\}
\]
ונראה ש־$(\Pp, \Ll)$ מישור אפיני.
\begin{proof}
	כדי להוכיח את הטענה עלינו לבדוק ששלוש האקסיומות אכן מתקיימות. \\
	האקסיומה הראשונה היא שלכל שתי נקודות ישר יחיד המכיל את שתיהן.
	מעבר על כל צמדי הנקודות ובדיקה ישירה מראה שהאקסיומה אכן מתקיימת, בפרט יש 4 נקודות ו־$\binom{4}{2} = 6$ צמדים ואכן גם $|\Ll| = 6$. \\
	האקסיומה השנייה היא שלכל ישר ונקודה יש ישר יחיד העובר בנקודה ומקביל לישר.
	נבחר לדוגמה את $P$ ואת $\{Q, R\}$, אז הישר $\{P, S\}$ מקיים את שתי התכונות, ובאופן דומה נוכל לבדוק את כל שאר המקרים. \\
	האקסיומה האחרונה היא שקיימות שלוש נקודות לא קולינאריות.
	אם נבחר את $P, Q, R$, אז אכן לכל ישר $l \in \Ll$ מתקיים $\{P, Q, R\} \not\subseteq l$.
\end{proof}

\subquestion{}
יהי $K$ שדה ו־$V$ מרחב וקטורי ממימד $3$ מעל $K$.
יהי $H_0 \le V$ ממימד $2$.
נסמן ב־$\Pp$ את קבוצת כל הישרים שלא ב־$H_0$, וב־$\Ll$ את קבוצת כל המישורים השונים מ־$H_0$. \\
נראה ש־$(\Pp, \Ll)$ מישור אפיני.
\begin{proof}
	נניח ש־$P, Q \in \Pp$ ונראה שעובר ביניהן ישר.
	נבחין כי $\{ 0 \} \in P, Q$ מהגדרתם כתת־מרחבים ממימד $1$.
	נניח ש־$P = \Sp\{ v \}, Q = \Sp \{ u \}$, אז $v, e \notin H_0$ בהגדרה ובהתאם $l = \Sp\{v, u\} \notin H_0$ ובהתאם $l \in \Ll$.
	כמובן מלינארית אנו יכולים להסיק שמישור זה יחיד.

	נניח ש־$P = \Sp\{ u \} \in \Pp$ נקודה ו־$l = \Sp\{v, w\} \in \Ll$.
	אם $u \in l$ אז $l$ מקיים את האקסיומה השנייה, ועתה נניח אחרת.
	נבחין כי במקרה זה שני ישרים מקבילים אם ורק אם $l \cap m \in H_0$, וכמובן משיקולי דרגה והגדרה יש אינסוף פתרונות כאלה, בפרט מישור כך שהוא מכיל את $u \notin H_0$.

	נשאר לנו למצוא שלוש נקודות לא קולינאריות, כלומר שלושה ישרים שאין להם מישור משותף, נבחר לצורך כך את $u \perp H_0$ ו־$P = \Sp\{ u \}$.
	עבור שתי הנקודות הנוספות נבחר שני וקטורים נוספים בלתי תלויים ב־$H_0, u$, בנפרד, מובטח לנו שיש כאלה.
\end{proof}

\question{}
יהי $\FF$ שדה ויהי $(E, V, t)$ מרחב אפיני. \\
תהי $P \in E$ ו־$(E, +_P, \cdot_P)$ מרחב וקטורי מעל $\FF$ שראשיתו $P$, נסמן כ־$E_P$. \\
נוכיח שההעתקה $v_P : E_P \to V$ היא איזומורפיזם של מרחבים וקטוריים.
\begin{proof}
	נוכיח תחילה ש־$v_P$ העתקה לינארית.
	נניח ש־$\alpha, \beta \in \FF$ וכן ש־$Q, R \in E_P$, אז מתקיים,
	\begin{align*}
		v_P(\alpha Q +_P \beta R)
		& = v(P, \alpha Q +_P \beta R) \\
		& = \alpha Q +_P \beta R - P \\
		& = \alpha \cdot_P + Q \beta \cdot_P R - 2P \\
		& = \alpha (Q - P) + P + \beta (R - P) + P - 2P \\
		& = \alpha v_P(Q) + \beta v_P(R)
	\end{align*}
	תוך שימוש בהגדרות המופיעות בסיכום.

	ראינו ש־$v_P$ היא העתקה הפיכה בהרצאה ולכן היא איזומורפיזם של מרחבים וקטוריים.
\end{proof}

\question{}
\subquestion{}
יהיו $W, W' \le V$ תתי־מרחבים אפיניים ו־$P, Q \in E$ נקודות.
נראה ישירות כי מתקיים,
\[
	P + W
	= Q + W'
	\iff W = W' \land Q - P \in W
\]
\begin{proof}
	נראה תחילה ש־$P + W = Q + W \iff Q - P \in W$. \\
	נניח ש־$P + W = Q + W$, אז,
	\[
		P + W - Q
		= Q + W - Q
		= \{ w + Q - Q \mid w \in W \}
		= \{ w \mid w \in W \}
		= W
	\]
	ולכן בפרט $P + Q \in W$. \\
	נניח ש־$Q - P \in W$, אז מתקיים,
	\[
		P + W
		= P - Q + Q + W
		= Q + W
	\]
	ונסיק את הטענה הראשונה.

	עתה נראה שאם $R + W = R + W'$ אז $W = W'$.
	יהי $w \in R + W$, אז קיים $w' \in W$ כך ש־$w = R + w'$, וידוע כי גם $w \in W'$ ולכן קיים גם $w'' \in W'$ כך ש־$R + w'' = w$.
	אז בהתאם מתקיים $w'' = w'$ ולכן $W = W'$.

	עתה נעבור להוכחת טענת השאלה, נניח ש$P + W = Q + W'$.
	מתקיים,
	\[
		P + W = Q + W'
		\iff P - Q + W = Q - Q + W'
		\iff P - P + W = Q - P + W'
	\]
	ולכן נובע ש־$W = W'$ ומטענת העזר הראשונה נובע שגם $P - Q \in W$.

	נעבור לכיוון השני ונניח ש־$W = W'$ וכן ש־$P - Q \in W$.
	אז הטענה נובעת ישירות מטענת העזר הראשונה.
\end{proof}

\subquestion{}
תהי $F = P + W \le E$ תת־יריעה עבור $P \in E$ ו־$W \le V$. \\
נראה שלכל $Q \in F$ מתקיים $F = Q + W$.
\begin{proof}
	נבחין כי $Q \in F \iff Q = P + w$ עבור $w \in W$.
	בהתאם $Q - P = w \in W$ ולכן מסעיף הקודם נובע $Q + W = P + W = F$.
\end{proof}

\question{}
\begin{definition}
	יהיו $(F_1, W_1)$ ו־$(F_2, W_2)$ שתי תת־יריעות. \\
	נאמר שהן מקבילות אם $W_1 = W_2$ ונסמן $F_1 \parallel F_2$.
\end{definition}
נוכיח את משפט אוקלידס: לכל $F \subseteq E$ תת־יריעה ו־$P \in E \setminus F$ קיימת $P \in F' \subseteq E$ תת־יריעה יחידה מקבילה ל־$F$.
\begin{proof}
	נניח שמתקיים $F = Q + W$ עבור $Q \in E$ ו־$W \le V$. \\
	בהתאם $F' = P + W$ היא תת־יריעה של $E$, מהגדרתה היא מקבילה ל־$F$.

	נותר אם כן להוכיח יחידות, נניח שגם $F'' \le E$ תת־יריעה מקבילה ל־$F$, ונסיק ש־$F'' = P + W = F'$ מהשאלה הקודמת.
\end{proof}

\question{}
תהיינה תת־יריעות $F_1, F_2 \le E$. \\
נראה כי $F_1 \parallel F_2 \iff \exists v \in V,\ F_1 + v = F_2$ וכן שאם $F \le E$ אז לכל $v, w \in V$ מתקיים $F + v = F + w \iff v - w \in W$.
\begin{proof}
	נסמן $F_1 = P_1 + W_1, F_2 = P_2 + W_2$. \\
	נניח ש־$F_1 + v = F_2$ עבור $v \in V$, אז מתקיים $P_1 + W_1 + v = P_2 + W_2$ ולכן נובע ש־$W_1 = W_2$ (משאלה 3) ובהתאם $F_1 \parallel F_2$.

	נניח ש־$F_1 \parallel F_2$, כלומר $W_1 = W_2$.
	נסמן $v = P_1 - P_2 \in V$, אז מתקיים $P_1 + W_1 + v = P_1 + P_2 - P_1 + W_1 = P_2 + W_2$, כלומר $F_1 + v = F_2$.

	נניח עתה ש־$F \le E$ ויהיו $v, w \in V$.
	נניח גם ש־$F = P + W$. \\
	אם $F + v = F + w$ אז $P + v + W = P + w + W$ ולכן $(P - v) - (P - w) \in W$. \\
	נניח בכיוון ההפוך ש־$v - w \in W$, אז תת־היריעות מקבילות ובהכרח $F + v = F + w$.
\end{proof}

\question{}
תהי $S \subseteq E$ ונגדיר,
\[
	\langle S \rangle
	= \bigcap_{S \subseteq F \le E} F
\]
כלומר נאמר ש־$S$ יוצרת את תת־היריעה $\langle S \rangle$. \\
נוכיח כי $\langle S \rangle \le E$ וכן שהיא מינימלית ביחס ההכלה ביחס לתת־היריעות המכילות את $S$.
\begin{proof}
	אם $S = \emptyset$ אז היא תת־יריעה והיא מינימלית ביחס ההכלה, לכן נניח ש־$S \ne \emptyset$, ותהי $P \in S$ נקודה כלשהי.
	מתקיים $P \in F$ לכל $S \subseteq F \le E$ ולכן $F = P + W$ עבור $W \le V$.
	נוכל אם כן להסיק שמתקיים,
	\[
		\langle S \rangle
		= P + \bigcap_{S - P \subseteq W \le V} W
	\]
	כלומר $\langle S \rangle = P + \langle S - P \rangle$, כאשר $\langle S - P \rangle$ הוא תת־מרחב וקטורי הנוצר על־ידי $S - P$.
	נסמן $W' = \langle S - P \rangle$ ונסיק ש־$S = P + W' \le E$. \\
	נשאר להוכיח שתת־יריעה זו היא מינימלית ביחס ההכלה, אך טענה זו נובעת ישירות ממינימליות ביחס להכלה של תת־מרחב וקטורי נוצר.
\end{proof}

\end{document}
