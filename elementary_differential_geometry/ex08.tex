\input{../article_base.tex}
\title{פתרון מטלה 8 --- גאומטריה דיפרנציאלית אלמנטרית, 80560}

\begin{document}
\maketitle
\maketitleprint[purple]

\question{}
תהי $S = S^2(0, 1)$ ספירת היחידה, נחשב את התבנית היסודית בהצגה מטריצאלית תוך שימוש בפרמטריזציה,
\[
	\varphi(\theta, \phi)
	= (\sin \phi \cos \theta, \sin \phi \sin \theta, \cos \phi)
\]
\begin{solution}
	\[
		D \varphi |_{(\theta, \phi)}
		= \begin{pmatrix}
			- \sin \phi \sin \theta & \cos \phi \cos \theta \\
			\sin \phi \cos \theta & \cos \phi \sin \theta \\
			0 & - \sin \phi
		\end{pmatrix}
	\]
	ונזכור כי הגדרנו,
	\[
		g_{ij}
		= \langle D_i \varphi_{(\theta, \phi)}, D_j \varphi_{(\theta, \phi)} \rangle
	\]
	ולכן,
	\begin{align*}
		G
		& = \begin{pmatrix}
			\langle D_1 \varphi_{(\theta, \phi)}, D_1 \varphi_{(\theta, \phi)} \rangle & \langle D_2 \varphi_{(\theta, \phi)}, D_1 \varphi_{(\theta, \phi)} \rangle \\
			\langle D_1 \varphi_{(\theta, \phi)}, D_2 \varphi_{(\theta, \phi)} \rangle & \langle D_2 \varphi_{(\theta, \phi)}, D_2 \varphi_{(\theta, \phi)} \rangle
		\end{pmatrix} \\
		& = \begin{pmatrix}
			\sin^2 \phi \sin^2 \theta + \sin^2 \phi \cos^2 \theta & 0 \\
			0 & \cos^2 \phi \cos^2 \theta + \cos^2 \phi \sin^2 \theta + \sin^2 \phi
		\end{pmatrix} \\
		& = \begin{pmatrix}
			\sin^2 \phi & 0 \\
			0 & 1
		\end{pmatrix}
	.\end{align*}
\end{solution}

\question{}
יהיו $0 < r < R$ ממשיים.

\subquestion{}
נמצא הצגה פרמטרית לטורוס במרחב האוקלידי המתקבל על־ידי סיבוב של המעגל במישור $y, z$ עם מרכז בנקודה $(0, R, 0)$ ורדיוס $r$ סביב ציר $z$.
\begin{solution}
	למעשה כבר מצאנו אחת במטלה הקודמת, נצטט,
	\[
		\varphi(u, v)
		= (\cos(v) \cdot (R + r \cos u), \sin(v) \cdot (R + r \cos u), r \sin u)
	\]
	אך זוהי פרמטריזציה כך ש־$\varphi(u, 0)$ הוא מעגל סביב $(R, 0, 0)$ ולכן נשנה ונגדיר,
	\[
		\varphi(u, v)
		= (\sin(v) \cdot (R + r \cos u), \cos(v) \cdot (R + r \cos u), r \sin u)
	\]
	הפעם מבדיקה ישירה נקבל,
	\[
		\varphi(\{ 0 \} \times [0, 2 \pi])
		= S((0, R, 0), r) \cap \{ 0 \} \times \RR^2
	\]
	כמבוקש.
\end{solution}

\subquestion{}
נמצא משוואה המגדירה את הטורוס.
\begin{solution}
	במטלה הקודמת הראינו כי,
	\[
		\varphi({[0, 2 \pi]}^2)
		= \{ (x, y, z) \in \EE^3 \mid {(x^2 + y^2 + z^2 + R^2 - r^2)}^2 = 4 R^2 (x^2 + y^2) \}
	\]
	ולכן המשוואה ${(x^2 + y^2 + z^2 + R^2 - r^2)}^2 = 4 R^2 (x^2 + y^2)$ היא משוואה של הטורוס.
\end{solution}

\subquestion{}
נגדיר את הפרמטריזציה,
\[
	\varphi(\theta, \phi)
	= ((R + r \cos \phi) \cos \theta, (R + r \cos \phi) \sin \theta, r \sin \phi)
\]
ונחשב את התבנית היסודית הראשונה שלה.
\begin{solution}
	\[
		D \varphi |_{(\theta, \phi)}
		= \begin{pmatrix}
			- (R + r \cos \phi) \sin \theta & - r \sin \phi \cos \theta \\
			(R + r \cos \phi) \cos \theta & -r \sin \phi \sin \theta \\
			0 & r \cos \phi
		\end{pmatrix}
	\]
	ולכן,
	\begin{align*}
		G
		& = \begin{pmatrix}
			\langle D_1 \varphi_{(\theta, \phi)}, D_1 \varphi_{(\theta, \phi)} \rangle & \langle D_2 \varphi_{(\theta, \phi)}, D_1 \varphi_{(\theta, \phi)} \rangle \\
			\langle D_1 \varphi_{(\theta, \phi)}, D_2 \varphi_{(\theta, \phi)} \rangle & \langle D_2 \varphi_{(\theta, \phi)}, D_2 \varphi_{(\theta, \phi)} \rangle
		\end{pmatrix} \\
		& = \begin{pmatrix}
			{(R + r \cos \phi)}^2 (\sin^2 \theta + \cos^2 \theta) & (R + r \cos \phi) r \sin \phi (\cos \theta \sin \theta - \sin \theta \cos \theta) \\
			(R + r \cos \phi) r \sin \phi (\cos \theta \sin \theta - \sin \theta \cos \theta) & r^2 \sin^2 \phi (\cos^2 \theta + \sin^2 \theta) + r^2 \cos^2 \phi
		\end{pmatrix} \\
		& = \begin{pmatrix}
			{(R + r \cos \phi)}^2 & 0 \\
			0 & r^2
		\end{pmatrix} \\
	\end{align*}
\end{solution}

\question{}
יהי $S$ היפרבולואיד הנתון על־ידי המשוואה $x^2 + y^2 - z^2 = 1$. \\
נחשב את התבנית היסודית הראשונה בהצגה מטריצאלית הפרמטריזציה,
\[
	\varphi(t, u)
	= (\cosh t \cos u, \cosh t \sin u, \sinh t)
\]
\begin{solution}
	\[
		D \varphi |_{(t, u)}
		= \begin{pmatrix}
			\sinh t \cos u & - \cosh t \sin u \\
			\sinh t \sin u & \cosh t \cos u \\
			\cosh t & 0
		\end{pmatrix}
	\]
	ולכן,
	\begin{align*}
		G
		& = \begin{pmatrix}
			\langle D_1 \varphi_{(\theta, \phi)}, D_1 \varphi_{(\theta, \phi)} \rangle & \langle D_2 \varphi_{(\theta, \phi)}, D_1 \varphi_{(\theta, \phi)} \rangle \\
			\langle D_1 \varphi_{(\theta, \phi)}, D_2 \varphi_{(\theta, \phi)} \rangle & \langle D_2 \varphi_{(\theta, \phi)}, D_2 \varphi_{(\theta, \phi)} \rangle
		\end{pmatrix} \\
		& = \begin{pmatrix}
			\sinh^2 t + \cosh^2 t & 0 \\
			0 & \cosh^2 t
		\end{pmatrix} \\
		& = \begin{pmatrix}
			\cosh(2t) & 0 \\
			0 & \cosh^2 t
		\end{pmatrix}
	\end{align*}
\end{solution}

\question{}
יהי $S$ היפרבולואיד הנתון על־ידי המשוואה $x^2 + y^2 - z^2 = -1$. \\
נחשב את התבנית היסודית הראשונה בהצגה מטריצאלית הפרמטריזציה,
\[
	\varphi(t, u)
	= (\sinh t \cos u, \sinh t \sin u, \cosh t)
\]
\begin{solution}
	\[
		D \varphi |_{(t, u)}
		= \begin{pmatrix}
			\cosh t \cos u & -\sinh t \sin u \\
			\cosh t \sin u & \sinh t \cos u \\
			\sinh t & 0
		\end{pmatrix}
	\]
	ולכן,
	\begin{align*}
		G
		& = \begin{pmatrix}
			\langle D_1 \varphi_{(\theta, \phi)}, D_1 \varphi_{(\theta, \phi)} \rangle & \langle D_2 \varphi_{(\theta, \phi)}, D_1 \varphi_{(\theta, \phi)} \rangle \\
			\langle D_1 \varphi_{(\theta, \phi)}, D_2 \varphi_{(\theta, \phi)} \rangle & \langle D_2 \varphi_{(\theta, \phi)}, D_2 \varphi_{(\theta, \phi)} \rangle
		\end{pmatrix} \\
		& = \begin{pmatrix}
			\cosh^2 t + \sinh^2 t & 0 \\
			0 & \sinh^2 t
		\end{pmatrix} \\
	\end{align*}
\end{solution}

\question{}
יהי $S$ החרוט הנתון על־ידי $x^2 + y^2 = z^2, z > 0$.
נחשב את התבנית היסודית הראשונה עבור הפרמטריזציה,
\[
	\varphi(\theta, r)
	= (r \cos \theta, r \sin \theta, r)
\]
\begin{solution}
	\[
		D \varphi |_{(\theta, r)}
		= \begin{pmatrix}
			- r \sin \theta & \cos \theta \\
			r \cos \theta & \sin \theta \\
			0 & 1
		\end{pmatrix}
	\]
	ולכן,
	\[
		G
		= \begin{pmatrix}
			r^2 & 0 \\
			0 & 2
		\end{pmatrix}
	\]
\end{solution}

\end{document}
