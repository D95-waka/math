\input{../article_base.tex}
\title{פתרון מבחן בית --- גאומטריה דיפרנציאלית אלמנטרית, 80560}

% chktex-file 23

\begin{document}
\maketitle
\maketitleprint[purple]

\tableofcontents

\question{}
יהי $(E, V)$ מרחב אפיני.

\subquestion{}
תהי $S \subseteq E$ קבוצה ותהי $S' = \langle S \rangle$ תת־היריעה הנוצרת על־ידי $S$.
נראה שמתקיים,
\[
	S'
	= \left\{ \sum_{i = 1}^n \lambda^i P_i \mid \sum_{i}^n \lambda^i = 1, P_1, \ldots, P_n \in S \right\}
\]
\begin{proof}
	נראה הכלה דו־כיוונית, נניח תחילה ש־$P = \lambda^1 P_1 + \cdots + \lambda^n P_n$ עבור $P_1, \ldots, P_n \in S, \lambda^1 + \cdots + \lambda^n = 1$, ונראה ש־$P \in S'$.
	נבחר $Q \in S$ שונה מ־$P_1, \ldots, P_n$, אם אין כזאת אז בכל מקרה סיימנו, ממשפט אינווריאנטיות לצירוף אפיני נקבל,
	\[
		P
		= Q + \lambda^1 (P_1 - Q) + \cdots + \lambda^n (P_n - Q)
	\]
	ולכן קיבלנו שמהגדרת $S' = Q + W$ עבור $W \le E$ אכן $P \in S'$.

	לכיוון ההפוך נראה שאם $P \in S'$ אז קיימות $P_1, \ldots, P_n, \lambda^1 + \cdots + \lambda^n = 1$ המקיימות $P = \lambda^1 P_1 + \cdots + \lambda^n P_n$.
	נקבע נקודה שרירותית $Q \in S$ וכן נסמן שוב $S' = Q + W$, נניח ש־$\dim W = n$ ולכן קיימות $P_1, \ldots, P_n \in S$ כך ש־$((P_1 - Q), \ldots, (P_n - Q)) \subseteq W$ בסיס שלו.
	נבחין כי אם $Q \in \{P_1, \ldots, P_n\}$ אז המרחב סופי ובפרט גם $P \in \{P_1, \ldots, P_n\}$ ולכן נוכל להניח אחרת. \\
	עתה נקבל $P \in S' \iff P - Q \in W$ ולכן $P - Q = \alpha^1 (P_1 - Q) + \cdots + \alpha^n (P_n - Q)$ עבור $\alpha^1, \ldots, \alpha^n \in \FF$.
	נסיק,
	\[
		P
		= (P - Q) + Q
		= Q + \alpha^1 (P_1 - Q) + \cdots + \alpha^n (P_n - Q)
	\]
	נסמן נקודה נוספת $P_{n + 1} \in S$ (שוב נניח שיש כזו אחרת סיימנו) ונבנה את מערכת המשוואות,
	\[
		\lambda^1 + \cdots + \lambda^{n + 1} = 1,
		\quad
		\sum_{i = 1}^{n + 1} \lambda^i (P_i - Q) = P - Q
	\]
	זוהי מערכת משוואות מדרגה מלאה ובשל קיום הצירוף $\alpha^1 P^1 + \cdots + \alpha^n P^n$ נסיק שיש פתרון למשוואה. \\
	קיבלנו הכלה דו־כיוונית ולכן הטענה חלה.
\end{proof}

\subquestion{}
נראה שאם $L \subseteq E$ קבוצת נקודות,
אז $L$ תת־יריעה אפינית אם ורק אם לכל $P, Q \in L$ מתקיים $\langle P, Q \rangle \subseteq L$.
\begin{proof}
	נניח ש־$L$ תת־יריעה אפינית, אז מהסעיף הקודם נובעת סגירות לצירופים אפיניים, בפרט לצירופים של שתי נקודות,
	\[
		\left\{ \sum_{i = 1}^2 \lambda^i P_i \mid \sum_{i}^2 \lambda^i = 1, P_1, P_2 \in L \right\}
		\subseteq L
	\]
	עבור $P, Q \in L$ נקבל ש־$\langle P, Q \rangle \subseteq L$ כהגדרת קבוצה זו.

	נניח בכיוון ההפוך שלכל $P, Q \in L$ גם $\langle P, Q \rangle \subseteq L$. \\
	תהי $P_0 \in L$ ונסמן $W = L - P_0$, נראה ש־$W \le V$ ובכך נקבל ש־$L$ תת־יריעה אפינית. \\
	יהי $u \in W$, ו־$\alpha \in \FF$, אז $\langle P_0, P_0 + u \rangle \subseteq L$, אבל $\langle P_0, P_0 + u \rangle = P_0 + \Sp\{ u \}$ ולכן $\Sp\{ u \} \subseteq L$ ו־$\alpha u \in W$.
	קיבלנו סגירות לכפל בסקלר, נעבור להראות סגירות לחיבור. \\
	נניח ש־$u, v \in W$ וש־$\operatorname{char} \FF \ne 2$ ונראה ש־$u + v \in W$.
	נסמן $P_1 = P_0 + u$ וכן $P_2 = P_0 + 2v$, נקבל שגם $\langle P_1, P_2 \rangle \subseteq L$, כלומר $P_1 + (2v - u) t \in L$ לכל $t \in \FF$, לכן גם $P_0 + u + (2v - u) t \in L$ ולכן $u + (v - u) t \in W$.
	נבחר $t = \frac{1}{2}$ ונקבל ש־$u + \frac{1}{2} v - \frac{1}{2} u = \frac{1}{2}(u + v) \in W$, ומסגירות לכפל בסקלר שמצאנו נקבל שגם $u + v \in W$. \\
	נסיק ש־$W$ הוא תת־מרחב לינארי, לכן $P_0 + W = L$ מרחב אפיני.

	נעיר שדרשנו בהוכחה זו שמציין השדה יהיה שונה מ־$2$, נראה שהטענה לא חלה במקרה זה.
	נגדיר $\FF = \FF_2$ וכן $E = V = \FF_2^2$, ולבסוף גם $L = \{ (0, 0), (1, 0), (0, 1) \}$ ונראה שלכל $P, Q \in E$ מתקיים $\langle P, Q \rangle \subseteq L$.
	הבדיקה נעשית במעבר על כלל האפשרויות,
	\begin{align*}
		& \langle (0, 0), (1, 0) \rangle
		= (0, 0) + \Sp\{ (1, 0) \}
		= \{ (0, 0), (1, 0) \} \\
		& \langle (0, 0), (0, 1) \rangle
		= \{ (0, 0), (0, 1) \} \\
		& \langle (1, 0), (0, 1) \rangle
		= \{ (1, 0), (0, 1) \}
	\end{align*}
	וקיבלנו ש־$L$ אכן מקיימת את התנאי, נראה ש־$L$ היא לא תת־יריעה אפינית.
	נניח בשלילה שהיא כן ולכן $L = (0, 0) + W = W$ עבור $W \le \FF^2$, כלומר בשל בחירת מרחב הנקודות נוכל לזהות את $L$ כמרחב לינארי, ובמקרה זה $(1, 0), (0, 1) \in W$ אבל $(1, 0) + (0, 1) = (1, 1) \notin W$ בסתירה.
\end{proof}

\question{}
יהי $V$ מרחב וקטורי ממימד סופי מעל $\FF$ ו־$V^\vee$ המרחב הדואלי של $V$.

\subquestion{}
נראה שלכל $v \in V$ מתקיים,
\[
	\forall l \in V^\vee,\ 
	\langle l, v \rangle = 0 \iff v = 0
\]
\begin{proof}
	נניח ש־$v = 0$ ויהי $l \in V^\vee$, כלומר $l : V \to \FF$ העתקה לינארית.
	אז מתקיים $l(v) = 0$ מהגדרת ההעתקה הלינארית.

	לכיוון ההפוך נניח שלכל $l \in V^\vee$ מתקיים $l(v) = 0$ ונניח בשלילה ש־$v \ne 0$.
	נרחיב את $( v )$ לבסיס $\Bb = (v, b_2, \ldots, b_n)$ הפורש את $V$.
	בהתאם קיימת העתקה לינארית $l : V \to \FF$ כך שמתקיים $l(v) = l(b_i) = 1$, אבל $l \in V^\vee$ ולכן מההנחה $\langle l, v \rangle = 0 \ne 1$ בסתירה.
\end{proof}

\subquestion{}
יהי $l \in V^\vee$, נוכיח ש־$l = 0$ אם ורק אם לכל $v \in V$ מתקיים $\langle l, v \rangle = 0$.
\begin{proof}
	נניח ש־$l = 0$, אז בהגדרה $\langle l, v \rangle = l(v) = 0$ לכל $v$.

	נניח ש־$\langle l, v \rangle = 0$ לכל $v \in V$ ונניח ש־$l \ne 0$, לכן קיים $u \in \im l$ כך ש־$u \ne 0$ וכן $l(u) \ne 0$ בסתירה.
\end{proof}

\subquestion{}
נראה שלכל $W \le V^\vee$ מתקיים,
\[
	\dim_\FF W + \dim_\FF W_0 = \dim_\FF V
\]
\begin{proof}
	הגדרנו $W_0 = \{ v \in V \mid \forall l \in W,\ l(v) = 0 \}$.
	נניח ש־$\Bb = (b_1, \ldots, b_n)$ בסיס של $V$ וכן נסמן $\Bb^\vee = (b^1, \ldots, b^n)$.
	נניח בלי הגבלת הכלליות ש־$(b^1, \ldots, b^k)$ עם $k \le n$ הוא בסיס סדור של $W$.
	אז מהגדרה מתקיים $l(b^i) = 0$ לכל $l \in W$ ו־$k < i \le n$, ולכן $b_i \in W_0$. באופן דומה נסיק ש־$b_i \notin W_0$ לכל $i \le k$ שכן קיים עד לכך ש־$b^i \in W$.
	קיבלנו אם כך ש־$(b_{k + 1}, \ldots, b_n)$ בסיס סדור של $W_0$.
\end{proof}

\subquestion{}
יהיו $S_1, S_2 \le V$ ונראה ש־${(S_1 + S_2)}^0 = S_1^0 \cap S_2^0$.
\begin{proof}
	יהי $l \in {(S_1 + S_2)}^0$, אז $l(v + u) = l(v) + l(u) = 0$ לכל $u \in S_1, v \in S_2$.
	נבחין כי $0 \in S_1, S_2$ כתת־מרחבים ולכן נקבל שגם $l(v) = 0, l(u) = 0$ לכל $u, v$ כאלה, ונסיק $l \in S_1^0, l \in S_2^0$ ובפרט נמצא בחיתוך.

	מהצד השני נניח ש־$l \in S_1^0 \cap S_2^0$, אז בפרט $l(u) = l(v) = 0$ לכל $u \in S_1, v \in S_2$, ולכן גם $l(u + v) = l(u) + l(v) = 0$ ונסיק ש־$l \in {(S_1 + S_2)}^0$.
\end{proof}

\subquestion{}
נראה שאם $L^1, L^2 \le V^\vee$ אז ${(L^1 \cap L^2)}_0 = L_0^1 + L_0^2$.
\begin{proof}
	מהסעיף הקודם נסיק $L^1 \cap L^2 = {(L_0^1 + L_0^2)}^0$, ולכן ${(L^1 \cap L^2)}_0 = {(L_0^1 + L_0^2)}^0_0 = L_0^1 + L_0^2$ מזהויות שהוכחנו.
\end{proof}

\question{}
עקום חלק עם פרמטריזציה לפי אורך $c : I \to \EE^3$ נקרא בי־רגולרי אם $\forall t \in I,\ c''(t) \ne 0$.
נסמן $v = c'$ ונגדיר את הנורמל על־ידי $n = \frac{c''}{|c''|}$ ואת הבי־נורמל על־ידי $b = v \wedge n$.
נבחין כי $(v, n, b)$ בסיס אורתונורמלי.

\subquestion{}
נראה שקיימים $\kappa, \tau \in \RR$ כך שמתקיים,
\[
	\begin{pmatrix}
		v' \\ n' \\ b'
	\end{pmatrix}
	=
	\begin{pmatrix}
		0 & \kappa & 0 \\
		- \kappa & 0 & \tau \\
		0 & - \tau & 0
	\end{pmatrix}
	\begin{pmatrix}
		v \\ n \\ b
	\end{pmatrix}
\]
\begin{proof}
	נבחין שהטענה שקולה לשוויון,
	\[
		\begin{pmatrix}
			v' \\ n' \\ b'
		\end{pmatrix}
		=
		\begin{pmatrix}
			0 + \kappa n + 0 \\
			- \kappa v + 0 + \tau b \\
			0 - \tau n + 0
		\end{pmatrix}
	\]
	כלומר הטענה שקולה לשוויונות,
	\[
		v' = \kappa n,
		\qquad
		n' = - \kappa v + \tau b,
		\qquad
		b' = - \tau n
	\]
	ראינו בכיתה ש־$\kappa = \kappa(t)$ העקמומיות בנקודה אכן קיימת, וש־$v' = \kappa n$.
	נזכיר כי טענה זו נובעת מהעובדה ש־$\lVert v' \rVert = 1$ ולכן $v' \perp v$ מהעובדה שהפרמטריזציה היא לפי אורך.

	נבחין שגם $\lVert v \rVert = 1, \lVert n \rVert = 1$ ולכן $\lVert b \rVert = 1 \cdot 1 - | v \cdot n | = 1$, וכן $v \cdot b = 0$ ולכן ניתן להסיק של־$b'$ יש רק רכיב אורתוגונלי.
	נקבל שגם $(v \cdot b)' = v' \cdot b + v \cdot b' = 0$ אבל $v' \cdot b = n \cdot b \cdot |c''| = 0$ ולכן נסיק ש־$b' \cdot v = 0$, ובהתאם מתקיים $b' = - \tau n$ לאיזשהו $\tau = \tau(t)$.
	נשים לב כי יכולנו להגדיר גם ללא המינוס.

	מתקיים,
	\[
		0
		= \begin{vmatrix}
			v & n & b
		\end{vmatrix}
		= - \begin{vmatrix}
			v & b & n
		\end{vmatrix}
		= \begin{vmatrix}
			b & v & n
		\end{vmatrix}
	\]
	כלומר גם $n = b \wedge v$ ולכן מחוקי גזירה,
	\[
		n'
		= b' \wedge v + b \wedge v'
		= (-\tau n) \wedge v + b \wedge (\kappa n)
		= -\tau (n \wedge v) + \kappa (b \wedge n)
		= \tau b - \kappa v
	\]
	כפי שרצינו.
\end{proof}

\subquestion{}
תהי $f : \EE^3 \to \EE^3$ העתקה אפינית שומרת מרחק וכיוון, כלומר $f(x) = Ax + b$ עבור $A \in SO_3(\RR)$ ו־$b \in \RR^3$. \\
נראה ש־$f \circ c$ ו־$c$ בעלות עקמומיות ופיתול משותפים.
\begin{proof}
	נגזור את $f \circ c$ ונסמן ב־$v_1$,
	\[
		v_1
		= (f \circ c)'
		= (f' \circ c) \cdot c'
		= A v
	\]
	שכן $f' \equiv A$.
	בהתאם גם $(f \circ c)'' = A v'$ ולכן אם נסמן $n_1, b_1$ הנורמל והבי־נורמל של $f \circ c$ אז נקבל,
	\[
		n_1
		= \frac{A v'}{|A v'|}
		= \frac{A v'}{|v'|}
		= A n
	\]
	נסמן עתה גם $\kappa_1, \tau_1$ העקמומיות והפיתול של $f \circ c$ ונקבל,
	\[
		v_1' = \kappa_1 n_1
		\iff A v' = \kappa_1 A n
		\iff v' = \kappa_1 n
	\]
	ונסיק ש־$\kappa = \kappa_1$.
	
	נעבור לבדיקה של $b_1$,
	\[
		b_1
		= v_1 \wedge n_1
		= (A v) \wedge (A n)
	\]
	כלומר,
	\[
		0
		= \begin{vmatrix}
			A v & A n & b_1
		\end{vmatrix}
		= A \begin{vmatrix}
			v & n & A^{-1} b_1
		\end{vmatrix}
	\]
	כאשר המעבר האחרון נובע ישירות מהגדרת העתקות לינאריות בהצגה מטריצאלית.
	נשתמש בהפיכות $A$ כדי להסיק,
	\[
		A \begin{vmatrix}
			v & n & A^{-1} b_1
		\end{vmatrix}
		= \begin{vmatrix}
			v & n & A^{-1} b_1
		\end{vmatrix}
		= \begin{vmatrix}
			v & n & b
		\end{vmatrix}
	\]
	ונקבל ש־$b_1 = A b$ ולכן,
	\[
		b_1' = - \tau_1 n_1
		\iff A b' = - \tau_1 A n
	\]
	ונסיק ש־$\tau_1 = \tau$ כפי שרצינו.
\end{proof}
נבחין שאם $f$ משנה כיוון אז לא נוכל לבצע את המעבר $-\tau A n = A (-\tau n)$, ונקבל בהתאם שהעקמומיות והפיתול משנים גם הם סימן.

\subquestion{}
נראה ש־$c(I)$ ניתנת לשיכון במישור אם ורק אם $\tau = 0$.
\begin{proof}
	נניח ש־$c(I)$ ניתנת לשיכון במישור.
	נסמן $n_0$ וקטור המקיים $n_0 \perp c(I)$ ונניח ש־$\lVert n_0 \rVert = 1$.
	נבחין ש־$v \perp n_0$ אחרת מערך הממוצע נקבל שיש ל־$c$ נקודה מחוץ למישור.
	מאותו שיקול בדיוק נקבל שגם $n \perp n_0$ ולכן $b \equiv \pm n_0$, נניח על־ידי שינוי $n_0$ ש־$b \equiv n_0$.
	נעיר ש־$b$ רציף ולכן לא יתכן שהוא מקבל גם $n_0$ וגם $- n_0$.
	אנו יודעים ש־$b' \equiv 0$ כנגזרת של קבועה וכן ש־$0 = b' = - \tau n$, אבל $n$ לא מתאפסת ולכן $\tau \equiv 0$ בלבד.

	בכיוון ההפוך נניח ש־$\tau \equiv 0$, אבל $n \ne 0$ ולכן $b' \equiv 0$ בלבד.
	בהתאם נסיק ש־$b$ קבוע, כלומר קיים וקטור $n_0$ קבוע כך ש־$c', c'' \perp n_0$ בכל $I$.
	מכאן נובע ש־$c(I)$ משוכן במישור. כדי להראות זאת נניח ש־$0 \in C(I)$ ויהי בסיס אורתוגונלי $(e_1, e_2, n_0)$, אז נקבל ש־$D_{n_0} c = 0$ ולכן $\pi_{n_0}$ פונקציה קבועה ובפרט $\pi_{n_0} \equiv 0$, כאשר $\pi_{n_0}$ ההטלה לציר.
\end{proof}

\subquestion{}
יהי $c : I \to \EE^3$ עקום רגולרי כלשהו.
נראה שמתקיים,
\[
	\kappa(t)
	= \frac{|c'(t) \wedge c''(t)|}{{|c'(t)|}^3}
\]
וכן שאם $c$ הוא בי־רגולרי אז גם,
\[
	\tau(t)
	= \frac{\begin{vmatrix} c'(t) & c''(t) & c^{(3)}(t) \end{vmatrix}}{{| c'(t) \wedge c''(t) |}^2}
\]
\begin{proof}
	נגדיר רפרמטריזציה לפי אורך של $c$, $\tilde{c} : J \to \EE^3$, ונגדיר את הדיפאומורפיזם $\varphi : J \to I$ כך ש־$c \circ \varphi = \tilde{c}$.
	בהתאם נובע ש־$\tilde{c}' = (c' \circ \varphi) \cdot \varphi'$ מכלל השרשרת.
	$\tilde{c}$ היא פרמטריזציה לפי אורך ולכן,
	\[
		\tilde{v}' = \tilde{\kappa} \tilde{n}
		\iff
		\tilde{c}'' = \tilde{\kappa} \frac{\tilde{c}''}{|\tilde{c}''|}
		\implies \tilde{\kappa} = \lVert \tilde{c}'' \rVert
	\]
	אז נובע,
	\[
		\tilde{\kappa}
		= \lVert \tilde{c}'' \rVert
		= \lVert ((c' \circ \varphi) \cdot \varphi')' \rVert
		= \lVert (c' \circ\varphi)' \varphi' + (c' \circ \varphi) \varphi'' \rVert
		= \lVert (c'' \circ \varphi) {(\varphi')}^2 + (c' \circ \varphi) \varphi'' \rVert
	\]
	אבל $\varphi' = \frac{1}{\lVert c' \circ \varphi \rVert}$ מהגדרתה ובהתאם $\varphi'' = \frac{- (c'' \circ \varphi) \varphi'}{{\lVert c' \circ \varphi\rVert}^2} = \frac{- (c'' \circ \varphi)}{{\lVert c' \circ \varphi\rVert}^3}$,
	\[
		\tilde{\kappa}
		= \frac{| \lVert c'' \circ \varphi \rVert \lVert c' \circ \varphi \rVert - (c' \circ \varphi) (c'' \circ \varphi) |}{{\lVert c' \circ \varphi \rVert}^3}
		= \frac{|c' \circ \varphi \wedge c'' \circ \varphi|}{{\lVert c' \circ \varphi \rVert}^3}
	\]
	כאשר הזהות האחרונה הוכחה בתרגיל.

	מסעיף א' $\tilde{b}' = - \tilde{\tau} \tilde{n}$.
	נחשב,
	\[
		\tilde{b}'
		= (\tilde{v} \wedge \tilde{n})'
		= \tilde{v}' \wedge \tilde{n} + \tilde{v} \wedge \tilde{n}'
		= \tilde{\kappa} \tilde{n} \wedge \tilde{n} + \tilde{v} \wedge (- \tilde{\kappa} \tilde{v} + \tilde{\tau} \tilde{b})
		= \tilde{\kappa} (\tilde{n} \wedge \tilde{n}) - \tilde{\kappa} (\tilde{v} \wedge \tilde{v}) + \tilde{\tau} (\tilde{v} \wedge \tilde{b})
		= \tilde{\tau} (\tilde{v} \wedge \tilde{b})
	\]

	אם הפרמטריזציה היא לפי אורך אז,
	\[
		-\tau n = b'
		\iff \tau n \cdot n
		= -n \cdot b'
		= -n (v \wedge n)'
		= -n (v' \wedge n + v \wedge n')
		= 0 - n (v \wedge n')
	\]
	כלומר $\tau = -n \cdot (v \wedge n')$.
	אז,
	\begin{align*}
		\tau
		& = -\frac{c''}{\lVert c'' \rVert} \left(c' \times \left(\frac{c''}{\lVert c'' \rVert}\right)'\right) \\
		& = -\frac{c''}{\lVert c'' \rVert} \left(c' \times \left( \frac{1}{\lVert c'' \rVert} c''' - \frac{\lVert c'' \rVert'}{{\lVert c'' \rVert}^2} c'' \right) \right) \\
		& = -\frac{c''}{\lVert c'' \rVert} \left(c' \times \left( \frac{1}{\lVert c'' \rVert} c''' - \frac{\lVert c'' \rVert'}{{\lVert c'' \rVert}^2} c'' \right) \right) \\
		& \overset{(1)}{=} -\frac{c''}{\lVert c'' \rVert} \left(c' \times \left( \frac{1}{\lVert c'' \rVert} c''' \right) \right) \\
		& = -\frac{c''}{{\lVert c'' \rVert}^2} (c' \times c''') \\
		& = \frac{(c' \times c'') \cdot c'''}{{\lVert c'' \rVert}^2} \\
		& \overset{(2)}{=} \frac{(c' \times c'') \cdot c'''}{{\lVert c' \wedge c'' \rVert}^2} \\
		& \overset{(3)}{=} \frac{\begin{vmatrix} c' & c'' & c''' \end{vmatrix}}{{\lVert c' \wedge c'' \rVert}^2}
	\end{align*}
	כאשר,
	\begin{enumerate}
		\item $c'' \cdot (c' \times c'') = 0$
		\item מהנוסחה הראשונה וההנחה שהפרמטריזציה לפי אורך
		\item זהות
	\end{enumerate}
	וקיבלנו שהנוסחה נכונה למקרה זה.

	נעבור למקרה הכללי, נניח ש־$\tilde{c} = c \circ \varphi$ רפרמטריזציה לפי אורך, ולכן,
	\[
		\tilde{\tau}
		= \frac{\begin{vmatrix} \tilde{c}' & \tilde{c}'' & \tilde{c}''' \end{vmatrix}}{{\lVert \tilde{c}' \wedge \tilde{c}'' \rVert}^2}
		= \frac{(\tilde{c}' \times \tilde{c}'') \cdot \tilde{c}'''}{{\lVert (c \circ \varphi)' \wedge (c \circ \varphi)'' \rVert}^2}
	\]
	נזכור כי מצאנו שמתקיים $(c \circ \varphi)' = (c' \circ \varphi) \varphi', \qquad (c \circ \varphi)'' = (c'' \circ \varphi) {(\varphi')}^2 + (c' \circ \varphi) \varphi''$ ולכן,
	\[
		(c \circ \varphi)'''
		= (c''' \circ \varphi) {(\varphi')}^3 + (c'' \circ \varphi) 2 \varphi' \varphi'' + (c'' \circ \varphi) \varphi' \varphi'' + (c' \circ \varphi) \varphi'''
		= (c''' \circ \varphi) {(\varphi')}^3 + 3 (c'' \circ \varphi) \varphi' \varphi'' + (c' \circ \varphi) \varphi'''
	\]
	ובהתאם נחשב,
	\[
		\tilde{c}' \wedge \tilde{c}''
		= (c \circ \varphi)' \wedge (c \circ \varphi)''
		= (c' \circ \varphi) \varphi \wedge ((c'' \circ \varphi) {(\varphi')}^2 + (c' \circ \varphi) \varphi'')
		= {(\varphi')}^3 (c' \circ \varphi) \wedge (c'' \circ \varphi)
	\]
	ולכן גם,
	\[
		(\tilde{c}' \times \tilde{c}'') \cdot \tilde{c}'''
		= {(\varphi')}^3 ((c' \circ \varphi) \wedge (c'' \circ \varphi)) \cdot ((c''' \circ \varphi) {(\varphi')}^3 + 3 (c'' \circ \varphi) \varphi' \varphi'' + (c' \circ \varphi) \varphi''')
		= {(\varphi')}^6 ((c' \circ \varphi) \wedge (c'' \circ \varphi)) \cdot (c''' \circ \varphi)
	\]
	נציב,
	\[
		\tau \circ \varphi
		= \frac{(\tilde{c}' \times \tilde{c}'') \cdot \tilde{c}'''}{{\lVert (c \circ \varphi)' \wedge (c \circ \varphi)'' \rVert}^2}
		= \frac{{(\varphi')}^6 ((c' \circ \varphi) \wedge (c'' \circ \varphi)) \cdot (c''' \circ \varphi)}{{(\varphi')}^{2 \cdot 3} {\lVert (c' \circ \varphi) \wedge (c'' \circ \varphi) \rVert}^2}
	\]
	וקיבלנו שהטענה נכונה גם במקרה הכללי.
\end{proof}

\question{}
נניח ש־$S \subseteq \EE^3$ משטח רגולרי ו־$f : U \to S$ פרמטריזציה מקומית ל־$p \in S$, ונניח ש־$f(u) = p$.
יהיו $c, d : I \to S$ עקומים רגולריים ונסמן $c = f \circ \gamma, d = f \circ \phi$, כאשר $\gamma(0) = \phi(0) = u$.

\subquestion{}
נראה ש־$f$ היא קונפורמית אם ורק אם $E = G, F = 0$, כאשר$E, F, G$ מקדמי התבנית היסודית הראשונה $I$.
\begin{proof}
	נגדיר את הזווית בין שני העקומים ב־$u$ על־ידי הזווית של $\gamma'(0), \phi'(0)$, זווית זו מוגדרת להיות $\cos \theta = \frac{\langle \gamma'(0), \phi'(0) \rangle}{\lVert \gamma'(0) \rVert \cdot \lVert \phi'(0) \rVert}$.

	נניח ש־$f$ היא קונפורמית, כלומר מתקיים,
	\[
		\arccos \frac{\langle c'(0), d'(0) \rangle}{\lVert c'(0) \rVert \cdot \lVert d'(0) \rVert}
		= \arccos \frac{\langle f'(\gamma((0))) \gamma'(0), f'(\phi((0))) \phi'(0) \rangle}{\lVert c'(0) \rVert \cdot \lVert d'(0) \rVert}
		= \arccos \frac{\langle \gamma'(0), \phi'(0) \rangle}{\lVert \gamma'(0) \rVert \cdot \lVert \phi'(0) \rVert}
	\]
	לכל שני עקומים $c, d$ כאלה.
	נבחין כי $\arccos$ חד־חד ערכית ועל, ולכן התנאי שקול לתנאי,
	\[
		\frac{\langle c'(0), d'(0) \rangle}{\lVert c'(0) \rVert \cdot \lVert d'(0) \rVert}
		= \frac{\langle \gamma'(0), \phi'(0) \rangle}{\lVert \gamma'(0) \rVert \cdot \lVert \phi'(0) \rVert}
	\]
	נבחין גם כי מתקיים,
	\[
		\langle c'(0), d'(0) \rangle
		= \langle D f |_{\gamma(0)} \cdot \gamma'(0), D f |_{\phi(0)} \cdot \phi'(0) \rangle
		= \langle D f |_u \cdot \gamma'(0), D f |_u \cdot \phi'(0) \rangle
		= I_p(\gamma'(0), \phi'(0))
	\]
	כלומר מתקיים,
	\[
		\frac{I_p(\gamma'(0), \phi'(0))}{I_p(\gamma'(0), \gamma'(0)) \cdot I_p(\phi'(0), \phi'(0))}
		= \frac{\langle \gamma'(0), \phi'(0) \rangle}{{\lVert \gamma'(0) \rVert}^2 \cdot {\lVert \phi'(0) \rVert}^2}
	\]

	עתה נציב מספר עקומים בזהות שקיבלנו.
	עבור עקומים לפי אורך ואנכים נקבל,
	\[
		\frac{I_p(\gamma'(0), \phi'(0))}{I_p(\gamma'(0), \gamma'(0)) \cdot I_p(\phi'(0), \phi'(0))}
		= 0
		\implies I_p(\gamma'(0), \phi'(0)) = 0
		\implies F = 0
	\]
	אז עבור עקומים כלליים נקבל שמתקיים,
	\[
		\frac{E \gamma_1'(0) \phi_1'(0) + G \gamma_2'(0) \phi_2'(0)}{E({(\gamma_1'(0))}^2 + {(\phi_1'(0))}^2) + G({(\gamma_2'(0))}^2 + {(\phi_2'(0))}^2)}
		= \frac{\langle \gamma'(0), \phi'(0) \rangle}{{\lVert \gamma'(0) \rVert}^2 \cdot {\lVert \phi'(0) \rVert}^2}
	\]
	אם $E \ne G$ נוכל לבנות עקומים שנגזרתם $(E - G, 1), (1, 1)$ ונקבל סתירה לשוויון האחרון, ולכן בהכרח $E = G$.

	נניח בכיוון ההפוך ש־$E = G, F = 0$ ונקבל תוך שימוש במהלכים זהים למהלך הקודם,
	\[
		\frac{\langle c'(0), d'(0) \rangle}{\lVert c'(0) \rVert \cdot \lVert d'(0) \rVert}
		= \frac{E \gamma_1'(0) \phi_1'(0) + G \gamma_2'(0) \phi_2'(0)}{E({(\gamma_1'(0))}^2 + {(\phi_1'(0))}^2) + G({(\gamma_2'(0))}^2 + {(\phi_2'(0))}^2)}
		= \frac{\gamma_1'(0) \phi_1'(0) + \gamma_2'(0) \phi_2'(0)}{{(\gamma_1'(0))}^2 + {(\phi_1'(0))}^2 + {(\gamma_2'(0))}^2 + {(\phi_2'(0))}^2}
	\]
	אבל האחרון אינו אלא,
	\[
		\frac{\langle \gamma'(0), \phi'(0) \rangle}{{\lVert \gamma'(0) \rVert}^2 \cdot {\lVert \phi'(0) \rVert}^2}
	\]
	ולכן $f$ קונפורמית.
\end{proof}

\subquestion{}
נאמר ש־$f : U \to S$ היא שומרת שטח אם לכל $R \subseteq U$ מתקיים $\vol_2(R) = \vol_2(f(R))$. \\
נראה ש־$f$ שומרת שטח אם ורק אם $EG - F^2 = 1$.
\begin{proof}
	נניח ש־$f$ היא שומרת שטח.
	נזכור שמתקיים,
	\[
		\vol_2(f(R))
		= \iint_{R} {(E G - F^2)}^{\frac{1}{2}}\ dl,
		\qquad
		\vol_2(R)
		= \vol(R)
		= \iint_R 1\ dl
	\]
	כלומר,
	\[
		\vol_2(f(R)) = \vol(R)
		\iff \iint_R {(EG - F^2)}^{\frac{1}{2}} - 1\ dl = 0
	\]
	ובהתאם ${(EG - F^2)}^\frac{1}{2} = 1$ $\lambda$־כמעט תמיד, עבור $\lambda$ מידת לבג על אופרטור הנפח $\vol_2$ מעל $\EE^3$.
	נזכור ש־$f$ רציפה ולכן הטענה נכונה תמיד.

	נניח בכיוון ההפוך ש־$EG - G^2 = 1$, אז נקבל שמתקיים,
	\[
		\vol_2(R)
		= \iint_R {(EG - F^2)}^{\frac{1}{2}}\ dl
		= \iint_R 1\ dl
		= \vol(R)
	\]
	כלומר $f$ משמרת שטח.
\end{proof}

\subquestion{}
נראה ש־$f$ קונפורמית ושומרת שטח אם ורק אם היא איזומטריה מקומית, כלומר שמתקיים,
\[
	\forall u \in U \forall v, w \in \RR^2,\ 
	\langle v, w \rangle = I_{f(u)}(D f |_u v, D f |_u w)
\]
\begin{proof}
	נניח ש־$f$ קונפורמית ושומרת שטח, אז מתקיים $E = G, F = 0$ וכן $E^2 = 1$, אבל $I$ תבנית חיובית לחלוטין ולכן $E = 1$ בלבד, כלומר $I_p = (\begin{smallmatrix}
		1 & 0 \\
		0 & 1
	\end{smallmatrix})$.
	אז $E_p = {(D_1 f |_p)}^2 = 1$ ולכן נסיק $D_1 f |_p \in \{ \pm \}$, הטענה נכונה גם על $D_2 f$. \\
	בהתאם מתקיים,
	\[
		I_{f(u)}(D f |_u v, D f |_u w)
		= v^t I_2 w
		= \langle v, w \rangle
	\]
	כפי שרצינו.

	בכיוון ההפוך נניח את השוויון ונקבל,
	\[
		I_{f(u)}(D f |_u v, D f |_u w)
		= E v^1 w^1 + F(v^2 w^1 + v^1 w^2) + G v^2 w^2
		= v^1 w^1 + v^2 w^2
		= \langle v, w \rangle
	\]
	ונקבל ש־$E = G = 1, F = 0$ ההצבה היחידה שנכונה תמיד, משקילות שמצאנו בסעיפים א' וב' נקבל ש־$f$ היא קונפורמית ומשמרת שטח.
\end{proof}

\question{}
\subquestion{}
יהיו $0 < r < R$ ויהי המשטח הפרמטרי $\varphi : {[0, 2 \pi]}^2 \to \EE^3$ המוגדר על־ידי,
\[
	\varphi(\theta, \phi)
	= ((R + r \cos \phi) \cos \theta, (R + r \cos \phi) \sin \theta, r \sin \phi)
\]
\begin{solution}
	נחשב את התבנית היסודית הראשונה של $\varphi$,
	\[
		D \varphi |_{(\theta, \phi)}
		= \begin{pmatrix}
			- (R + r \cos \phi) \sin \theta & - r \sin \phi \cos \theta \\
			(R + r \cos \phi) \cos \theta & -r \sin \phi \sin \theta \\
			0 & r \cos \phi
		\end{pmatrix}
	\]
	ולכן,
	\begin{align*}
		I
		& = \begin{pmatrix}
			\langle D_1 \varphi_{(\theta, \phi)}, D_1 \varphi_{(\theta, \phi)} \rangle & \langle D_2 \varphi_{(\theta, \phi)}, D_1 \varphi_{(\theta, \phi)} \rangle \\
			\langle D_1 \varphi_{(\theta, \phi)}, D_2 \varphi_{(\theta, \phi)} \rangle & \langle D_2 \varphi_{(\theta, \phi)}, D_2 \varphi_{(\theta, \phi)} \rangle
		\end{pmatrix} \\
		& = \begin{pmatrix}
			{(R + r \cos \phi)}^2 (\sin^2 \theta + \cos^2 \theta) & (R + r \cos \phi) r \sin \phi (\cos \theta \sin \theta - \sin \theta \cos \theta) \\
			(R + r \cos \phi) r \sin \phi (\cos \theta \sin \theta - \sin \theta \cos \theta) & r^2 \sin^2 \phi (\cos^2 \theta + \sin^2 \theta) + r^2 \cos^2 \phi
		\end{pmatrix} \\
		& = \begin{pmatrix}
			{(R + r \cos \phi)}^2 & 0 \\
			0 & r^2
		\end{pmatrix} \\
	\end{align*}
	עתה נעבור לחישוב השטח על־ידי נוסחת השטח התלויה בתבנית הראשונה,
	\begin{align*}
		\vol_2(\im \varphi)
		& = \int_{0}^{2 \pi} \int_{0}^{2 \pi} \sqrt{EG - F^2}\ d \theta\ d \phi \\
		& = \int_{0}^{2 \pi} \int_{0}^{2 \pi} \sqrt{{(R + r \cos \phi)}^2 r^2}\ d \theta\ d \phi \\
		& = \int_{0}^{2 \pi} \int_{0}^{2 \pi} (R + r \cos \phi) r\ d \theta\ d \phi \\
		& = r \int_{0}^{2 \pi} 1\ d \theta \cdot \int_{0}^{2 \pi} R + r \cos \phi\ d \phi \\
		& = 2 \pi r \int_{0}^{2 \pi} R + r \cos \phi\ d \phi \\
		& = 2 \pi r (R \phi + r \sin \phi) \mid_{\phi = 0}^{\phi = 2 \pi} \\
		& = 4 \pi^2 r R
	\end{align*}
	מצאנו ששטח הטורוס הפרמטרי הוא $4 \pi^2 r R$.
\end{solution}

\subquestion{}
יהיו $0 \le \phi_0 \le \phi_1 \le \pi$,
נחשב את השטח של $\varphi : [0, 2 \pi] \times [\phi_0, \phi_1] \to \EE^3$ המוגדר על־ידי,
\[
	f(\theta, \phi)
	= (\sin \phi \cos \theta, \sin \phi \sin \theta, \cos \phi)
\]
\begin{solution}
	זהו חלק של ספירת היחידה $S^2$ הכלוא בין המישורים $z_0 = \cos \phi_0, z_1 = \cos \phi_1$.
	נחשב את התבנית היסודית הראשונה,
	\[
		D f
		= \begin{pmatrix}
			-\sin \phi \sin \theta & \cos \phi \cos \theta \\
			\sin \phi \cos \theta & \cos \phi \sin \theta \\
			0 & - \sin \phi
		\end{pmatrix}
	\]
	ולכן,
	\begin{align*}
		& E
		= {(D_1 f)}^2
		= {(- \sin \phi \sin \theta)}^2 + {(\sin \phi \cos \theta)}^2 + 0^2
		= \sin^2 \phi \\
		& F
		= (D_1 f)(D_2 f)
		= - \sin \phi \sin \theta \cos \phi \cos \theta + \sin \phi \cos \theta \cos \phi \sin \theta + 0
		= 0 \\
		& G
		= {(\cos \phi \cos \theta)}^2 + {(\cos \phi \sin \theta)}^2 + \sin^2 \phi
		= \cos^2 \phi + \sin^2 \phi
		= 1
	\end{align*}
	כלומר,
	\[
		I
		= \begin{pmatrix}
			\sin^2 \phi & 0 \\
			0 & 1
		\end{pmatrix}
	\]
	נעבור לחישוב השטח,
	\begin{align*}
		\vol_2(\im \varphi)
		& = \int_{\phi_0}^{\phi_1} \int_0^{2 \pi} \sqrt{EG - F^2}\ d \theta\ d \phi \\
		& = \int_{\phi_0}^{\phi_1} \int_0^{2 \pi} \sqrt{\sin^2 \phi}\ d \theta\ d \phi \\
		& = \int_{0}^{2 \pi} 1\ d \theta \cdot \int_{\phi_0}^{\phi_1} \sin \phi\ d \phi \\
		& = 2 \pi \cdot (- \cos \phi) \mid_{\phi = \phi_0}^{\phi = \phi_1} \\
		& = 2 \pi (-\cos \phi_1 + \cos \phi_0)
	\end{align*}
	כלומר מצאנו ששטח החתך הוא $2 \pi (\cos \phi_0 - \cos \phi_1)$.

	נבחין שקיבלנו שהשטח הוא $2 \pi (z_1 - z_0)$, כלומר השטח לא תלוי בערכם אלא רק בהפרש שלהם, כלומר בגובה שלהם.
\end{solution}

\question{}
יהיו $0 < r < R$ ויהי המשטח הפרמטרי $\varphi : {(0, 2 \pi)}^2 \to \EE^3$ המוגדר על־ידי,
\[
	\varphi(\theta, \phi)
	= ((R + r \cos \phi) \cos \theta, (R + r \cos \phi) \sin \theta, r \sin \phi)
\]
נסמן את הטורוס הנוצר על־ידי הפרמטריזציה ב־$T = \varphi({(0, 2 \pi)}^2)$.

\subquestion{}
נחשב את מקדמי התבנית היסודית השנייה.
\begin{solution}
	מצאנו את ערך הנגזרת והתבנית היסודית הראשונה של הטורוס בשאלה 5,
	\[
		D \varphi |_{(\theta, \phi)}
		= \begin{pmatrix}
			- (R + r \cos \phi) \sin \theta & - r \sin \phi \cos \theta \\
			(R + r \cos \phi) \cos \theta & -r \sin \phi \sin \theta \\
			0 & r \cos \phi
		\end{pmatrix}
	\]
	וכן,
	\[
		I
		= \begin{pmatrix}
			{(R + r \cos \phi)}^2 & 0 \\
			0 & r^2
		\end{pmatrix}
	\]
	וכן
	ונחשב גם את הנורמל,
	\[
		n(u)
		= \frac{D_1 f(u) \wedge D_2 f(u)}{| D_1 f(u) \wedge D_2 f(u) |}
	\]
	ולכן נחשב את $D_1 f(u) \wedge D_2 f(u)$,
	\[
		D_1 f(u) \wedge D_2 f(u)
		= \begin{pmatrix}
			r (R + r \cos \phi) \cos \phi \cos \theta \\
			r (R + r \cos \phi) \cos \phi \sin \theta \\
			r (R + r \cos \phi) \sin \phi
		\end{pmatrix}
	\]
	נבחין שדילגנו על שלב הפישוט לצורך הקריאות.
	בהתאם גם,
	\[
		|D_1 f(u) \wedge D_2 f(u)|
		= r(R + r \cos \phi) \sqrt{\cos^2 \phi \cos^2 \theta + \cos^2 \phi \sin^2 \theta + \sin^2 \phi}
		= r(R + r \cos \phi)
	\]
	ולכן נקבל,
	\[
		n(\theta, \phi)
		= \begin{pmatrix}
			\cos \phi \cos \theta \\
			\cos \phi \sin \theta \\
			\sin \phi
		\end{pmatrix}
	\]
	וכן,
	\[
		D_1 n
		= \begin{pmatrix}
			- \cos \phi \sin \theta \\
			\cos \phi \cos \theta \\
			0
		\end{pmatrix},
		\qquad
		D_2 n
		= \begin{pmatrix}
			- \sin \phi \cos \theta \\
			- \sin \phi \sin \theta \\
			\cos \phi
		\end{pmatrix}
	\]
	נעבור לחישוב מקדמי התבנית היסודית השנייה,
	\begin{align*}
		& L
		= - D_1 f \cdot D_1 n
		= - (R + r \cos \phi) \sin \theta \cos \phi \sin \theta - (R + r \cos \phi) \cos \theta \cos \phi \cos \theta
		= - (R + r \cos \phi) \cos \phi \\
		& M
		= - D_1 f \cdot D_2 n
		= (R + r \cos \phi) \sin \theta \sin \phi \cos \theta - (R + r \cos \phi) \cos \theta \sin \phi \sin \theta + 0
		= 0 \\
		& N
		= - D_2 f \cdot D_2 n
		= - r \sin^2 \phi \cos^2 \theta - r \sin^2 \phi \sin^2 \theta - r \cos^2 \phi
		= - r \sin^2 \phi - r \cos^2 \phi
		= - r
	\end{align*}
	כלומר מצאנו שמתקיים,
	\[
		I\!I
		= \begin{pmatrix}
			- (R + r \cos \phi) \cos \phi & 0 \\
			0 & -r
		\end{pmatrix}
	\]
\end{solution}

\subquestion{}
נחשב את עקמומיות גאוס ואת העקמומיות הממוצעת.
\begin{solution}
	נשתמש בנוסחות שמצאנו בשאלה 7 כדי לחשב את הערכים הללו.
	\[
		K
		= \frac{\det I\!I}{\det I}
		= \frac{r (R + r \cos \phi) \cos \phi}{r^2 {(R + r \cos \phi)}^2}
		= \frac{\cos \phi}{r (R + r \cos \phi)}
	\]
	ועל־ידי שימוש בנוסחה שקיבלנו עבור עקמומיות ממוצעת,
	\[
		H
		= \frac{1}{2} \tr(I\!I \cdot I^{-1})
		= \frac{1}{2} \tr \begin{pmatrix}
			- (R + r \cos \phi) \cos \phi & 0 \\
			0 & -r
		\end{pmatrix}
		\begin{pmatrix}
			{(R + r \cos \phi)}^{-2} & 0 \\
			0 & r^{-2}
		\end{pmatrix}
		= -\frac{1}{2} \left(\frac{\cos \phi}{R + r \cos \phi} + \frac{1}{r}\right)
	\]
\end{solution}

\subquestion{}
נמצא נקודות על הטורוס שבהן עקמומיות גאוס מקיימת $K > 0, K = 0, K < 0$.
\begin{solution}
	נשתמש בערך המפורש שמצאנו בסעיף הקודם,
	\[
		K = 0
		\iff \frac{\cos \phi}{r (R + r \cos \phi)} = 0
		\iff \cos \phi = 0
	\]
	ולכן $\phi = \frac{\pi}{2}$ מקיים את הטענה, נבחר $(\pi, \frac{\pi}{2})$ כנקודה בה העקמומיות היא אפס, נבחין כי יכולנו להגיע לתוצאה זו גם באופן גאומטרי לגמרי, זו נקודה שנמצאת בדיוק במגע שבין טורוס לרצפה, שם העקמומיות מאוזנת.

	נסיק מהחישוב הקודם שגם $K < 0 \iff \cos \phi < 0$ ונבחר את הנקודה $(\pi, \pi)$, זוהי נקודה בחלק הפנימי של הטורוס.
	לבסוף גם נבחר $(\pi, \frac{\pi}{3})$ כנקודה בה מתקיים $K > 0$, זוהי נקודה בחלק החיצוני של הטורוס.
\end{solution}

\question{}
תהי $n : U \to S^2$ העתקת גאוס של $f : U \to \EE^3$, המוגדרת על־ידי,
\[
	n = \frac{D_1 f \wedge D_2 f}{| D_1 f \wedge D_2 f |}
\]
ונגדיר $W : T_p S \to T_p S$ על־ידי,
\[
	W(D_i f)
	= -D_i n
\]
עבור $i \in \{1, 2\}$.

\subquestion{}
נראה ש־$W$ הוא אופרטור צמוד לעצמו של $T_p S$, כלומר,
\[
	I_p(W(u), v)
	= I_p(u, W(v))
\]
לכל $u, v \in T_p S$.
\begin{proof}
	נסמן $u = D |_p f (u_x, u_y), v = D |_p f (v_x, v_y)$ ולכן,
	\[
		W(u)
		= W(D |_p f (u_x, u_y))
		= - D |_p n (u_x, u_y)
	\]
	ולכן גם,
	\[
		W(v)
		= - D |_p n (v_x, v_y)
	\]
	ולכן,
	\begin{align*}
		I_p(W(u), v)
		& = \langle W(u), v \rangle \\
		& = \langle - D |_p n (u_x, u_y), D |_p f (v_x, v_y) \rangle \\
		& = \langle -D_1 n(u_x), D_1 f(v_x) \rangle + \langle -D_2 n(u_y), D_1 f(v_x) \rangle + \langle -D_1 n(u_x), D_2 f(v_y) \rangle + \langle -D_2 n(u_y), D_2 f(v_y) \rangle \\
		& = u_x v_x \langle -D_1 n, D_1 f \rangle + u_y v_x \langle -D_2 n, D_1 f \rangle + u_x v_y \langle -D_1 n, D_2 f \rangle + u_y v_y \langle -D_2 n, D_2 f \rangle \\
		& = u_x v_x \langle D_1 f, -D_1 n \rangle + u_y v_x \langle D_2 f, -D_1 n \rangle + u_x v_y \langle D_1 f, -D_2 n \rangle + u_y v_y \langle D_2 f, -D_2 n \rangle \\
		& = \langle - D |_p f (u_x, u_y), D |_p n (v_x, v_y) \rangle \\
		& = \langle u, W(v) \rangle \\
		& = I_p(u, W(v))
	\end{align*}
	וקיבלנו שהטענה אכן נכונה.
\end{proof}

\subquestion{}
נראה כי לכל $u, v \in T_p S$ מתקיים,
\[
	I\!I_p(u, v)
	= I_p(W(u), v)
	= I_p(u, W(v))
\]
\begin{proof}
	מספיק להראות את הטענה לכל איבר בנפרד, נסמן,
	\[
		I\!I_p
		= \begin{pmatrix}
			L & M \\
			M & N
		\end{pmatrix},
		\qquad
		I_p
		= \begin{pmatrix}
			E & F \\
			F & G
		\end{pmatrix}
	\]
	ולכן נרצה להראות שמתקיים,
	\[
		L(u, v) = E(W(u), v),
		\quad
		M(u, v) = F(W(u), v),
		\quad
		N(u, v) = G(W(u), v)
	\]
	נשים לב כי $\langle n, D_i f \rangle = 0$ מהגדרת $n$ כאנך של המישור המשיק, אם נגזור את הביטוי נקבל,
	\[
		D_j \langle n, D_i f \rangle = D_j 0 = 0
	\]
	אז תוך שימוש בנגזרת מכפלה נקבל,
	\[
		D_j \langle n, D_i f \rangle
		= \langle D_j n, D_i f \rangle + \langle n, D_j D_i f \rangle = 0
		\implies \langle D_j n, D_i f \rangle = - \langle n, D_j D_i f \rangle
	\]
	עבור $i, j \in \{1, 2\}$.
	עבור בחירת $i = 1, j = 1$ נקבל,
	\[
		E(W( \cdot ), \cdot )
		= \langle D_1 n, D_1 f \rangle
		= \langle n, D_1^2 f \rangle
		= L
	\]
	כלומר מצאנו שהטענה מתקיימת עבור $E, L$, ותוך שימוש בשוויון זה נוכל להסיק את הטענה לכל שאר המקדמים גם כן.
\end{proof}

\subquestion{}
נראה כי המטריצה אשר מייצגת את $W$ בבסיס $(D_1 f, D_2 f)$ היא,
\[
	{[W]}_{(D_1 f, D_2 f)}
	= \begin{pmatrix}
		E & F \\
		F & G
	\end{pmatrix}^{-1}
	\begin{pmatrix}
		L & M \\
		M & N
	\end{pmatrix}
\]
\begin{proof}
	בסעיף הקודם מצאנו שמתקיים,
	\[
		\begin{pmatrix}
			L & M \\
			M & N
		\end{pmatrix}
		= \begin{pmatrix}
			E & F \\
			F & G
		\end{pmatrix}
		\begin{pmatrix}
			W(D_1 f) \\
			W(D_2 f)
		\end{pmatrix}
		\implies
		\begin{pmatrix}
			E & F \\
			F & G
		\end{pmatrix}^{-1}
		\begin{pmatrix}
			L & M \\
			M & N
		\end{pmatrix}
		= \begin{pmatrix}
			W(D_1 f) \\
			W(D_2 f)
		\end{pmatrix}
	\]
	כאשר השתמשנו בחיוביות בהחלט של $I$ במעבר האחרון.
\end{proof}

\subquestion{}
יהי $(e_1, e_2)$ בסיס אורתונורמלי של $W$ ונניח שהוא מקבל את הערכים העצמיים $(\kappa_1, \kappa_2)$ בהתאמה.
נראה שלכל $u = e_1 \cos \theta + e_2 \sin \theta \in T_p S$ מתקיים,
\[
	\kappa_{\operatorname{normal}}(u)
	= \kappa_1 \cos^2 \theta + \kappa_2 \sin^2 \theta
\]
\begin{proof}
	נראה את הטענה בשתי דרכים. הדרך הראשונה היא גאומטרית, אנו יודעים שהתבנית היסודית השנייה מייצגת עקמומיות בנקודה, ואם נשתמש במעבר מדיפרנציאל להצגה כיוונית נקבל בדיוק את הגדרת הנורמל,
	כלומר,
	\[
		\kappa_n(u)
		= I\!I(u)
		= I(W(u), u)
		\tag{1}
	\]
	מתקיים על־פי ההנחה,
	\[
		W(e_1)
		= \kappa_1 e_1,
		\qquad
		W(e_2)
		= \kappa_2 e_2
	\]
	ולכן מלינאריות $W$ גם,
	\[
		W(u)
		= \kappa_1 \cos \theta + \kappa_2 \sin \theta
	\]
	ולכן תוך שימוש בשוויון $(1)$ מתקבל,
	\[
		\kappa_n(u)
		= \langle \kappa_1 e_1 \cos \theta + \kappa_2 e_2 \sin \theta, e_1 \cos \theta + e_2 \sin \theta \rangle
		= \kappa_1 e_1 e_1 \cos^2 \theta + (\kappa_1 + \kappa_2) e_1 e_2 \sin \theta \cos \theta + \kappa_2 e_2 e_2 \sin^2 \theta
	\]
	אבל $(e_1, e_2)$ בסיס אורתונורמלי ולכן נקבל,
	\[
		\kappa_n(u)
		= \kappa_1 \cos^2 \theta + \kappa_2 \sin^2 \theta
	\]
	וקיבלנו את נכונות הטענה.

	כמובטח, נעבור לדרך השנייה, היא למעשה הבסיס לדרך הראשונה.
	הגדרנו את הנורמל לעקומה כנורמל $\kappa_n(u)$ העקום הנוצר מחיתוך המשטח עם המישור האפיני הנוצר על־ידי $n$ ו־$u$.
	אילו נגדיר $c : I \to S$ עקום עם פרמטריזציה לפי אורך בהתאם נקבל שאם $c(0) = p$ אז $c'(0) = u$ ולכן גם $c' \cdot n = 0$, כלומר $n$ מתלכד עם $c''$.
	אז $c'' = \kappa_n n$ ותוך שימוש בפיתוח טיילור שראינו בכיתה נוכל להסיק שאכן $\kappa_n = I\!I(u)$ בדיוק.
\end{proof}

\subquestion{}
נניח ש־$\kappa_1 \ge \kappa_2$ ונסיק שאלו הם ערכי העקמומיות המקסימלית והמינימלית בהתאמה.
\begin{proof}
	נבחין כי מהסעיף הקודם קיימת התאמה בין זוויות ווקטורים הנורמליים ב־$T_p S$, לכן מספיק לבחון אותם.
	נגדיר את,
	\[
		h(\theta) = \kappa_n(u(\theta)) = \kappa_n(e_1 \cos \theta + e_2 \sin \theta) = \kappa_1 \cos^2 \theta + \kappa_2 \sin^2 \theta
	\]
	נבחין כי בהתאם להגדרתה $h$ מקבלת כל ערך עקמומיות נורמלית ולכן מספיק לחקור אותה כפונקציה $h : [0, 2 \pi] \to \RR$.
	בנוסף זוהי פונקציה $\pi$־מחזורית ולכן נבדוק רק את התחום $[0, \pi]$.
	נגזור ונקבל,
	\[
		h'(\theta)
		= -2 \kappa_1 \cos \theta \sin \theta + 2 \kappa_2 \sin \theta \cos \theta
		= \sin(2 \theta) (\kappa_2 - \kappa_1)
	\]
	ולכן $h' = 0 \iff \theta = 0, \frac{\pi}{2}$ ומבדיקה ישירה נקבל ש־$\theta = 0$ מקסימום ובה מתקבל $h(0) = \kappa_1$ וכן $\theta = \frac{\pi}{2}$ מינימום ושם $h(\frac{\pi}{2}) = \kappa_2$. \\
	נסיק ש־$\kappa_1, \kappa_2$ הן העקמומיויות הראשיות וכן $e_1, e_2$ הכיוונים הראשיים.
\end{proof}

\subquestion{}
נגדיר את עקמומיות גאוס $K$ להיות המכפלה $\kappa_1 \cdot \kappa_2$ עבור העקמומיויות הראשיות.
נגדיר את העקמומיות הממוצעת $H = \frac{\kappa_1 + \kappa_2}{2}$. \\
נסיק שמתקיים,
\[
	K = \det\left(
	\begin{pmatrix}
		E & F \\
		F & G
	\end{pmatrix}^{-1}
	\begin{pmatrix}
		L & M \\
		M & N
	\end{pmatrix}
	\right)
	= \frac{LN - M^2}{EG - F^2}
\]
וכן,
\[
	H
	= \frac{1}{2} \tr\left(
	\begin{pmatrix}
		E & F \\
		F & G
	\end{pmatrix}^{-1}
	\begin{pmatrix}
		L & M \\
		M & N
	\end{pmatrix}
	\right)
	= \frac{1}{2} \cdot \frac{LG - 2MF + NE}{EG - F^2}
\]
\begin{proof}
	ניזכר בטענה מלינארית 1, אם $M \in M_2(\RR)$ לכסינה עם $\lambda_1, \lambda_2$ ערכים עצמיים, אז מתקיים $\det M = \lambda_1 \cdot \lambda_2$,
	בפרט אם $T \in \aut(\RR^2)$ אופרטור לינארי ו־$M$ מטריצת הייצוג שלו באיזשהו בסיס, אז $\det M$ עדיין מקיימת את הטענה.
	מצאנו שבבסיס $(D_1 f, D_2 f)$ המטריצה המייצגת של $W$ היא,
	\[
		J
		= \begin{pmatrix}
			E & F \\
			F & G
		\end{pmatrix}^{-1}
		\begin{pmatrix}
			L & M \\
			M & N
		\end{pmatrix}
	\]
	וכן מצאנו בסעיף הקודם ש־$K = \kappa_1 \kappa_2$ היא מכפלת הערכים העצמיים, ולכן,
	\[
		K
		= \det J
	\]
	נעבור לחישוב נוסחה ישירה,
	\[
		K
		= \det J
		= \det \begin{pmatrix}
			E & F \\
			F & G
		\end{pmatrix}^{-1}
		\det \begin{pmatrix}
			L & M \\
			M & N
		\end{pmatrix}
		= \frac{LN - M^2}{EG - F^2}
	\]
	כאשר השתמשנו בכפליות הדטרמיננטה ודטרמיננטת מטריצה הופכית.
	
	נעבור לעקמומיות הממוצעת.
	עוד טענה מלינארית היא ש־$\tr(M) = \lambda_1 + \lambda_2$, כלומר העקבה היא תמיד סכום הערכים העצמיים, לכן,
	\[
		H
		= \frac{1}{2} (\kappa_1 + \kappa_2)
		= \frac{1}{2} \tr J
	\]
	וקיבלנו את המבוקש.
	נשתמש בנוסחה ידועה למטריצה הופכית מסדר $2$ ונקבל,
	\begin{align*}
		\frac{1}{2} \tr J
		& = \frac{1}{2} \tr \left( \frac{1}{EG - F^2} \begin{pmatrix}
				G & -F \\
				-F & E
			\end{pmatrix}
			\begin{pmatrix}
				L & M \\
				M & N
			\end{pmatrix}
		\right) \\
		& = \frac{1}{2} \frac{1}{EG - F^2} \tr
		\begin{pmatrix}
			GL - FM & GM - FN \\
			-FL + LE & -FM + EN
		\end{pmatrix} \\
		& = \frac{1}{2} \frac{LG - 2MF + NE}{EG - F^2}
	\end{align*}
	וקיבלנו את המבוקש.
\end{proof}

\subquestion{}
נראה שאם התבנית היסודית השנייה מתאפסת בכל נקודה אז המשטח הוא חלק ממישור.
\begin{proof}
	מסעיף ב' נובע שלכל $p \in S$ ולכל $u, v \in T_p S$ מתקיים,
	\[
		0
		= I\!I_p(u, v)
		= \langle W(u), v \rangle
		= \langle -n u, v \rangle
	\]
	נסיק אם כך ש־$D_1 n, D_2 n \perp D |_p f$, או במילים אחרות $D_1 n, D_2 n \perp T_p S$, אבל הגדרנו את $n$ כך ש־$n \perp T_p S$ ו־$\lVert n \rVert = 1$ ולכן $D_1 n, D_2 n \perp n$ ונקבל ש־$D |_p n = 0$ בלבד.
	טענה זו נכונה לכל $p \in S$ ולכן $n$ היא קבועה, ונסמן $n \equiv n_0$.
	בהתאם גם $D_1 f, D_2 f \perp n_0$ בכל נקודה, ותוך שימוש בטיעון שקול לשאלה 3 סעיף ג' נקבל ש־$S$ ניתנת לשיכון במישור.
\end{proof}

\end{document}
