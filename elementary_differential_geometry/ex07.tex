\input{../article_base.tex}
\title{פתרון מטלה 7 --- גאומטריה דיפרנציאלית אלמנטרית, 80560}

\DeclareMathOperator\eval{eval}

\begin{document}
\maketitle
\maketitleprint[purple]

\question{}
תהי $S^2 \subseteq \EE^3$ ספירת היחידה.

\subquestion{}
נמצא פרמטריזציות רגולריות המכסות את היריעה.
\begin{solution}
	נגדיר את ההעתקה,
	\[
		\varphi(x, y) = \left( \frac{2 x}{1 + x^2 + y^2}, \frac{2 y}{1 + x^2 + y^2}, \frac{-1 + x^2 + y^2}{1 + x^2 + y^2} \right)
	\]
	העתקה זו היא המיפוי של נקודה $(x, y) \in \EE^2$ לספירה על־ידי שימוש ישיר בהגדרת הספירה של רימן, אותה נתאר מיד. \\
	נסמן $N = (0, 0, 1)$ ותהי נקודה כלשהי $(x, y) \in \RR^2$.
	נגדיר את $\varphi(x, y)$ להיות נקודת החיתוך של הישר $L = (1, 0, 0) t + (x, y, 0) s$ עבור $t + s = 1$ עם $S^2$.
	נשים לב כי בהכרח יש נקודת חיתוך מלבד $N$ ל־$L$ כנביעה מרציפות ובדיקת הנורמה.
	נקבל $\lVert L \rVert = 1 \iff \lVert (x s, y s, t) \rVert = 1 \iff x^2 s^2 + y^2 s^2 + t^2 = 1$, תוצאת שוויון זה היא בדיוק $\varphi(x, y)$.
	נסיק אם כך ישירות מתהליך זה ש־$\varphi$ היא חד־חד ערכית, ונסיק ש־$\varphi \restriction \im \varphi = \bar{\varphi}$ היא דיפאומורפיזם.
	משימוש במשפט הפונקציה ההפוכה נוכל גם להסיק ש־$\im \varphi = S^2 \setminus N$, על־ידי הגדרת ישר בין נקודה ו־$N$ ולקיחת החיתוך עם המישור.
	נסמן $\psi : S^2 \setminus \{ N \} \to \RR^2$ כך ש־$\psi = \varphi^{-1}$.

	נוכל לקבל באופן דומה גם להגדיר הטלה כזו עם $N' = (0, 0, -1)$, ונקבל בהתאמה העתקה $\phi : S^2 \setminus \{ N' \} \to \RR^2$ דומה שהיא דיפאומורפיזם.
	אם כל לכל נקודה $P \in S$ אם $P = N$ אז נבחר את $U = S \setminus \{ N' \}$ ואת $\phi$, ואחרת את $U = S \setminus \{ N \}$ ואת $\psi$, ונקבל ש־$S$ יריעה.
\end{solution}

\subquestion{}
נמצא נוסחה מפורשת ל־$\psi$.
\begin{solution}
	נניח ש־$P = (x, y, z)$ ונמצא את $\psi(x, y, z) = (u, v)$.
	נבחין ש־$L = (0, 0, 1) t + (x, y, z) s$ ונרצה למצוא את $L^{-1}(\RR^2 \times \{ 0 \})$.
	כלומר נבדוק מתי $t + zs = 0$, כלומר נחשוד בערך $t = -zs$ ולכן במקרה זה $L = (x s, ys, 0)$, אבל גם $t + s = 1$ ולכן $s = 1 + zs \iff s = \frac{1}{1 - z}$.
	נציב ונקבל,
	\[
		L(t, s)
		= (\frac{x}{1 - z}, \frac{y}{1 - z}, 0)
	\]
	כלומר $u = \frac{x}{1 - z}$ וכן $v = \frac{y}{1 - z}$.
\end{solution}

\question{}
נגדיר את הקבוצה,
\[
	S = \{ (x, y, z) \in \EE^3 \mid x^2 + y^2 = 1 \}
\]
ונראה שהיא משטח רגולרי.
\begin{proof}
	תהי $P \in S$, כלומר $P = (x, y, z)$ כך ש־$x^2 + y^2 = 1$.
	אם $x \ne 0$, נגדיר את ההעתקה $\varphi : (0, 2 \pi) \times (-1, 1) \to S$ על־ידי,
	\[
		\varphi(u, v) = (\cos u, \sin u, z + v)
	\]
	נבחין כי,
	\[
		D \varphi |_{(u, v)}
		= \begin{pmatrix}
			- \sin u & 0 \\
			\cos u & 0 \\
			0 & 1
		\end{pmatrix} 
	\]
	ונקבל ש־$\deg D \varphi |_{(u, v)} = 2$, ולכן $\varphi$ הפיכה ממשפט הפונקציה ההפיכה.
	נבחין כי $\varphi$ על, זאת על־ידי בחירת $v = z$ וכן $u = \Arg(x, y)$, ולכן נקבל ש־$\varphi$ דיפאומורפיזם (חלק).
	נסיק ש־$\varphi$ פרמטריזציה מקומית עבור $P$.

	נעבור למקרה $x = 0$, כלומר $P = (0, 1, z)$.
	במקרה זה נגדיר $\varphi = (-\pi, \pi) \times (-1, 1) \to S$ באותו אופן בדיוק של קודם, ולכן נקבל שוב דיפאומורפיזם חלק, אך הפעם $P \in {(\dom \varphi)}^\circ$, ולכן $\varphi$ פרמטריזציה מקומית של $P$.

	נסיק ש־$S$ הוא משטח רגולרי.
\end{proof}

\question{}
תהי הנקודה $P_0 = (R, 0, 0)$ ונניח ש־$C = S(P_0, r) \cap (\RR \times \{ 0 \} \times \RR)$ עבור $0 < r < R$.
יהי $T \subseteq \EE^3$ הטורוס המתקבל כגוף סיבוב של $C$ סביב ציר $z$.

\subquestion{}
נראה ש־$T$ משטח פרמטרי רגולרי.
\begin{proof}
	נגדיר את ההעתקה,
	\[
		\varphi : {[0, 2 \pi)}^2 \to T, % chktex 9
		\qquad
		\varphi(u, v)
		= (\cos v \cdot (R + r \cos u), \sin v \cdot (R + r \sin u), r \sin u)
	\]
	ונראה שהיא חד־חד ערכית על וגזירה.

	חד־חד ערכיות נובעת ישירות מחד־חד ערכיות $(\cos x, \sin x)$ עם $v, u$. \\
	תהי $(x, y, z) \in T$.
	אז כגוף סיבוב נובע ש־$(x, y, z) = (x', z', 0) (\cos v, \sin v, 1)$, וכן $(x', z') \in C$ ולכן קיים $u$ כך ש־$(x', z') = r (\cos u, \sin u)$, ומהגדרה נקבל בדיוק ש־$\varphi(u, v) = (x, y, z)$.

	לבסוף כהרכבה של פונקציות גזירות $\varphi$ גזירה ולכן דיפאומורפיזם, נסיק ש־$T$ יריעה פרמטרית חלקה.
\end{proof}

\subquestion{}
נראה שמתקיים,
\[
	T
	= \{ (x, y, z) \in \EE^3 \mid {(x^2 + y^2 + z^2 + R^2 - r^2)}^2 = 4 R^2 (x^2 + y^2) \}
\]
\begin{proof}
	נשתמש בפרמטריזציה ונקבל שהטענה שקולה לטענה,
	\[
		T
		= \{ 0 \le u, v < \pi \mid K = 4 R^2 ({(\cos v \cdot (R + r \cos u))}^2 + {(\sin v \cdot (R + r \cos u))}^2) \}
	.\]
	עבור,
	\begin{align*}
		K
		& = {\left({(\cos v \cdot (R + r \cos u))}^2 + {(\sin v \cdot (R + r \cos u))}^2 + r^2 \sin^2 u + R^2 - r^2 \right)}^2 \\
		& = {\left(\cos^2 v \cdot {(R + r \cos u)}^2 + \sin^2 v \cdot {(R + r \cos u)}^2 + r^2 \sin^2 u + R^2 - r^2 \right)}^2 \\
		& = {\left({(R + r \cos u)}^2 + r^2 \sin^2 u + R^2 - r^2 \right)}^2 \\
		& = {\left( R^2 + 2 r R \cos u + r^2 \cos^2 u + r^2 \sin^2 u + R^2 - r^2 \right)}^2 \\
		& = {\left( 2 r R \cos u + 2 R^2 \right)}^2 \\
		& = 4 R^2 {\left( r \cos u + 2 R \right)}^2 \\
		& = 4 R^2 (\cos^2 v + \sin^2 v) {\left(r \cos u + 2 R \right)}^2 \\
		& = 4 R^2 (x^2 + y^2)
	\end{align*}
	כלומר מצאנו שהטענה אכן מתקיימת.
\end{proof}

\end{document} % chktex 17
