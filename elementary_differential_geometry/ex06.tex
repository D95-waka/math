\input{../article_base.tex}
\title{פתרון מטלה 6 --- גאומטריה דיפרנציאלית אלמנטרית, 80560}

\DeclareMathOperator\eval{eval}

\begin{document}
\maketitle
\maketitleprint[purple]

\question{}
תהי $\wedge : \RR^3 \times \RR^3 \to \RR^3$ המכפלה החיצונית, כלומר ההעתקה המתאימה עבור $(x, y)$ ל־$\varphi = z \mapsto \det(x, y, z)$ את הווקטור היחיד $v$ המקיים $l_v = \varphi$.

\subquestion{}
נראה ש־$\wedge$ בי־לינארית.
\begin{proof}
	נזכור כי מתקיים $\det(x + x', y, z) = \det(x, y, z) + \det(x', y, z)$ ולכן אם $(x \wedge y) = u, (x' \wedge y) = u'$ אז,
	\[
		u \cdot z = \det(x, y, z),
		\quad
		u' \cdot z = \det(x', y, z)
		\implies
		(u + u') \cdot z
		= \det(x + x', y, z)
	\]
	התהליך זהה עבור $y$.
\end{proof}

\subquestion{}
נראה ש־$x \wedge y = - y \wedge x$.
\begin{proof}
	ידוע שמתקיים $\det(x, y, z) = - \det(y, x, z)$ ולכן הטענה נובעת ישירות תוך שימוש במהלך של הסעיף הקודם.
\end{proof}

\subquestion{}
נראה ש־$x \cdot (x \wedge y) = 0 = (x \wedge y) \cdot y$.
\begin{proof}
	ידוע כי $(x \wedge y) \cdot y = \det(x, y, y) = 0$ כדטרמיננטה של מטריצה לא הפיכה.
	הצד השני נובע מסימטריה של מכפלה פנימית ממשית.
\end{proof}

\subquestion{}
נראה ש־$x \wedge y = 0$ אם ורק אם $\{ x, y \}$ תלויה לינארית.
\begin{proof}
	נניח ש־$x \wedge y = 0$, אז נקבל $\det(x, y, z) = 0$ לכל $z$, אם נבחר $z \notin \Sp\{ x, y \}$ נקבל שבהכרח $x, y$ פרופורציונליים.

	בכיוון ההפוך הטענה נובעת ישירות מדטרמיננטה של מטריצה לא הפיכה.
\end{proof}

\subquestion{}
נוכיח שאם $x \wedge y \ne 0$ אז $(x, y, x \wedge y)$ בסיס סדור חיובי.
\begin{proof}
	נבחין כי $\lVert x \wedge y \rVert \ne 0$ ולכן $\det(x, y, x \wedge y) \ne 0$ ולכן מטריצה זו הפיכה ובהתאם מרחב העמודות שלה הוא בלתי־תלוי לינארית.
\end{proof}

\subquestion{}
נראה ש־$(x \wedge y) \wedge z = (x \cdot z) y - (y \cdot z) x$.
\begin{proof}
	מתקיים $(x \wedge y) \cdot w = \det(x, y, w)$ וכן $((x \wedge y) \wedge z) \cdot w = \det(x \wedge y, z, w)$, נפתח את הגדרת הדטרמיננטה ונסיים.
\end{proof}

\subquestion{}
נראה שתקיים,
\[
	(x \wedge y) \cdot (z \wedge w)
	= \begin{vmatrix}
		x \cdot z & x \cdot w \\
		y \cdot z & y \cdot w
	\end{vmatrix}
\]
\begin{proof}
	ישירות מפתיחת הדטרמיננטה.
\end{proof}

\subquestion{}
נראה שמתקיים,
\[
	{|x \wedge y|}^2
	= {|x|}^2 {|y|}^2 - {(x \cdot y)}^2
\]
\begin{proof}
	נציב בנוסחה מהשאלה הקודמת ונקבל,
	\[
		(x \wedge y) \cdot (x \wedge y)
		= \begin{vmatrix}
			x \cdot x & x \cdot y \\
			y \cdot x & y \cdot y
		\end{vmatrix}
		= {|x|}^2 {|y|}^2 - {(x \cdot y)}^2
	\]
\end{proof}

\subquestion{}
נסמן $(i, j, k)$ הבסיס הסטנדרטי של $\RR^3$.
נראה שמתקיים $i \wedge j = k, j \wedge k = i, k \wedge i = j$.
\begin{proof}
	מתקיים $(i \wedge j) \cdot w = \det(i, j, w) = 1 \cdot \begin{vmatrix} 1 & w_2 \\ 0 & w_3 \end{vmatrix} = w_3 = k \cdot w$.
	שני המקרים הנוספים שקולים.
\end{proof}

\subquestion{}
נראה שמתקיים,
\[
	x \wedge y
	= \begin{pmatrix}
		x^2 y^3 - x^3 y^2 \\
		x^3 y^1 - x^1 y^3 \\
		x^1 y^2 - x^2 y^1
	\end{pmatrix}
\]
\begin{proof}
	באופן דומה לסעיף הקודם,
	\begin{align*}
		(x \wedge y) \cdot w
		& = \det(x\ y\ w) \\
		& = \begin{vmatrix}
			x^1 & y^1 & w^1 \\
			x^2 & y^2 & w^2 \\
			x^3 & y^3 & w^3
		\end{vmatrix} \\
		& = w^1 \begin{vmatrix}
			x^2 & y^2 \\
			x^3 & y^3
		\end{vmatrix}
		- w^2 \begin{vmatrix}
			x^1 & y^1 \\
			x^3 & y^3
		\end{vmatrix}
		+ w^3 \begin{vmatrix}
			x^1 & y^1 \\
			x^2 & y^2
		\end{vmatrix} \\
		& = w^1 (x^2 y^3 - x^3 y^2) - w^2 (x^1 y^3 - x^3 y^1) + w^3 (x^1 y^2 - x^2 y^1) \\
		& = \begin{pmatrix}
			x^2 y^3 - x^3 y^2 \\
			x^3 y^1 - x^1 y^3 \\
			x^1 y^2 - x^2 y^1
		\end{pmatrix} \cdot w
	\end{align*}
	ישירות משימוש בכללי חישוב דטרמיננטה.
\end{proof}

\question{}
תהיינה $A \in M_2(\RR), B \in M_{1 \times 2}(\RR)$ עבור $A$ סימטרית, ו־$c \in \RR$. \\
שניונית במישור $\AAA^2$ היא קבוצה מהצורה,
\[
	Q(\iota) = \{ x \in \AAA^2 \mid {\iota(x)}^t A \iota(x) + B \iota(x) + c = 0 \}
\]
נגיד גם ששתי שניוניות $Q, Q' \subseteq \AAA^2$ הן שקולות אפינית אם קיימת $f \in GA_2(\RR)$ העתקה אפינית הפיכה כך ש־$Q(\id) = Q'(f)$. \\
נראה ששניונית שקולה אפינית לאחת מהאפשרויות הבאות: אליפסה, היפרבולה, פרבולה, זוג ישרים אנכים, זוג ישרים מקבילים, נקודה או הקבוצה הריקה.
\begin{proof}
	נגדיר את התבנית $q(x) = x^t A x + B x$ כך ש־$Q = \{ q(x) + c = 0 \}$, אז $x^t A x + B x$ תבנית ריבועית ולכן קיים בסיס כך שהיא שווה ל־$x^t \diag(c_1, c_2) x$, כאשר $c_1, c_2 \in \{1, 0, -1\}$.
	אם במצב זה $P$ מטריצת המעבר המתאימה אז $Q(P)$ היא שקולה שניונית שקולה, כלומר נוכל להניח בלי הגבלת הכלליות שמתקיים,
	\[
		Q = \{ (x, y) \in \AAA^2 \mid c_1 x^2 + c_2 y^2 + c = 0 \}
	\]
	במצב זה $c \in \RR$, אבל נוכל להרכיב מטריצה סקלרית $\diag(\frac{1}{c}, \ldots, \frac{1}{c})$ ולקבל ש־$Q$ שקול לשניונית בה $c \in \{1, 0, -1\}$ בלבד, ולכן נניח גם כך בלי הגבלת הכלליות.
	נבחין כי בהתאם לערכי $c_1, c_2$, $Q$ היא קבוצת הפתרונות של אחת המשוואות הבאות בלבד,
	\[
		x^2 + y^2 + c = 0,
		\quad
		x^2 - y^2 + c = 0,
		\quad
		x^2 + y = 0,
		\quad
		x^2 + c = 0
	\]
	עבור $c \in \{1, 0, -1\}$.

	נעבור לבחון כל מקרה ובכך לסיים את ההוכחה.
	אם $Q = \{ x^2 + y^2 + c = 0 \}$ אז אם $c = 0$ אז נקבל $Q = \{ 0 \}$, אם $c = 1$ אז נקבל $Q = \emptyset$.
	אם $c = -1$ אז נקבל ש־$Q$ הוא מעגל, כלומר שכל שניונית מצורה זו היא אליפסה.

	נניח ש־$Q = \{ x^2 - y^2 + c = 0 \}$.
	אם $c \in \{-1, 1\}$ אז נקבל ש־$Q$ היא היפרבולה, ואחרת נקבל ש־$Q$ היא ישרים מצטלבים.

	נניח ש־$Q = \{ x^2 + y = 0 \}$, אז נקבל ש־$Q$ היא פרבולה.

	לבסוף נניח ש־$Q = \{ x^2 + c = 0 \}$ אז אם $c = -1$ נקבל ישרים מקבילים, אם $c = 0$ אז נקבל ש־$Q$ שקולה לישר בודד, ואם $c = 1$ אז נקבל ש־$Q = \emptyset$ שוב.
\end{proof}
נסמן $A = (\begin{smallmatrix} \alpha & \beta \\ \beta & \gamma \end{smallmatrix})$ ונמצא תנאי שקובע אילו צורות לא מנוונות (אליפסה היפרבולה או פרבולה) מתקבלות.
\begin{solution}
	נסמן $\varepsilon_1$ להיות הסימן של $\alpha$ ונקבל,
	\[
		(x\ y) A {(x, y)}^t
		= \alpha x^2 + 2 \beta x y + \gamma y^2
		= \varepsilon_1 {\left(\sqrt{\varepsilon_1 \alpha} x + \frac{\varepsilon_1 \beta}{\sqrt{\varepsilon_1 \alpha}} y\right)}^2 + \left(- \varepsilon_1 \frac{\beta^2}{\alpha} + \gamma\right) y^2
	\]
	מצאנו שאם שני המקדמים חיוביים, כלומר אם $\alpha > 0$ וגם $- \beta^2 + \gamma > 0$ אז מתקבלת אליפסה, כלומר נקבל אליפסה אם ורק אם $\alpha > 0, \gamma > \beta^2$. \\
	באופן דומה אם $\alpha > 0, \gamma < \beta^2$ נקבל היפרבולה ואם $\beta^2 = \gamma$ נקבל פרבולה.
\end{solution}

\question{}
נמיין שניוניות במישור אוקלידי, כלומר הפעם נבחן שניוניות עד־כדי שקילות אוקלידית, כלומר נגיד ש־$Q$ ו־$Q'$ שקולות אוקלידית אם $f \in GA_2(\RR)$ שומרת מרחק המקיימת $Q(f) = Q'$.
\begin{solution}
	המקרה דומה לשאלה הקודמת, אך הפעם לא נוכל לבצע פקטור גודל, כלומר ש־$Q = \{ a x^2 + b y^2 + c = 0 \}$ עבור $a, b, c \in \RR$.
	בשל היכולת לחלק או לכפול את המשוואה בלי לשנות את מרחב תוצאותיה, נוכל להניח ש־$c \in \{1, 0, -1\}$.
	בהתאם נקבל ש־$Q$ שקול לפתרון אחת המשוואות,
	\[
		ax^2 + by^2 + c = 0,
		\quad
		ax^2 - by^2 + c = 0,
		\quad
		ax^2 + by = 0,
		\quad
		ax^2 + c = 0
	\]
	באופן שקול למקרה הקודם.
\end{solution}

\question{}
נמיין אפינית ואוקלידית את השניוניות הבאות ונמצא להן משוואה סטנדרטית.

\subquestion{}
נגדיר $Q = \{ 4x^2 - 4xy + y^2 - 4x - 8y - 4 = 0 \}$.
\begin{solution}
	נבחין כי,
	\[
		4x^2 - 4xy + y^2 - 4x - 8y - 4
		= {(2x - y - 1)}^2 - 10y - 5
	\]
	אז מצאנו ש־$Q$ מתאימה למשוואת פרבולה $x^2 - y = 0$.

	נוכל להסיק אם כך שהיא שקולה אוקלידית למשוואה מהצורה $ax^2 + by^2 + c = 0$.
\end{solution}

\subquestion{}
נגדיר $Q = \{ 5x^2 + 8xy + 5y^2 - 3x + y - 2 = 0 \}$.
\begin{solution}
	מחפיפה של $(\begin{smallmatrix} 5 & 4 \\ 4 & 5 \end{smallmatrix})$ נקבל את ההעתקה $w \mapsto (\begin{smallmatrix} 5^{-1 / 2} & 0 \\ - \frac{4}{3 \sqrt{5}} & \frac{3}{\sqrt{5}} \end{smallmatrix}) w$. \\
	ממנה נסיק ש־$Q \cong \{ x^2 + y^2 - \frac{3}{\sqrt{5}} x + \frac{8}{\sqrt{5}} y - 2 = 0 \} = \{ {(x - \frac{3}{2 \sqrt{5}})}^2 + {(y + \frac{4}{\sqrt{5}})}^2 - 5 - \frac{13}{20} = 0 \}$.
	נסיק ש־$Q$ מתאים למשוואה הסטנדרטית $x^2 + y^2 - 1 = 0$ וזוהי משוואת אליפסה.

	באופן שקול נקבל שגם $a x^2 + b y^2 - 1 = 0$ משוואה סטנדרטית האוקלידית.
\end{solution}

\end{document}
