\input{../article_base.tex}
\title{גאומטריה דיפרנציאלית אלמנטרית --- סיכום}
\setcounter{secnumdepth}{2}

\usepackage{fancyhdr}
\pagestyle{fancy}
\renewcommand{\headrulewidth}{0pt}

\begin{document}
\maketitle
\maketitleprint[purple]

\tableofcontents

\section{שיעור 1 --- 20.10.2025}
\subsection{מבוא}
גאומטריה היא אבן יסוד של החברה שלנו, והיא לוקחת חלק בכל תהליך בנייה תכנון ומדידה.
לאורך ההיסטוריה היה חקר של גאומטריה באיזשהו אופן נאיבי, אך אנו נעסוק בחקר של הגאומטריה באופן האקסיומטי שלה.
אנו נעסוק בחקר של צורות חלקות, כלומר שאפשר ללטף אותן, תוך שימוש בכלים שראינו באנליזה.
הרעיון בקורס הוא לגשת בצורה אלמנטרית לבעיות לאו דווקא מורכבות בגישה שהיא גאומטרית.
הצורות שנחקור הן יריעות, ככל הנראה יריעות חלקות.

\subsection{הגישה הסינתטית}
המתמטיקה המודרנית מתבססת על תורת הקבוצות, לכן עלינו לספק הגדרה קבוצתית הולמת למושג המישור.
\begin{definition}[ישרים מקבילים]
	שני ישרים נקראים מקבילים אם הם מתלכדים או אינם נחתכים.
\end{definition}
\begin{definition}[קולינאריות]
	נאמר שקבוצה של נקודות הן קולינאריות כאשר כל הנקודות שייכות לישר אחד.
\end{definition}
\begin{definition}[מישור אפיני]
	זוג סדור $(\Pp, \Ll)$ כאשר $\Pp$ קבוצה שאת ערכיה נכנה נקודות ו־$\Ll$ קבוצה של קבוצות של נקודות, אותן נכנה ישרים.
	זוג סדור זה יקרא מישור אפיני אם הוא מקיים את התכונות הבאות,
	\begin{enumerate}
		\item לכל שתי נקודות יש ישר יחיד המכיל את שתיהן
		\item לכל ישר ונקודה קיים ישר יחיד מקביל לישר העובר דרך הנקודה
		\item קיימות שלוש נקודות שאינן קולינאריות
	\end{enumerate}
\end{definition}
נעבור למשפט יסודי שמדגים את אופי המישור האפיני.
\begin{theorem}[מספר נקודות מינימלי במישור אפיני]
	יהי מרחב אפיני $(\Pp, \Ll)$, אז ב־$\Pp$ לפחות $4$ נקודות.
\end{theorem}
\begin{proof}
	יהיו $P, Q, R \in \Pp$ נקודות שונות ולא קולינאריות.
	נסמן גם $l = \langle P, Q \rangle, m = \langle P, R \rangle$ שני הישרים העוברים דרך הנקודות המתאימות.
	נסמן את $l' \parallel l \ni R, m \parallel m' \ni P$, וגם את $\{ S \} = l' \cap m'$ עבור $S \in P$.
	אנו טוענים כי $S$ קיימת וכי היא נקודה רביעית.

	נטען טענת עזר, והיא ש־$l' \not\parallel m'$.
	אילו $l' \parallel m'$ אז מטרנזיטיביות יחס השקילות המושרה מיחס ההקבלה היה נובע כי $l \parallel l' \parallel m' \parallel m$, אבל אז מהתכונה השנייה של מישור אפיני היה מתקבל ש־$l = m$ בסתירה לבחירת $P, Q, R$.

	אם $S \in \{ P, Q \}$ אז היה נובע ש־$l = l'$ ולכן גם $R \in l$, בסתירה.
	אם באופן שקול $S \in \{ P, R \}$ אז נקבל סתירה דומה, ולכן נותר להניח ש־$S$ קיימת ושונה מ־$P, Q, R$.
\end{proof}
שני התרגילים הבאים יאפשרו לנו לתרגל את הגישה הסינתטית,
\begin{exercise}
	הוכיחו כי כל ישר מכיל לפחות שתי נקודות שונות.
\end{exercise}
\begin{exercise}
	הוכיחו כי יחס ההקבלה בין ישרים הוא יחס שקילות.
\end{exercise}
נבחן את המודל אשר כולל את $P, Q, R, S$ ואת הישרים $l, l', m, m', \langle Q, R \rangle, \langle P, S \rangle$.
זהו המודל המינימלי אשר עומד בהגדרת המישור האפיני, ולמעשה מהווה הדוגמה הפשוטה ביותר לאחד כזה.

\subsection{הגישה האנליטית}
עתה כאשר בחנו את המישור מבחינה סינתטית אנו יכולים לעבור לבחון את המרחב באופן אנליטי.
\begin{definition}[מודל אנליטי]
	יהי $\FF$ שדה ונסמן $\Pp = \FF^2$ וכן את הישרים שהם קבוצת השורשים של משוואות מהצורה $ax + by + c = 0$ עבור $a, b, c \in \FF$ ו־$a, b \ne 0$.
	במקרה זה ישרים מקבילים אם ורק אם $a, b$ המגדירים את הישרים שווים.
\end{definition}

\subsection{מרחבים אפיניים}
נראה עתה את ההגדרה שתאפשר לנו לדון במרחבים, בנקודות ובכיוונים, קרי ווקטורים.
\begin{definition}[מרחב אפיני]
	יהי $\FF$ שדה.
	מרחב אפיני נתון על־ידי שלשה $(E, V, t)$ כאשר $E$ קבוצה של נקודות, $V$ מרחב וקטורי מעל $\FF$, \\
	ו־$t : E \times V \to E$ אשר מסומנת גם $(P, v) \mapsto P + v$.
	$t$ מלשון translation, היא פונקציית ההזזה, מקיימת את התכונות הבאות,
	\begin{enumerate}
		\item אסוציאטיביות: $(P + v) + w = P + (v + w)$ לכל $P \in E, v, w \in V$
		\item איבר נייטרלי: $P + 0 = P$ לכל $P \in E$
		\item חד־חד ערכיות ברכיב השני: לכל $P, Q \in E$ קיים $v \in V$ יחיד כך שמתקיים $P + v = Q$, נסמן $v = \overrightarrow{P Q}$
	\end{enumerate}
\end{definition}
\begin{notation}
	נסמן את ההשמה החלקית של $t$ על־ידי $t_P$ עבור $P \in E$ נתונה, כלומר,
	\[
		t_P(v)
		= t(P, v)
		= P + v
	\]
\end{notation}
\begin{example}
	יהיו $I \subseteq \RR$ קטע ו־$f : I \to \RR$ רציפה. \\
	נסמן $E = \{ F : I \to \RR \mid F' = f \}$ וכן $V = \RR$, ולבסוף גם $(F, c) \mapsto F + c$. \\
	אז זהו מרחב אפיני, והמימד שלו הוא בדיוק 1.
\end{example}

\section{שיעור 2 --- 21.10.2025}
\subsection{מרחבים אפיניים --- המשך}
נמשיך לראות דוגמות למרחבים אפיניים.
\begin{example}
	נבחר את,
	\[
		E = \{ (x^1, \ldots, x^n) \in \FF^n \mid x^1 + \cdots + x^n = 1 \}
	\]
	יחד עם,
	\[
		V = \{ (\xi^1, \ldots, \xi^n) \in \FF^n \mid \xi^1 + \cdots + \xi^n = 0 \}
	\]
	ופונקציית ההזזה,
	\[
		t(x, \xi)
		= x + \xi
		= (x^1 + \xi^1, \ldots, x^n + \xi^n)
	\]
	זהו מרחב אפיני, ההוכחה שזהו המצב מושארת לקורא.
\end{example}
\begin{example}
	אם $V$ מרחב וקטורי מעל $\FF$ אז $(V, V, t)$ עבור $t : V \times V \to V$ המוגדרת על־ידי סכום, הוא מרחב אפיני.
\end{example}
בהינתן מרחב אפיני אנו יכולים לבנות פונקציה $v : E \times E \to V$ המסומנת על־ידי $v(P, Q) = Q - P$, היא הפונקציה שמתאימה לשתי נקודות את הווקטור היחיד שמקיים $P + w = Q$.
זוהי הפונקציה שמקיימת את התכונה השלישית של הגדרת המרחב האפיני, ובזמן שהיא שוברת באיזשהו מקום את הסימטריה שבין הנקודות, היא פותחת פתח לדיון אודות הרעיון של וקטורים.
\begin{proposition}[תכונות של פונקציית ההפרש]
	אם $v : E \times E \to V$ פונקציית ההפרש אז מתקיים,
	\begin{enumerate}
		\item לכל $P, Q, R \in E$ מתקיים $(Q - P) + (R - Q) = R - P$
		\item לכל $P \in E$ הפונקציה $v_P : E \to V$ המוגדרת על־ידי $v_P(Q) = Q - P = v(P, Q)$ היא פונקציה חד־חד ערכית ועל
	\end{enumerate}
\end{proposition}
\begin{proof}
	\begin{enumerate}
		\item ישירות מאקסיומות מרחב אפיני,
		\[
			P + ((Q - P) + (R - Q))
			= (P + (Q - P)) + (R - Q)
			= Q + (R - Q)
			= R
		\]
		\item עבור $w \in V$ תהי $Q = P + w$ אז,
			\[
				v_P(Q)
				= Q - P
				= v
			\]
			ולכן הפונקציה היא על.
			נניח ש־$v_P(Q) = v_P(R)$ עבור $R \in E$, אז,
			\[
				Q - P
				= R - P
				\implies Q = P + (Q - P)
				= P + (R - P)
				= R
			\]
			וקיבלנו חד־חד ערכיות.
			\qedhere
	\end{enumerate}
\end{proof}
\begin{proposition}
	עבור $P \in E$ הפונקציות $v_P$ ו־$t_P$ הן הופכיות אחת לשנייה.
\end{proposition}
\begin{proof}
	\[
		E \overset{v_P}{\mapsto} V
		\overset{t_P}{\mapsto} E
	\]
	לכל $Q \in E$ מתקיים,
	\[
		Q
		\mapsto Q - P
		\mapsto P + (Q - P)
		= Q
	\]
	וכן,
	\[
		V \overset{t_P}{\mapsto} E
		\overset{v_P}{\mapsto} V
	\]
	ומתקיים,
	\[
		v
		\mapsto P + v
		\mapsto (P + v) - P
		= v
	\]
\end{proof}
עתה אנו רוצים להגדיר מרחב וקטורי מעל המרחב האפיני שלנו, נבחן את $E \times E$ ונמפה את הנקודות ל־$V \times V$ על־ידי שימוש ב־$v_P \times v_P$.
נבחן את,
\[
	E \times E
	\xrightarrow{v_P \times v_P} V \times V
	\xrightarrow{+} V
	\xrightarrow{t_P} E
	\xleftarrow{+} E \times E
\]
כלומר, נבחן את המיפוי,
\[
	(Q, R)
	\mapsto (Q - P, R - P)
	\overset{+}{\mapsto}  (Q - P) + (R - P)
	\overset{t_P}{\leftarrow} P + (Q - P) + (R - P)
\]
את המבנה הזה נהוג לכנות $E_P = (E, P, +_P, \cdot_P)$ וזהו אכן מרחב וקטורי.

\subsection{תתי־מרחבים אפיניים}
\begin{definition}[תת־מרחב אפיני]
	יהי מרחב אפיני $(E, V)$.
	קבוצה $L \subseteq E$ תיקרא תת־מרחב אפיני אם $L = \emptyset$ או שקיימים $P \in L$ ו־$W \le V$ כך שמתקיים,
	\[
		L
		= P + W
		= \{ P + w \mid w \in W \}
	\]
	נקרא גם יריעה אפינית או יריעה לינארית, ולמעשה נשתמש בשמות אלה יותר.
\end{definition}
\begin{example}
	נבחן את $E = \RR^2$ ונגדיר את,
	\[
		L = \{ (x, y) \in \RR^2 \mid x + y = 1 \}
	\]
	נבחין כי $L$ הוא לא תת־מרחב של המרחב הלינארי $E$, אך אנו לא בוחנים את $E$ ואת $L$ כמרחבים לינאריים, אלא כמרחבים אפיניים.
	במקרה זה אם נבחר את $W = \Sp\{ (1, -1) \} = \{ (x, y) \in \RR^2 \mid x + y = 0 \} \le \RR^2$ וכן את $P = (0, 1)$ אז נקבל $L = P + W$.
\end{example}
\begin{remark}
	אם $L = P + W$ תת־מרחב אפיני, אז,
	\[
		W
		= L - P
		= \{ Q - P \mid Q \in L \}
	\]
	בהתאם גם $Q \in L \implies Q = P + w$ עבור $w \in W$ כלשהו, נובע ש־$Q - P = w \in W$.
\end{remark}
\begin{theorem}[יחידות תת־מרחב לינארי פורס]
	$P + W = Q + W'$ עבור $P, Q \in E$ ו־$W, W' \le V$ אם ורק אם $W = W'$ ו־$Q - P \in W$.
\end{theorem}
ההוכחה מושארת במסגרת התרגילים הבאים.
\begin{exercise}
	הוכיחו כי $P + W = Q + W$ אם ורק אם $Q - P \in W$.
\end{exercise}
\begin{exercise}
	הוכיחו כי אם $R + W = R + W'$ אז נובע ש־$W = W'$.
\end{exercise}
\begin{definition}[מרחב משיק]
	$W = W(L)$ נקרא מרחב הכיוונים או המרחב המשיק של $L$. \\
	בהתאם נסמן $\dim_{\FF} L = \dim_{\FF} W$ כמימד תת־המרחב.
\end{definition}
\begin{exercise}
	הוכיחו כי חיתוך של תתי־יריעות הוא תת־יריעה.
\end{exercise}
\begin{definition}
	אם $S \subseteq E$ קבוצה של נקודות, אז נאמר ש־$L$ הוא תת־היריעה האפינית הנוצרת על־ידי $S$ אם $L$ הוא היריעה המינימלית במימדה המכילה את כלל הנקודות.
\end{definition}
\begin{example}
	אם $E = \RR^2$ אז תת־היריעה הנוצרת על־ידי $\{ (0, 0) \}$ היא היריעה $\{ (0, 0) \}$, תת־היריעה הנוצרת על־ידי $\{ (0, 1), (1, 0) \}$ היא היריעה $L = \{ (x, y) \in \RR^2 \mid x + y = 0 \}$.
\end{example}
\begin{definition}
	קבוצה של נקודות תיקרא בלתי־תלויה אפינית אם אין נקודה ששייכת למרחב האפיני שנוצר על־ידי יתר הנקודות.
\end{definition}
\begin{example}
	במרחב $\RR^3$ הקבוצות הבאות בלתי־תלויות אפינית:
	\[
		\{ (0, 1, 0) \},
		\quad
		\{ (0, 1, 0), (1, 0, 0) \},
		\quad
		\{ (0, 1, 0), (1, 0, 0), (0, 0, 1) \}
	\]
	אך לא יכול להיות שתהיה קבוצה בגודל $4$ כזו אם הנקודות הן לא קולינאריות.
\end{example}
\begin{theorem}
	יהי $(E, V)$ מרחב אפיני.
	תהי $(P_1, \ldots, P_r)$ סדרת נקודות ב־$E$ ותהי $(\lambda^1, \ldots, \lambda^n)$ סדרת סקלרים ב־$\FF$ עם התכונה ש־$\lambda^1 + \cdots + \lambda^n = 1$.
	אז לכל $P_0, P_0' \in E$ מתקיים,
	\[
		P_0 + \lambda^1 (P_1 - P_0) + \lambda^2 (P_2 - P_0) + \cdots + \lambda^r (P_r - P_0)
		= P_0' + \lambda^1 (P_1 - P_0') + \lambda^2 (P_2 - P_0') + \cdots + \lambda^r (P_r - P_0')
	\]
\end{theorem}
\begin{proof}
	\begin{align*}
		& P_0 + \lambda^1 (P_1 - P_0) + \lambda^2 (P_2 - P_0) + \cdots + \lambda^r (P_r - P_0) \\
		= & P_0' + (P_0 - P_0') + \lambda^1 ((P_1 - P_0') + (P_0' - P_0)) + \cdots + \lambda^r ((P_r - P_0') + (P_0' - P_0)) \\
		= & P_0' + (P_0 - P_0') + (\lambda^1 + \cdots + \lambda^r) (P_0' - P_0) + \lambda^1 (P_1 - P_0') + \cdots + \lambda^r (P_r - P_0')
	\end{align*}
\end{proof}
\begin{notation}
	נסמן את הנקודה היחידה הזו שאיננה תלויה בראשית בסימון $\lambda^1 P_1 + \cdots + \lambda^r P_r$. \\
	זה נקרא צירוף אפיני והוא התחליף שלנו לצירופים לינאריים, והוא אף סגור להם.
\end{notation}

\listoftheorems[title=הגדרות ומשפטים,ignoreall,show={theorem,definition},swapnumber,onlynamed={proposition,lemma}]

\end{document}
