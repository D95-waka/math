\input{../article_base.tex}
\title{גאומטריה דיפרנציאלית אלמנטרית --- סיכום}
\setcounter{secnumdepth}{2}

\usepackage{fancyhdr}
\pagestyle{fancy}
\renewcommand{\headrulewidth}{0pt}

\DeclareMathOperator\eval{eval}

\begin{document}
\maketitle
\maketitleprint[purple]

\tableofcontents

\section{שיעור 1 --- 20.10.2025}
\subsection{מבוא}
גאומטריה היא אבן יסוד של החברה שלנו, והיא לוקחת חלק בכל תהליך בנייה תכנון ומדידה.
לאורך ההיסטוריה היה חקר של גאומטריה באיזשהו אופן נאיבי, אך אנו נעסוק בחקר של הגאומטריה באופן האקסיומטי שלה.
אנו נעסוק בחקר של צורות חלקות, כלומר שאפשר ללטף אותן, תוך שימוש בכלים שראינו באנליזה.
הרעיון בקורס הוא לגשת בצורה אלמנטרית לבעיות לאו דווקא מורכבות בגישה שהיא גאומטרית.
הצורות שנחקור הן יריעות, ככל הנראה יריעות חלקות.

\subsection{הגישה הסינתטית}
המתמטיקה המודרנית מתבססת על תורת הקבוצות, לכן עלינו לספק הגדרה קבוצתית הולמת למושג המישור.
\begin{definition}[ישרים מקבילים]
	שני ישרים נקראים מקבילים אם הם מתלכדים או אינם נחתכים.
\end{definition}
\begin{definition}[קולינאריות]
	נאמר שקבוצה של נקודות הן קולינאריות כאשר כל הנקודות שייכות לישר אחד.
\end{definition}
\begin{definition}[מישור אפיני]
	זוג סדור $(\Pp, \Ll)$ כאשר $\Pp$ קבוצה שאת ערכיה נכנה נקודות ו־$\Ll$ קבוצה של קבוצות של נקודות, אותן נכנה ישרים.
	זוג סדור זה יקרא מישור אפיני אם הוא מקיים את התכונות הבאות,
	\begin{enumerate}
		\item לכל שתי נקודות יש ישר יחיד המכיל את שתיהן
		\item לכל ישר ונקודה קיים ישר יחיד מקביל לישר העובר דרך הנקודה
		\item קיימות שלוש נקודות שאינן קולינאריות
	\end{enumerate}
\end{definition}
נעבור למשפט יסודי שמדגים את אופי המישור האפיני.
\begin{theorem}[מספר נקודות מינימלי במישור אפיני]
	יהי מרחב אפיני $(\Pp, \Ll)$, אז ב־$\Pp$ לפחות $4$ נקודות.
\end{theorem}
\begin{proof}
	יהיו $P, Q, R \in \Pp$ נקודות שונות ולא קולינאריות.
	נסמן גם $l = \langle P, Q \rangle, m = \langle P, R \rangle$ שני הישרים העוברים דרך הנקודות המתאימות.
	נסמן את $l' \parallel l \ni R, m \parallel m' \ni P$, וגם את $\{ S \} = l' \cap m'$ עבור $S \in P$.
	אנו טוענים כי $S$ קיימת וכי היא נקודה רביעית.

	נטען טענת עזר, והיא ש־$l' \not\parallel m'$.
	אילו $l' \parallel m'$ אז מטרנזיטיביות יחס השקילות המושרה מיחס ההקבלה היה נובע כי $l \parallel l' \parallel m' \parallel m$, אבל אז מהתכונה השנייה של מישור אפיני היה מתקבל ש־$l = m$ בסתירה לבחירת $P, Q, R$.

	אם $S \in \{ P, Q \}$ אז היה נובע ש־$l = l'$ ולכן גם $R \in l$, בסתירה.
	אם באופן שקול $S \in \{ P, R \}$ אז נקבל סתירה דומה, ולכן נותר להניח ש־$S$ קיימת ושונה מ־$P, Q, R$.
\end{proof}
שני התרגילים הבאים יאפשרו לנו לתרגל את הגישה הסינתטית,
\begin{exercise}
	הוכיחו כי כל ישר מכיל לפחות שתי נקודות שונות.
\end{exercise}
\begin{exercise}
	הוכיחו כי יחס ההקבלה בין ישרים הוא יחס שקילות.
\end{exercise}
נבחן את המודל אשר כולל את $P, Q, R, S$ ואת הישרים $l, l', m, m', \langle Q, R \rangle, \langle P, S \rangle$.
זהו המודל המינימלי אשר עומד בהגדרת המישור האפיני, ולמעשה מהווה הדוגמה הפשוטה ביותר לאחד כזה.

\subsection{הגישה האנליטית}
עתה כאשר בחנו את המישור מבחינה סינתטית אנו יכולים לעבור לבחון את המרחב באופן אנליטי.
\begin{definition}[מודל אנליטי]
	יהי $\FF$ שדה ונסמן $\Pp = \FF^2$ וכן את הישרים שהם קבוצת השורשים של משוואות מהצורה $ax + by + c = 0$ עבור $a, b, c \in \FF$ ו־$a, b \ne 0$.
	במקרה זה ישרים מקבילים אם ורק אם $a, b$ המגדירים את הישרים שווים.
\end{definition}

\subsection{מרחבים אפיניים}
נראה עתה את ההגדרה שתאפשר לנו לדון במרחבים, בנקודות ובכיוונים, קרי ווקטורים.
\begin{definition}[מרחב אפיני]
	יהי $\FF$ שדה.
	מרחב אפיני נתון על־ידי שלשה $(E, V, t)$ כאשר $E$ קבוצה של נקודות, $V$ מרחב וקטורי מעל $\FF$, \\
	ו־$t : E \times V \to E$ אשר מסומנת גם $(P, v) \mapsto P + v$.
	$t$ מלשון translation, היא פונקציית ההזזה, מקיימת את התכונות הבאות,
	\begin{enumerate}
		\item אסוציאטיביות: $(P + v) + w = P + (v + w)$ לכל $P \in E, v, w \in V$
		\item איבר נייטרלי: $P + 0 = P$ לכל $P \in E$
		\item חד־חד ערכיות ברכיב השני: לכל $P, Q \in E$ קיים $v \in V$ יחיד כך שמתקיים $P + v = Q$, נסמן $v = \overrightarrow{P Q}$
	\end{enumerate}
\end{definition}
\begin{notation}
	נסמן את ההשמה החלקית של $t$ על־ידי $t_P$ עבור $P \in E$ נתונה, כלומר,
	\[
		t_P(v)
		= t(P, v)
		= P + v
	\]
\end{notation}
\begin{example}
	יהיו $I \subseteq \RR$ קטע ו־$f : I \to \RR$ רציפה. \\
	נסמן $E = \{ F : I \to \RR \mid F' = f \}$ וכן $V = \RR$, ולבסוף גם $(F, c) \mapsto F + c$. \\
	אז זהו מרחב אפיני, והמימד שלו הוא בדיוק 1.
\end{example}

\section{שיעור 2 --- 21.10.2025}
\subsection{מרחבים אפיניים --- המשך}
נמשיך לראות דוגמות למרחבים אפיניים.
\begin{example}
	נבחר את,
	\[
		E = \{ (x^1, \ldots, x^n) \in \FF^n \mid x^1 + \cdots + x^n = 1 \}
	\]
	יחד עם,
	\[
		V = \{ (\xi^1, \ldots, \xi^n) \in \FF^n \mid \xi^1 + \cdots + \xi^n = 0 \}
	\]
	ופונקציית ההזזה,
	\[
		t(x, \xi)
		= x + \xi
		= (x^1 + \xi^1, \ldots, x^n + \xi^n)
	\]
	זהו מרחב אפיני, ההוכחה שזהו המצב מושארת לקורא.
\end{example}
\begin{example}
	אם $V$ מרחב וקטורי מעל $\FF$ אז $(V, V, t)$ עבור $t : V \times V \to V$ המוגדרת על־ידי סכום, הוא מרחב אפיני.
\end{example}
נעצור עתה ונגיד שמרחב אפיני באיזשהו מובן הוא מרחב וקטורי, אבל ללא הקונספט של ראשית, ובכך הוא מאפשר גמישות רבה יותר, בהמשך נראה שהקשר בין המונחים חזק אף יותר, ושאחד משרה את השני. \\
המרחב האפיני מורכב מפונקציית התרגום, אנו רוצים לשאול את השאלה ההפוכה עתה, מהו הווקטור היחיד שמתרגם נקודה לנקודה אחרת.
בהתאם, ניגש להגדרה הבאה.
\begin{definition}[פונקציית הפרש]
	יהי מרחב אפיני $(E, V, t)$, פונקציה $v : E \times E \to V$ תיקרא פונקציית הפרש אם לכל $P, Q \in E$ מתקיים,
	\[
		t(P, v(P, Q)) = Q
	\]
	כלומר היא הפונקציה שמתאימה לנקודות $P$ ו־$Q$ את הווקטור היחיד $w$ המקיים $P + w = Q$. \\
	נסמן גם $v(P, Q) = Q - P$.
\end{definition}
\begin{notation}
	נגדיר $v_P : E \to V$ להשמה החלקית $v_P(Q) = v(P, Q) = Q - P$.
\end{notation}
\begin{remark}
	אם $v, v' : E \times E \to V$ שתי פונקציות הפרש, אז מתקיים,
	\[
		\forall P, Q \in E,\ v(P, Q) = v'(P, Q)
	\]
	ישירות מהגדרת המרחב האפיני, לכן נאמר על $v$ שהיא פונקציית ההפרש היחידה והייחודית למרחב.
\end{remark}
\begin{proposition}[תכונות של פונקציית ההפרש]
	אם $v : E \times E \to V$ פונקציית ההפרש אז מתקיים,
	\begin{enumerate}
		\item לכל $P, Q, R \in E$ מתקיים $(Q - P) + (R - Q) = R - P$
		\item לכל $P \in E$ הפונקציה $v_P : E \to V$ המוגדרת על־ידי $v_P(Q) = Q - P = v(P, Q)$ היא פונקציה חד־חד ערכית ועל
	\end{enumerate}
\end{proposition}
\begin{proof}
	\begin{enumerate}
		\item ישירות מאקסיומות מרחב אפיני,
		\[
			P + ((Q - P) + (R - Q))
			= (P + (Q - P)) + (R - Q)
			= Q + (R - Q)
			= R
		\]
		\item עבור $w \in V$ תהי $Q = P + w$ אז,
			\[
				v_P(Q)
				= Q - P
				= v
			\]
			ולכן הפונקציה היא על.
			נניח ש־$v_P(Q) = v_P(R)$ עבור $R \in E$, אז,
			\[
				Q - P
				= R - P
				\implies Q = P + (Q - P)
				= P + (R - P)
				= R
			\]
			וקיבלנו חד־חד ערכיות.
			\qedhere
	\end{enumerate}
\end{proof}
נבחין כי בזמן שפונקציית ההפרש שוברת את הסימטריה שהתרגלנו אליה בפונקציית התרגום, אך היא מהווה משלים שלה, הטענה הבאה מציגה לנו את הקשר ההדוק שבין הרעיונות.
\begin{proposition}
	עבור $P \in E$ הפונקציות $v_P$ ו־$t_P$ הן הופכיות אחת לשנייה.
\end{proposition}
\begin{proof}
	\[
		E \overset{v_P}{\mapsto} V
		\overset{t_P}{\mapsto} E
	\]
	לכל $Q \in E$ מתקיים,
	\[
		Q
		\mapsto Q - P
		\mapsto P + (Q - P)
		= Q
	\]
	וכן,
	\[
		V \overset{t_P}{\mapsto} E
		\overset{v_P}{\mapsto} V
	\]
	ומתקיים,
	\[
		v
		\mapsto P + v
		\mapsto (P + v) - P
		= v
	\]
\end{proof}
עתה אנו רוצים להגדיר מרחב וקטורי מעל המרחב האפיני שלנו, נבחן את $E \times E$ ונמפה את הנקודות ל־$V \times V$ על־ידי שימוש ב־$v_P \times v_P$.
נבחן את,
\[
	E \times E
	\xrightarrow{v_P \times v_P} V \times V
	\xrightarrow{+} V
	\xrightarrow{t_P} E
	\xleftarrow{+} E \times E
\]
כלומר, נבחן את המיפוי,
\[
	(Q, R)
	\mapsto (Q - P, R - P)
	\overset{+}{\mapsto}  (Q - P) + (R - P)
	\overset{t_P}{\leftarrow} P + (Q - P) + (R - P)
\]
מכאן יש לנו הפתח להגדרה הבאה.
את המבנה הזה נהוג לכנות $E_P = (E, P, +_P, \cdot_P)$ וזהו אכן מרחב וקטורי.
\begin{definition}[מרחב וקטורי מושרה מנקודה]
	יהי $(E, V, t)$ מרחב אפיני ותהי $P \in E$ נקודה כלשהי.
	עבור $+_P : E \times E \to E$ המוגדרת על־ידי,
	\[
		\forall Q, R \in E,\ Q +_P R = Q + R - P
	\]
	ו־$\cdot_P : \FF \times E \to E$ המוגדרת על־ידי,
	\[
		\forall \alpha \in F, Q \in E,\ \alpha \cdot_P Q = \alpha \cdot (Q - P) + P
	\]
	המרחב $(E, P, +_P, \cdot_P)$ הוא מרחב וקטורי המושרה מהמרחב האפיני והנקודה.
\end{definition}
\begin{exercise}
	הוכיחו כי זהו אכן מרחב וקטורי.
\end{exercise}

\subsection{תתי־מרחבים אפיניים}
כבר ראינו שמרחב אפיני באיזשהו עולם מתנהג ומדבר בשפה של מרחבים וקטוריים, ובדיוק כמו בהם, גם כאן נרצה לעסוק בתת־מרחבים, נתחיל בהגדרת תת־המרחב האפיני.
\begin{definition}[תת־מרחב אפיני]
	יהי מרחב אפיני $(E, V)$.
	קבוצה $L \subseteq E$ תיקרא תת־מרחב אפיני אם $L = \emptyset$ או שקיימים $P \in L$ ו־$W \le V$ כך שמתקיים,
	\[
		L
		= P + W
		= \{ P + w \mid w \in W \}
	\]
	נקרא גם יריעה אפינית או יריעה לינארית, ולמעשה נשתמש בשמות אלה יותר.
\end{definition}
\begin{example}
	נבחן את $E = \RR^2$ ונגדיר את,
	\[
		L = \{ (x, y) \in \RR^2 \mid x + y = 1 \}
	\]
	נבחין כי $L$ הוא לא תת־מרחב של המרחב הלינארי $E$, אך אנו לא בוחנים את $E$ ואת $L$ כמרחבים לינאריים, אלא כמרחבים אפיניים.
	במקרה זה אם נבחר את $W = \Sp\{ (1, -1) \} = \{ (x, y) \in \RR^2 \mid x + y = 0 \} \le \RR^2$ וכן את $P = (0, 1)$ אז נקבל $L = P + W$.
\end{example}
\begin{remark}
	אם $L = P + W$ תת־מרחב אפיני, אז,
	\[
		W
		= L - P
		= \{ Q - P \mid Q \in L \}
	\]
	בהתאם גם $Q \in L \implies Q = P + w$ עבור $w \in W$ כלשהו, נובע ש־$Q - P = w \in W$.
\end{remark}
\begin{theorem}[יחידות תת־מרחב לינארי פורס]
	$P + W = Q + W'$ עבור $P, Q \in E$ ו־$W, W' \le V$ אם ורק אם $W = W'$ ו־$Q - P \in W$.
\end{theorem}
ההוכחה מושארת במסגרת התרגילים הבאים.
\begin{exercise}
	הוכיחו כי $P + W = Q + W$ אם ורק אם $Q - P \in W$.
\end{exercise}
\begin{exercise}
	הוכיחו כי אם $R + W = R + W'$ אז נובע ש־$W = W'$.
\end{exercise}
\begin{definition}[מרחב משיק]
	$W = W(L)$ נקרא מרחב הכיוונים או המרחב המשיק של $L$. \\
	בהתאם נסמן $\dim_{\FF} L = \dim_{\FF} W$ כמימד תת־המרחב.
\end{definition}
\begin{exercise}
	הוכיחו כי חיתוך של תתי־יריעות הוא תת־יריעה.
\end{exercise}
\begin{definition}
	אם $S \subseteq E$ קבוצה של נקודות, אז נאמר ש־$L$ הוא תת־היריעה האפינית הנוצרת על־ידי $S$ אם $L$ הוא היריעה המינימלית במימדה המכילה את כלל הנקודות.
\end{definition}
\begin{example}
	אם $E = \RR^2$ אז תת־היריעה הנוצרת על־ידי $\{ (0, 0) \}$ היא היריעה $\{ (0, 0) \}$, תת־היריעה הנוצרת על־ידי $\{ (0, 1), (1, 0) \}$ היא היריעה $L = \{ (x, y) \in \RR^2 \mid x + y = 0 \}$.
\end{example}
\begin{definition}
	קבוצה של נקודות תיקרא בלתי־תלויה אפינית אם אין נקודה ששייכת למרחב האפיני שנוצר על־ידי יתר הנקודות.
\end{definition}
\begin{example}
	במרחב $\RR^3$ הקבוצות הבאות בלתי־תלויות אפינית:
	\[
		\{ (0, 1, 0) \},
		\quad
		\{ (0, 1, 0), (1, 0, 0) \},
		\quad
		\{ (0, 1, 0), (1, 0, 0), (0, 0, 1) \}
	\]
	אך לא יכול להיות שתהיה קבוצה בגודל $4$ כזו אם הנקודות הן לא קולינאריות.
\end{example}
\begin{theorem}
	יהי $(E, V)$ מרחב אפיני.
	תהי $(P_1, \ldots, P_r)$ סדרת נקודות ב־$E$ ותהי $(\lambda^1, \ldots, \lambda^n)$ סדרת סקלרים ב־$\FF$ עם התכונה ש־$\lambda^1 + \cdots + \lambda^n = 1$.
	אז לכל $P_0, P_0' \in E$ מתקיים,
	\[
		P_0 + \lambda^1 (P_1 - P_0) + \lambda^2 (P_2 - P_0) + \cdots + \lambda^r (P_r - P_0)
		= P_0' + \lambda^1 (P_1 - P_0') + \lambda^2 (P_2 - P_0') + \cdots + \lambda^r (P_r - P_0')
	\]
\end{theorem}
\begin{proof}
	\begin{align*}
		& P_0 + \lambda^1 (P_1 - P_0) + \lambda^2 (P_2 - P_0) + \cdots + \lambda^r (P_r - P_0) \\
		= & P_0' + (P_0 - P_0') + \lambda^1 ((P_1 - P_0') + (P_0' - P_0)) + \cdots + \lambda^r ((P_r - P_0') + (P_0' - P_0)) \\
		= & P_0' + (P_0 - P_0') + (\lambda^1 + \cdots + \lambda^r) (P_0' - P_0) + \lambda^1 (P_1 - P_0') + \cdots + \lambda^r (P_r - P_0')
	\end{align*}
\end{proof}
\begin{notation}
	נסמן את הנקודה היחידה הזו שאיננה תלויה בראשית בסימון $\lambda^1 P_1 + \cdots + \lambda^r P_r$. \\
	זה נקרא צירוף אפיני והוא התחליף שלנו לצירופים לינאריים, והוא אף סגור להם.
\end{notation}

\section{שיעור 3 --- 27.10.2025}
\subsection{העתקות אפיניות}
עד כה יש לנו את המרחב האפיני $(E, V)$, ונגדיר את המודל הסטנדרטי.
\begin{definition}[מודל אפיני סטנדרטי]
	נסמן $\AAA^n = (\AAA^n, \RR^n)$ כאשר $\AAA^n = \RR^n$, ונסמן את הערכים בו בעזרת $(x, \xi) \in \AAA^n$.
	פונקציית הצירוף תוגדר על־ידי חיבור.
\end{definition}
עתה נעבור לעסוק בהעתקות משמרות מבנה.
\begin{definition}[העתקה אפינית]
	נניח ש־$(E, V), (F, U)$ מעל שדה משותף $\FF$.
	העתקה אפינית, המסומנת $E \to F$, נתונה על־ידי זוג פונקציות $f : E \to F$ ו־$\varphi : V \to U$ העתקה לינארית,
	כך שמתקיים,
	\[
		\forall P \in E \forall v \in V,\ f(P + v) = f(P) + \varphi(v)
	\]
\end{definition}
\begin{remark}
	תוך שימוש בהגדרה הדואלית שלנו נסיק שמתקיים, $f(P + v) = f(P) + \varphi(v) \iff f(P + v) - f(P) = \varphi(v)$.
	נובע אם כך ש־$\varphi$ נקבעת ביחידות עבור הפונקציה $f$.
\end{remark}
\begin{notation}
	נסמן $df = \varphi$ ונאמר ש־$\varphi$ היא הדיפרנציאל של $f$.
\end{notation}
הסיבה שאנחנו קוראים ל־$\varphi$ כך היא שפונקציות $f$ שונות יכולות להיות בעלות דיפרנציאל זהה, והן תיבדלנה בקבוע בלבד, כלומר הדיפרנציאל מתנהג כפי שהיינו מצפים מדיפרנציאל באנליזה.
נעבור למספר דוגמות להעתקות אפיניות.
\begin{example}
	פונקציה קבועה, כלומר $f(P) = f(Q)$ לכל $P, Q \in E$.
	במקרה זה $\varphi \equiv 0_V$ או ש־$\varphi = 0_{\Hom(V, U)}$.
\end{example}
\begin{example}
	הזזות. נבחן את המקרה $E = F, V = U$, כלומר בבחינת אנדומורפיזם, לכל $w \in V$ נגדיר את ההעתקה $t_w : E \to E$ על־ידי $P \mapsto P + w$. \\
	נבדוק שהיא אכן אפינית,
	אם $P \in E, v \in V$ אז,
	\[
		t_w(P + v) - t_w(P)
		= (P + v) + w - (P + w)
		= (P + w) + v - (P + w)
		= v
	\]
	כאשר,
	\[
		(P + v) + w
		= P + (v + w)
		= P + (w + v)
		= (P + w) + v
	\]
	ישירות מהאקסיומה השנייה ושימוש בקומוטטיביות החיבור במרחבים וקטוריים.
	כלומר $\varphi(v) = v$ לכל $v \in V$, או בסימון שלנו $dt_w = \id_V$.
\end{example}
\begin{example}
	פונקציות הומותטיות (homothecy).
	יהיו $O \in F, \lambda \in \FF$, אז נגדיר את הפונקציה $h_{O, \lambda} : E \to F$ על־ידי ההכפלה של וקטור פי $\lambda$ במרחקו מ־$O$, כלומר,
	\[
		h_{O, \lambda}(P) = O + \lambda(P - O)
	\]
	ומתקיים $dh_{O, \lambda} = \lambda \id_V$.
\end{example}
\begin{exercise}
	הוכיחו כי פונקציה הומותטית היא העתקה אפינית.
\end{exercise}
\begin{example}
	נניח ש־$E = \AAA^n$ וש־$F = \AAA^m$.
	נגדיר את ההעתקה $\AAA^n \to \AAA^m$ על־ידי,
	\[
		x \mapsto A \cdot x + b
	\]
	עבור $A \in \operatorname{Mat}_{n, m}(\FF)$ ו־$b \in F$.
	במקרה זה הדיפרנציאל הוא ההעתקה הלינארית המיוצגת על־ידי $A$.
\end{example}
\begin{exercise}
	הוכיחו שהרכבה של העתקות אפיניות היא אפינית. \\
	הוכיחו שגם $d(g \circ f) = dg \circ df$.
\end{exercise}
\begin{definition}[איזומורפיזם אפיני]
	$f: E \to F$ תיקרא איזומורפיזם אפיני אם קיימת העתקה אפינית $g : F \to E$ כך שמתקיים,
	\[
		g \circ f = \id_E
		\quad
		f \circ g = \id_F
	\]
	במקרה ש־$E = F$ נקרא להעתקה אוטומורפיזם אפיני, ונסמן ב־$\aut(E)$ להיות חבורת האוטומורפיזמים מעל המרחב האפיני $E$.
\end{definition}

\subsection{יוצרים ובסיסים}
\begin{definition}[תת־יריעה נוצרת]
	נניח ש־$S \subseteq E$ תת־קבוצה, אז $\langle S \rangle \le E$ היא תת־היריעה הנוצרת על־ידי $S$, והיא חיתוך כל היריעות המכילות את $S$,
	\[
		\forall S,\ \subseteq L \le E \implies \langle S \rangle \subseteq L
	\]
	משפחה $S$ של נקודות נקראת יוצרת של $E$ אם $\langle S \rangle = E$.
\end{definition}
\begin{definition}[בסיס אפיני]
	$\{ P_0, \ldots, P_n \}$ תיקרא סדרה בלתי־תלויה אפינית כאשר $\dim \langle P_0, \ldots, P_n \rangle = n$ ונכנה את $\{ P_0, \ldots, P_n \}$ בסיס אפיני ובלתי־תלוי אפינית.
\end{definition}
נעבור עתה לדבר על קורדינטות.
\begin{definition}[מערכת יחוס]
	יהי $E$ ממימד סופי מעל $\FF$.
	מערכת יחוס מעל $E$ נתונה על־ידי זוג $(O, \Bb)$ כאשר $O \in E$ ו־$\Bb = (b_1, \ldots, b_n)$ בסיס סדור של $V$.
\end{definition}
\begin{proposition}
	בהינתן מערכת יחוס $(O, \Bb)$ לכל $P \in E$ קיימת הצגה יחידה $P = O + \sum_{i = 1}^n b_i x^i$ עבור $x \in \FF^n$.
\end{proposition}
\begin{definition}[קורדינטה]
	ל־$x = (x^1, \ldots, x^n)$ היחיד כך ש־$P = O + \Bb x$ נקרא הקורדינטות של $P$ במערכת הייחוס $(O, \Bb)$.
\end{definition}

\section{שיעור 4 --- 28.10.2025}
\subsection{קורדינטות --- המשך}
\begin{definition}[מפה ופרמטריזציה למרחב וקטורי]
	אם $V$ מרחב וקטורי מעל $\FF$ כך ש־$\dim V = n$, אז נכנה את ההעתקה $x : V \to \FF^n$ מפה, העתקה כמובן תלויה בקורדינטה.
	המיפוי ההפוך $\FF^n \to V$ יכונה פרמטריזציה של $V$.
\end{definition}
\begin{theorem}[מרחב וקטורי מושרה]
	תהי $V$ קבוצה ותהי $\operatorname{Coor}(V) = \{ f : V \to \FF^n \mid f \text{ is bijection} \}$ עבור $n \in \NN$ כלשהו,
	כך שמתקיים שלכל $x, y \in \operatorname{Coor}(V)$ מתקיים,
	\[
		y \circ x^{-1} \in GL_n(\FF)
	\]
	בתנאים אלה ניתן להגדיר על הקבוצה $V$ מבנה של מרחב וקטורי מעל $\FF$ יחיד עם התכונה שלכל $x \in \operatorname{Coor}(V)$ הוא איזומורפיזם לינארי.
\end{theorem}
\begin{proof}
	תהי $x \in \operatorname{Coor}(V)$.
	נגדיר את החיבור על־ידי,
	\[
		+_V : V \times V \to V,
		\quad
		v +_V u = x^{-1}(x(v) +_{F^n} x(u))
	\]
	וכן,
	\[
		\cdot_V : \FF \times V \to V,
		\quad
		\alpha \cdot_V u = x^{-1}(\alpha x(u))
	\]
	נזכור ש־$x$ הוא איזומורפיזם לינארי ולכן,
	\[
		x(v + w)
		= x(x^{-1}(x(w) + x(v)))
		= x(v) + x(w)
	\]
	ובאופן דומה,
	\[
		x(\alpha u)
		= x(x^{-1}(\alpha x(u)))
		= \alpha \cdot x(u)
	\]
	ונשאר לנו להראות שהפונקציות שקיבלנו הן יחידות, כלומר שאין משמעות לבחירת $x$.
	נניח ש־$y \in \operatorname{Coor}(V)$, ונרצה להראות שמתקיים,
	\[
		x^{-1}(x(v) + x(w))
		= y^{-1}(y(v) + y(w))
	\]
	ובאופן דומה שוויון של הכפל.
	נסמן $y \circ x^{-1} = \lambda_Q$.
	נפעיל על שני הצדדים את הפונקציה $y$ ונקבל,
	\[
		y(v) + y(w)
		= y(x^{-1}(x(v) + x(w)))
		= \lambda_Q(x(v) + x(w))
	\]
	אבל $\lambda_Q$ היא לינארית ולכן נקבל,
	\[
		y(v) + y(w)
		= \lambda_Q(x(v)) + \lambda_Q(x(w))
		= y(v) + y(w)
	\]
	ומצאנו שאכן יש שוויון.
\end{proof}
\begin{definition}[מפה אפינית]
	יהי $(E, V)$ מרחב אפיני $n$־מימדי.
	מערכת קורדינטות על $A$ היא איזומורפיזם $x : E \to \AAA^n$ אפיני. \\
	במקרה זה $x(u) = b + A u$ עבור $b \in \AAA^n$ ו־$A \in \aut(V, \FF^n)$.
	נגדיר גם את הקבוצה $GA_n(\FF)$ כקבוצת ה־$x$־ים הללו.
\end{definition}
למשפט שראינו יש אנלוגיה לגרסה האפינית.
הפעם במקום נקבע שתי נקודות כנקודות האפס ובכך נקבל שמתקיימים תנאי המשפט עבור המרחבים הווקטוריים.

\section{שיעור 5 --- 3.11.2025}

\subsection{מרחבים אפיניים ממשיים}
בחלק זה והלאה נעסוק בממשיים, כלומר מעתה $\FF = \RR$.
הדבר הראשון שנעסוק בו יהיה הנורמה.
\begin{definition}[נורמה]
	יהי $V$ מרחב וקטורי מעל $\RR$.
	נורמה מעל $V$ היא פונקציה $\lVert \cdot \rVert : V \to \RR_{\ge 0}$ המקיימת את התכונות,
	\begin{enumerate}
		\item חיוביות בהחלט: $\forall v \in V,\ 0 \le \lVert v \rVert$ וכן $v = 0 \iff \lVert v \rVert = 0$
		\item הומוגניות: $\forall v \in V, \alpha \in \RR,\ \lVert \alpha v \rVert = |\alpha| \cdot \lVert v \rVert$
		\item אי־שוויון המשולש: $\forall v, u \in V,\ \lVert v + u \rVert \le \lVert v \rVert + \lVert u \rVert$
	\end{enumerate}
\end{definition}
נראה מספר דוגמות לנורמות במקרה $V = \RR^p$.
\begin{example}
	${\lVert x \rVert}_{\infty} = \max\{ |x_1|, \ldots, |x_p| \}$, נורמת הסופרימום או נורמת אינסוף.
\end{example}
\begin{example}
	${\lVert x \rVert}_1 = |x_1| + \cdots + |x_p|$ היא נורמת 1.
\end{example}
\begin{example}
	הנורמה האוקלידית, ${\lVert x \rVert}_2 = \sqrt{x_1^2 + \cdots + x_p^2}$.
\end{example}
במרחבים סופיים מעל $\RR$ ישנו משפט הגורס כי כל הנורמות שקולות, כלומר לדוגמה ${\lVert x \rVert}_{\infty} \le {\lVert x \rVert}_1 \le p {\lVert x \rVert}_{\infty}$. \\
באופן אנלוגי גם ${\lVert x \rVert}_{\infty} \le {\lVert x \rVert}_2 \le \sqrt{p} {\lVert x \rVert}_{\infty}$.
\begin{definition}[מטריקה]
	עבור קבוצה $X$ נגדיר פונקציית מרחק, או מטריקה, כפונקציה $\rho : X^2 \to \RR_{\ge 0}$ כפונקציה המקיימת את התכונות:
	\begin{enumerate}
		\item חיוביות בהחלט: $\rho(x, y) = 0 \iff x = y$
		\item סימטריה: $\forall x, y \in X,\ \rho(x, y) = \rho(y, x)$
		\item אי־שוויון המשולש: $\forall x, y, z \in X,\ \rho(x, z) \le \rho(x, y) + \rho(y, z)$
	\end{enumerate}
\end{definition}
כל נורמה משרה מטריקה, כלומר כל מרחב נורמי הוא בפרט מרחב מטרי.
יהי $(E, V)$ מרחב אפיני מעל $\RR$.
אם על $V$ מוגדרת נורמה אז על $E$ מוגדרת פונקציית מרחק.
כלומר, נוכל להשרות מרחק גם על מרחב אפיני.
\begin{definition}[כדור]
	במרחב מטרי כללי $(X, \rho)$ נגדיר כדור (פתוח) על־ידי,
	\[
		B(x, r) = \{ y \in X \mid \rho(x, y) < r \}
	\]
	ונסמן לעתים את הכדור גם על־ידי $B_r(x) = B(x, r)$.
\end{definition}
במקרה של נורמה כמובן נקבל את הטענה השקולה שמתקיים,
\[
	B(x, r) = \{ y \in \RR^p \mid \lVert x - y \rVert < r \}
\]
נזכיר גם את ההגדרות המשלימות לכדור פתוח.
\begin{definition}[כדור סגור וספירה]
	נגדיר את הכדור הסגור על־ידי, $\overline{B}(x, r) = \overline{B(x, r)} = \{ y \in X \mid \rho(x, y) \le r \}$. \\
	באופן דומה נגדיר את הספירה, $S(x, r) = \partial B(x, r) = \{ y \in X \mid \rho(x, y) = r \}$.
\end{definition}
נגדיר גם התכנסות במרחב אפיני.
\begin{definition}[סדרה וסדרה מתכנסת]
	אם $(E, V)$ מרחב אפיני מעל $\RR$ וכן ${\{ P_n \}}_{n = 1}^\infty \subseteq E$ סדרת נקודות. \\
	נאמר שהסדרה מתכנסת לנקודה $P \in E$ כאשר $\lim_{n \to \infty} \lVert P_n - P \rVert = 0$ במובן הממשי.
\end{definition}
ובהתאם נצטט משפט חשוב שיעזור לנו.
\begin{theorem}[התכנסות והתכנסות קורדינטה קורדינטה]
	במקרה ש־$E = \RR^p$ אם ${\{ x_n \}}_{n = 1}^\infty, l \subseteq \RR^p$ סדרה ונקודה, אז מתקיים,
	\[
		x_n \to l
		\iff \forall i \le p,\ x_n^i \to l^i
	\]
	כלומר, הסדרה מתכנסת אם ורק אם היא מתכנסת קורדינטה קורדינטה.
\end{theorem}
המשפט נובע ישירות מהטענה כי $|x_n^i - l^i| \le \lVert x_n - l \rVert \le C |x_n^i - l^i|$.

\subsection{עקומים במרחב אפיני ממשי}
הקונספט של עקומים במרחב הוא קונספט שקצת קשה לעתים לדבר עליו. עקום הרי הוא רעיון מאוד כללי.
בשל כך, נתחיל בדיון על מסילות. לפני שניגש להגדרה הפורמלית נאמר שהמטרה שלנו היא לאפיין אובייקטים שהם קשירים מסילתית במרחב, וכן מהווים באיזשהו מובן תמונה של קטע, זאת אומרת שהם מתנהגים בערך כמו חוט שזז במרחב.
\begin{definition}[מסילה]
	מסילה (עקום פרמטרי) ב־$\AAA^n$ היא פונקציה $\alpha : I \to \AAA^n$, עבור $I \subseteq \RR$ קטע וכך ש־$\alpha$ גזירה. \\
	כלומר כשלכל $t \in I$ מתקיים,
	\[
		\lim_{s \to 0} \frac{\alpha(t + s) - \alpha(t)}{s} = L
	\]
	הוא גבול מוגדר וסופי.
	נסמן גבול זה ב־$\dot{\alpha}(t) = \alpha'(t)$ את ערך הנגזרת בנקודה כפונקציה של $t \in I$. \\
	המסילה תיקרא \textbf{רגולרית} כאשר $\alpha'(t) \ne 0$ לכל $t \in I$.
\end{definition}
כמובן, עתה משראינו את ההגדרה, נעבור לדוגמות.
\begin{example}
	כל הבאים הם מסילות:
	\begin{enumerate}
		\item ישרים (פרמטריים):
			אם $L \le E$ ישר אז $L = P + \{ t v \mid t \in \RR \}$ עבור $P \in E, v \in V$. \\
			בהתאם נוכל להגדיר פונקציה $\alpha : \RR \to \RR^n$ כך ש־$\alpha(t) = P + t v$.
			נבחין כי,
			\[
				\alpha'(t)
				= \lim_{s \to 0} \frac{\alpha(t + s) - \alpha(t)}{s}
				= \lim_{s \to 0} \frac{P + (t + s) v - P - t v}{s}
				= v
			\]
			ולכן המסילה רגולרית ובפרט $\alpha'(t) = v$.
		\item נגדיר את $\alpha : \RR \to \RR^2$ על־ידי $\alpha(t) = (r \cos t, r \sin t)$ כאשר $0 < r \in \RR$. זוהי מסילה כך שתמונתה היא מעגל ברדיוס $r$ במישור.
			הפעם נקבל ש־$\alpha'(t) = (-r \sin t, r \cos t)$, כלומר גם הפעם המסילה היא רגולרית. \\
			דרך פשוטה במיוחד לראות זאת היא על־ידי בחינת $\lVert \alpha'(t) \rVert = r$. \\
			באותו אופן נקבל שגם $\ddot{\alpha}(t) = \alpha''(t) = -r(\cos t, \sin t)$.
		\item במרחב נגדיר את המסילה $\alpha : \RR \to \RR^3$ על־ידי $\alpha(\cos t, \sin t, t)$, זוהי למעשה ספירלה.
			גם הפעם נוכל לראות כי זוהי מסילה רגולרית.
	\end{enumerate}
\end{example}
בעולם של יריעות, בפרט של מסילות, לא מעניינות אותנו תכונות שתלויות בפרמטריזציה, כלומר במסילה כפונקציה התלויה ב־$t$.
אנו מבקשים לעסוק בתמונה, במסלול של המסילה, כלומר באובייקט $\im \alpha \subseteq E$.
המקרה הראשון שנתרכז בו הוא האורך של עקומה.
\begin{definition}[אורך של מסילה]
	תהי $\alpha : [a, b] \to \RR^n$ מסילה רגולרית.
	נגדיר את האורך של המסילה באופן הבא,
	\[
		L(\alpha)
		= \int_{a}^{b} \lVert \alpha'(t) \rVert\ dt
	\]
\end{definition}
זוהי הגדרה שאולי הגיונית גאומטרית, אבל נרצה להראות שהיא אכן מקיימת את הקונספט של מרחק.
נגדיר הגדרה שבהמשך תוכיח את עצמה כשקולה.
\begin{definition}
	תהי $\Pp = (t_0, \ldots, t_k)$ חלוקה של $[a, b]$, כלומר $t_0 = a, t_k = b$ ומתקיים $t_n < t_{n + 1}$.
	בהינתן $\alpha : [a, b] \to \RR^p$ מסילה, אז נגדיר את הישר הפוליגונלי כמסילה שנוצרת על־ידי סדרת הנקודות $(\alpha(t_0), \ldots, \alpha(t_k))$.
	אז נגדיר את האורך בהינתן החלוקה $\Pp$ על־ידי,
	\[
		L_{\alpha}(\Pp)
		= \sum_{j = 1}^k \lVert \alpha(t_j) - \alpha(t_{j - 1}) \rVert
	\]
\end{definition}
ובהתאם ננסח את המשפט שמקשר את ההגדרות.
\begin{theorem}
	אם $\alpha : [a, b] \to \RR^p$ אז מתקיים,
	\[
		\forall \varepsilon > 0 \exists \delta > 0, \forall \Pp,\ (\lambda(\Pp) < \delta) \implies (|L(\alpha) - L_{\alpha}(\Pp)| < \varepsilon)
	\]
	כלומר עבור על חלוקה $\Pp$ של $[a, b]$ כך ש־$\Delta(\Pp) < \delta$ המרחק בין שני סוגי המרחק חסומים על־ידי $\varepsilon$.
\end{theorem}
אומנם את ההוכחה לא נביא, אך נרמוז ונגיד שאם נבחן את ההגדרה של האינטגרל לפי קושי, ונשתמש במשפט לגרנז', נוכל להוכיח את הטענה.

\section{שיעור 6 --- 4.11.2025}

\subsection{קמירות במרחבים אפיניים}
\begin{definition}[קמירות במרחב אפיני]
	עבור מרחב אפיני $(E, V)$ מעל הממשיים, צירוף אפיני $\lambda^0 P_0 + \cdots + \lambda^k P_k$ עבור $\lambda^i \in \RR, P_i \in E$ וכך ש־$\lambda^0 + \cdots + \lambda^k = 1$,
	יקרא קמור (convex) כאשר $0 \le \lambda^i$ לכל $0 \le i \le k$.
\end{definition}
\begin{definition}[קטע אפיני]
	בהינתן $P, Q \in E$ הקטע $[P, Q] \subseteq E$ המוגדר על־ידי,
	\[
		[P, Q]
		= \{ R \in E \mid R = \lambda P + \mu Q \mid \lambda + \mu = 1, 0 \le \lambda, \mu \}
	\]
\end{definition}
\begin{definition}[קבוצה קמורה]
	קבוצה $C \subseteq E$ של נקודות תיקרא קמורה (convex) אם $[P, Q] \subseteq C$ לכל $P, Q \in C$.
\end{definition}
באופן טבעי נוכל להגדיר קטע $I \subseteq \RR$ כקבוצה קמורה של ממשיים.
נרצה להגדיר הגדרה גאומטרית לקונספט של קמירות אפינית, ואז בהתאם להוכיח שהיא שקולה לסגור הקטעים של קבוצה.
\begin{definition}[סגור קמור אפיני]
	אם $A \subseteq E$ קבוצה של נקודות, אז נאמר שהקבוצה $C \subseteq E$ המקיימת,
	\[
		C = \bigcap_{\substack{A \subseteq K \subseteq E \\ \text{$K$ is convex}}} K
	\]
\end{definition}

\section{שיעור 7 --- 10.11.2025}
\subsection{פרמטריזציה לפי אורך}
אנו עוסקים בעקומים פרמטריים רגולריים, קרי במסילות רגולריות, $\alpha : I \to \AAA^n$.
הדרישה שלנו היא ש־$\alpha$ תהיה גזירה אינסוף פעמים, כלומר חלקה, ונדרוש ש־$\alpha'(t) \ne 0$ לכל $t \in I$.
נוכל גם לדבר על $A = \alpha(I) \subseteq \AAA^n$, וזהו למעשה העקום עצמו, נקרא לאובייקט זה גם מסלול.
בהינתן שתי מסילות אנו רוצים להבין מתי מתקיים $\im(\alpha) = \im(\tilde{\alpha})$ עבור שתי מסילות, כלומר אנחנו מנסים להבין מתי שתי מסילות הלכה למעשה מייצגות אותו אובייקט ביקום.
\begin{definition}[דיפאומורפיזם]
	פונקציה $\varphi : X \to Y$ תיקרא דיפאומורפיזם אם היא הפיכה, גזירה, והפיכתה גזירה. \\
	דיפאומורפיזם חלק יהיה דיפאומורפיזם כך שהוא גזיר אינסוף פעמים.
\end{definition}
\begin{remark}
	ממשפט ההעתקה ההפוכה נובע שאם דיפאומורפיזם הוא חלק אז גם הפונקציה ההפיכה שלו היא דיפאומורפיזם חלק. \\
	באופן דומה רגולריות של הדיפאומורפיזם הוא תכונה שחלה על הפונקציה ההפיכה גם כן.
\end{remark}
\begin{definition}[רפרמטריזציה]
	בהינתן $\alpha : I \to \AAA^n$ ו־$\varphi : J \to I$ מסילה חלקה רגולרית, $I, J$ קטעים, ו־$\varphi$ דיפאומורפיזם.
	במצב זה נאמר ש־$\tilde{\alpha} = \alpha \circ \varphi$ היא פרמטריזציה שקולה ל־$\alpha$, עוד נאמר ש־$\tilde{\alpha}$ היא רפרמטריזציה של $\alpha$.
\end{definition}
\begin{example}
	נניח ש־$\alpha : I \to \AAA^2$ נתונה על־ידי $\alpha(t) = {(\alpha^1(t), \alpha^2(t))}^T$.
	נגדיר $\varphi : J \to I$ על־ידי $u \mapsto 2 u$.
	אז $\tilde{\alpha} : J \to \AAA^2$ פונקציה המוגדרת על־ידי $\tilde{\alpha}(t) = {(\alpha^1(2t), \alpha^2(2t))}^2$.
	בהתאם נקבל שגם $\tilde{\alpha}'(t) = 2 \alpha'(2t)$.
\end{example}
\begin{remark}
	במקרה הדו־מימדי אם $\psi = \varphi^{-1}$ אז מתקיים, $1 = (\varphi \circ \psi)' = \varphi(\psi(t)) \cdot \psi'(t)$. ולכן בפרט $\psi'$ לא מתאפסת ושומרת על סימן.
\end{remark}
\begin{notation}
	אם $\psi' > 0$ אז נאמר שהיא שומרת על כיוון, אחרת נאמר שהיא משנה כיוון.
\end{notation}
\begin{theorem}[קיום פרמטריזציה לפי אורך]
	יהי עקום פרמטרי $c : I \to \AAA^n$ עקום חלק רגולרי.
	אז קיימת $\tilde{c} : J \to \AAA^n$ פרמטריזציה שקולה ל־$c$ כך ש־$\lVert \tilde{c}' \rVert \equiv 1$.
	פרמטריזציה זו נקראת פרמטריזציה לפי אורך.
\end{theorem}
\begin{proof}
	נסמן $t_0 \in I$ שרירותי.
	נגדיר את $\psi(t) = \int_{t_0}^{t} \lVert c'(u) \rVert\ du$, ונקבל כך את פונקציית האורך מ־$t_0$ בכל נקודה.
	לפי FTC (המשפט היסודי של החשבון האינפיניטסימלי) $\psi$ גזירה ומתקיים ומתקיים $\psi'(t) = \lVert c'(t) \rVert > 0$.
	כלומר $\psi$ רגולרית ומונוטונית, ולכן הפיכה.
	נגדיר $J = \psi(I)$, אז $J$ קטע כתמונה של פונקציה רציפה בקטע.
	יתר־על־כן, $\psi$ היא דיפאומורפיזם, ואף דיפאומורפיזם חלק, נסמן $\varphi = \psi^{-1}$.
	נגדיר $\tilde{c} = c \circ \varphi$, זוהי רפרמטריזציה של $c$, ונותר לבדוק ש־$\lVert \tilde{c}'(t) \rVert = 1$ לכל $t \in J$.
	מהגדרת $\tilde{c}$ מתקיים,
	\[
		\tilde{c}'(t)
		= c'(\varphi(t)) \cdot \varphi'(t)
		= c'(\varphi(t)) \cdot \frac{1}{\psi'(\varphi(t))}
	\]
	ולכן $\lVert \tilde{c}'(t) \rVert = \lVert c'(\varphi(t)) \rVert \cdot \frac{1}{\lVert \psi'(\varphi(t)) \rVert} = 1$.
\end{proof}
\begin{example}
	נגדיר את $c : [0, 2 \pi] \to \AAA^2$ על־ידי $c(t) = {(r \cos t, r \sin t)}^t$ כאשר $r > 0$.
	אז מתקיים $c'(t) = {(- r \sin t, r \cos t)}^t$ ולכן $\lVert c'(t) \rVert = r$ לכל $t$.
	בהתאם גם $L(c) = \int_{0}^{2 \pi} \lVert c'(t) \rVert\ dt = 2 \pi r$ ולכן נגדיר את $J = [0, 2 \pi r]$ ובהתאם נקבל מהמשפט שאם $\varphi(t) = \frac{t}{r}$,
	אז נוכל להגדיר את $\tilde{c}(s) = (c \circ \varphi)(s) = c(\varphi(s)) = c(\frac{s}{r})$, כלומר $\tilde{c}(t) = {(r \cos(\frac{t}{r}), r \sin(\frac{t}{r}))}^t$.
\end{example}

\subsection{עקמומיות}
עתה נרצה לדון בהבדל שבין אובייקטים במישור לבין אובייקטים בעקומות.
אם $c : I \to \AAA^2$ לדוגמה רגולרית, אז נשים לב שנוכל להגדיר את הישר המשיק בנקודה, $l(t) = c(t_0) + t c'(t_0)$ לכל $t \in \RR$, ולקבל מישור אפיני.
אם $c$ פרמטריזציה לפי אורך, אז בפרט נקבל שהווקטור המגדיר את המשיק הוא נורמלי.
נרצה למצוא את הווקטור האורתוגונלי שלו, במטרה להבין את ההתנהגות של הפונקציה ביחס לשני הווקטורים הללו.
בהתאם נוכל להבין כמה עקום מתעקם בהתאם לקשר בין הנגזרת השנייה לבין האורתוגונלי לנגזרת.
\begin{definition}[בסיס אורתוגונלי של נגזרת]
	תהי מסילה $c : I \to \AAA^2$ רגולרית לפי אורך ותהי $t_0 \in I$. \\
	נסמן $v(t) = c'(t)$ וכן $n(t) = \begin{pmatrix} 0 & -1 \\ 1 & 0 \end{pmatrix} c'(t)$, אז $(v(t), n(t))$ הוא בסיס אורתוגונלי של הנגזרת בנקודה.
	אז $\lVert c'(t) \rVert = 1$ וכן $c'(t) \cdot c'(t) = 1$, אז $c''(t) \cdot c'(t) + c'(t) \cdot c''(t) \equiv 0$, ונסיק ש־$2 c''(t) c'(t) \equiv 0$ אם ורק אם $c''(t) \perp c'(t)$.
	אז נגדיר $k : I \to \RR$ על־ידי $c''(t) = k(t) n(t)$, פונקציה זו נקראת העקמומיות (המכוונת) של $c$.
\end{definition}

\section{שיעור 8 --- 11.11.2025}
\subsection{עקמומיות --- המשך}
נדון במסילה $c(t) - {(t, 9 - t^2)}^t$ עבור $t \in [-3, 3]$.
אז $c'(t) = {(1, -2t)}^t$ ובהתאם $\lVert c'(t) \rVert = \sqrt{1 + 4t^2}$ ובאופן דומה $c''(t) = {(0, 2)}^t$.
אבל $c$ היא לא לפי אורך ולכן לא נוכל להשתמש בה בפשטות.

\section{שיעור 9 --- 17.11.2025}
\subsection{עקמומיות}

\section{שיעור 10 --- 18.11.2025}
\begin{theorem}
	תהי $k : I \to \RR$ פונקציה גזירה $r \ge 0$ פעמים.
	בהינתן נקודה $s_0 \in I$ ונקודה $P_0 = (P^1, P^2) \in \EE^2$ ו־$v_0 = {(v_0^1, v_0^2)}^t \in \RR^2$ כך ש־$\lVert v_0 \rVert = 1$,
	אז קיימת מסילה $c : I \to \EE^2$ לפי אורך עם פונקציית עקמומיות הנתונה על־ידי $k$.
	אם $c(s_0) = P_0, c'(s_0) = v_0$ אז $c$ גם יחידה.
	יתר־כל־כן,
	\[
		c(s)
		= \begin{pmatrix}
			P^1 + \int_{s_0}^{s} \left(\cos\left( \int_{s_0}^{t} k(u)\ du + \theta_0\right)\right)\ dt \\
			P^2 + \int_{s_0}^{s} \left(\sin\left( \int_{s_0}^{t} k(u)\ du + \theta_0\right)\right)\ dt
		\end{pmatrix}
	\]
	כאשר $0 \le \theta_0 < 2 \pi$.
\end{theorem}
\begin{proof}
	נתחיל בהוכחת הקיום.
	נגדיר $c(s_0) = P_0$ ונגדיר את הנגזרת בהתאם לדרישות המופיעות במשפט.
	נשים לב ש־$\lVert c'(s) \rVert = 1$ לכל $s \in I$ ו־$c''(s) = l(s) n(s)$ ישירות מחישוב.

	נניח כי גם $d : I \to \EE^2$ עומדת בתנאים, ונתבונן בפונקציה $f(s) = (c^1)'(s) - (d^1)'(s)$ וב־$g(s) = (c^2)'(s) - (d^2)'(s)$.
	אז נקבל ש־$f'(s) = -k(s) g(s)$ ו־$g'(s) = k(s) f(s)$, ולכן $\frac{1}{2} (f^2 + g^2)(s) = 0$ ולכן $f^2 + g^2$ פונקציה קבועה.
	אבל $f(s_0) = g(s_0) = 0$ ולכן גם $f = 0 = g$ בלבד.
\end{proof}
\begin{theorem}
	תהי $c : I \to \EE^2$ מסילה רגולרית לפי אורך.
	אז קיימת $\theta : I \to \RR$ כך שמתקיים $c'(s) = {(\cos \theta(s), \sin \theta(s))}^t$, כך ש־$\theta$ יחידה עד כדי הזזות ב־$2 \pi k$.
\end{theorem}

\section{שיעור 11 --- 24.11.2025}
\subsection{תרגילים}

\section{שיעור 12 --- 25.11.2025}
\subsection{מרחבים דואליים}
\begin{definition}[מרחב דואלי]
	יהי $V$ מרחב וקטורי ממימד $n \in \NN$ מעל שדה $\FF$.
	נגדיר,
	\[
		V^{\vee} = \hom_\FF(V, \FF)
	\]
	ונקרא ל־$V^\vee$ המרחב הדואלי של $V$.
	לאיבר במרחב הדואלי נקרא תבנית לינארית (linear form) או פונקציונל (functional).
\end{definition}
\begin{example}
	$V^\vee = {(\FF_{\operatorname{col}^n})}^\vee = \{ M_{n \times 1}(\FF) \}$, ההעתקות הלינאריות ממימד $n$ למימד $1$.
\end{example}
\begin{proposition}
	במקרה זה אם $l \in V^\vee$ אז קיים וקטור $a = {(a_1, \ldots, a_n)}^t \in V$ כך שמתקיים $l(x) = a^t x$ ונוכל לסמן $l = l_a$.
\end{proposition}
\begin{proof}
	נסמן $a_j = l(e_j)$ עבור $1 \le j \le n$.
	נגדיר $a = {(a_j)}_{i = 1}^n$, אז,
	\[
		l(x)
		= l(e_1 x^1 + \cdots + e_n x^n)
		= x^1 l(e_1) + \cdots + x^n l(e_n)
		= x^1 a_1 + \cdots + x^n a_n
		= a^t x
		= l_a(x)
	\]
	ומצאנו שאכן $l = l_a$.
\end{proof}
\begin{remark}
	בהתאם נקבל $\FF_{\operatorname{row}}^n \simeq {(\FF_{\operatorname{col}}^n)}^\vee$.
	כלומר באיזשהו מובן מרחב דואלי הוא גם דואלי במובן הסימון.
\end{remark}
\begin{example}
	יהי $V$ ו־$\Bb = (b_1, \ldots, b_n) \subseteq V$ בסיס סדור.
	אם $v = \Bb x$ אז נוכל להגדיר $l^j(v) = x_j$, כאשר $l^j$ היא תבנית לכל $j$.
\end{example}
\begin{example}
	עבור $S \ne 0$ כלשהי נגדיר את הקבוצה,
	\[
		\Ff
		= \Ff(S, \FF)
		= \{ f : S \to \FF \}
	\]
	עבור $f, g \in \Ff$ נגדיר $(f + g)(s) = f(s) + g(s)$ ובאופן דומה $(k \cdot f)(s) = k \cdot g(s)$ עבור $k \in \FF$.
	נקבל אם כך ש־$(\Ff, +, \cdot)$ מרחב לינארי מעל $\FF$.
	
	עבור $s_0 \in S$ נגדיר,
	\[
		\eval_{s_0} : \Ff(S) \to \FF,
		\quad
		f \mapsto f(s_0)
	\]
	אז $\eval_{s_0} \in \Ff^\vee$.
\end{example}
\begin{example}
	נניח ש־$S = I \subseteq \RR$ ונניח ש־$V = C^1(I)$,
	אז $f \mapsto f'(s_0)$ אף היא פונקציונל לינארי.

	באופן דומה אם $V = \Rr([a, b])$ מרחב הפונקציות האינטגרביליות רימן, נוכל להגדיר גם את $f \mapsto \int_a^b f\ dx$, גם זו תבנית.
\end{example}

\section{שיעור 13 --- 1.12.2025}
\subsection{מרחבים דואליים}
נניח ש־$V$ מרחב וקטורי ממימד סופי מעל $\FF$.
תורה זו מוגדרת גם עבור מקרים של מימד לא סופי, אבל ישנם מספר הבדלים שנציין אך לא נחקור.
הגדרנו את $V^\vee = \hom(V / \FF)$ כמרחב הדואלי של $V$.
\begin{definition}[בסיס דואלי]
	יהי $\Bb = (b_1, \ldots, b_n)$ בסיס של $V$.
	נגדיר $b^i \in V^\vee$ לכל $i \le n$ על־ידי $b^i(b_j) = \delta_{ij}$ לכל $j \le n$.
	נקרא אפריורית לקבוצה $\Bb^\vee = (b^1, \ldots, b^n)$ בסיס דואלי של $V$.
\end{definition}
\begin{remark}
	אם $v \in V$ אז מתקיים $v = \Bb x$ עבור $x \in \FF_{\text{col}}^n$.
	בהתאם $b^i(v) = x^i$.
\end{remark}
\begin{theorem}[קיום בסיס דואלי]
	$\Bb^\vee$ בסיס של $V^\vee$.
\end{theorem}
\begin{proof}
	נתחיל באי־תלות.
	נניח ש־$a_1, \ldots, a_n \in \FF$ כך ש־$a_1 b^1 + \cdots + a_n b^n = 0_{V^\vee}$, אז,
	\[
		0_\FF
		= 0_{V^\vee}(b_j)
		= \left(\sum_{i = 1}^n a_i b^i\right)(b_j)
		= \sum_{i = 1}^n a_i b^i(b_j)
		= \sum_{i = 1}^n a_i \delta_{ij}
		= a_j
	\]
	וקיבלנו אי־תלות.

	נעבור להוכיח פרישה.
	עבור $l \in V^\vee$ תבנית יהיו $a_i = l(b_i)$ לכל $i \le n$.
	נראה ש־$l = a_1 b^1 + \cdots + a_n b^n$.
	מספיק לבדוק ששתי הפונקציות מתלכדות על איברי הבסיס $\Bb$, זאת ישירות מלינאריות.
	\[
		\left(\sum_{i = 1}^n a_i b^i\right)(b_j)
		= a_j
		= l(b_j)
	\]
	וקיבלנו שאכן הבסיס $\Bb^\vee$ פורש את $V^\vee$.
\end{proof}
נבחין כי נוכל ליצור צימוד $V^\vee \times V \to \FF$ על־ידי $(l, v) \mapsto l(v)$ ונסמן את הפעולה הזאת באופן דומה למכפלה פנימית על־ידי $\langle l, v \rangle$.
נקבל שמתקיים $b_1 b^1(v) + \cdots + b_n b^n(v) = v = \sum_{i = 1}^n b_i \langle b^i, v \rangle$ וכן $l = b^1 l(b_1) + \cdots + b^n l(b_n) = \sum_{i = 1}^n b^i \langle l, b_i \rangle$.
\begin{theorem}[שקילות לאיפוס במרחב דואלי]
	לכל וקטור $v \in V$ מתקיים $v = 0_V \iff \forall l \in V^\vee,\ l(v) = 0_\FF$. \\
	לכל $l \in V^\vee$ מתקיים $l = 0_{V^\vee} \iff \forall v \in V,\ l(v) = 0_\FF$.
\end{theorem}
\begin{theorem}[קיום בסיס לבסיס דואלי]
	לכל $(b^1, \ldots, b^n)$ בסיס של $V^\vee$ קיים $\Bb = (b_1, \ldots, b_n)$ בסיס של $V$ כך שמתקיים,
	\[
		\Bb^\vee = (b^1, \ldots, b^n)
	\]
\end{theorem}
\begin{proof}
	תהי תבנית $\varphi : V \to \FF_{\text{col}}^n$ המוגדרת על־ידי $\varphi(v) = {(b^1(v), \ldots, b^n(v))}^t$.
	נבחין כי $v \in \ker \varphi$ אם ורק אם $b^i(v) = 0$ לכל $i \le n$, אבל בהתאם הטענה מתקיימת רק כאשר $v = 0$, לכן $\dim \ker \varphi = 0$ ונובע ש־$\varphi$ איזומורפיזם לינארי.
	לכן אם $(b_1, \ldots, b_n)$ תמונתה $(e_1, \ldots, e_n)$ כך שהאחרון הוא הבסיס הסטנדרטי של $\FF^n$ אז $b^i(b_j) = \delta_{ij}$ בדיוק כפי שרצינו.
\end{proof}
\begin{definition}[מאפסים]
	לכל $S \subseteq V$ נגדיר,
	\[
		S^0 = \{ l \in V^\vee \mid \forall s \in S,\ l(s) = 0 \} \subseteq V^\vee
	\]
	ונקרא ל־$S^0$ המאפס של $S$.
\end{definition}
\begin{theorem}[תכונות מאפס]
	לכל $S, T \subseteq V$ מתקיים:
	\begin{enumerate}
		\item $S^0 \subseteq V^\vee$, כלומר זהו אופרטור למרחב הדואלי
		\item $S^0 = \Sp^0(S)$, ולכן מספיק לדבר על קבוצה פורשת במקום על תתי־מרחבים
		\item $S \subseteq T \implies T^0 \subseteq S^0$
	\end{enumerate}
\end{theorem}
\begin{theorem}
	יהי $U \le V$, אז $\dim U + \dim U^0 = \dim V$.
\end{theorem}
\begin{proof}
	יהי $(b_1, \ldots, b_n)$ בסיס של $V$ כך שגם $(b_1, \ldots, b_k)$ בסיס של $U$, כאשר $k \le n$. \\
	נתבונן ב־$B^\vee = (b^1, \ldots, b^n)$, ונטען ש־$(b^{k + 1}, \ldots, b^n)$ בסיס של $U^0$.
	יהי $l \in V^0$ אז $l = b^1 l(b_1) + \cdots + b^n l(b_n)$ אבל $l(b_i) = 0$ לכל $i \le k$ ונובע ש־$l \in \Sp(b^{k + 1}, \ldots, b^n)$.
	נובע אם כך ש־$\dim U + \dim U^0 = \dim V$.
\end{proof}
\begin{definition}[קבוצת האפסים]
	תהי $L \subseteq V^\vee$, אז נגדיר,
	\[
		L_0 = \{ v \in V \mid \forall l \in L,\ l(v) = 0 \}
	\]
	קבוצה זו תיקרא קבוצת האפסים של $L$.
\end{definition}
\begin{theorem}
	לכל $L, M \subseteq V^\vee$ מתקיים,
	\begin{enumerate}
		\item $L_0 \le V$
		\item $L_0 = \Sp_0(L)$
		\item $L \subseteq M \implies M_0 \subseteq L_0$
	\end{enumerate}
\end{theorem}
\begin{theorem}
	יהי $W \subseteq V^\vee$, אז מתקיים $\dim W + \dim W_0 = \dim V$.
\end{theorem}
אם $S \subseteq V$ אז מהו ${(S^0)}_0$?
נקבל $S \subseteq {(S^0)}_0 = \Sp S$.
בהתאם נקבל שמתקיים $L \subseteq {(L_0)}^0 = \Sp(L)$.
\begin{definition}[העתקה דואלית]
	תהי $f : V \to W$ העתקה לינארית בין מרחבים, ותהי $l \in W^\vee$ תבנית.
	אז נגדיר את ההעתקה הדואלית של $f$ להיות $f^\vee = l \circ f$, כאשר בהתאם $f^\vee : W^\vee \to V^\vee$. \\
	בשפת סימון מכפלה פנימית נקבל $\langle f^\vee(l), v \rangle = \langle l, f(v) \rangle$.
\end{definition}

\section{שיעור 14 --- 2.12.2025}
\subsection{מאפסים}
נמשיך לדון במאפסים ותכונותיהם.
\begin{corollary}
	אם $W_1, W_2 \le V$ אז מתקיים $W_1 = W_2 \iff W_1^0 = W_2^0$.
\end{corollary}
כלומר אנו יכולים לדון בתתי־מרחבים על־ידי שימוש במאפסים, ישנה שקילות שמאפשרת לנו להרחיב את הדיון שלנו גם בתתי־מרחבים באופן כללי.
\subsection{העתקות דואליות}
נמשיך את הדיון שלנו על העתקות אלה.
\begin{theorem}
	יהיו $V, W$ מרחבים לינאריים כך ש־$\Bb = (b_1, \ldots, b_n), \Dd = (d_1, \ldots, d_m)$ בסיסים בהתאמה ו־$\Bb^\vee, \Dd^\vee$ הבסיסים הדואליים בהתאמה.
	אם $f : V \to W$ לינארית ו־$A = [f]$ בבסיסים $\Bb, \Dd$ בהתאמה אז $[f^\vee] = A^t$ בבסיסים $\Dd^\vee, \Bb^\vee$.
\end{theorem}
יהי $V$ מרחב אוקלידי, כלומר מרחב לינארי ממימד סופי מעל $\RR$ עם מכפלה פנימית $\langle \cdot, \cdot \rangle : V \times V \to \RR$.
לכל $v \in V$ קיימת העתקה,
\[
	l_v : V \to \RR,
	\quad
	l_v(w) = \langle v, w \rangle
\]
כלומר $l_v = \langle v, \cdot \rangle$.
נשים לב ש־$l_v \in V^\vee$.
המשמעות היא שקיים צימוד מלא בין $V$ ל־$V^\vee$, על־ידי $v \mapsto l_v$, ולכן $V \simeq V^\vee$. \\
בהתאם לדוגמה ב־$\RR^3$ אם נגדיר $l(v) = \det(x\ y\ v)$ עבור $x, y \in \RR_{\text{col}}^3$ קבועים, נקבל העתקה לינארית ולכן לכל $z \in \RR^3$ קיים $x \wedge y \in \RR^3$ יחיד כך ש־$l(z) = 0$.
זוהי למעשה המכפלה החיצונית, המכפלה הווקטורית.

\section{שיעור 15 --- 8.12.2025}
\subsection{תבניות בי־לינאריות}
יהי $V$ מרחב וקטורי מעל שדה $\FF$.
\begin{definition}[תבנית בי־לינארית]
	תבנית בי־לינארית על $V$ היא פונקציה $g : V \times V \to \FF$ כך שהפונקציות $x \mapsto g(v, x), x \mapsto g(x, v)$ עבור $v \in V$ כאשר הן $V \to \FF$ הן לינאריות.
\end{definition}
\begin{example}
	אם $l^1, l^2 \in \hom(V, \FF)$ אז גם $(v, w) \mapsto l_{1}(v) l^2(w)$ אף היא תבנית בי־לינארית.
\end{example}
\begin{example}
	עבור $V = M_{n \times m}(\FF)$, ו־$A \in M_m(\FF)$ נגדיר $g_A(X, Y) = \tr(X^t A Y)$, זוהי פונקציה משמרת לינאריות מורכבת על־ידי פונקציונל לינארי, ולכן תבנית בי־לינארית.
\end{example}
\begin{example}
	כאשר $V = \FF_{\operatorname{col}}^n$ ו־$A \in M_n(\FF)$ אז $g_A(x, y) = x^t A y$, אז זוהי תבנית בי־לינארית כמקרה פרטי של הדוגמה הקודמת.
\end{example}
במקרה שבו $n = 2$ ו־$A = \diag(1, 1) = \id$ אז נקבל $g_E(x, y) = x^t y = x \cdot_E y = x^1 y^1 + x^2 y^2$, זוהי המכפלה הפנימית הסטנדרטית כאשר $\FF = \RR$ או $\FF = \CC$.
אם לעומת זאת $A = \diag(1, -1)$ נקבל $g_L(x, y) = x \cdot_L y = x^1 y^1 - x^2 y^2$, זוהי מכפלה פנימית חשובה בפיזיקה.
ישנו גם המקרה $A = (\begin{smallmatrix} 0 & 1 \\ 1 & 0 \end{smallmatrix})$ אז נקבל $g_H(x, y) = x^1 y^2 + x^2 y^1$, נקראת התבנית ההיפרבולית.
אם נגדיר $A = (\begin{smallmatrix} 0 & 1 \\ -1 & 0 \end{smallmatrix})$ אז נקבל $g_S(x, y) = x^1 y^2 - x^2 y^1 = \det(x\ y)$, כלומר זוהי דטרמיננטה של מטריצה ריבועית כאשר עמודותיה הן הווקטורים $x, y$.

\subsection{מטריצת גרם}
או באנגלית מטריצת Gram.
אם $V$ מרחב וקטורי ממימד סופי מעל $\FF$ ו־$\Bb = (b_1, \ldots, b_n)$ בסיס סדור, $g : V \times V \to \FF$ תבנית בי־לינארית.
\begin{definition}[מטריצת גרם]
	אם $g_{ij} = g(b_i, b_j)$ אז המטריצה $G = [g_{ij}] \in M_n(\FF)$ נקראת מטריצת גרם של $g$ בבסיס $\Bb$.
\end{definition}
אם $v \in V$ אז נוכל לכתוב $v = \Bb x$ עבור וקטור קורדינטות $x$, באופן דומה אם $w \in V$ אז נוכל לסמן $w = \Bb y$.
במצב זה,
\[
	g(v, w)
	= g\left(\sum_{i = 1}^n b_i x^i, \sum_{j = 1}^n b_j y^j\right)
	= \sum_{i = 1}^n \sum_{j = 1}^n x^i y^j g(b_i, b_j)
	= x^t G y
\]
\begin{proposition}
	אם $g$ תבנית בי־לינארית ו־$G$ מטריצת גרם בבסיס $\Bb$ שלה, אז לכל $v = \Bb x, w = \Bb y$ מתקיים $g(v, w) = x^t G y$ ובהתאם $G$ מגדירה ביחידות את $g$.
\end{proposition}
עתה נרצה לשאול את השאלה המתבקשת מה קורה ל־$G$ כאשר אנו משנים את הבסיס.
תהי $P \in GL_n(\FF)$, היא תהיה מטריצת מעבר $\Bb P = \Bb'$ כלומר לכל וקטור קורדינטות $x$ נקבל $x = P x'$.
אז במקרה זה $g(v, w) = x^t G y = {(P x')}^t G (P y') = {(x')}^t (P^t G P) y'$.
\begin{proposition}
	אם $\Bb, \Bb'$ בסיסים כך ש־$\Bb P = \Bb'$ מטריצת מעבר, וכן $g$ תבנית בי־לינארית, אז אם $G$ מטריצת גרם מעל $\Bb$ של $g$, אז $P^t G P$ היא מטריצת גרם בבסיס $\Bb'$.
\end{proposition}
\begin{definition}[מטריצות חופפות]
	נאמר ששתי מטריצות $A, B \in M_n(\FF)$ הן חופפות (congruent) אם קיימת $P \in GL_n(\FF)$ כך ש־$B = P^t A P$.
\end{definition}
\begin{exercise}
	הוכיחו כי זהו יחס שקילות.
\end{exercise}
אם $g$ תבנית בי־לינארית על $V$ אז נגדיר $g^t(v, w) = g(w, v)$.
\begin{definition}[תבנית בי־לינארית סימטרית]
	$g$ נקראת סימטרית אם $g = g^t$ ואנטי־סימטרית כאשר $g = - g^t$.
\end{definition}
\begin{exercise}
	$g$ היא סימטרית אם ורק אם $G$ סימטרית וכן ואנטי־סימטרית אם ורק אם $G$ אנטי־סימטרית.
\end{exercise}
\begin{definition}[תבנית אורתוגונלית]
	תהיינה $V$ מרחב וקטורי ו־$g : V \times V \to \FF$ תבנית בי־לינארית סימטרית.
	נסמן $u \perp v$ כאשר $g(v, w) = 0$, ונאמר ש־$u, v$ אורתוגונליות ביחס ל־$g$. \\
	אם $U, W \subseteq V$ אז נסמן $U \perp W$ כאשר $\forall u \in U, w \in W,\ u \perp w$. \\
	נסמן אף $U^\perp = \{ w \in V \mid \forall u \in V,\ u \perp w \}$
\end{definition}
\begin{definition}[גרעין של תבנית בי־לינארית]
	הגרעין של $g$ מוגדר על־ידי $V^\perp = \ker g$.
\end{definition}
\begin{definition}[ניוון]
	נאמר ש־$g$ לא מנוונת כאשר $\ker g = \{ 0 \}$. \\
	$U \le V$ נקרא לא מנוון כאשר $g |_{U \times U}$ לא מנוונת, אחרת $U$ נקרא איזוטרופי.
\end{definition}
נניח ש־$V = \RR^2$ ו־$g = g_L$, כלומר $g(x, y) = x^1 y^1 - x^2 y^2$.
במקרה זה נקבל $g(x, x) = {(x^1)}^2 - {(x^2)}^2$, ובהתאם $g(x, x) = 0$ כאשר $x_1 = x_2$, לכן $U = \{ x \mid |x_1 - x_2| = 0 \}$ מנוונת (מאוד).
אם נבחן את $U = \{ x \mid x_1 = 0 \lor x_2 = 0 \}$ אז נקבל שרק $g(0) = 0$ ולכן זוהי קבוצה לא מנוונת.

\subsection{תבנית ריבועית}
נניח ש־$V$ ו־$g$ תבנית בי־לינארית סימטרית על $V$.
\begin{definition}[תבנית ריבועית]
	נגדיר את התבנית הריבועית $q : V \to \FF$ על־ידי $q(v) = g(v, v)$.
	$q$ זו נקראת התבנית הריבועית המתאימה ל־$g$.
\end{definition}
נשים לב שאם $k \in \FF, v \in V$ אז $q(k v) = k^2 q(v)$.
אם $G = {[g]}_\Bb = [ g_{ij} ]$, אז מתקיים $q(v) = x^t G x = \sum_{1 \le i, j \le n} x^i x^j g_{ij}$.

בדוגמות שראינו קודם נקבל $q_E(x) = {(x^1)}^2 + {(x^2)}^2$.
נקבל כך גם $q_L(x) = {(x^1)}^2 - {(x^2)}^2$, ולבסוף גם $q_H(x) = 2 x^1 x^2$.
בפרט אנו רואים שבמקרה של $g_H$ ישנם וקטורים שמקבלים $0$ ב־$q_H$, כלומר היא באמת לא מתנהגת כמו נורמה.

\section{שיעור 16 --- 9.12.2025}
\subsection{לכסון תבניות בי־לינאריות}
נניח ש־$V$ מרחב וקטורי סוף־מימדי מעל השדה $\FF$, ו־$g : V \times V \to \FF$ תבנית בי־לינארית סימטרית.
נרצה לעסוק בשאלת הלכסון בהקשר של $g$, כלומר האם יש בסיס כך ש־$G$ המטריצה המייצגת של $g$ היא אלכסונית.
\begin{theorem}[נוסחת הפולריזציה]
	אם $\Char \FF \ne 2$ מתקיים,
	\[
		g(v, w) = \frac{1}{2} (q(v + w) - q(v) - q(w))
	\]
\end{theorem}
\begin{exercise}
	הוכיחו משפט זה.
\end{exercise}
\begin{definition}[בסיס אורתוגונלי]
	בסיס $\Bb = (b_1, \ldots, b_n)$ נקרא אורתוגונלי כאשר $g(b_i, b_j) = 0$ לכל $1 \le i \ne j \le n$.
\end{definition}
\begin{theorem}[קיום בסיס אורתוגונלי]
	לכל $V$ ו־$g$ סימטרית וכאשר $\Char \FF \ne 2$ קיים בסיס אורתוגונלי.
\end{theorem}
\begin{proof}
	אם $g = 0$ אז כל בסיס הוא אורתוגונלי.
	אחרת נוכיח באינדוקציה על מימד $V$, כלומר נניח נכונות עבור עבור מרחבים ממימד קטן מ־$n$ ונסיק את נכונות הטענה עבור $n$ כאשר $g \ne 0$.
	אם $g \ne 0$ אז קיים $b \in V$ כך ש־$0 \ne q(b) = g(b, b)$, טענה זו נכונה שכן אם $g \ne 0$ אז קיימים $v, w$ כך ש־$g(v, w) \ne 0$, אז גם $g(w, v) = g(v, w)$ ונוכל לבחור $g(v + w, v + w)$.

	יהי $U = \Sp\{ b \}$ ונתבונן ב־$U^\perp$ ונטען שמתקיים $V = U \bigoplus U^\perp$.
	לכל וקטור $v \in V$ יהי $v_0 = v - \frac{g(b, v)}{g(b, b)} b$ ונשים לב כי $v_0 \in U^\perp$, נבדוק,
	\[
		g(b, v_0)
		= g(b, v) - g(b, b) \frac{g(b, v)}{g(b, b)}
		= 0
	\]
	ולכן אכן מצאנו שמתקיים $v_0 \in U^\perp$.
	נוכל אם כן לכתוב $v = v_0 + \frac{g(b, v)}{g(b, b)} b$ ולכן $V = U + U^\perp$, נותר להראות $U \cap U^\perp = \emptyset$.
	נניח ש־$v \in U \cap U^\perp$, אז $v = k b$ לאיזשהו $k \in \FF$.
	בהתאם נקבל $g(b, v) = g(b, k b) = k g(b, b)$, אבל $v \in U^\perp$ ולכן גם $g(b, v) = 0$, כלומר $k = 0$ ולכן $v = 0$.

	נראה ש־$\dim U^\perp = \dim V - 1$.
	לפי הנחת האינדוקציה לצמצום של $g$ על $U^\perp$ קיים $(b_1, \ldots, b_n)$ בסיס אורתוגונלי, אם $b_1 = b$ הבסיס $\Bb = (b_1, \ldots, b_n)$ מקיים את טענת המשפט.
\end{proof}
\begin{corollary}
	אם $D = {[ g ]}_\Bb = \diag(d_1, \ldots, d_n)$ עבור $d_i \in \FF$ וכן $v = \Bb x, u = \Bb y$ אז מתקיים,
	\[
		g(v, u)
		= x^t D y
		= x^1 d_1 y^1 + \cdots + x^n d_n y^n
	\]
	בהתאם גם $q(v) = {(x^1)}^2 d_1 + \cdots + {(x^n)}^2 d_n$.
\end{corollary}
מצאנו שלכל $g$ יש בסיס אורתוגונלי, כלומר שגם $g(b_i, b_j) = 0$ לכל $i \ne j$, אבל במצב זה מטריצת גרם בהכרח אלכסונית, ננסח זאת כמסקנה.
\begin{corollary}
	כאשר $\Char \FF \ne 2$ לכל $A \in M_n(\FF)$ קיימות $P \in GL_n(\FF)$ ו־$D \in M_n(\FF)$ אלכסונית כך ש־$D = P^t A P$.
\end{corollary}
נעבור לחיפוש אחר דרכים למציאת בסיס מלכסן כזה.
\begin{definition}[בסיס אורתונורמלי]
	אם $\FF = \RR, \CC$ אז בסיס $\Bb = (b_1, \ldots, b_n)$ נקרא אורתוגונלי אם $g(b_i, b_j) \in \{ 1, 0, -1 \}$ לכל $1 \le i, j \le n$.
\end{definition}
\begin{example}
	כאשר $\FF = \RR$ נגדיר,
	\[
		A
		= \begin{pmatrix}
			-1 & -2 & 4 \\
			-2 & 2 & 1 \\
			4 & 2 & -1
		\end{pmatrix}
	\]
	ולכן נקבל $q({(x, y, z)}^t) = -x^2 - 4xy + 2y^2 + 8xz + 4yz + z^2$, ועתה נרצה לבצע השלמה לריבוע.
	למעשה יכולנו לחשב מפולינום כזה בדיוק את המטריצה המתאימה לו, כלומר יש התאמה חד־חד ערכית ועל בין מטריצות ותבניות ריבועיות (ולכן גם תבניות בי־לינאריות).
	אז נחשב ונקבל $q(x, y, z) = -{(x + 2y + 4z)}^2 + 6y^2 + 15z^2 - 12yz = -{(x + 2y + 4z)}^2 + 6 {(y - z)}^2 + 9z^2$, כל זאת על־ידי תהליך של השלמה לריבוע.
	נסמן $u = x + 2y + 4z, v = y - z, w = z$ ונקבל $q(u, v, w) = -u^2 + 6v^2 + 9w^2$, בהתאם,
	\[
		\begin{pmatrix}
			u \\ v \\ w
		\end{pmatrix}
		= \begin{pmatrix}
			1 & 2 & -4 \\
			0 & 1 & -1 \\
			0 & 0 & 1
		\end{pmatrix}
		\begin{pmatrix}
			x \\ y \\ z
		\end{pmatrix}
	\]
	כלומר אם עבדנו במקור בבסיס $\Bb$ עתה מצאנו בסיס מלכסן $\Bb'$, וכן ידוע לנו ש־$\Bb' = Q \Bb$ עבור המטריצה שחישבנו זה עתה, אז אם $P Q = I$ אז $P$ היא מטריצת המעבר המלכסנת.
\end{example}
אילו לא היו לנו ריבועים בביטוי המקורי, אז היינו יכולים להשתמש במעבר מהצורה $x = u + v, y = u - v$ ולקבל בסיס בו יש ריבועים וכן שהוא ניתן לשינוי כפעולה הפיכה.
אנו יודעים שכפל של מטריצה במטריצה אלמנטרית מימין מאפשרת לנו לבצע פעולות שורה, אם נכפול במטריצה המשוכפלת משמאל נקבל את אותה הפעולה אבל על העמודות, ובהתאם אין זה מפתיע אותנו שמתקבל בדיוק $D = P^t A P$.
נוכל להשתמש בפעולות שעשינו במהלך מציאת הבסיס המלכסן והן מרכיבות הלכה למעשה את $P$.

\section{שיעור 17 --- 15.12.2025}
\subsection{משטחים חלקים}
נעבור לדבר על משטחים חלקים, כלומר אובייקטים במרחב שעבור כל נקודה שלהם יש סביבה של הנקודה המתנהגת כמו מרחב אפיני.
עד כה דיברנו על מסילות, הן באיזשהו מובן עקום ממימד $1$, ובאמת הגדרנו אותן כחלקות ולכן כמתנהגות באופן לינארי בסביבות מאוד קטנות.
בהתאם נגדיר,
\begin{definition}[משטח חלק]
	$S \subseteq \EE^3$ תיקרא משטח חלק (או רגולרי) אם לכל $P \in S$, קיימות $V \subseteq \EE^3, U \subseteq \EE^2$ פתוחות כך שקיימת $f : U \to S \cap V$ חלקה המקיימת,
	\begin{enumerate}
		\item $f$ הומיאומורפיזם
		\item $f$ חד־חד ערכית לכל $u \in U$
	\end{enumerate}
\end{definition}
\begin{example}
	מישורים אפיניים, כלומר אם $U = \RR^2, V = \RR^3$ וכן $S = \{ P + u \mid u \in U \}$.
\end{example}
\begin{example}
	גרף של פונקציה,
	אם $U \subseteq \EE^2$ פתוחה ו־$f : U \to \EE^1$ חלקה אז $S = \{ (x, y, z) \in \EE^3 \mid z = f(x, y) \}$.
\end{example}

\section{שיעור 18 --- 16.12.2025}
\subsection{משטחים רגולריים}
\begin{theorem}[שקילות למשטח רגולרי]
	$S \subseteq \EE^3$ תיקרא משטח רגולרי אם מתקיימים אחד מבין התנאים הבאים:
	\begin{enumerate}
		\item לכל $P \in S$ קיימת $P \in V \subseteq \EE^2$ סביבה פתוחה ו־$V \subseteq \EE^2$ פתוחה כך ש־$f : U \to \EE^3$ המקיימת,
			\begin{enumerate}
				\item $f(U) \cong V \cap S$, כלומר $f$ היא הומיאומורפיזם
				\item $\forall u \in U,\ \deg D f |_u = 2$ לכל $x \in \dom f$
			\end{enumerate}
		\item לכל $P \in S$ קיימות $P \in V \subseteq \EE^3$ סביבה פתוחה כך שמתקיים,
			\[
				S = \{ (x, y, z) \in \EE^3 \mid F(x, y, z) = 0 \}
			\]
			עבור פונקציה המקיימת,
			\begin{enumerate}
				\item $F$ חלקה ב־$V$
			\end{enumerate}
		\item לכל $P \in S$ קיימת $P \in V \subseteq \EE^3$ סביבה פתוחה וקיימת $U \subseteq \EE^3$ קבוצה פתוחה כך ש־$g : U \to S \cap V$ חלקה כך שלכל $(x, y, z) \in S \cap V$ מתקיים $z = g(x, y)$
	\end{enumerate}
\end{theorem}
\begin{example}
	עבור $S = S(0, 1)$ ספירת היחידה נבין שנוכל לכסות את הקבוצה על־ידי $f(x, y) = \pm \sqrt{x^2 + y^2}$ ולכן זהו משטח רגולרי.
	מהצד השני נוכל גם להגדיר את $F(x, y, z) = x^2 + y^2 + z^2 - 1$ ולקבל מהאפיון השקול למשטח שהספירה היא אכן משטח.
\end{example}

\section{שיעור 19 --- 29.12.2025}
\subsection{מישור משיק}
נבחן את השלשה הסדורה $(U, f, V)$ אשר מהווה פרמטריזציה מקומית של משטח רגולרי, אז $U \subseteq \EE^2$ וכן $V \subseteq \EE^3$.
אם נבחן את $T_u V = \{ u \} \times \RR^2$ אז נקבל קבוצה ב־$\EE^3$, זהו המישור המשיק של הנקודה $u \in U$.
נוכל באופן שקול להסתכל על $T_P \EE^3$, ואם $f : U \to S \cap V$ כך ש־$f(u) = P$, אז נרצה לבחון את $T_P S = D f |_u(T_u V)$, אבל זוהי הצגה פרמטרית ולא גאומטרית.
\begin{definition}[מישור משיק]
	אם $S \subseteq \EE^3$ משטח רגולרי, אז נגדיר את המישור המשיק לנקודה $P \in S$ על־ידי $T_P S = D f |_u (T_u V)$ עבור $P \in U \subseteq \EE^2$ פתוחה, $V \subseteq \EE^3$ פתוחה ו־$f : U \to V$ פרמטריזציה.
	במקרה זה גם $P = f(u)$ וכן $T_u V = \RR^2$.
\end{definition}
\begin{theorem}[אפיון שקול למישור משיק]
	אם $S$ משטח רגולרי וכן $P \in S$ אז מתקיים,
	\[
		T_P S
		= \{ c'(t_0) \mid c : I \to S \text{ regular path}, c(t_0) = P \}
	\]
	כלומר מישור משיק מתלכד עם ערכי הנגזרות של מסילות העוברות דרך הנקודה.
\end{theorem}

\section{שיעור 20 --- 30.12.2025}
\subsection{התבנית היסודית הראשונה}
\begin{definition}[התבנית היסודית הראשונה]
	נניח ש־$S \subseteq \EE^3$ משטח רגולרי ו־$P \in S$.
	נגדיר את התבנית היסודית הראשונה בתור המישור היחיד העובר ב־$P$ ובעל דיפרנציאל זהה ל־$P$ ב־$S$. \\
	עבור $\langle , \rangle : \RR^3 \times \RR^3 \to \RR$ המכפלה הפנימית הסטנדרטית,
	נגדיר $I_p = \langle , \rangle \restriction_{T_P S \times T_P S}$ להיות התבנית היסודית הראשונה של $S$ ב־$P$.
	כאשר,
	\[
		T_P(S)
		= \{ P + f'(t_0) \mid f : I \to S, f(t_0) = P \}
	\]
	המישור המשיק ל־$P$ ב־$S$.
\end{definition}
\begin{example}
	אם $f : \EE^2 \to \EE^3$ המוגדרת על־ידי $f(u^1, u^2) = P + u^1 \xi + u^2 \eta$ כאשר $\xi, \eta$ בלתי־תלויים ב־$\RR^3$.
\end{example}
\begin{example}
	אם $S$ נתון על־ידי המשוואה $z = 0$ אז נוכל להגדיר $U = \EE^2, V = \EE^3$ וכן $f : (u^1, u^2) \mapsto u^1 e_1 + u^2 e_2$ ונקבל שהתבנית ניתנת להצגה עם מטריצת גרם $G = I_2$.
\end{example}

\section{שיעור 21 --- 5.1.2026}
\subsection{התבנית היסודית הראשונה --- המשך}
כרגע אנו עוסקים במשטחים במרחב האוקלידי, כרגיל נסמן $U \subseteq \EE^2$ עם $f : U \to \EE^3$ כפרמטריזציה מקומית.
בשיעור הקודם דיברנו על התבנית היסודית הראשונה, כלומר לכל $p \in U$ התאמנו פונקציית מכפלה פנימית $p \mapsto I_p$, זוהי הפונקציה אשר פועלת על המישור המשיק ל־$p$ במשטח הנתון.
כמובן הגדרנו גם את המישור המשיק של $p$ על־ידי שימוש בפרמטריזציה שלו, כלומר $D f |_p : \RR^2 \to \RR^3 = T_{f(p)} \RR^3$, וזהו תת־מרחב לינארי.
באופן מהותי זהו הייצוג של המישור האוקלידי על יריעה דו־מימדית, ובה אנו יכולים שוב לשאול שאלות על גדלים זוויות וכדומה.

נזכור גם כי כבר דיברנו על תבניות בי־לינאריות סימטריות, ואף ראינו שקיימת לה תבנית ריבועית, היא פונקציה שמייחסת סקלר לכל וקטור, הוא הריבוע של הנורמה של הווקטור.
לבסוף גם הזכרנו שאם אנו עובדים בקורדינטה, אז נוכל להגדיר את מטריצת גרם (התלויה בקורדינטה) של התבנית הבי־לינארית.
עבור $g = I_p$ והבסיס הסטנדרטי $(e_1, e_2)$ הפורש את $\RR^2$, אז נקבל,
\[
	(e_1, e_2) \xrightarrow{D f |_p} (D f |_p(e_1), D f |_p (e_2))
\]
אלו הם וקטורים בלתי־תלויים מהגדרת המשטח, ועלינו להבין את התנהגותם.
נסמן $f(u^1, u^2) = (f^1(u^1, u^2), f^2(u^1, u^2), f^3(u^1, u^2)) = (x, y, z)$ אז נקבל,
בהתאם נוכל לקבל שמטריצת הייצוג היא,
\[
	g_{11}(p) = D_1 f |_p \cdot D_1 f |_p,
	\quad
	g_{1 2} = D_1 f |_p \cdot D_2 f |_p = g_{2 1},
	\quad
	g_{2 2}(p) = D_2 f |_p \cdot D_2 f |_p
\]
נשים לב ש־$g_{ij}(p)$ היא בעצמה פונקציה המקבלת שני פרמטרים ומחזירה סקלר, כלומר $\RR^2 \to \RR$, ובהתאם נוכל לקבל שאם מתקיים,
\[
	G
	= \begin{pmatrix}
		g_{1 1} & g_{1 2} \\
		g_{2 1} & g_{2 2}
	\end{pmatrix}
\]
אז הלכה למעשה $G(p) \in M_2(\RR)$ היא תבנית בי־לינארית.
כלומר בבסיס סטנדרטי נקבל $G : U \to M_2(\RR)$.
בהתאם נוכל להגדיר מטריקה חדשה במרחב.
\begin{definition}[מטריקה רימנית במישור]
	תהי $U \subseteq \EE^2$ פתוחה.
	מטריקה רימנית על $U$ היא פונקציה $g : U \to M_2(\RR)$ כאשר נסמן,
	\[
		g
		= \begin{pmatrix}
			g_{1 1} & g_{1 2} \\
			g_{2 1} & g_{2 2}
		\end{pmatrix}
	\]
	כך ש־$g_{i j} : U \to \RR$, אשר מקיימת את התכונה ש־$g_{i j}$ חלקה, וכן ש־$g(p)$ היא תבנית בי־לינארית חיובית בהחלט, לכל $p \in U$.
\end{definition}
אם נשכח לרגע את המשטח, בהגדרה שראינו זה עתה ישנו מבנה קיים בפרמטריזציה כללית, ועל־ידי שימוש בתבנית יסודית הראשונה נוכל ליצור הלכה למעשה מטריקה רימנית כזו.
כלומר אנחנו משתמשים במכפלה הפנימית של $\RR^2$ כדי להגדיר לכל נקודה תבנית בי־לינארית, וכך נקבל בדיוק מטריקה רימנית. \\
זה גם אומר שנוכל על־ידי שינוי קורדינטה להגיע למצב שבו $G(p) = \id$ בדיוק, זאת שכן היא חיובית בהחלט.

\subsection{אורך ושטח של עקומים ומשטחים}
נניח ש־$S \subseteq \EE^3$ משטח חלק, ויהי עקום $c : I \to \EE^3$ כך ש־$c(I) \subseteq S$ ו־$I \subseteq \EE^1$.
אם גם $f : U \to f(U)$ פרמטריזציה מקומית אז נוכל לבחון אז קיימת $\varphi$ המקיימת $c = f \circ \varphi$, כלומר נוכל להשתמש בפרמטריזציה כדי לייצג את העקום.
ניזכר גם שהגדרנו את האורך של עקום, אם $I = [a, b]$ אז,
\[
	L(c)
	= \int_a^b \lVert c'(t) \rVert\ dt
\]
אבל בהתאם נוכל לקבל,
\[
	c'(t)
	= D f |_{\varphi(t)} \circ \varphi'(t)
\]
ובהצגה מטריצאלית נחשב ונקבל,
\[
	D_1 f^1(\varphi(t)) \cdot \varphi'(t) + D_2 f^1(\varphi(t)) \varphi^2(t)
	+ D_1 f^2(\varphi(t)) \cdot \varphi'(t) + D_2 f^2(\varphi(t)) \varphi^2(t)
	+ D_1 f^3(\varphi(t)) \cdot \varphi'(t) + D_2 f^3(\varphi(t)) \varphi^2(t)
\]
ניזכר כי התבנית היסודית הראשונה בנויה כך ש־$G = D_1 f^1 D_1 f^1 + \cdots$ ולכן מתקיים,
\[
	\lVert c'(t) \rVert
	= g_{1 1}(\varphi(t)) {(\varphi^1(t))}^2 + g_{2 1}(\varphi(t)) \varphi^1(t) \varphi^2(t) + g_{2 2}(\varphi(t)) {(\varphi^2(t))}^2
\]
כלומר ישנו קשר הדוק בין תבנית יסודית ראשונה ובין המרחק של עקום במשטח.

\section{שיעור 22 --- 6.1.2026}
\subsection{אורך ושטח על משטח}
נמשיך בדיון שלנו על מדידת מרחקים של עקומים משוכנים במשטחים.
המטרה שלנו היא במקום לבנות את המסילות מ־$I$ ל־$S$, לבנות אותן לפרמטריזציה $U$.
בהתאם אם $c = f \circ \varphi$ עבור $f : U \to S$ ו־$\varphi : I \to U$ אז נקבל,
\[
	c'(t)
	= D f |_{\varphi(t)}(\varphi'(t))
	= \begin{pmatrix}
		D_1 f (\varphi(t)) & D_2 f (\varphi(t))
	\end{pmatrix}
	\begin{pmatrix}
		\varphi_1'(t) \\
		\varphi_2'(t)
	\end{pmatrix}
\]
אבל נזכור כי אנו מחפשים את ${\lVert c'(t) \rVert}^2 = \langle c'(t), c'(t) \rangle$, ובהתאם,
\[
	{\lVert c'(t) \rVert}^2
	= {(c'(t))}^t c'(t)
	= \begin{pmatrix}
		\varphi_1'(t) &
		\varphi_2'(t)
	\end{pmatrix}
	\begin{pmatrix}
		D_1 f (\varphi(t)) \\
		D_2 f (\varphi(t))
	\end{pmatrix}
	\begin{pmatrix}
		D_1 f (\varphi(t)) & D_2 f (\varphi(t))
	\end{pmatrix}
	\begin{pmatrix}
		\varphi_1'(t) \\
		\varphi_2'(t)
	\end{pmatrix}
	= \begin{pmatrix}
		g_{1 1} \varphi(t) & g_{1 2} \varphi(t) \\
		g_{2 1} \varphi(t) & g_{2 2} \varphi(t)
	\end{pmatrix}
\]

\subsection{שטח של משטח}
להגדיר שטח זה קשה.
\begin{remark}
	נזכור שמתקיים ${|v \wedge w|}^2 = {|v|}^2 {|w|}^2 - {(v \cdot w)}^2$.
\end{remark}
נשים לב כי אפשר לייצג שטח על־ידי,
\[
	\vol_2(f(R))
	= \int_R \lVert D_1 f(u) \wedge D_2 f(u) \rVert\ du
\]
ולכן,
\[
	{\lVert D_1 f(u) \wedge D_2 f(u) \rVert}^2
	= \langle D_1 f(u), D_1 f(u) \rangle \cdot \langle D_2 f(u), D_2 f(u) \rangle - {\langle D_1 f(u), D_2 f(u) \rangle}^2
	= E(u) G(u) - F^2(u)
\]
עבור $I(p) = (\begin{smallmatrix} E & F \\ F & G \end{smallmatrix})$.
נסיק שמתקיים,
\[
	\vol_2(f(R))
	= \int_R \sqrt{E(u) G(u) - F^2(u)}\ du
\]
\begin{definition}[פונקציה שומרת שטח]
	יהיו $S_1, S_2 \subseteq \EE^3$ משטחים (פרמטריים).
	$F : S_1 \to S_2$ נקראת שומרת שטח אם עבור $U \subseteq \EE^2$ ו־$f_1 : U \to S_1, f_2 : U \to S_2$ פרמטריזציות כך ש־$f_2 = F \circ f_1$, מתקיים,
	\[
		\forall R \subseteq U,\ 
		\vol_2(f_1(U)) = \vol_2(f_2(U))
	\]
\end{definition}
במקרה זה נקבל שגם $E_1(u) G_1(u) - F_1^2(u) = F_2(u) G_2(u) - F_2^2(u)$.
לדוגמה כאשר $S = S^2(\RR)$ ספירת היחידה וכן $U = (0, 2 \pi) \times (0, \pi)$ נגדיר,
\[
	f : U \to S,
	\qquad
	f(\theta, \phi)
	= {(\sin \phi \cos \theta, \sin \phi \sin \theta, \cos \phi)}^t
\]
בהתאם המטריצה המייצגת את התבנית היסודית הראשונה היא $E = \sin^2 \phi, F = 0, G = 1$.

\section{שיעור 23 --- 12.1.2026}
\subsection{עקמומיות}
אנו יודעים שבעוד שהנגזרת הראשונה מעידה על סדר השינוי של פונקציה, הנגזרת השנייה היא שערוך טוב לקצב השינוי בשיפוע, כלומר היא מאפשרת לנו לתארך את העקמומיות של גרף.
נניח ש־$f : U \to \EE^3$ משטח פרמטרי, ותהי $(a, b) \in U$, אז מנוסחת טיילור נקבל,
\[
	f((a, b) + (h, k)) - f(a, b)
	= D_1 f (a, b) \cdot h + D_2 f(a, b) \cdot k + \frac{1}{2} (D_1^2 f(a, b) \cdot h^2 + D_1 D_2 f(a, b) \cdot h k + D_2^2 f(a, b) \cdot k^2) + R(h, k)
\]
אבל נוכל לכתוב בפורמט של תבנית ריבועית,
\[
	f((a, b) + (h, k)) - f(a, b)
	= D_1 f (a, b) \cdot h + D_2 f(a, b) \cdot k + \frac{1}{2} (h, k) \begin{pmatrix}
		D_1^2 f & D_1 D_2 f \\
		D_2 D_1 f & D_2^2 f
	\end{pmatrix} 
	\begin{pmatrix} h \\ k \end{pmatrix} + R(h, k)
\]
אם נכתוב $n = \frac{D_1 f \wedge D_2 f}{|D_1 f \wedge D_2 f|}$ אז נקבל,
\[
	n (f((a, b) + (h, k)) - f(a, b))
	= 0 + \frac{1}{2} (n, k) \begin{pmatrix}
		D_1^2 f & D_1 D_2 f \\
		D_2 D_1 f & D_2^2 f
	\end{pmatrix} n + R(h, k) {(n, k)}^t n
\]
\begin{definition}[התבנית היסודית השנייה]
	יהי $f : U \to \EE^3$ משטח פרמטרי (מקומי) רגולרי.
	התבנית היסודית השנייה על המשטח מוגדרת על־ידי,
	\[
		L = D_1^2 f \cdot n,
		\quad
		M = D_1 D_2 f \cdot n = D_2 D_1 f \cdot n,
		\quad
		N = D_2^2 f \cdot n
	\]
	כאשר,
	\[
		n : U \to S^2,
		\qquad
		n(u)
		= \frac{D_1 f(u) \wedge D_2 f(u)}{|D_1 f(u) \wedge D_2 f(u)|}
	\]
\end{definition}
\begin{example}
	נבחן את המישור.
	נגדיר $f(t, s) = p + t u + s v$ עבור $p \in \EE^3, u, v \in \RR^3$.
	במקרה זה $D_1 f \equiv u, D_2 f \equiv v$.
	בהתאם נקבל שגם $D_1^2 f \equiv D_2^2 \equiv D_1 D_2 f \equiv 0$.
	נקבל שגם $n(t, s) = \frac{u \wedge v}{|u \wedge v|}$ ובהתאם $L = M = N = 0$.
\end{example}
\begin{remark}
	מתקיים $D_i f \cdot n : U \to \EE^1$ ולכן $D_1 f \cdot n = 0$ מהגדרת הנורמל.
	במצב זה נקבל שגם $D_1^2 f \cdot n + D_1 f \cdot D_1 n = 0$ ולכן נובע ש־$L = -D_1 f \cdot D_1 n$, באופן דומה נקבל שגם $M = - D_1 f \cdot D_2 n = -D_2 f \cdot D_1 n$.
	לבסוף גם נקבל ש־$N = - D_2 f \cdot D_2 n$.
\end{remark}
נעבור לשימוש בכלי חדש זה.
אם $I \xrightarrow{\phi} U \xrightarrow{f} \EE^3$ פירוק של $c : I \to \EE^3$ מסילה רגולרית, ונניח שזוהי פרמטריזציה לפי אורך שלה.
במקרה זה מתקיים $\lVert c' \rVert \equiv 1$ וכן $c(0) = P$ ו־$c'' \cdot c' \equiv 0$.
אז נסיק ש־$n c = 0$ ו־$c'' = a n + b (n \wedge c)$, נובע ש־$c'' = (c'' \cdot n) + (c'' \cdot n \wedge c') n \wedge c'$.
אבל $c = f \circ \phi$ ולכן $c' = df \circ \phi'$ ולכן,
\[
	(D_1 f, D_2 f) \cdot (\phi_1', \phi_2')
	= D_1 f \phi_1' + D_2 f \phi_2'
\]
נקבל לבסוף שמתקיים $c'' = \kappa_{\operatorname{normal}} n + \kappa_{\operatorname{geodesic}}(\cdots)$.

\section{שיעור 24 --- 13.1.2026}
נמשיך בדיון של השיעור הקודם.
נניח ש־$S \subseteq \EE^3$ משטח פרמטרי חלק, ותהי $f : U \to \EE^3$ פרמטריזציה מקומית, כלומר $U \subseteq \EE^2$ פתוחה.
נסמן עבור $P \in S$ את המישור המשיק ב־$T_P S$, את התבנית הראשונה על־ידי $I_P$ ואת התבנית השנייה על־ידי ${I\!I}_P$.
\begin{example}
	נעסוק בגרף של פונקציה.
	נניח ש־$f : U \to \EE^1$ היא פונקציה סקלרית.
	נגדיר את המשטח $S = G_f = \{ (x, y, f(x, y)) \mid (x, y) \in U \}$.
	נגדיר $F : U to \EE^3$ הפרמטריזציה של $S$ ולכן נקבל,
	\[
		D_1 F = {(1, 0, D_1 f)}^t,
		\qquad
		D_2 F = {(0, 1, D_2 f)}^t
	\]
	ולכן,
	\[
		n = {(-D_1 f, -D_2 f, 1)}^t \cdot \frac{1}{\sqrt{1 + {(D_1 f)}^2 + {(D_2 f)}^2}}
	\]
	נעבור לחישוב לקראת התבנית היסודית השנייה,
	\[
		D_1^2 F = {(0, 0, D_1^2 f)}^t,
		\quad
		D_2^2 F = {(0, 0, D_2^2 f)}^t,
		\quad
		D_1 D_2 F = {(0, 0, D_1 D_2 f)}^t
	\]
	עתה נצטרך לכפול סקלרית ב־$n$,
	\[
		L = D_1^2 f,
		\quad
		M = D_1 D_2 f,
		\quad
		N = D_2^2 f
	\]
	וקיבלנו,
	\[
		{I\!I}_F
		= \begin{pmatrix}
			D_1^2 f & D_1 D_2 f \\
			D_2 D_1 f & D_2^2 f
		\end{pmatrix} \cdot \frac{1}{\sqrt{1 + {(D_1 f)}^2 + {(D_2 f)}^2}}
	\]
	וזהו בדיוק ההסיאן של $f$.
\end{example}

\section{שיעור 25 --- 19.1.2026}
\subsection{עקמומיות}
תהי $S \subseteq \EE^3$ יריעה במרחב, 

\listoftheorems[title=הגדרות ומשפטים,ignoreall,show={theorem,definition},swapnumber,onlynamed={proposition,lemma}]

\end{document}
