\newcounter{english}
\input{../article_base.tex}
\title{Exercise 7 Answer Sheet --- Logic Theory (2), 80424}

\DeclareMathOperator{\PA}{PA}
\DeclareMathOperator{\Coll}{Coll}
\DeclareMathOperator{\Ind}{Ind}
\DeclareMathOperator{\Sat}{Sat}

\begin{document}
\maketitle
\maketitleprint[yellow]

\question{}
For any $\Gamma \subseteq \operatorname{form}_{L_{\PA}}$ let $I \Gamma$ be the induction scheme over $\Gamma$. \\
For $\varphi(z, y, t_0, \ldots, t_{n - 1}) \in \operatorname{form}_{L_{\PA}}$ we define
\[
	\Coll(\varphi) = \forall z \forall t_0 \ldots \forall t_{n - 1} (\forall x \le z \exists y \varphi \leftrightarrow \exists w \forall x \le z \exists y \le w \varphi)
\]
By the last exercise, $\PA \vdash \Coll(\varphi)$ for any such $\varphi$. \\
Let $\Coll = \{ \Coll(\varphi) \mid \varphi \in \operatorname{form}_{L_{\PA}} \}$, $Q$ be the first 9 axioms of $\PA$, and $\PA_0 = Q \cup I \Sigma_0 \cup \Coll$. \\
We will show that $\PA_0 \vdash \PA$.
\begin{proof}
	For $Q \subseteq \PA$ we assumed that $\PA_0 \vdash Q$, then it is sufficient to show that $\PA_0 \vdash \Ind(\varphi)$ for any $\varphi \in \operatorname{form}_{L_{\PA}}$. \\
	If $\varphi \in \Sigma_0$ then $\PA_0 \vdash \Ind(\varphi) \ni I \Sigma_0$. \\
	In the previous exercises we have shown that for any $\varphi$ there is $n$ such that $\varphi \in \Sigma_n^0$, and that there is $\psi \in \Sigma_n$, for which, $\PA \vdash (\varphi \leftrightarrow \psi)$.
	We will prove in induction that $\PA_0 \vdash (\varphi \leftrightarrow \psi)$ as well, and that $\PA_0 \vdash \Ind(\varphi)$ in such case.
	The case of $n = 0$ was proven in the last exercise. \\
	Let us assume that $\varphi \in \Sigma_n(\PA)$ and $\psi \in \Sigma_n$ such that $\PA \vdash (\varphi \leftrightarrow \psi)$ and $\psi = \exists v \phi$ for $\phi \in \Pi_{n - 1}$.
	By the induction hypothesis $\psi$ fulfills the claim, meaning that $\PA_0 \vdash \Ind(\psi)$.
	We will show that $\PA_0 \vdash (\varphi \leftrightarrow \psi)$ and that $\PA_0 \vdash \Ind(\varphi)$. \\
	As for the first part, by $\Coll(\phi)$ and negation we infer that $\PA_0 \vdash (\varphi \leftrightarrow \psi)$ as intended, and it is left to show that $\PA_0 \vdash \Ind(\psi)$.
	The proof is the same as of question 1 of exercise 6. \\
	Let $\Mm \models \PA_0$ be some model and we assume that $\exists x \phi(x, y, z)$ satisfies the induction condition, meaning that $\Mm \models \phi(x, 0, c)$ for some $c \in M$;
	as well,
	\[
		\Mm \models \forall y (\exists x \varphi(x, y, c)) \to (\exists x \varphi(x, S(y), c))
	\]
	We want to show that $\Mm \models \forall y \exists x \varphi(x, y, c)$. 

	Let $b \in M$ be some arbitrary element, we want to show that,
	\[ \label{eq:1}
		\Mm \models \exists x \varphi(x, b, c)
		\tag{1}
	\]
	thus proving the last statement.
	By using $\Coll \subseteq \PA_0$ we infer that
	\[ \label{eq:2}
		\Mm \models \forall b' \le b \exists x \varphi(x, b', c)
		\tag{2}
	\]
	if and only if exists $d \in M$ such that,
	\[ \label{eq:3}
		\Mm
		\models \forall b' \le b\ \exists x \le d\ \varphi(x, b', c)
		\tag{3}
	\]
	In the set-theoretical sense, let $X = \{ x \mid \Mm \models \exists b' \le b,\ \varphi(x, b', c) \}$, this is a finite set as the order induced on $M$ by $\le^\Mm$ is a well-order.
	Let us select $d = \sup_{\le^\Mm} X$, this $d$ fulfills (\ref{eq:3}), thus also proving (\ref{eq:2}), but as $b \in M$ is arbitrarily chosen, (\ref{eq:1}) holds as well, completing our proof.
\end{proof}

\question[3]
We will show that $\Sat_0$ is primitive-recursive.
\begin{proof}
	In the lectures we saw that $\Sat_0$ is total recursive. \\
	It follows that $\Sat_0$ is a $\Sigma_1^0$ formula, meaning that it is equivalent to some $\Sigma_0$ formula, $\varphi$.
	By the presented proof, we also know that $\varphi(x, y) = \exists z \psi(x, y, z)$ for $x$ code for a $\sigma_0$-formula,
	$y$ a finite-support assignment and $z$ an inference sequence to represent the formula and the assignment.
	We will bound $z$, showing that $\varphi$ is indeed $\Sigma_0$.
	The sequence $z$ fulfills, $z = \# \langle (f_i, s_i, \epsilon_i) \mid i < n \rangle$ for some $n \in \NN$, and it is derived by definition that $z \le \# (f, s, \epsilon)$ for some such values.
	Let us choose $s = \# \sigma, f = \# x$ and fixing some arbitrary large enough $\epsilon \ge \TT, \FF$.
	As both $x$ and $\sigma$ are variables and the fact that they bind each element of the sequence (by 7.3 and 7.4) we infer that they are indeed a bound to $\psi$.
\end{proof}

\question{}
Let $t$ be a term and $\sigma$ some finite-supported assignment. \\
We will prove that $t^{\hat{\NN}}(\sigma) \le K^{\lceil t \rceil + 1}$ for $K = \max\{ \sigma(v_i) \mid i \in \NN \} + 2$.
\begin{proof}
	We will prove the claim by induction over the structure of the term.
	Note that the proof is equivalent the case of trees and sequences, then we shall use trees without diverge from past assumptions.

	For $t = 0$ (the only constant), $\lceil t \rceil = 1$ and $K \ge 1$, then $0^{\hat{\NN}}(\sigma) = 0^{\hat{\NN}} = 0 \le 1 \le K^1$. \\
	For $t = v_i$ for some variable $v_i$ ($i \in \NN$), then $t^{\hat{\NN}}(\sigma) = \sigma(v_i)$.
	From the other side, $K \ge \sigma(v_i)$ as a maximum value, and $\lceil t \rceil + 1 \ge 2$, then $\sigma(v_i) \le {\sigma(v_i)}^{2} \le K^2$.

	Let us move to the induction step, we assume that the claim holds for $s_0, s_1 \in \operatorname{term}_{\PA}$ and for some assignment $\sigma$.
	Either $t = S(s_0), t = s_0 + s_1$ or $t = s_0 \cdot s_1$. \\
	For $t = S(s_0)$ we get $t^{\hat{\NN}}(\sigma) = S(s_0^{\hat{\NN}}(\sigma)) \le K^{\lceil s_0 \rceil + 1} + 1$ by the induction hypothesis.
	By our definition of numbering also $\lceil s_0 \rceil \cdot 13 \le \lceil t \rceil$, in particular $K^{\lceil s_0 \rceil + 1} + 1 \le K^{\lceil t \rceil + 1}$. \\
	The remaining cases are similar, we will prove it for $t = s_0 + s_1$,
	\[
		t^{\hat{\NN}}(\sigma) = s_0^{\hat{\NN}}(\sigma) + s_0^{\hat{\NN}}(\sigma) \le K^{\lceil s_0 \rceil + 1} + K^{\lceil s_1 \rceil + 1}
	\]
	We assume without loss of generality that $\lceil s_0 \rceil \le \lceil s_1 \rceil$, then,
	\[
		K^{\lceil s_0 \rceil + 1} + K^{\lceil s_1 \rceil + 1}
		\le 2K^{\lceil s_1 \rceil + 1}
	\]
	By our definition of Gödel numbering $\lceil t \rceil \ge 13^{\lceil s_0 \rceil} \lceil s_1 \rceil$, then the claim follows almost immediately.
\end{proof}

\end{document}
