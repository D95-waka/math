\newcounter{english}
\input{../article_base.tex}
\title{Exercise 1 Answer Sheet --- Logic Theory (2), 80424}

\begin{document}
\maketitle
\maketitleprint{}

\question{}
Let us assume that $D$ is an ultrafilter on given set $X$.

\subquestion{}
Suppose that there is some $s \in D$ such that $|s| < \omega$, we will show that $D$ is principal.
\begin{proof}
	We will assume that $s$ is minimal in relation to cardinality in $D$, otherwise let us choose the minimal such $s$ by using the well ordering principal.
	There is a unique such $s \in D$, otherwise $|s| = |s'|$ and $|s \cap s'| < |s|$, a contradiction to the minimality of the cardinality of $s$. \\
	For every $x \in D$, there is some $y \in D$ such that $y \subseteq x \cap s$, but by $s$ minimality, $y = x \cap s$, hence $s \subseteq x$.
	In the other direction, if $s \subseteq x$, then $x \in D$ by filter properties. \\
	Therefore $D = \{ x \in \Pp(X) \mid s \subseteq x \}$, $D$ is a principal ultrafilter defined by $s$.
\end{proof}

\subquestion{}
We will show that if $D \subseteq \Pp(I)$ is a principal ultrafilter, $\langle \Mm_i \mid i \in I \rangle$ is a sequence of $L$-structures, then $\Nn = \prod_{i \in I} \Mm_i / D \cong M_i$ for some $i \in I$.
\begin{proof}
	Let $s \subseteq I$ be the defining set for $D$, and let us fix $i \in s$.
	Let $f : N \to M_i$ such that $f({[g]}_D) = g(i)$.
	$f$ is well defined, as every for every $h, h' \in [h]$, it is follows that $h \restriction s = h' \restriction s$ by $D$'s definition as principal.

	We will show that $f$ is an isomorphism between $\Nn$ and $\Mm_i$.
	$L$-constants are being preserved directly from definition of ultra-products.
	Assume $F \in \operatorname{Func}_{L, n}$, and $t_0, \dots, t_{n - 1} \in \operatorname{term}_L$, then,
	\[
		f(F^\Nn(t_0^\Nn, \dots, t_{n - 1}^\Nn))
		= f([\{ F^{\Mm_j}(t_0^{\Mm_j}, \dots, t_{n - 1}^{\Mm_j}) \mid j \in I \}])
		= F^{\Mm_i}(t_0^{\Mm_i}, \dots, t_{n - 1}^{\Mm_i})
	\]
	meaning $f$ preserves terms.
	Let $R \in \operatorname{rel}_{L, n}$, and let $t_0, \dots, t_{n - 1} \in \operatorname{term}_L$, then,
	\[
		\Nn \models R(t_0, \dots, t_{n - 1})
		\iff \{ j \in I \mid \langle t_0^{\Mm_j}, \dots, t_{n - 1}^{\Mm_j} \rangle \in R^{\Mm_j} \} \in D
		\implies \langle t_0^{\Mm_j}, \dots, t_{n - 1}^{\Mm_j} \rangle \in R^{\Mm_i}
	\]
	We conclude that $f$ is indeed a structures isomorphism.
\end{proof}

\question{}
We will show that if $F$ is a filter on a set $X$ such that if $s \in F$ then $s$ is infinite, then there is a non-principal ultrafilter extending $F$.
\begin{proof}
	Initially, we claim that there is no $x \in F$ such that $|x|, |x^C| \ge \omega$, when $x^C = X \setminus x$.
	If there is one, then $\emptyset \subseteq x \cap x^C = \emptyset$ is in $F$, in contradiction to $F$ being a filter.
	We define the ultrafilter $U = \{ x \subseteq X \mid |x| \ge \omega \}$, this is indeed an ultrafilter as noted in lecture.
	it is sufficient to show that $F \subseteq U$.
	Indeed, if $s \in F$ then $|s| \ge \omega$, implies that $s \in U$.

	We found an ultrafilter extending $F$, it is required to prove that $U$ is not a principal ultrafilter as well.
	We will assume otherwise, then it is follows that there is $S \subseteq X$ such that $\forall x, S \subseteq x \iff x \in U$.
	Let $a \in S$ be an element, then $b = X \setminus \{a\}$ is a set such that $|b| \ge \omega$, implies that $b \in U$, a contradiction as $S \not\subseteq b$.
\end{proof}

\question{}
\subquestion{}
Let $\Nn = (\NN; R^\Nn)$ be graph such that $(a, b) \in R^\Nn \iff |b - a| = 1$.
Let $D$ be a non-principal ultrafilter on $\NN$ (An example of one is the construction from exercise 2).
Let $\Mm = \Nn^\NN / D$.
We will prove that $\Mm$ is not path-connected.
\begin{proof}
	Let us define $a = [0], b = [id]$, in the sense that the constant valued function $f(n) = 0$, it derived that $f \in a$.
	We claim that there is no finite path between $a, b \in N$.
	We assume otherwise, and define $\langle [v_i] \mid i < n \rangle$ for some $n < \omega$, such that $([v_i], [v_{i + 1}]) \in R^\Nn$ for every $i < n$, and $[v_0] = a, [v_{n - 1}] = b$.
	$D$ is non-principal then there is some $i < \omega$ such that $v_i(i)$ testifies to the relation, and by the last exercise $i$ is as large as we desire, then we assume $i > n$.
	But $v_0(i) = 0, v_{n - 1}(i) = i$, and by the relation definition $v_j(i) = j$ for every $0 \le j \le n$, it is implies that $i = v_{n - 1}(i) < n < i$, a contradiction.
	We conclude that $\Mm$ is not path-connected.
\end{proof}

\subquestion{}
Let $\Rr = (\RR; +, \cdot, 0, 1, <)$.
Let $D$ be a non-principal ultrafilter on $\NN$, and let $\Rr^* = \Rr^\NN / D$. \\
We will show that in $\Rr^*$ there is an infinitesimal element, namely $\epsilon \in R^*$ such that $\forall n \in \NN, 0 < \epsilon < \frac{1}{n}$.
\begin{proof}
	Let $d \in D$ be an unbounded subset of $\NN$, and let $r : \NN \to \RR$ such that $(r \restriction d)(n) = \frac{1}{n + 1}$,
	We define $\epsilon = {[r]}_{\Rr^*}$.
	We observe that $1^{\Rr^*} = [\{ \langle n, 1 \rangle \mid n \in \NN \}]$ by definition, and it is follows that ${(\frac{1}{n})}^{\Rr^*} = [\{ \langle n, \frac{1}{n} \rangle \mid n \in \NN \}]$.
	Let $m \in \NN$ be some number, we will show that $\Rr^* \models \epsilon < \frac{1}{n}$,
	\[
		\Rr^* \models \epsilon < \frac{1}{n}
		\iff \{ n \in \NN \mid \RR \models \epsilon < \frac{1}{m} \} \in D
		\impliedby \{ n \in d \mid \epsilon(n) = \frac{1}{m + 1} < \frac{1}{m} \} \in D
	\]
	and by $d$'s definition the statement indeed holds.
	Lastly we will note that $\Rr^* \models 0 < \epsilon$, it is derived by the same method.
\end{proof}

\question{}
We will define a filter $F$ on $X$ as $\sigma$-complete if and only if for every $\{ s_\alpha \mid \alpha < \omega \} \subseteq F$, $\bigcap_{\alpha < \omega} s_\alpha \in F$.

\subquestion{}
We will show that if $F$ if a $\sigma$-complete ultrafilter and $|X| = \omega$ then $F$ is principal.
\begin{proof}
	An answer using AC without necessity, \\
	Let $\{ s_\alpha \mid \alpha < \omega \} \subseteq F$ be some collection, and let us define $t_0 = s_0$ and for every $\alpha < \omega$, $t_{\alpha + 1} = s_\alpha \cap t_\alpha$.
	We also define $t_\omega = \bigcap_{\alpha < \omega} t_\alpha$, this is indeed a set, and $\in F$ in particular, due to $F$ being $\sigma$-complete.
	The item $t_\omega$ is a supremum in relation of $\supseteq$ in the chain $\langle t_\alpha \mid \alpha < \omega \rangle$, and it follows that every chain has such supremum.
	By Zorn's lemma we conclude that there is $s_m \in F$ which is maximal according to the relation $\supseteq$, then $\forall s \in F, s_m \subseteq s$.
	If $s_m \subseteq s$ for any $s \subseteq X$, then by filters properties $s \in F$, concluding that $F = \{ s \subseteq X \mid s_m \subseteq s \}$.

	We can also use the well-founded order of $\subseteq$ of $X$ and take a minimal of it, omitting the part of Choice axiom.

	We can also assume that it is non-principal, therefore for every $x \in X$, $X \setminus \{x\} \in F$ (as it is an ultrafilter), then $\bigcap_{\alpha < \omega} X \setminus \{ x_\alpha \} = \emptyset \in F$, a contradiction.
\end{proof}

\subquestion{}
We will provide an example of a $\sigma$-complete ultrafilter non-principal filter on an infinite set.
\begin{solution}
	Let $\mu < \kappa$ be cofinal cardinals such that $\omega_1 < \mu$.
	We define $U = \{ \delta < \kappa \mid \mu < \delta \}$.
	This is indeed an ultrafilter as concluded from the cofinality of $\mu$, and it is non-principal as derived from extension of question 2.
	It is clear that $U$ is $\sigma$-complete, as $\omega_1$ intersections are preserving cofinality for $\mu$.
\end{solution}

\subquestion{}
We will show that if $\Mm_i = (X_i, <_i)$ are nonempty well-ordered sets for $i \in I$ and $D$ is a $\sigma$-complete ultrafilter on $I$, then $\Nn = \prod_{i \in I} \Mm_i / D$ is well-ordered.
\begin{proof}
	Using Löwenheim–Skolem we can assume without loss of generality that $\Nn$ is a countable model.
	Let $\{ [f_j] \in N \mid j < J \}$ be some collection, we will show that there if $[f] \in Y$ minimal in $<^\Nn$.
	Let $f(i) = \min_{<^{\Mm_i}} \{ f_j(i) \mid j < J \}$, this definition holds as $<^{\Mm_i}$ is a well-order.
	By $\sigma$-completeness we deduce that $\{ i < I \mid f_j(i) = f(i), j < J \} = \bigcap_{j < J} \{ i < I \mid f_j(i) = f(i) \}$,
	implying that indeed $[f] \in N$, concluding our claim.
\end{proof}

\subquestion{}
We will show that the statement of part c does not hold without the assumption that $D$ is $\sigma$-complete.
\begin{solution}
	Let $\Nn = {(\NN; <)}^{\NN} / D$ for the ultrafilter defined in question 2.
	It is clear that $\Nn$ fulfills the requirements, and that $D$ is not $\sigma$-complete.
	We will construct a set and show there is no minimum in it.
	Let $s = \{ [f] : \NN \to \NN \mid \forall n \in \NN f(n) \le f(n + 1) \} \subseteq N$, and let $[f] \in s$, we choose the function $[g] \in s$ such that $f(n) = g(n)$ for $n > M \in \NN$, and $g(n) = f(M)$ otherwise.
	Its follows that $[g] <^{\Nn} [f]$, therefore $f$ is not minimal, we conclude that indeed $s$ witnessing that $<^{\Nn}$ is not a well-order.
\end{solution}

\end{document}
