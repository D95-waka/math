\newcounter{english}
\input{../article_base.tex}
\title{Exercise 9 Answer Sheet --- Logic Theory (2), 80424}

\DeclareMathOperator{\PA}{PA}
\DeclareMathOperator{\Coll}{Coll}
\DeclareMathOperator{\Ind}{Ind}
\DeclareMathOperator{\Sat}{Sat}

\begin{document}
\maketitle
\maketitleprint[yellow]

\question{}
\subquestion{}
We will show that the set,
\[
	H
	= \{ e \in \NN \mid (e, 0) \in \dom U \}
\]
is recursively-enumerable but not recursive.
\begin{proof}
	$H$ is $\Sigma_1^0$ by the definition of $U$, implies that it is recursively-enumerable, then it is sufficient to show that it is not recursive.
	Let $f : \NN \to \NN$ be the function defined by,
	\[
		f(x) = x
	.\]
	This is indeed a recursive function (even primitive-recursive), and for every $e \in D$, when $D = \{ e \in \NN \mid (e, e) \in \dom U \}$, we get $f(e) = e \in H$, as $e$ witnessing bound larger than $0$.
	It follows that $f$ is a recursive reduction of $D$ to $H$ and implies that $H$ is not recursive.
\end{proof}

\subquestion{}
We will conclude that $H^C = \NN \setminus H$ is not recursively-enumerable.
\begin{proof}
	$H$ is a relation, meaning that if it were to be $\Pi_1^0$ it would be $\Delta_1^0$ and recursive.
	If $H^C$ is recursively-enumerable then $H^C = \{ e \in \NN \mid (e, 0) \in \dom(\lnot U) \}$ is $\Sigma_1^0$, and implying by negation that $H = \lnot H^C$ (in the sense of relations) is $\Pi_1^0$.
	But $H$ is recursively-enumerable and $\Sigma_1^0$, meaning that it is indeed $\Delta_1^0$, in contradiction to its being not recursive.
\end{proof}

\subquestion{}
Let us define the set,
\[
	T = \{ e \in \NN \mid \{ x \mid (e, x) \in \dom U \} = \NN \}
\]
We will show that $T$ is not recursively-enumerable.
\begin{proof}
	We assume in contradiction that $T$ is recursively-enumerable. \\
	By the equivalence to recursively-enumerable functions proposition we can assume that there is a function $f : \NN \to \NN$ such that $f$ is primitive-recursive and $T = \{ f(x) \mid x \in \NN \}$.
	Let us define $g : \NN^2 \to \NN$ by,
	\[
		g(n, x) = U(f(n), x) + 1
	\]
	$g$ is indeed total recursive function as $U$ is recursive and $f$ is total such that $U(f(n), \cdot)$ is total recursive function.
	$h(x) = g(x, x) + 1 = U(f(x), x) + 1$ is a total recursive function.
	This means that $\# h \in T$, and there is $m \in \NN$ such that $f(m) = \# h$.
	\[
		h(m)
		= U(f(m), m) + 1
		= U(\# h, m) + 1
		= h(m) + 1
	\]
	This is contradiction by diagonalization.
\end{proof}

\subquestion{}
We will show that $T^C = \NN \setminus T$ is not recursively-enumerable.
\begin{proof}
	$f : H^C \to T^C$ recursive and $f^{-1}(T^C) = H^C$ is recursively-enumerable by the contradiction assumption.
	We want to show that $f(e) \in T^C$ for any $e \in H^C$.

	Any $e \in H^C$ represents recursive function such that it is not defined in $0$, in particular it is not total (as $0$ witnessing that).
	We conclude that $e \in T^C$, and that $H^C \subseteq T^C$.
	We assume in contradiction that $T^C$ is recursively-enumerable, then there is recursive and total $f : \NN \to \NN$ such that $T^C = \{ f(n) \mid n \in \NN \}$.
	We define $f : \NN \to \NN$ by,
	\[
		f(x)
		= \begin{cases}
			x & x \in H^C \\
			\# c_{U(x, 0)} & x \in H
		\end{cases}
	.\]
	TODO
\end{proof}

\question{}
We will show that any infinite recursively-enumerable set contains an infinite recursive subset.
\begin{proof}
	Let $X$ be some recursively-enumerable set.
	Let $f : \NN \to \NN$ be a total primitive-recursive function such that $X = f(\NN)$.
	We saw that $\max$ is primitive-recursive, then,
	\[
		g(x) = \max \{ f(n) \mid n < x \}
	\]
	is primitive-recursive as well.
	Note that $f$ is unbounded, therefore $g$ is unbounded as well.
	We define $h : \NN \to \{ 0, 1\}$ by,
	\[
		h(x)
		= \begin{cases}
			1 & g(x) = f(x) \\
			0 & \text{otherwise}
		\end{cases}
	\]
	This is primitive-recursive function as the cases operation is primitive, and $h = \chi_Y$ for $Y \subseteq X$ recursive as we wanted.
\end{proof}

\question{}
We define the sets $A, B \subseteq \NN$ as recursively-separated if there is some recursive set $C \subseteq \NN$ such that $A \subseteq C$ and $B \cap C = \emptyset$. \\
We will find an example of two disjoint recursively-enumerable subsets of $\NN$ which cannot be recursively-separated.
\begin{solution}
	Is this a topology problem?
	This example will show that $\NN$ with the topology induced by recursiveness is not normal.

	Let $f : \NN \to \NN$ be the function such that $f(\NN) = H$ from the first question.
	We define $A = \{ f(2n) \mid n \in \NN \}$ and $B = \{ f(2n + 1) \mid n \in \NN \}$, these are clearly recursively-enumerable but not recursive.
	We also see that $A \cap B = \emptyset$ directly by definition.

	Let us assume in contradiction that there is a recursive $C \subseteq \NN$ such that $A \subseteq C$ and $B \cap C = \emptyset$.
	But the set of recursive functions $R$ is recursive and the reduction of recursive function is recursive, then $R \cap C = A$ is recursive, in contradiction to question 1.
\end{solution}

\end{document}
