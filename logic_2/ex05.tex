\newcounter{english}
\input{../article_base.tex}
\title{Exercise 5 Answer Sheet --- Logic Theory (2), 80424}

\begin{document}
\maketitle
\maketitleprint[yellow]

\question{}
\subquestion{}
Suppose that $\varphi(z, y, t_0, \ldots, t_{n - 1})$ is a formula in the language of $L_{\operatorname{PA}}$.
We will show that,
\[
	\operatorname{PA} \vdash \forall z \forall t_0 \ldots \forall t_{n - 1} (\forall x \le z \exists y \varphi \leftrightarrow \exists w \forall x \le z \exists y \le w \varphi)
\]
\begin{proof}
	Let us fix $z, t_0, \ldots, t_{n - 1}$, then is suffice to show that $\operatorname{PA} \vdash \forall x \le z \exists y \varphi \leftrightarrow \exists w \forall x \le z \exists y \le w \varphi$.
	We assume toward a contradiction that the formula does not hold, then $\forall x \le z \exists y \varphi$ as well $\lnot \exists w \forall x \le z \exists y \le w \varphi$ holds as well.
	The latter is equivalent to $\forall w \exists x \le z \forall y \le w (\lnot \varphi)$.
	We fix $w$ and it derives that $\exists x \le z \forall y \le w (\lnot \varphi)$ holds.
	Let us fix such $x$ as well (such that it is disjoint from previous fixations according to KE).
	From $\forall x \le z \exists y \varphi$ we can derive that for our fixed $x$ the statement holds, meaning that $\exists y \varphi$, and $\forall y \le w (\lnot \varphi)$.
	We finally fix some $y$ and then $\varphi$ holds, but $w$ can be arbitrarily fixed, we assume that $y \le w$ and it follows that $\lnot \varphi$, a contradiction.
\end{proof}

\subquestion{}
Suppose that $\psi(y_0, y_1, t_0, \ldots, t_{n - 1})$ is some formula in the language $L_{\operatorname{PA}}$.
We will show that,
\[
	\operatorname{PA}
	\vdash \forall t_0 \ldots \forall t_{n - 1} (\exists y_0 \exists y_1 \psi \leftrightarrow \exists w \exists y_0 \le w \exists y_1 \le w \psi)
\]
\begin{proof}
	The proof is very similar to part a,
	Let $t_0, \ldots, t_{n - 1}$ be some values, and let us assume toward a contradiction that $\exists y_0 \exists y_1 \psi$,
	and that $\lnot \exists w \exists y_0 \le w \exists y_1 \le w \psi \equiv \forall w \forall y_0 \le w \forall y_1 \le w (\lnot \psi)$ hold.
	We fix $y_0, y_1$ such that $\psi$ holds, and fix some $w \ge \sup\{ y_0, y_1 \}$, hence $\lnot \psi$ is derived by the second formula, resulting in a contradiction.
\end{proof}

\question{}
We define $\Sigma_n(\operatorname{PA})$ as the set of formulas $\psi(x_0, \ldots, x_{n - 1})$, such that for some $\Sigma_n$ formula $\psi'(x_0, \ldots, x_{n - 1})$,
\[
	\operatorname{PA}
	\vdash \forall x_0, \ldots, \forall x_{n - 1} (\psi \leftrightarrow \psi')
\]

\subquestion{}
We will show that the class of $\Sigma_0 (\operatorname{PA})$ formulas in PA is closed under boolean operations and bounded quantifiers.
\begin{proof}
	Let $\psi, \varphi \in \Sigma_0(\operatorname{PA})$ be some formulas.
	There exist $\varphi', \psi'$ such that,
	\[
		\operatorname{PA}
		\vdash \forall x_0, \ldots, \forall x_{n - 1} (\psi \leftrightarrow \psi'),
		\forall x_0, \ldots, \forall x_{n - 1} (\varphi \leftrightarrow \varphi')
	\]
	Then it follows that,
	\[
		\operatorname{PA}
		\vdash \forall x_0, \ldots, \forall x_{n - 1} ((\psi \leftrightarrow \psi') \land (\varphi \leftrightarrow \varphi'))
	\]
	Therefore by an identity from the previous course,
	\[
		\operatorname{PA}
		\vdash \forall x_0, \ldots, \forall x_{n - 1} ((\psi \land \varphi) \leftrightarrow (\psi' \land \varphi'))
	\]
	Namely, $\varphi \land \psi \in \Sigma_n(\operatorname{PA})$.

	By another identity it also derives that,
	\[
		\varphi \leftrightarrow \varphi' \equiv (\lnot \varphi) \leftrightarrow (\lnot \varphi')
	\]
	We deduce that $\Sigma_n(\operatorname{PA})$ is closed under boolean operations.

	We use the fact that $\Sigma_0$ is closed under bounded quantifiers and deduce immediately that $\Sigma_0(\operatorname{PA})$ is closed under bounded quantifiers as well.
\end{proof}

\subquestion{}
We will show that for some $n > 0$, the class of $\Sigma_n (\operatorname{PA})$ formulas are closed under bounded quantifiers, existential quantifiers, disjunctions and conjunctions.
\begin{proof}
	Let $\varphi \in \Sigma_n(\operatorname{PA})$, and suppose $\varphi' \in \Sigma_n$ testifies to that.
	$\forall v \le w \varphi' \in \Sigma_n^0$, then by the last exercise we deduce that there is $\psi \in \Sigma_n$ such that $\forall v \le w \varphi \equiv \forall v \le w \varphi' \equiv \psi$.
	$\varphi$ testifies to $\forall v \le w \varphi \in \Sigma_n(\operatorname{PA})$.
	The proof is identical for bounded existential quantifier.

	We will show that the class in closed under existential quantifiers.
	Suppose that $\varphi \in \Sigma_n(\operatorname{PA})$, and let this be testified by $\varphi' \in \Sigma_n$.
	There is $\phi \in \Pi_{n - 1}$ such that $\varphi' = \exists v \phi$.
	We define $\psi = \exists u \varphi$, then by the second part of the last question,
	\[
		\operatorname{PA}
		\vdash \forall t_0 \ldots \forall t_{n - 1} (\exists u \exists v \varphi \leftrightarrow \exists w \exists v \le w \exists u \le w \varphi)
	\]
	Then $\psi \in \Sigma_n(\operatorname{PA})$.

	We claim that $\Sigma_n(\operatorname{PA})$ is also closed under disjunction and conjunction,
	as the proof is the same as the case of $n = 0$.
\end{proof}

\subquestion{}
We will show that for $n > 0$, the class $\Pi_n(\operatorname{PA})$ is closed under bounded quantifiers, universal quantifiers, disjunctions and conjunctions.
\begin{proof}
	The proof for closeness under bounded quantifiers, as well disjunctions and conjunctions is the same as of the above, we move to show closeness under universal quantifiers.
	Let $\varphi \in \Pi_n(\operatorname{PA})$ and $\varphi' \in \Pi_n$ be the formula to testify that.
	We define $\psi = \forall v \varphi$, note that $\psi \equiv \forall v \varphi'$.
	$\varphi' = \forall u \phi$ for some $\phi \in \Sigma_{n - 1}$.
	Using part a of question 1 we deduce that $\forall u \psi \equiv \forall w \forall v \le w \forall u \le w \phi$, it is implied that indeed $\psi \in \Pi_n(\operatorname{PA})$.
\end{proof}

\subquestion{}
We will show that the negation of a $\Sigma_n(\operatorname{PA})$ formula is $\Pi_n(\operatorname{PA})$.
\begin{proof}
	In the previous exercise we had shown that the negation of $\Sigma_n^0$ formula is a $\Pi_n^0$ formula and vice versa.
	We also know that every $\Sigma_n$ formula is in particular a $\Sigma_n^0$ formula.
	Let $\varphi \in \Sigma_n(\operatorname{PA})$ and $\varphi' \in \Sigma_n$ be formulas such that,
	\[
		\operatorname{PA} \vdash \forall x_0, \ldots, \forall x_{n - 1} (\varphi \leftrightarrow \varphi')
	\]
	It follows that,
	\[
		\operatorname{PA} \vdash \forall x_0, \ldots, \forall x_{n - 1} ((\lnot \varphi) \leftrightarrow (\lnot \varphi'))
	\]
	But by the statement above, $\lnot \varphi' \in \Pi_n^0$ and there is $\varphi'' \in \Pi_n$ such that $\varphi' \equiv \lnot \varphi''$,
	which implies that,
	\[
		\operatorname{PA} \vdash \forall x_0, \ldots, \forall x_{n - 1} ((\lnot \varphi) \leftrightarrow \varphi'')
	\]
	We deduce that indeed $\varphi \in \Pi_n(\operatorname{PA})$.
\end{proof}

\subquestion{}
We will show that every formula is in $\Sigma_n(\operatorname{PA})$ or in $\Pi_n(\operatorname{PA})$ for some $n \in \NN$.
\begin{proof}
	Similarly to the last exercise, we will prove the claim by induction over the structure of the formula.
	For any atomic formula the claim holds for $n = 0$ from part a.
	For negation we have shown in the last part that there is closeness, and in parts b and c we had shown closeness for conjunctions.

	Lastly, for universal quantifiers we either use definition under $\Sigma_n(\operatorname{PA})$, and question 1 part a for formulas from $\Pi_n(\operatorname{PA})$.
	We handle existential quantifiers similarly.
\end{proof}

\question{}
TODO

\end{document}
