\newcounter{english}
\input{../article_base.tex}
\title{Exercise 5 Answer Sheet --- Logic Theory (2), 80424}

\begin{document}
\maketitle
\maketitleprint[yellow]

\question{}
\subquestion{}
Suppose that $\varphi(z, y, t_0, \ldots, t_{n - 1})$ is a formula in the language of $L_{\operatorname{PA}}$.
We will show that,
\[
	\operatorname{PA} \vdash \forall z \forall t_0 \ldots \forall t_{n - 1} (\forall x \le z \exists y \varphi \leftrightarrow \exists w \forall x \le z \exists y \le w \varphi)
\]
\begin{proof}
	Let us fix $z, t_0, \ldots, t_{n - 1}$, then is suffice to show that $\operatorname{PA} \vdash \forall x \le z \exists y \varphi \leftrightarrow \exists w \forall x \le z \exists y \le w \varphi$.
	We assume toward a contradiction that the formula does not hold, then $\forall x \le z \exists y \varphi$ as well $\lnot \exists w \forall x \le z \exists y \le w \varphi$ holds as well.
	The latter is equivalent to $\forall w \exists x \le z \forall y \le w (\lnot \varphi)$.

	We want to show that there is an upper bound to $y$ in the formula $\forall x \le z \exists y \varphi$.
	This will be proven using induction over $z$.
	For the case $z = 0$ there is only the case $x = 0$ and the $y$ that fulfills $\exists y \varphi$ for $x = 0$ is an upper bound.
	We assume that $m$ is an upper bound, meaning that $\forall x \le z \exists y \le m \varphi$ and check the case for $\forall x \le z + 1 \exists y \varphi$.
	There is $y$ witnessing the case $x = z + 1$, and either $m \le y$ or $y \le m$, we define $m' = \max\{y, m\}$ and it follows that it is an upper bound for the case $z + 1$.

	Back to the initial claim, there is $m$ by the induction above such that it is a bound for $y$ in the formula $\forall x \le z \exists y \varphi$.
	We fix $w = m$ and it derives that $\exists x \le z \forall y \le w (\lnot \varphi)$ holds, and we fix such $x \le z$, then $\forall \le w (\lnot \varphi)$.
	But by the first formula $\exists y \le w \varphi$ in contradiction to this claim.
\end{proof}

\subquestion{}
Suppose that $\psi(y_0, y_1, t_0, \ldots, t_{n - 1})$ is some formula in the language $L_{\operatorname{PA}}$.
We will show that,
\[
	\operatorname{PA}
	\vdash \forall t_0 \ldots \forall t_{n - 1} (\exists y_0 \exists y_1 \psi \leftrightarrow \exists w \exists y_0 \le w \exists y_1 \le w \psi)
\]
\begin{proof}
	The proof is very similar to part a,
	Let $t_0, \ldots, t_{n - 1}$ be some values, and let us assume toward a contradiction that $\exists y_0 \exists y_1 \psi$,
	and that $\lnot \exists w \exists y_0 \le w \exists y_1 \le w \psi \equiv \forall w \forall y_0 \le w \forall y_1 \le w (\lnot \psi)$ hold.
	We fix $y_0, y_1$ such that $\psi$ holds, and fix some $w \ge \sup\{ y_0, y_1 \}$, hence $\lnot \psi$ is derived by the second formula, resulting in a contradiction.
\end{proof}

\question{}
We define $\Sigma_n(\operatorname{PA})$ as the set of formulas $\psi(x_0, \ldots, x_{n - 1})$, such that for some $\Sigma_n$ formula $\psi'(x_0, \ldots, x_{n - 1})$,
\[
	\operatorname{PA}
	\vdash \forall x_0, \ldots, \forall x_{n - 1} (\psi \leftrightarrow \psi')
\]

\subquestion{}
We will show that the class of $\Sigma_0 (\operatorname{PA})$ formulas in PA is closed under boolean operations and bounded quantifiers.
\begin{proof}
	Let $\psi, \varphi \in \Sigma_0(\operatorname{PA})$ be some formulas.
	There exist $\varphi', \psi'$ such that,
	\[
		\operatorname{PA}
		\vdash \forall x_0, \ldots, \forall x_{n - 1} (\psi \leftrightarrow \psi'),
		\forall x_0, \ldots, \forall x_{n - 1} (\varphi \leftrightarrow \varphi')
	\]
	Then it follows that,
	\[
		\operatorname{PA}
		\vdash \forall x_0, \ldots, \forall x_{n - 1} ((\psi \leftrightarrow \psi') \land (\varphi \leftrightarrow \varphi'))
	\]
	Therefore by an identity from the previous course,
	\[
		\operatorname{PA}
		\vdash \forall x_0, \ldots, \forall x_{n - 1} ((\psi \land \varphi) \leftrightarrow (\psi' \land \varphi'))
	\]
	Namely, $\varphi \land \psi \in \Sigma_n(\operatorname{PA})$.

	By another identity it also derives that,
	\[
		\varphi \leftrightarrow \varphi' \equiv (\lnot \varphi) \leftrightarrow (\lnot \varphi')
	\]
	We deduce that $\Sigma_n(\operatorname{PA})$ is closed under boolean operations.

	We use the fact that $\Sigma_0$ is closed under bounded quantifiers and deduce immediately that $\Sigma_0(\operatorname{PA})$ is closed under bounded quantifiers as well.
\end{proof}

\subquestion{}
We will show that for some $n > 0$, the class of $\Sigma_n (\operatorname{PA})$ formulas are closed under bounded quantifiers, existential quantifiers, disjunctions and conjunctions.
\begin{proof}
	Let $\varphi \in \Sigma_n(\operatorname{PA})$, and suppose $\varphi' \in \Sigma_n$ testifies to that.
	$\forall v \le w \varphi' \in \Sigma_n^0$, then by the last exercise we deduce that there is $\psi \in \Sigma_n$ such that $\forall v \le w \varphi \equiv \forall v \le w \varphi' \equiv \psi$.
	$\varphi$ testifies to $\forall v \le w \varphi \in \Sigma_n(\operatorname{PA})$.
	The proof is identical for bounded existential quantifier.

	We will show that the class in closed under existential quantifiers.
	Suppose that $\varphi \in \Sigma_n(\operatorname{PA})$, and let this be testified by $\varphi' \in \Sigma_n$.
	There is $\phi \in \Pi_{n - 1}$ such that $\varphi' = \exists v \phi$.
	We define $\psi = \exists u \varphi$, then by the second part of the last question,
	\[
		\operatorname{PA}
		\vdash \forall t_0 \ldots \forall t_{n - 1} (\exists u \exists v \varphi \leftrightarrow \exists w \exists v \le w \exists u \le w \varphi)
	\]
	Then $\psi \in \Sigma_n(\operatorname{PA})$.

	We claim that $\Sigma_n(\operatorname{PA})$ is also closed under disjunction and conjunction,
	as the proof is the same as the case of $n = 0$.
\end{proof}

\subquestion{}
We will show that for $n > 0$, the class $\Pi_n(\operatorname{PA})$ is closed under bounded quantifiers, universal quantifiers, disjunctions and conjunctions.
\begin{proof}
	The proof for closeness under bounded quantifiers, as well disjunctions and conjunctions is the same as of the above, we move to show closeness under universal quantifiers.
	Let $\varphi \in \Pi_n(\operatorname{PA})$ and $\varphi' \in \Pi_n$ be the formula to testify that.
	We define $\psi = \forall v \varphi$, note that $\psi \equiv \forall v \varphi'$.
	$\varphi' = \forall u \phi$ for some $\phi \in \Sigma_{n - 1}$.
	Using part a of question 1 we deduce that $\forall u \psi \equiv \forall w \forall v \le w \forall u \le w \phi$, it is implied that indeed $\psi \in \Pi_n(\operatorname{PA})$.
\end{proof}

\subquestion{}
We will show that the negation of a $\Sigma_n(\operatorname{PA})$ formula is $\Pi_n(\operatorname{PA})$.
\begin{proof}
	In the previous exercise we had shown that the negation of $\Sigma_n^0$ formula is a $\Pi_n^0$ formula and vice versa.
	We also know that every $\Sigma_n$ formula is in particular a $\Sigma_n^0$ formula.
	Let $\varphi \in \Sigma_n(\operatorname{PA})$ and $\varphi' \in \Sigma_n$ be formulas such that,
	\[
		\operatorname{PA} \vdash \forall x_0, \ldots, \forall x_{n - 1} (\varphi \leftrightarrow \varphi')
	\]
	It follows that,
	\[
		\operatorname{PA} \vdash \forall x_0, \ldots, \forall x_{n - 1} ((\lnot \varphi) \leftrightarrow (\lnot \varphi'))
	\]
	But by the statement above, $\lnot \varphi' \in \Pi_n^0$ and there is $\varphi'' \in \Pi_n$ such that $\varphi' \equiv \lnot \varphi''$,
	which implies that,
	\[
		\operatorname{PA} \vdash \forall x_0, \ldots, \forall x_{n - 1} ((\lnot \varphi) \leftrightarrow \varphi'')
	\]
	We deduce that indeed $\varphi \in \Pi_n(\operatorname{PA})$.
\end{proof}

\subquestion{}
We will show that every formula is in $\Sigma_n(\operatorname{PA})$ or in $\Pi_n(\operatorname{PA})$ for some $n \in \NN$.
\begin{proof}
	Similarly to the last exercise, we will prove the claim by induction over the structure of the formula.
	For any atomic formula the claim holds for $n = 0$ from part a.
	For negation we have shown in the last part that there is closeness, and in parts b and c we had shown closeness for conjunctions.

	Lastly, for universal quantifiers we either use definition under $\Sigma_n(\operatorname{PA})$, and question 1 part a for formulas from $\Pi_n(\operatorname{PA})$.
	We handle existential quantifiers similarly.
\end{proof}

\question{}
\subquestion{}
We will show that PA proves that $\forall x \exists y > 0 (\forall 0 < z \le x (z \mid y))$.
\begin{proof}
	Let us notate $\varphi(x) = \exists y > 0 (\forall 0 < z \le x (z \mid y))$, We want to prove that $\forall x \varphi$ by induction over $x$.
	For $x = 0$ we will arbitrarily set $y = 1$, and the formula holds as there is not $0 < z \le 0$.

	Let us assume that $\operatorname{PA} \models \varphi(x)$ for some $x$, we will show that $\varphi(x + 1)$ as well.
	There is $y$ such that it satisfies the formula $\varphi(x)$, then we define $y' = y \cdot (x + 1)$.
	Let $0 < z \le x + 1$, then if $z < x + 1$ by the induction hypothesis $z \mid y$ and in particular $z \mid y \cdot (x + 1) = y'$.
	If $z = x + 1$ then $z \mid x + 1$ and in particular $z \mid y'$ as well.
	Then $y'$ satisfies $\varphi(x + 1)$, and the induction step is concluded.
\end{proof}

\subquestion{}
We will show that PA proves that for all $x$, if $y$ satisfies $\varphi$ of the last part,
and $x' < x'' \le x$ then $\alpha = (x' + 1) \cdot y + 1$ and $\beta = (x'' + 1) \cdot y + 1$ are co-primes ($\gcd(\alpha, \beta) = 1$).
\begin{proof}
	$\alpha, \beta$ are co-primes in PA if and only if $\operatorname{PA} \vdash \forall n ((n \mid \alpha \land n \mid \beta) \rightarrow n = 1)$.
	By KE we assume otherwise, and let $n > 1$ be a number that divides both $\alpha$ and $\beta$.
	Then $n \mid \beta - \alpha = (x'' - x') \cdot y$.
	$x'' - x' < x$ and by part a it is implied that $n \mid y$.
	We infer that $n \mid y \cdot (x' + 1)$ as well, and by the assumption in contradiction also $n \mid (x' + 1) \cdot y + 1$, then $n \mid \alpha - (x' + 1) \cdot y = 1$.
	We conclude that $n = 1$, but $n \ne 1$, a contradiction.
\end{proof}

\subquestion{}
We will prove that for every formula $\psi(x, y_0, \ldots, y_{n - 1})$, PA proves that for all $y_0, \ldots, y_{n - 1}$ and all $w, k, m$ such that $m$ satisfies $\varphi(k)$,
meaning that $\forall 0 < z \le k (z \mid m)$, and $w \le k$, then there is $n$ such that,
\begin{enumerate}
	\item If $x < w$ then $\psi(x, y_0, \ldots, y_{n - 1})$ if and only if $(x + 1) \cdot m + 1 \mid n$
	\item If $p$ is a prime and $p \mid n$ then there is $x < w$ such that $\psi(x, y_0, \ldots, y_{n - 1})$ and $p \mid (x + 1) \cdot m + 1$.
\end{enumerate}
\begin{proof}
	We will prove the claim by induction over $w$.

	For the basis we set $w = 0$.
	In this case there is not $x < 0$, then the first statement is vacuously true.
	We fix $n = 1$ and thus the second statement is also vacuously true.

	Let us assume that the statement holds for $w$, such that $n_0$ satisfies both of the claims, and let us examine $w + 1$.
	If we would choose $n_0$ for $w + 1$, then the first statement holds for any $x < w$, and it remains to examine $x = w$.
	$(w + 1) \cdot m + 1 \nmid n_0$ by the induction hypothesis, then if $\lnot \psi(w, y_0, \ldots, y_{n - 1})$ we can choose $n = n_0$.
	If $\psi(w, y_0, \ldots, y_{n - 1})$ then we choose $n = n_0 \cdot (w + 1) \cdot m + 1$.

	For the second statement, let $p \mid n$ for the $n$ we found above.
	By the second part of the question, there is some unique $x < w + 1$ such that $p \mid (x + 1) \cdot m + 1 \mid n$.
	By the induction hypothesis $(x + 1) \cdot m + 1 \mid n_0 \mid n$ then $\psi(x, y_0, \ldots, y_{n - 1})$ is true, concluding the statement.
\end{proof}

\question{}
\subquestion{}
We will show that there is a coded empty set, namely exist $n, m, k$ such that $\lnot FS(x, n, m, k)$ for all $x$.
\begin{proof}
	By the last question it is suffice to fix $n = 0$, as the formula to represent the empty set if $\psi(x) = \perp$ (or some other sentence like $0 = 1$).
	We also choose $k = 1$.
\end{proof}

\subquestion{}
We will show that there are coded singletons, for all $y$ there are $n, m, k$ such that for all $x$, $FS(x, m, n, k) \iff x = y$.
\begin{proof}
	We define $\psi(x, y) = x = y$, as well $k > y$ some fixed value, then by question 3 part c, when $y$ is playing the role of $y_0$ of 3c,
	there is $n$ such that $\psi(x, y)$ is and only if $x < k \land x = y $, or more simply, $x = y$.
\end{proof}

\subquestion{}
We will show that PA proves that there are coded unions for coded finite sets.
For all $n_1, m_1, k_1$ and $n_2, m_2, k_2$ there are $n, m, k$ such that for all $x$, $FS(x, m, n, k) \iff FS(x, m_1, n_1, k_1) \lor FS(x, m_2, n_2, k_2)$.
\begin{proof}
	We define $k = \max\{k_1, k_2\}$, as well $m = k! $ and $n = n_1 \cdot n_2$.
	It follows that $FS(x, m, n, k) \iff (x + 1) \cdot m + 1 \mid n = n_1 \cdot n_2 \iff (x + 1) \cdot m + 1 \mid n_1 \lor (x + 1) \cdot m + 1 \mid n_2 \iff FS(x, m_1, n_1, k_1) \lor FS(x, m_2, n_2, k_2)$.

	We could also define $\psi(x) = FS(x, m_1, n_1, k_1) \lor FS(x, m_2, n_2, k_2)$ directly and use question 3 part c.
\end{proof}

\subquestion{}
We will show that PA proves that there are coded intersections for coded finite sets.
For all $n_1, m_1, k_1$ and $n_2, m_2, k_2$ there are $n, m, k$ such that for all $x$, $FS(x, m, n, k) \iff FS(x, m_1, n_1, k_1) \land FS(x, m_2, n_2, k_2)$.
\begin{proof}
	We define $k = \min\{k_1, k_2\}$ as well $m = k! $ and $n = \gcd(n_1, n_2)$.
	The proof is almost identical to that of the last part.

	We could also define $\psi(x) = FS(x, m_1, n_1, k_1) \land FS(x, m_2, n_2, k_2)$ directly and use question 3 part c.
\end{proof}

\end{document}
