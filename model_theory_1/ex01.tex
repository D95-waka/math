\newcounter{english}
\input{../article_base.tex}
\title{Solution to Exercise 0 --- Model Theory (1), 80616}

\begin{document}
\maketitle
\maketitleprint[yellow]

\question{}
\begin{definition}
	A formula $\varphi$ is called \textit{basic Horn} formula if,
	\[
		\varphi = (\theta_0(\bar{x}) \land \cdots \land \theta_{m - 1}(\bar{x})) \to \theta_m(\bar{x})
	\]
	where $\theta_i$ is atomic for $i \le m$.
\end{definition}
\begin{remark}
	In the case $m = 0$ we get $\varphi = \theta_0(\bar{x})$.
	In the case that $\theta_m = \perp$, $\varphi \equiv \lnot \theta_0(\bar{x}) \lor \cdots \lnot \theta_{m - 1}(\bar{x})$.
\end{remark}
\begin{definition}
	The set of Horn formulas is the minimal set of formulas containing the basic Horn formulas and closed under conjugation and quantification.
\end{definition}

\subquestion{}
Let $\psi(x_0, \ldots, x_{n - 1})$ be a Horn formula and let $F \subseteq \Pp(I)$ be a filter.
Let $\langle \Mm_i \mid i \in I \rangle$ be a sequence of structures and $a_j \in \prod_{i \in I} \Mm_i$ for $j < n$ such that,
\[
	\{ i \in I \mid \Mm_i \models \psi(a_0(i), \ldots, a_{n - 1}(i)) \} \in F
\]
We will show that $\Nn = \prod_{i \in I} \Mm_i / F \models \psi({[a_0]}_F, \ldots, {[a_{n - 1}]}_F)$.
\begin{proof}
	Let us prove by induction over Horn set.

	Assume that $\varphi(\bar{x}) = P(\bar{x})$ and $\{ i \in I \mid \Mm_i \models P(\bar{x}) \} \in F$, then by definition $\Nn \models \varphi$.
	Let us assume that $\varphi = (\theta_0 \land \cdots \land \theta_{n - 1}) \to \theta_n$ for atomic $\theta_n$, then the claim holds directly from definition as in the last part.

	The case where $\theta_m = \perp$ is equivalent as $\varphi \equiv \theta_0 \lor \cdots \lor \theta_{m - 1}$.

	We move to assume that the claim holds for $\varphi, \psi$ and prove for $\varphi \land \psi$.
	This part of the proof is identical to the proof of Łoś theorem,
	\[
		\{ i \in I \mid \Mm_i \models \varphi(\bar{a}) \land \psi(\bar{a}(i)) \}
		= \{ i \in I \mid \Mm_i \models \varphi(\bar{a}(i)) \} \cap \{ i \in I \mid \Mm_i \models \psi(\bar{a}(i)) \} \in F
		\implies \Nn \models \varphi, \psi
		\implies \Nn \models \varphi \land \psi
	\]
	where the second equation is derived from filter definition.

	We move to the case $\varphi = \exists x\ \psi$ for $\psi$ that fulfills the claim.
	If $J = \{ i \in I \mid \Mm_i \models \varphi(\bar{a}(i)) \} \in F$ then for $j \in J$ we define $b_j \in M_j$ as a witness to $\Mm_j \models \psi(b_j, \bar{a}(j))$, meaning that,
	\[
		\{ i \in I \mid \Mm_i \models \psi(b(i), \bar{a}(i)) \} = J \in F
	\]
	where $b(i) = b_i$ for $i \in J$ and arbitrary otherwise.
	Then by the induction hypothesis $\Nn \models \psi([b], \bar{[a]})$ and therefore $\Nn \models \varphi(\bar{[a]})$.

	Lastly we will assume that $\varphi = \forall x\ \psi$ for $\psi$ that fulfills the claim and show that $\varphi$ does as well. \\
	If $J = \{ i \in I \mid \Mm_i \models \varphi(\bar{a}(i)) \} \in F$, then if $b : I \to \bigcup M_i$ some choice function then,
	\[
		\{ i \in I \mid \Mm_i \models \psi(b(i), \bar{a}) \} \supseteq J \in F
	\]
	therefore $\Nn \models \psi([b], \bar{[a]})$, then $\Nn \models \varphi(\bar{[a]})$ as required.
\end{proof}

\subquestion{}
We will find a language and a sentence $\varphi$ such that for any set $I$ and filter $F$, \\
there is $\langle \Mm_i \mid i \in I \rangle$ such that $\Nn = \prod_{i \in I} \Mm_i / F \models \varphi$ if and only if $F$ is an ultrafilter.
\begin{solution}
	Define $L = \{ = \}$ as the trivial language, and $\varphi = \exists x \exists y\ (x \ne y \land \forall z\ (z = x \lor z = y))$. \\
	Let $I$ be some indices set and $F \subseteq \Pp(F)$.
	Let $M = \{0, 1\}$ and $\Nn = \Mm^I / F$.

	If $F$ is an ultrafilter then by Łoś theorem $\Nn \models \varphi$.
	Otherwise there is a set $J \subseteq I$ such that $J \notin F, I \setminus J \notin F$.
	Define,
	\[
		f(x)
		= \begin{cases}
			0 & x \in J \\
			1 & x \in I \setminus J
		\end{cases}
	\]
	then $\Nn \models [f] \ne [c_0]$ as well $\Nn \models [f] \ne [c_1]$, and thus $\Nn \models \lnot \varphi$.
\end{solution}

\subquestion{}
Let $\Mm_n$ be finite models in the language of equality, let $\Uu \subseteq \Pp(\omega)$ be a non-principal ultrafilter, and define $\Mm_{\omega} = \prod \Mm_n / \Uu$.
We will show that $M_{\omega}$ is either finite or uncountable.
\begin{proof}
	Let us assume for contradiction that $|M_{\omega}| = \omega$ and let $\langle f_n : \omega \to \bigcup M_i \mid n < \omega \rangle$ be sequence such that $\{ [f_n] \} = M_{\omega}$.
	Let $B_0 = \{ n < \omega \mid f_0(n) \ne f_1(n) \} \in \Uu$, and by recursion for each $B_k$ we define,
	\[
		B_{k + 1} = B_k \cap \{ n < \omega \mid \forall i \le k,\ f_i(n) \ne f_{k + 1}(n) \} \in \Uu
	\]
	Then $\{ B_k \} \subseteq \Uu$ and $B_k \supseteq B_{k + 1}$ for any $k$, and let $B = \inf B_n$.
	If $B = \emptyset$ then there is minimal $k < \omega$ such that $B_k \ne \emptyset$, but $B_k, B_{k + 1} \in \Uu$, therefore $\emptyset \in \Uu$, a contradiction.
	Then there is $b \in B$, then $S = \{ f_n(b) \mid n < \omega \}$ is a set such that $|S| = \aleph_0$, but $S \subseteq M_b$ and $|M_b| < \aleph_0$.
	We conclude that $|M_{\omega}|$ cannot be countable.

	We will show that $|M_{\omega}|$ if and only if $|M_n| < K$ for $n \in J \in \Uu$ and $K < \omega$.
	Let us assume that $\forall n \in J,\ |M_n| < K$, and define,
	\[
		\varphi_K = \forall x_0 \cdots \forall x_{K - 1}\ \left(\bigvee_{i < j < K} x_i = x_j\right)
	\]
	then $\Mm_n \models \varphi_K$ for any $n \in J$, therefore by Łoś we get $\Mm_{\omega} \models \varphi_K$ as well, in particular $|M_{\omega}| < K$. \\
	In the other direction the claim holds similarly by Łoś.

	We will show that $|M_{\omega}| = 2^{\aleph_0}$ if $|M_{\omega}|$ is uncountable.
	By the last claim, $\Mm_{\omega} \models \lnot \varphi_K$ for any $K < \omega$, meaning that $|M_{\omega}|$ is unbounded.
	Let us assume that $h : \omega \to \omega$ is a function such that $|M_{h(n)}| \ge n$.
	We can also define $h_n : n \to M_n$ by the last cardinality inequality (and more choice).
	It is known that the cardinality of functions $\omega \to \omega$ that strictly increasing is $2^{\aleph_0}$, then it suffices to show that any such function $g$ can be mapped uniquely to $[f] \in M_{\omega}$.
	We can define $f' = \{ \langle h(g(n)), h_{h(g(n))}(g(n)) \}$ and,
	\[
		f(n)
		= \begin{cases}
			f'(n) & n \in \dom f' \\
			h_n(0) & \text{otherwise}
		\end{cases}
	\]
	Then $[f] \in M_{\omega}$.
	If $g, g'$ are two different strictly increasing functions, and $f, f'$ are their respective constructions, then $[f] \ne [f']$ directly from definition.
	We deduce that $2^{\aleph_0} \le |M_{\omega}| \le |\omega^{\omega}| = |2^{\aleph_0}|$.
\end{proof}

\question{}
\subquestion{}
Suppose that $T$ has quantifier elimination and that there us a model $\Mm_0$ that embeds in every model of $T$. \\
We will show that $T$ is complete.
\begin{proof}
	
\end{proof}

\end{document}
