\newcounter{english}
\input{../article_base.tex}
\title{Solution to Exercise 3 --- Model Theory (1), 80616}

\begin{document}
\maketitle
\maketitleprint[yellow]

\question{}
\subquestion{}
We will show that there is a collection $X \subseteq \Pp(\omega)$ such that $|X| = 2^{\aleph_0}$ and if $x \ne y \in X$ then $x \cap y$ is bounded.
\begin{proof}
	For any $A \subseteq \omega$ infinite let us define $Y(A) = \{ \sum_{n \in A \cap m} 2^n \mid m < \omega \}$, then $X(A) \in \Pp(\omega)$.
	It follows that $Y : \omega^\omega \to \Pp(\omega)$ is a function, note that it is injective, as if $A \ne B \in \dom Y$ then there is $m \in A \setminus B$, witnessing $Y(A) \ne Y(B)$.
	The injectivity of $Y$ implies that $|\im Y| = |\Pp(\omega)|$, and let $X = \im Y$.

	We move to show that if $A' \ne B' \in X$ then $A' \cap B' < \alpha$ for some $\alpha < \omega$.
	Let $A, B \subseteq \omega$ be elements such that $Y(A) = A', Y(B) = B'$.
	$A \ne B \implies \exists m < \omega,\ k \in A, k \notin B$ without loss of generality.
	Let $k < m < \omega$, then it suffices to show that $\sum_{n < m} 2^n (\indicator_A(n) - \indicator_B(n)) \ne 0$,
	but as a result from number theory and by the fact that $A \cap m \ne B \cap m$ the expression indeed does not nullify.
	We deduce that $|A' \cap B'| \le k < \omega$ as wished.
\end{proof}

\subquestion{}
Let $\Ll = \{0, S, \le\} \cup \{ R_X \mid X \subseteq \omega \}$ be an uncountable language such that $0$ is constant symbol, $S$ is unary function symbol,
$\le$ is binary relation symbol and $R_X$ is an unary relation symbol for all $X \subseteq \omega$. \\
Let $\Aa$ be a model over $\Ll$ with $A = \omega, 0^\Aa = 0, S^\Aa(n) = n + 1, x \le^\Aa y \iff x \in y$ and $R_X^\Aa(n) \iff n \in X$, and let $T = \Th(\Aa)$. \\
We will show that $T$ is $\omega$-categorical, that it has infinitely many nonequivalent formulas and that there are infinitely many non-isolate types in $S_1(T)$.
\begin{proof}
	Let $\Mm \models T$ be some countable model, we will show that $\Mm \cong \Aa$.
	We define recursively $f : A \to M$ by $f(0) = 0^\Mm$ and $f(n + 1) = S^\Mm(f(n))$.
	It follows that $f$ preserves $0, S$, it remains to show it also preserves $\le, R_X$.
	The claim that $f$ preserves $\le$ can be shown using double induction by fixing each $n$ and proving $n \le m \implies f(n) \le f(m)$.
	Definability of $n \in \Aa$ as $S^n(0)$ and the fact for each $X$, $R_X^\Aa(n) \iff S^n(0) \in X$ implies that $R_X$ is preserved under $f$ as well. \\
	$\Aa$ fulfills the axiom scheme of induction, meaning that $T$ does as well. Each $X \subseteq A$ is definable using $R_X$,
	meaning that the sentence $(0 \in \NN \land (x \in \NN \to S(x) \in \NN)) \to \forall x \in \NN$ exists in $T$ (to be precise it is symbolically appears and not the actual form written) where $\NN$ is some $X$.

	Let $\varphi_n(x) = (x = S^{\underline{n}}(0))$, as an informal way to define that $\varphi_0(x) = (x = 0), \varphi_1 = (x = S(0))$ and so on.
	$\Aa \models \forall x\ x \ne S(x)$, then we can deduce that $\varphi_n \not\equiv \varphi_m$ for all $n < m < \omega$.
	Then $\{ \varphi_n \mid n < \omega \} \subseteq T$ is a set of non-equivalent formulas.

	We have shown that $f$ is model embedding, and now we will show that it is also surjective.
	Let $\gamma \in M$ be a non-standard element, namely $\Mm \models \lnot \varphi_n(\gamma)$ for all $n < \omega$.
	By $f$'s construction, $\gamma \notin \im f$, and therefore also $\underline{n} \gamma \in M \setminus \im f$ as well.
	Let $X_0 \in X$ where $X$ is from the last part, then $X_0' = \{ \underline{n} \gamma \mid n \in X_0 \}$ is subset of $M$.
	Let $X_0 \ne X_1 \in X$ be some other set, and let us define $X_1'$ accordingly.
	Let $g : X \to M^{< \omega}$ by $X_1 \mapsto X_0' \cap X_1'$, but $|M^{< \omega}| = |M| = \omega$ but $g$ is injective and $|X| = 2^{\omega}$, a contradiction.
	We deduce that $f$ is a bijection.

	Let $g : \omega \to \{0, 1\}$, $|g^{-1}(1)| = |g^{-1}(0)| = \omega$ be a function and let $Q = {\{ q_n \}}_{n = 1}^\infty$ be the set of the primes.
	Let $X_q$ be the set such that $x \in X_q \iff q | x$, note that divisibility is definable, and let $R^i = R_{X_{q_i}}$ for any $i < \omega$.
	\[
		p_g'(x)
		= \{ x \ne \underline{n} \mid n < \omega \}
		\cup \{ R^i(x) \mid g(i) = 1 \}
		\cup \{ \lnot R^i(x) \mid g(i) = 0 \}
	\]
	$p_g'$ is consistent from the compactness theorem, and let $p_g$ be its closure, such that $p_g$ is complete and consistent.
	$p_g$ cannot be isolated as otherwise $p_g'$ can be isolated as well, a contradiction to $g$'s definition by taking the minimal $i < \omega$ such that $R^i$ does not show up in isolating formula.
	There are infinitely many such functions $g$, implying that there are also infinitely many such types $p_g$.
	To be precise there are $2^{\aleph_0}$ such functions, then $2^{\aleph_0}$ such non-isolated types.
\end{proof}

\question{}
Let $\omega \le \kappa$ be some cardinal.
We say that a model $\Mm$ is $\kappa$-saturated if for every $A \subseteq M$ with $|A| < \kappa$, any type in $S_1(A)$ is realized.

\subquestion{}
Let $T$ be a consistent complete theory over a language of size $\le \kappa$ with an infinite model. \\
We will show that there is a $\kappa^+$ saturated model of $T$ of cardinality $2^\kappa$.
\begin{proof}
	By Löwenheim-Skolem theorem we can assume that there is a model $\Mm \models T$ of cardinality $\kappa$.
	We will construct recursively a $\kappa$-saturated model, by recursion on $|A|$ for $A \subseteq M$.
	For $|A| = \emptyset$ we get $A = \emptyset$, $|\form| \le \kappa$ implies that $S_1(\emptyset) = S_1(T)$ is of cardinality $\le 2^\kappa$ as well.
	Let $\Mm_0' \models T \cup \{ p(c_p) \mid p \in S_1(\emptyset) \}$ be a model of the language $\Ll \cup \{ c_p \mid p \in S_1(\emptyset) \}$ and let $\Mm_0$ be its reduction to $\Ll$.
	$|M_0| \le 2^{\kappa}$ and it is $1$-saturated.

	Let us assume that $\alpha < \kappa$ is some cardinal and that $\Mm_{\alpha} \models T$ is a $\alpha^+$-saturated model with $|M_{\alpha}| \le 2^{\kappa}$.
	$|\Pp_{= \alpha}(M_{\alpha})| < \kappa$ and $|\form_{\Ll(A)}| < 2^\kappa$ then $|S_1(A)| < 2^\kappa$ for any such $A$ as well, meaning that,
	\[
		\Sigma_{\alpha} = \bigcup_{\substack{A \subseteq M_{\alpha} \\ |A| = \alpha}} S_1(A)
	\]
	is of cardinality $\le 2^\kappa$ as well.
	We enrich $\Ll$ by $\{ c_p \mid p \in \Sigma_{\alpha} \}$ and define $\Mm_{\alpha^+}' \models T \cup \{ p(c_p) \mid p \in \Sigma_{\alpha} \}$ be a models, $\Mm_{\alpha^+}$ be its reduction to $\Ll$.
	Then $\Mm_{\alpha^+}$ is $\alpha^{++}$-saturated and $|M_{\alpha^+}| \le 2^\kappa$.

	$\kappa^+$ is regular, thus $\Mm_{\kappa^+} = \bigcup_{\alpha < \kappa^+} \Mm_{\alpha}$ is of cardinality $2^\kappa$ and $\kappa^+$-saturated as wished.
\end{proof}

\subquestion{}
We will show that if $\Mm$ and $\Nn$ are elementary equivalent $\kappa$-saturated models of cardinality $\kappa$ then $\Mm \cong \Nn$,
and that any partial elementary map $f : A \to B$ with $A \subseteq M, B \subseteq N, |A| < \kappa$ can be extended to an isomorphism.
\begin{proof}
	The proof is similar to the case of $\omega$-saturation, the second statement implies the first one and it suffices to show it.
	The proof is by induction on ordinals $< \kappa$, the proof for the case of $\kappa = \omega$ can be used as a base for the induction.

	Without loss of generality we can assume $A, B = \alpha$ for $\alpha < \kappa$ by the well order principle.
	The case of successor ordinal is trivial from the case $\omega$.
	Let us assume that the statement is true for any $\alpha < \beta < \kappa$.
	Then there exists $f_{\alpha}$ and if $\alpha < \gamma < \beta$, $f_{\gamma} \restriction \alpha = f_{\alpha}$, then let us define $f_{\beta} = \bigcup_{\alpha < \beta} f_{\alpha}$.
	The induction step is completed therefore the statement holds for any $\alpha < \kappa$, and by an identical step there is also a function $f = f_{\kappa}$ fulfilling our requirements.
\end{proof}

\subquestion{}
We will show that if $\Mm \models T$ is $\kappa$-saturated model of a complete theory $T$, then every model $\Nn \models T$ of size $\kappa$ can be embedded into $\Mm$.
\begin{proof}
	We will construct an elementary embedding by recursion.
	Let $f_0 = \{ \langle d^\Nn, d^\Mm \rangle \mid d \in L, d \text{ is constant symbol} \}$, this is an elementary embedding by $T$'s completeness.

	Let $f_{\delta} : A \to B$ be a partial elementary embedding, $A \subseteq N, B \subseteq M$, $A = {\{ a_i \}}_{i < \delta}$.
	Let $a \in N \setminus A$ and let $p = tp(a / N_A)$	be the complete type such that $\Nn \models p(a)$.
	Then $q = f_{\delta}(p) =  p_{f(a_0), \ldots}^{a_0, \ldots}$ is a type $\in S_1(B)$.
	$p$ is consistent, therefore any $p' \subseteq p$ finite is consistent, $f_{\delta}(p') \subseteq q$ is consistent by truth-value perseverance of $f_{\delta}$, then $q$ is finitely satisfiable and therefore satisfiable.
	$f_{\delta}$ preserves truth-value of sentences in $L(A)$ and $p$ is complete, then $q$ is complete as well.
	By $\kappa$-saturation of $\Mm$ there exists $b \in M$ such that $\Mm \models q(b)$, let $f_{\delta + 1} = f_{\delta} \cup \{ \langle a, b \rangle \}$.

	Let $\delta$ be a limit ordinal and let us assume that $f_{\alpha}$ is defined for $\alpha < \delta$, such that $f_{\alpha} \subseteq f_{\beta}$ for all $\alpha < \beta < \delta$.
	Then $f_{\delta} = \bigcup_{\alpha < \delta} f_{\alpha}$ is defined and acts as elementary embedding as any term or formula are of finite length and thus embedded correctly in some $\alpha < \delta$.

	We get that $f_{\alpha}$ exists for each $\alpha \le \kappa$, note that by the definition of $\kappa$-saturation this process is not guaranteed continue after reaching $\kappa$.
	Note that by using the well order principle we can assume that $N = \kappa$ and therefore $f_{\kappa}$ is a total function, and therefore an embedding $\Nn \hookrightarrow \Mm$.
\end{proof}

\end{document}
