\newcounter{english}
\input{../article_base.tex}
\title{Solution to Exercise 0 --- Model Theory (1), 80616}

\begin{document}
\maketitle
\maketitleprint[yellow]

\question{}
Let $L = \{ P \}$ a language where $P$ is unary relation.
Define,
\[
	\varphi_n
	= \exists x_0 \ldots \exists x_n \left( \bigwedge_{i \le n} P(x_i) \land \bigwedge_{i < j \le n} x_i \ne x_j \right),
	\quad
	\psi_n
	= \exists x_0 \ldots \exists x_n \left( \bigwedge_{i \le n} \lnot P(x_i) \land \bigwedge_{i < j \le n} x_i \ne x_j \right)
\]
and let $T = \{ \varphi_n, \psi_n \mid n < \omega \}$. \\
We will show that $\cl_{\vdash} T$ is $\omega$-categorical.
\begin{proof}
	Let $\Mm \models T$ be some model.
	It can be proved by direct induction that $|P^\Mm| = \omega$ as well as $|\lnot P^\Mm| = \omega$.
	Let us construct $f : \omega \to M$ such that $f(n) \in P^\Mm$ for any $n < \omega$.
	$\Mm \models \varphi_0 \iff \Mm \models \exists x\ P(x)$ then let $f(0)$ be such witness.
	Let us assume that $f \restriction n$ is defined, then $\Mm \models \varphi_{n + 1}$, then by the pigeonhole principle there is some $a \in \Mm$ such that $a \notin f '' n$, and let $f(n + 1) = a$.
	For the sake of convenience let us redefine $f$ as $2 \times \omega \to M$ injective function such that $f(0, n)$ is the same as $f(n)$ and $f(1, n) \notin P^\Mm$.
	By CSB we can assume that $f$ is bijection as well, and by the selection of $\Mm$ as an arbitrary model of $T$ we can deduce that for any $\Mm, \Nn \models T$, $\Mm \cong \Nn$ by composition of functions as was constructed.
\end{proof}

\question{}
Let $L = \{ c_n \mid n < \omega \}$ be language consists of constant symbols.
Let us define the theory $T = \{ c_i \ne c_j \mid i < j < \omega \}$. \\
We will show that there are countably many non-isomorphic countable models of $T$, and that $T$ is complete.
\begin{proof}
	Let us define the model $\Mm_n$ such that $M = \omega$ and,
	\[
		c_i^\Mm = i + n
	\]
	for any $i < j < \omega$,
	\[
		c_i^\Mm
		= i + n
		\ne j + n
		= c_j^\Mm
	\]
	therefore $\Mm_n \models T$.
	$\Mm_n \models k \ne c_i$ for all $i < \omega$, in particular $\Mm_n \models \exists x\ x \ne c_i$.
	It is implied that also,
	\[
		\Mm_n
		\models \exists x_0 \ldots \exists x_{k - 1} \left( \bigwedge_{i < j < k} x_i \ne x_j \land x_i \ne c_l \right)
		= \varphi_l^k
	\]
	for all $l < \omega$.
	Finally, $\Mm_n \not\models \varphi_l^k$ for any $k > n$, we deduce that $\Mm_n \not\cong \Mm_m$ for any $n \ne m$.

	We move to show that $T$ is complete.
	Let us assume toward a contradiction that $\varphi$ is a sentence such that $\varphi \notin T$ and $T \cup \{ \varphi \}$ is consistent.
	By construction of Henkin models we can deduce that $\Mm_0 \models \varphi$, but $\Mm_0$ is minimal, namely if $\Nn \models T$ then $\Mm_0 \subseteq \Nn$,
	then by definition $T \models \varphi$, a contradiction.
\end{proof}

\question{}
We will show that $T = \Th(\NN, +, \cdot)$ has $2^{\aleph_0}$ non-isomorphic countable models.
\begin{proof}
	Observe the fact that numbers are definable in $T$, by formula as such,
	\[
		\varphi_{n}(x) = \forall y\ x \cdot y = \overbrace{y + \cdots + y}^{\text{$n$ times}}
	\]
	If $\Mm \models T$ then we denote by $\underline{n}$ the single element of $M$ that fulfills $\varphi_{n}$. \\
	By the fact that $\exists x\ \varphi_n(x) \in T$ it follows that $\{ \underline{n} \mid n < \omega \} \subseteq M$ for any such model. \\
	We also denote by $x \mid y$ the formula $\exists z\ x \cdot z = y$.

	Let $P \subseteq \NN$ be the set of prime numerals, namely $\varphi(x) = \forall y,\ (y \mid x \to (y = x \lor y = \underline{1}))$.
	We add new constant symbol $c$ to the language of $T$, and let $P' \subseteq P$ be some infinite set of primes.
	Let us define a new theory,
	\[
		T' = T \cup \{ \underline{p} \mid c \mid p \in P' \} \cup \{ \underline{p} \nmid c \mid p \notin P' \}
	\]
	For any $T_0' \subseteq T'$ finite, either $T_0' \subseteq T$ and satisfiable or $\{ c \mid \underline{p} \mid p \in P_0' \} \in T_0'$ for some finite $P_0' \subseteq P'$,
	and then $\NN \models \prod_{p \in P_0'} \underline{p}$.
	From the compactness theorem we conclude that $T'$ is satisfiable and let $\Mm_{P'}' \models T'$ be a witness.
	We can now remove the constant symbol $c$ and get a model $\Mm_{P'} \models T$.

	By downwards Löwenheim Skolem theorem we can assume that $|M_{P'}| = \omega$ for any $P' \subseteq P$.
	Let us assume that $|\{ \Mm_{P'} \mid P' \subseteq P, |P'| = \omega \}| < 2^{\omega}$, then there must be model $\Nn$ such that it has non-countable non-standard elements, in contradiction to being countable.
	Then $|\{ \Mm_{P'} \mid P' \subseteq P, |P'| = \omega \}| \ge 2^{\omega}$.
\end{proof}

\question{}
Let $\kappa \ge \omega$ be some cardinal and let $L$ be some language.
Let $T$ be a $\kappa$-categorical $L$-theory such that it has no finite models.
We will show that $T$ can be incomplete.
\begin{solution}
	Let $L = \{ c_{\alpha} \mid \alpha < \delta \} \cup \{ P \}$ for $\kappa < \delta$, where $c_{\alpha}$ is a constant symbol and $P$ is unary relation.
	\[
		T = \{ c_{\alpha} \ne c_{\beta} \mid \alpha < \beta < \delta \}
	\]
	It follows from the definition of $T$ that if $\Mm \models T$ then $|M| \ge \delta > \kappa$, therefore there are no models of $T$ of cardinality $\kappa$, then the theory is vacuously $\kappa$-categorical.
	$T$ is not complete, as $P(c_0) \notin T$ as well $\lnot P(c_0) \notin T$.
\end{solution}

\question{}
Let $T = \Th(\QQ, \le)$ be DLO\@. \\
We will show that $T$ is not $\kappa$-categorical for some uncountable cardinal $\kappa$.
\begin{proof}
	Define $(\RR, \le)$ and $(\RR + \QQ, \le)$, these are both models of DLO, and let us assume that $f : \RR + \QQ \to \RR$ is model isomorphism and thus also an order isomorphism.
	Let $y = f(\langle 1, 0 \rangle)$, then,
	\[
		|\{ x \ge y \mid x \in \RR \}|
		= 2^{\omega}
	\]
	but
	\[
		|\{ f^{-1}(x) \ge^{\RR + \QQ} \langle 1, 0 \rangle \mid x \in \RR \}|
		\le |\QQ|
		= \omega
	\]
	a contradiction, then $(\RR, \le) \not\cong (\RR + \QQ, \le)$.
\end{proof}

\end{document}
