\newcounter{english}
\input{../article_base.tex}
\title{Solution to Exercise 4 --- Model Theory (1), 80616}

\DeclareMathOperator{\acl}{acl}
\DeclareMathOperator{\dcl}{dcl}

\begin{document}
\maketitle
\maketitleprint[yellow]

\question{}
\begin{definition}[algebraic formula]
	Let $\Mm$ be a structure. A formula $\varphi(x) \in \form_{L_\Mm}$ is called algebraic if,
	\[
		|\{ a \in M \mid \Mm \models \varphi(a) \}| < \aleph_0
	.\]
\end{definition}
\begin{definition}[definable element]
	An element $a \in M$ is called definable if there is a formula $\psi(x)$ such that $\Mm \models \psi(x) \iff x = a$ for any $x \in M$.
\end{definition}
\begin{definition}[acl and dcl]
	If $\Mm$ is a structure and $A \subseteq M$ a set of elements, then let,
	\[
		\acl_\Mm(A) = \bigcup \{ \varphi(M) \mid \varphi \in \form_{L_A}, \varphi \text{ is algebraic} \},
		\quad
		\dcl\Mm(A) = \{ a \in M \mid a \text{ is definable using an $L_A$ formula} \}
	.\]
	be the set of elements that has a formula with parameters from $A$ that is algebraic and are true for them, and the set of elements uniquely definable by formulas with parameters.
\end{definition}

\subquestion{}
Let $A \subseteq M$ for $\Mm \prec \Nn$.
We will show that $\acl_\Mm(A) = \acl_\Nn(A)$ and $\dcl_\Mm(A) = \dcl_\Mm(A)$.
\begin{proof}
	Let $a \in M$ be $\in \acl_\Mm(A)$ and let $\varphi \in \form_{L_A}$ be an algebraic formula to witness it.
	$\Mm \models \varphi(a) \iff \Nn \models \varphi(a)$ by the given elementary embedding and the fact that $A \subseteq N$. \\
	Let $a \in \acl_\Nn(A)$ and assume toward contradiction that $a \notin \acl_\Mm(A)$, meaning that there is $\varphi(x)$ over $L_A$ such that $\Nn \models \varphi(a)$ and $\varphi$ is algebraic.
	Let $B = A \cup \varphi(M)$, denote $\varphi(M) = \{ b_i \mid i < N \}$ for $N < \omega$,
	\[
		\psi(x)
		= \varphi \land \left(\bigwedge_{i < N} x \ne b_i\right)
	.\]
	Then $\psi$ is algebraic over both $\Mm$ and $\Nn$ and $\psi(M) = \emptyset$, while $a \in \psi(N)$.
	Let $\phi = \exists x\ \psi$, then $\Nn \models \phi$ as $a$ witnesses exactly this, while $\Mm \models \lnot \phi$ in contradiction to $\Mm \models \phi \iff \Nn \models \phi$.

	Let us move to $\dcl$.
	TODO
\end{proof}

\end{document}
