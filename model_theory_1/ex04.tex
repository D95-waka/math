\newcounter{english}
\input{../article_base.tex}
\title{Solution to Exercise 4 --- Model Theory (1), 80616}

\DeclareMathOperator{\acl}{acl}
\DeclareMathOperator{\dcl}{dcl}

\begin{document}
\maketitle
\maketitleprint[yellow]

\question{}
\begin{definition}[algebraic formula]
	Let $\Mm$ be a structure. A formula $\varphi(x) \in \form_{L_\Mm}$ is called algebraic if,
	\[
		|\{ a \in M \mid \Mm \models \varphi(a) \}| < \aleph_0
	\]
\end{definition}
\begin{definition}[definable element]
	An element $a \in M$ is called definable if there is a formula $\psi(x)$ such that $\Mm \models \psi(x) \iff x = a$ for any $x \in M$.
\end{definition}
\begin{definition}[acl and dcl]
	If $\Mm$ is a structure and $A \subseteq M$ a set of elements, then let,
	\[
		\acl_\Mm(A) = \bigcup \{ \varphi(M) \mid \varphi \in \form_{L_A}, \varphi \text{ is algebraic} \},
		\quad
		\dcl\Mm(A) = \{ a \in M \mid a \text{ is definable using an $L_A$ formula} \}
	\]
	be the set of elements that has a formula with parameters from $A$ that is algebraic and are true for them, and the set of elements uniquely definable by formulas with parameters.
\end{definition}

\subquestion{}
Let $A \subseteq M$ for $\Mm \prec \Nn$.
We will show that $\acl_\Mm(A) = \acl_\Nn(A)$ and $\dcl_\Mm(A) = \dcl_\Mm(A)$.
\begin{proof}
	Let $a \in M$ be $\in \acl_\Mm(A)$ and let $\varphi \in \form_{L_A}$ be an algebraic formula to witness it.
	$\Mm \models \varphi(a) \iff \Nn \models \varphi(a)$ by the given elementary embedding and the fact that $A \subseteq N$. \\
	Let $a \in \acl_\Nn(A)$ and assume toward contradiction that $a \notin \acl_\Mm(A)$, meaning that there is $\varphi(x)$ over $L_A$ such that $\Nn \models \varphi(a)$ and $\varphi$ is algebraic.
	Let $B = A \cup \varphi(M)$, denote $\varphi(M) = \{ b_i \mid i < N \}$ for $N < \omega$,
	\[
		\psi(x)
		= \varphi \land \left(\bigwedge_{i < N} x \ne b_i\right)
	\]
	Then $\psi$ is algebraic over both $\Mm$ and $\Nn$ and $\psi(M) = \emptyset$, while $a \in \psi(N)$.
	Let $\phi = \exists x\ \psi$, then $\Nn \models \phi$ as $a$ witnesses exactly this, while $\Mm \models \lnot \phi$ in contradiction to $\Mm \models \phi \iff \Nn \models \phi$.

	The case for $\dcl$ is equivalent, we take $\varphi(x)$ such that $\varphi(M) = \{ a \}$ and therefore $\varphi(N) = \{ a \}$ as otherwise we can define $\varphi \land (x \ne a)$.
\end{proof}
We conclude that the relative model of $\acl$ and $\dcl$ can be omitted.

\subquestion{}
Let $A \subseteq M$, we will show that $\acl(\acl(A)) = \acl(A)$ and $\dcl(\dcl(A)) = \dcl(A)$.
\begin{proof}
	Let $a \in \dcl(\dcl(A))$ be an element and let $\varphi(x)$ be its witness.
	Suppose that $\{ b_i \mid i < N \}$ is the set of elements that $\varphi$ consists of such that they are not in $A$.
	$b_i \in \dcl(A)$ for any $i$, and let $\theta_i(x)$ witness that.
	Let $\{ y_i \mid i < N \}$ be a set of distinct and disjoint to $\varphi$ variables, and let us define,
	\[
		\varphi' = \varphi_{y_0, \ldots, y_{N - 1}}^{b_0, \ldots, b_{N - 1}},
		\quad
		\phi(x) = \exists y_0 \cdots \exists y_{N - 1}\ \left(\bigwedge_{i < N - 1} \theta(y_i)\right) \land \varphi'
	\]
	then $\phi \in \form_{L_A}$ and $\phi(M) = \{ a \}$, therefore $a \in \dcl(A)$ and thus $\dcl(\dcl(A)) = \dcl(A)$.

	The case of $\acl$ is equivalent.
\end{proof}

\subquestion{}
Let $L = \{0, 1, +, \cdot\}$ and $K$ be an algebraically closed field and let $A \subseteq K$. \\
We will compute $\acl(A)$ and $\dcl(A)$.
\begin{solution}
	$K$ has quantifier elimination therefore we can omit the discussion about quantifiers.
	Each formula $\varphi(x) \in \form_{L_A}$ is (without loss of generality) of the form,
	\[
		\varphi
		= (\bigwedge_{i < n} p_i(x) = 0) \land (\bigwedge_{j < m} p_{n + j}(x) \ne 0)
	\]
	where $p_i(x)$ is a polynomial over $A$.
	An element $a \in K$ is $\in \dcl(A)$ if it is the unique solution of some such polynomial.
	Each polynomial in algebraically closed field can be deconstructed to product of linear polynomials, meaning that $p_i(x)$ has a single solution if $p_i(x) = (x - a)$.
	We conclude that $\dcl(A) = \cl_{+, \cdot}(A)$.

	By similar proposition, $\acl(A) = \langle A \rangle \le K$, meaning the subfield generated by $A$.
\end{solution}

\question{}
\subquestion{}
Let $\kappa$ be an infinite cardinal and let $\lambda \le \kappa$ be the minimal cardinal such that $\kappa^\lambda > \kappa$.
Consider the order $(\kappa^{< \lambda}, <_l)$ the lexicographic order on $\kappa^{< \lambda}$. \\
We will show that this order has more than $\kappa$ many cuts and construct dense linear order of cardinality $\kappa$ with $> \kappa$ different cuts.
\begin{proof}
	A cut is defined as a collection $X \subseteq \kappa^{< \lambda}$ such that if $y \in \kappa^{< \lambda}$ and there is $x \in X$ such that $y < x$, then $y \in X$.
	Let $\delta \in \kappa^\lambda$ be a sequence, and let,
	\[
		\eta_{\delta} = \{ x \in \kappa^{< \lambda} \mid \exists \alpha < \lambda,\ x < \delta \restriction \alpha \}
	\]
	notice that if $x < \delta \restriction \alpha$ then for any $\alpha < \beta$ also $x < \delta \restriction \beta$, thus the definition of $\epsilon_{\delta}$ holds.
	We also note that $\epsilon_{\delta}$ is a cut of $\epsilon^{\delta}$ as $y < x < \delta \restriction \alpha \implies y < \delta \restriction \alpha$. \\
	Let $\delta \ne \gamma \in \kappa^{\lambda}$ and assume that $\beta$ is the least ordinal such that $\delta(\beta) \ne \gamma(\beta)$ and $\delta(\beta) > \gamma(\beta)$ without loss of generality.
	It follows that $\delta \restriction \beta \in \eta_{\delta}$ but $\delta \restriction \beta \notin \eta_{\gamma}$, therefore $\eta_{\delta} \ne \eta_{\gamma}$.
	The injection $\delta \mapsto \epsilon_{\delta}$ witness $|\kappa| < |\kappa^{\lambda}| \le |H|$ for $H \subseteq \kappa^{< \lambda}$ the collection of cuts.

	Let $X = \{ f \in \kappa^{\lambda} \mid \exists \alpha < \lambda \forall \alpha < \beta < \lambda,\ f(\beta) = 0 \}$, then there is a bijection of $X$ and $\kappa^{< \lambda}$ and therefore $|X| = \kappa$.
	The proof that $|H_X| > \kappa$ remains the same in this case, and we will show that $\langle X, <_l \rangle$ is dense.
	If $f, g \in X$ with $f <_l g$ then $f$ is not an initial segment of $g$ and therefore there exists least $\alpha$ such that $f(\alpha) < g(\alpha)$.
	We can define,
	\[
		h(x)
		= \begin{cases}
			f(x) & x \ne \alpha + 1 \\
			f(x) + 1 & x = \alpha + 1
		\end{cases}
	\]
	Then $f <_l h$ but $h(\alpha) < g(\alpha)$ therefore $f <_l h <_l g$, as intended.
\end{proof}

\subquestion{}
Let $T$ be a theory and let us assume that there is $\Mm \models T$,
a sequence $\langle \bar{a}_n \mid n < \omega \rangle \subseteq M^k$ and a formula $\varphi(\bar{x}, \bar{y})$ such that for any $n \ne m$, $\Mm \models \varphi(\bar{a}_n, \bar{a}_m) \iff n < m$. \\
We will show that $T$ is not $\kappa$-stable for any $\kappa \ge |T|$.
\begin{proof}
	We will show that there is $A \subseteq \Cc_T$ with $|A| \le \kappa$, such that $\kappa < |S_1(A)|$.
	Note that $\Mm \prec \Cc_T$, then $\Cc_T \models \varphi(\bar{a}_n, \bar{a}_m) \iff n < m$.

	If $T$ is $\omega$-stable then it is $\kappa$-stable for any $\kappa \ge |T|$, then it is sufficient to show that $T$ is not $\omega$-stable.
	By the equivalency to $\delta$-stableness theorem, $T$ is $\omega$-stable if and only if $|S_1(\emptyset)| \le \omega$ and $\forall n \in \NN,\ |S_n(N)| \le \omega$ for any $\Nn \models T$.
	Therefore if $|S_k(M)| > \omega$ then $T$ is not $\kappa$-stable.

	We intend to define lexicographic order over $\Mm$ using $\varphi$, assume that $\langle \bar{b}_n \mid n < m \rangle$ for some $m < \omega$.
	Let us define the formula that extends $\varphi$'s order to $m$-tuples against $m'$-tuples,
	\[
		\psi_{m, m'}(\bar{x}_0, \ldots, \bar{x}_{m - 1}, \bar{y}_0, \ldots, \bar{y}_{m' - 1})
		= (m < m') \lor \varphi(\bar{x}_0, \bar{y}_0) \lor \cdots \lor \varphi(\bar{x}_{m - 1}, \bar{y}_{m - 1})
	\]
	Notice that $\Mm \models \varphi(\langle \bar{a}_{n_i} \mid i < m \rangle, \langle \bar{a}_{n_j} \mid j < m' \rangle) \iff \langle n_i \mid i < m \rangle <_l \langle n_j \mid j < m' \rangle$.

	Let $\langle \omega^{< \omega}, < \rangle$ be a dense linear order from the last part, and let $H$ be the collection of cuts, we saw that $|H| > \omega$.
	Let $F : \omega \to M^k$ be the map defined by $n \mapsto \bar{a}_n$ and $G : \omega^{< \omega} \to {(M^k)}^{< \omega}$ defined as $G(f)(i) = F(f(i))$. \\
	Notate $H_\Mm = G(H)$, the collection of cuts of $M^k$ in relation to the order induced by $\psi$.
	Given $X \in H_\Mm$, let us define $p_X \in S_1(M)$ as,
	\[
		p(\bar{x})
		= \cl_\vdash \{ \varphi(\bar{y}, \bar{x}) \mid \bar{y} \in X \}
	\]
	Namely, $\bar{x}$ is an $k$-tuple of elements such that acts as the supremum of $X$.
	It follows that there are $|H_\Mm| = |H| > \omega$ such types, then $|S_1(M)| > \omega$ as well, concluding our proof.
\end{proof}

\subquestion{}
We will show that if $L$ is a language with $|L| \ge \kappa$ and $T$ is a theory such that $\forall n < \omega,\ |S_n(\emptyset)| \le \kappa$,
then there is $L' \subseteq L$ with $|L'| = \kappa$ such that for any $\varphi(x_0, \ldots, x_{n - 1}) \in \form_L$,
there is an $L'$-formula $\varphi'(x_0, \ldots, x_{n - 1})$ with $T \models \forall \bar{x}\ (\varphi \leftrightarrow \varphi')$.
\begin{proof}
	Let us observe $tp(c) = \{ \varphi(x) \mid T \models \varphi(c) \} \in S_1(\emptyset)$ for some constant symbol $c \in L$.
	If there were $\{ c_{\alpha} \mid \alpha < \lambda \}$ for $\lambda > \kappa$ then by $|\{ tp(c_{\alpha}) \mid \alpha < \lambda \}| \le |S_1(\emptyset)| \le \kappa$,
	we would get that up to equality in $T$ there are at least $\kappa$ constants, then let us add those constants to $L'$.

	In similar manner, if $R \in L$ is an $n$-placed relation symbol, we can define $p_R(\bar{x}) = \{ \varphi(\bar{x}) \mid T \models \varphi(\bar{x}) \to R(\bar{x}) \}$.
	We get up to $\kappa$ $n$-placed relation symbols $\{ R_{\alpha}^n \mid \alpha < \kappa \}$ such that there is $\alpha < \kappa$ for which $T \models \forall \bar{x}\ (R \leftrightarrow R_{\alpha}^n)$.
	We enrich $L'$ with $\{ R_{\alpha}^n \mid n < \omega, \alpha < \kappa \}$, note that $\kappa \cdot \kappa = \kappa$ therefore $|L'| = \kappa$.

	We move to function symbols. Let $F$ be an $n$-placed function symbol, and let,
	\[
		p_F(\bar{x}) = \{ \exists y\ \varphi(\bar{x}, y) \mid T \models \varphi(\bar{x}, y) \to (F(\bar{x}) = y) \}
	\]
	and define in the same sense $\{ F_{\alpha}^n \}$, the enrichment of $L'$ by them finishes the proof.
\end{proof}

\question{}
We say that $\Mm$ is minimal if $\Mm$ has no proper elementary substructure.

\subquestion{}
Let $T$ be a countable and complete theory.
We will show that if $T$ has a prime model, then eny minimal model is prime.
\begin{proof}
	TODO
\end{proof}

\end{document}
