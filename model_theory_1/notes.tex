\input{../article_base.tex}
\title{תורת המודלים 1 --- סיכום}
\setcounter{secnumdepth}{2}

\usepackage{fancyhdr}
\pagestyle{fancy}
\renewcommand{\headrulewidth}{0pt}

\begin{document}
\maketitle
\maketitleprint[yellow]

\tableofcontents

\setcounter{section}{-1}
\section{שיעור הכנה}
\subsection{מעט תורת הקבוצות}
\begin{definition}[מונה]
	סודר $\alpha$ נקרא מונה אם לכל $\beta < \alpha$ אין העתקה על $f : \beta \to \alpha$ (שקול לאי־קיום פונקציה חד־חד ערכית).
\end{definition}
\begin{example}
	כל הסודרים הסופיים הם מונים, וכך גם $\omega$.
\end{example}
\begin{example} 
	$\omega + 1, \omega + 2, \ldots$ הם לא מונים כי נוכל לבנות פונקציה $f : \omega + n \to \omega$ חד־חד ערכית.
\end{example}
נגדיר לדוגמה גם את $\omega_1 = \aleph_1$ להיות המונה הבא אחרי $\omega$.
\begin{theorem}[אי־חסימות מונים]
	לכל מונה $\kappa$ יש מונה $\mu > \kappa$.
\end{theorem}
\begin{proof}
	בהנחת אקסיומת הבחירה נסדר את $\Pp(\kappa)$ בסדר טוב בטיפוס סדר $\alpha$.
	אז אין העתקה על מ־$\kappa$ ל־$\alpha$.
	יהי $\mu > 0$ הסודר הראשון כך שאין העתקה על מ־$\kappa$ ל־$\mu$ ונטען כי $\mu$ מונה. \\
	אם $\mu$ איננו מונה, אז יש $\beta < \mu$ והעתקה חד־חד ערכית ועל $g : \kappa \to \beta$, והרכבת הפונקציות מספקת סתירה.
\end{proof}
ישנה גם הוכחה ללא אקסיומת הבחירה אבל לא נביא אותה בסיכום זה.
\begin{definition}[מונה עוקב]
	המונה הראשון שגדול ממונה $\kappa$ נקרא העוקב של $\kappa$ ומסומן $\kappa^+$.
\end{definition}
\begin{remark}
	אם $A$ קבוצת מונים, אז גם $\bigcup A$ מונה.
\end{remark}
אנו יכולים לבחון את $\aleph_0 = \omega$ וכן את $\aleph_1 = \aleph_0^+$ וכן הלאה, ולבסוף נוכל להגדיר גם את $\aleph_{\omega} = \sup\{ \aleph_n \mid n < \omega \}$, וכן $\aleph_{\omega + 1} = \aleph_{\omega}^+$.
\begin{theorem}[היררכיית אלף]
	כל מונה הוא $\aleph_{\alpha}$ עבור איזשהו סודר $\alpha$.
\end{theorem}
\begin{proof}
	נניח ש־$\kappa$ מונה, אז $\kappa \le \aleph_{\kappa}$ (ניתן להוכחה באינדוקציה טרנספיניטית).
	לכן קיים $\gamma$ הסודר הראשון כך ש־$\kappa \le \aleph_{\gamma}$.
	אם $\kappa < \aleph_{\gamma}$ אז נחלק למקרים.
	אם $\gamma = \delta + 1$ אז $\aleph_{\gamma} = \aleph_{\delta}^+$ אבל $\aleph_{\delta} < \kappa < \aleph_{\delta + 1}$ ואז $\kappa = \aleph_{\delta}$.
	אם $\gamma$ גבולי, אז $\aleph_{\gamma} = \sup\{ aleph_{\beta} \mid \beta < \gamma \}$ ולכן יש $\beta < \gamma$ כך ש־$\kappa \le \aleph_{\beta}$ כסתירה.
	לכן נסיק ש־$\kappa = \aleph_{\gamma}$.
\end{proof}
\begin{corollary}
	אם $\alpha$ סודר ו־$\kappa \le \alpha$ מונה ומקסימלי מבין המונים $\le \alpha$, אז $|\alpha| = |\kappa| = \kappa$.
\end{corollary}
\begin{proof}
	באינדוקציה.
\end{proof}
\begin{definition}[מונה סדיר]
	מונה אינסופי $\kappa$ יקרא סדיר (regular) אם אין $\mu < \kappa$ ופונקציה $f : \mu \to \kappa$ כך ש־$\sup \rng f = \kappa$.
\end{definition}
ניצוק תוכן להגדרה זו.
\begin{proposition}
	מונה $\kappa$ הוא סדיר אם ורק אם אין פירוק של $\kappa$ כאיחוד של קבוצות $\kappa = \bigcup \{ A_i \mid i < \mu \}$ כך ש־$\mu < \kappa$ וכן $|A_i| < \kappa$.
\end{proposition}
\begin{example}
	$\omega$ הוא סדיר, תחת אקסיומת הבחירה גם $\omega_1$ הוא סדיר.
	נניח ש־$f : \mu \to \omega_1$ עבור $\mu < \omega_1$ וכן $\sup \rng f = \bigcup \{ f(\delta) \mid \delta < \mu \}$ כאשר $f(\delta)$ בן־מניה.
	אבל מאקסיומת הבחירה איחוד בן־מניה של קבוצות בנות־מניה הוא גם בן־מניה.
\end{example}
\begin{definition}[מונה סדיר וחריג]
	מונה $\kappa$ יקרא חריג אם הוא אינסופי ואינו סדיר.
\end{definition}
\begin{example}
	$\aleph_{\omega}$ הוא מונה חריג.
	נגדיר $f(n) = \omega_n = \aleph_n$ כאשר $f : \omega \to \aleph_{\omega}$.
\end{example}
\begin{proposition}
	לכל מונה אינסופי $\kappa$ מתקיים $|\kappa| = |\kappa \times \kappa|$.
\end{proposition}
\begin{proof}
	נספק סקיצה כללית.
	נוכיח באינדוקציה על מונים אינסופיים. \\
	ל־$\omega$ זה ידוע וקל. \\
	נניח ש־$\kappa$ מונה כך שהטענה נכונה למונים קטנים ממנו.
	נגדיר סדר טוב על $\kappa \times \kappa$ באופן הבא,
	\begin{align*}
		\langle \alpha, \beta \rangle \le (\gamma, \delta)
		\iff & (\max\{ \alpha, \beta \} < \max\{ \gamma, \delta \}) \\
		\lor & (\max\{ \alpha, \beta \} = \max\{ \gamma, \delta \} \land \alpha < \gamma) \\
		\lor & (\max\{ \alpha, \beta \} = \max\{ \gamma, \delta \} \land \alpha = \gamma \land \beta \le \delta)
	\end{align*}
	נשים לב כי מתחת ל־$\langle \alpha, \beta \rangle$ יש פחות מ־$\kappa$ איברים,
	\[
		\le |\alpha + 1| \times |\beta + 1|
		\le |\mu_1 \times \mu_2|
		\le |\mu \times \mu|
		< \kappa
	\]
	עבור $\mu = \max(\mu_1, \mu_2) < \kappa$.
	הסדר שהגדרנו איזומורפי לסודר $\delta \le \kappa$ ולכן $\kappa \le |\kappa \times \kappa| \le \kappa$.
\end{proof}
\begin{corollary}
	לכל מונה $\kappa$ מתקיים $|\kappa^{< \omega}| = \kappa$.
\end{corollary}
\begin{theorem}[מונה עוקב הוא סדיר]
	אם $\kappa$ מונה אז $\kappa^+$ מונה סדיר.
\end{theorem}
\begin{proof}
	נניח בשלילה שלא ותהי $f : \mu \to \kappa^+$ כך ש־$\bigcup \{ f(\alpha) \mid \alpha < \mu \} = \kappa^+$. \\
	באמצעות בחירה לכל $\alpha$ נבחר $H_{\alpha} : \kappa \twoheadrightarrow f(\alpha) + 1$ וכן $H(\alpha, \beta) = H_{\alpha}(\beta)$ עבור $H : \mu \times \kappa \twoheadrightarrow \kappa^+$, וזו כמובן סתירה.
\end{proof}

\section{שיעור 1 --- 19.10.2025}
\subsection{רקע}
תורת המודלים היא תחום בלוגיקה שעוסקת בניתוח של תורות ושל מודלים המתקבלים מהם.
נראה דוגמה למשפט שנובע מתחום זה.
\begin{example}
	משפט אקס־גרוטנדיק, הגורס כי אם פונקציה $f : \CC^n \to \CC^n$ כך שכל קורדינטה שלה היא פולינום ב־$n$ משתנים.
	נניח ש־$f$ חד־חד ערכית, אז $f$ היא גם על.
\end{example}
זהו משפט מוזר מאוד ומפתיע.
הדרך להוכיח אותו היא כזו,
נניח שיש לנו סדרה של פולינומים כך שהם חד־חד ערכיים ולא על, אז הכישלון שנקבל הוא על־ידי פסוק מסדר ראשון בשפת תורת החוגים $\varphi$ כך ש־$\CC \models \varphi$.
\[
	\exists a_0, \ldots, \exists a_{N}\ \forall \bar{x} \forall \bar{y}\ (a_0 x_0 \cdots = a_0 y_0 \cdots) \to \bar{x} = \bar{y}
	\land \exists \bar{z} \forall \bar{x} \lnot \bigwedge_{i < N} a_0 \bar{x} = z_i
\]
נבחין כי מתקיימת העובדה שנוכיח בהמשך,
\begin{remark}
	התורה של שדה סגור אלגברית ממציין נתון היא שלמה.
	בפרט כל שדה סגור אלגברית ממציין 0 מקיים את $\varphi$.
\end{remark}
מההערה ושלמות נסיק שכל שדה מספיק סגור אלגברית ממימד מספיק גדול מקיים את $\varphi$.
בפרט ל־$p$ ראשוני מספיק גדול $\overline{\FF}_p \models \varphi$.
נסתכל על מקדמים של הפולינום הבעייתי $a_0, \ldots, a_{N}$ ונקבל שהם שייכים ל־$\FF_p[a_0, \ldots, a_{N}] = \tilde{\FF}$ שדה סופי כלשהו.
נניח ש־$z_{0}, \ldots, z_{n - 1}$ מעידה על הפולינומים האלו, אז,
\[
	\tilde{\FF}[z_0, \ldots, z_{n - 1}] = \tilde{\tilde{\FF}} \subseteq \overline{\FF}_p 
\]
אז $f \restriction \tilde{\tilde{\FF}}$ חד־חד ערכית ולכן על ולכן $\bar{z}$ מתקבל כסתירה. \\
הרעיון המגניב הוא שהצלחנו למצוא טענה מאוד מורכבת על־ידי שימוש במודלים שונים מאותו עולם.

בקורס עצמו אנחנו נוכיח טענות בעולם של תורת המודלים, המשפטים המרכזיים הם:
\begin{itemize}
	\item משפט Vaught: תהי $T$ תורה בת־מניה שלמה, אז לא יתכן של־$T$ יש בדיוק שני מודלים לא איזומורפיים בני־מניה עד כדי איזומורפיזם
	\item משפט מורלי (Morley): יהי $\kappa$ מונה לא בן־מניה, $T$ תורה מעל שפה בת־מניה, אז $T$ היא $\aleph_1$־קטגורית אם ורק אם $T$ היא $\kappa$־קטגורית
\end{itemize}

\subsection{תזכורת למושגים והגדרות}
\begin{definition}[שפה]
	אוסף של סימני קבועים יחסים ופונקציות.
\end{definition}
\begin{definition}[שמות עצם]
	שמות עצם הם אובייקט סינטקטי שמורכב מסימני פונקציה קבועים ומשתנים.
\end{definition}
\begin{definition}[משתנה חופשי]
	משתנים חופשיים, נסמן $\varphi(x_0, \ldots, x_{n - 1})$ כאשר המשתנים $x_0, \ldots, x_{n - 1}$ חופשיים ב־$\varphi$. \\
	נוכל גם לדבר על המשתנים החופשיים של פסוק, ונסמן באופן דומה $t(x_0, \ldots, x_{n - 1})$.
\end{definition}
\begin{definition}[פסוק]
	פסוק הוא נוסחה ללא משתנים חופשיים.
\end{definition}
\begin{definition}[השמה]
	בהינתן נוסחה $\varphi(x_0, \ldots, x_{n - 1})$ ומבנה $\Aa$, $a_0, \ldots, a_{n - 1} \in A$,
	אז $\Aa \models \varphi(a_0, \ldots, a_{n - 1})$ בהתאם להגדרת האמת והחישוב הרקורסיבית שראינו בקורסים קודמים.
\end{definition}
\begin{definition}[הומומורפיזם של מבנים]
	בהינתן שני מבנים $\Aa, \Bb$ בשפה $L$, אז נסמן פונקציה $f : \Aa \to \Bb$ כפונקציה בין העולמות כך שהיא הומומורפיזם,
	כלומר היא מכבדת פונקציות קבועים ויחסים במובן הבא,
	\[
		\bar{a} \in R^\Aa
		\implies f(\bar{a}) \in R^\Bb
	\]
	שיכון הוא מקרה בו גם הכיוון השני מתקיים. \\
	איזומורפיזם הוא שיכון שהוא גם על. \\
	אוטומורפיזם הוא איזומורפיזם בין מבנה לעצמו.
\end{definition}
\begin{definition}[תת־מבנה]
	נסמן תת־מבנה של מבנים על־ידי $\Aa \subseteq \Bb$ אם $\id : \Aa \to \Bb$ שיכון.
	בפרט הקבוצה $A$ סגורה תחת הפונקציות של $B$ ומכילה את כל הקבועים.
\end{definition}
\begin{theorem}[משפט הקומפקטיות]
	נניח ש־$\Sigma$ קבוצת פסוקים בשפה $L$ כך שלכל $\Sigma_0 \subseteq \Sigma$ סופית היא ספיקה, אז $\Sigma$ ספיקה.
\end{theorem}
\begin{definition}[תורה]
	תורה היא קבוצת פסוקים סגורה למסקנות.
	תורה $T$ היא עקבית אם $\perp \notin T$, ממשפט השלמות הגדרה זו שקולה לקיום מודל ל־$T$. \\
	תורה $T$ היא שלמה אם לכל פסוק $\varphi$ מתקיים $\varphi \in T$ או $\lnot \varphi \in T$.
\end{definition}
לדוגמה אם $\Aa$ מבנה, אז $\operatorname{Th}(\Aa)$ שלמה.
\begin{definition}[שקילות]
	$\Aa \equiv \Bb$ אם $\operatorname{Th}(\Aa) = \operatorname{Th}(\Bb)$ ו־$\Aa \cong \Bb$ אם יש איזומורפיזם.
	מתקיים $\Aa \cong \Bb \implies \Aa \equiv \Bb$.
\end{definition}
\begin{definition}
	$f : \Aa \to \Bb$ נקראת שיכון אלמנטרי אם לכל נוסחה $\varphi(x_0, \ldots, x_{n - 1})$ ו־$a_0, \ldots, a_{n - 1} \in A$ אז,
	\[
		\Aa \models \varphi(a_0, \ldots, a_{n - 1})
		\iff \Bb \models \varphi(f(a_0), \ldots, f(a_{n - 1}))
	\]
	אם $f = \id$ אז נגיד ש־$\Aa \prec \Bb$ תת־מודל אלמנטרי.
\end{definition}
\begin{remark}
	נניח ש־$\langle \Aa_n \mid n < \omega \rangle$ שרשרת מבנים כך ש־$\Aa_n \subseteq \Aa_{n + 1}$,
	אז יש דרך אחת להגדיר את איחוד המבנים $\Aa_{\omega} = \bigcup_{n < \omega} \Aa_n$ כך ש־$\Aa_n \subseteq \Aa_{\omega}$.
	נעיר כי גם אם נוסיף את ההנחה ש־$\Aa_n \equiv \Aa_{n + 1}$ לא בהכרח נקבל שגם $\Aa_{\omega} \equiv \Aa_n$. \\
	לדוגמה עבור $L = \{ \le \}$ ו־$\Aa_n = \{ z \in \ZZ \mid -n \le z \}$ אז $\Aa_{\omega} = \ZZ$ אבל התורות אכן שונות.
\end{remark}
\begin{definition}[קטגוריות]
	נאמר שתורה $T$ היא $\kappa$־קטגורית אם לכל $\Aa, \Bb \models T$ אז מתקיים,
	\[
		|A| = |B|
		\implies \Aa \cong \Bb
	\]
\end{definition}
\begin{remark}
	סודר $\alpha$ נקרא מונה אם לא קיים $\beta < \alpha$ ופונקציה $f : \beta \to \alpha$ על. \\
	לכל מונה שונה מ־0 קיים מונה גדול יותר ומינימלי המסומן $\kappa^+$ ומכונה המונה העוקב של $\kappa$. \\
	נסמן ${(\aleph_0)}^+ = \aleph_0$.
\end{remark}
\begin{theorem}
	נניח ש־$\langle \Aa_n \mid n < \omega \rangle$ כך ש־$\Aa_n \prec \Aa_{n + 1}$ אז $\Aa_n \prec \Aa_{\omega}$.
\end{theorem}
\begin{proof}
	קודם כל נשים לב לעובדה השימושית הבאה, אם $\Mm \prec \Nn \prec \Kk$ אז $\Mm \prec \Kk$.
	נובע שלכל $n < m$ מתקיים $\Aa_n \prec \Aa_m$.
	נוכיח את הטענה באינדוקציה על מבנה הנוסחה, לכל $n < \omega$ ולכל $a_0, \ldots, a_{m - 1} \in \Aa_n$ מתקיים,
	\[
		\Aa_n \models \psi(a_0, \ldots, a_{m - 1})
		\Aa_{\omega} \models \psi(a_0, \ldots, a_{m - 1})
	\]
	עבור $\psi$ אטומית הטענה נובעת מכך שאלו הם תתי־מבנים.
	אם הטענה נכונה עבור $\psi$ היא נכונה גם עבור שלילה וכך גם לקשרים הבינאריים. \\
	נניח ש־$\varphi = \exists x_0\ \psi$ כאשר $\varphi = \varphi(x_1, \ldots, x_{m - 1})$.
	אם $\Aa_n \models \varphi(a_1, \ldots, a_{m - 1})$ אז $\Aa_n \models \exists x_0\ \psi(x_0, \ldots, a_{m - 1})$ ולכם יש $a_0 \in \Aa_n$ כך שמתקיים $\Aa_n \models \psi(a_0, \ldots, a_{m - 1})$.
	מהנחת האינדוקציה נקבל שגם $\Aa_{\omega} \models \psi(a_0, \ldots, a_{m - 1})$ ולכן $\Aa_{\omega} \models \exists x_0\ \psi(x_0, a_0, \ldots, a_{m - 1})$. \\
	בכיוון השני נניח ש־$\Aa_{\omega} \models \exists x_0\ \psi(x_0, a_1, \ldots, a_{m - 1})$.
	לכן קיים $b \in A_{\omega}$ כך שמתקיים $\Aa_{\psi} \models \psi(b, a_1, \ldots, a_{m - 1})$ ובהתאם קיים $k < \omega$ כלשהו כך ש־$n \le k$ ומתקיים $b \in A_k$.
	מהנחת האינדוקציה $\Aa_{\omega} \models \psi(b, a_1, \ldots, a_{m - 1})$ ולכן מאינדוקציה $\Aa_k \models \psi(b, a_1, \ldots, a_{m - 1})$ ולבסוף גם,
	\[
		\Aa_n \prec \Aa_k \exists x_0\ \psi(x_0, a_1, \ldots, a_{m - 1})
	\]
	ונסיק שמתקיים גם,
	\[
		\Aa_n \models \exists x_0\ \psi(x_0, a_1, \ldots, a_{m - 1})
	\]
	כפי שרצינו.
\end{proof}
\begin{theorem}[מבחן טרסקי־ווט]
	נניח ש־$\Mm \subseteq \Nn$ תת־מבנה כך שלכל נוסחה $\varphi(x, x_0, \ldots, x_{n - 1})$ ופרמטרים $a_0, \ldots, a_{n - 1} \in M$ כך שמתקיים,
	\[
		\Nn \models \exists x\ \varphi(x, a_0, \ldots, a_{n - 1})
		\implies \exists b \in M,\ \Nn \models \varphi(b, a_0, \ldots, a_{n - 1})
	\]
	אם ורק אם תקיים $\Mm \prec \Nn$.
\end{theorem}
\begin{proof}
	אם $\Mm \prec \Nn$ ומתקיים,
	\[
		\Nn \models \exists x\ \varphi(x, a_0, \ldots, a_{n - 1})
		\implies \Mm \models \exists x\ \varphi(x, a_0, \ldots, a_{n - 1})
	\]
	ולכ קיים $b \in M$ כך שמתקיים $\varphi^\Mm(b, a_0, \ldots, a_{n - 1})$ ולכן בהכרח גם $\Nn \models \varphi(b, a_0, \ldots, a_{n - 1})$.

	נעבור לכיוון השני, ושוב נוכיח באמצעות אינדוקציה על מבנה הנוסחה $\varphi(x_0, \ldots, x_{n - 1})$, שכן $a_0, \ldots, a_{n - 1} \in M$ אז,
	\[
		\Mm \models \varphi(a_0, \ldots, a_{n - 1})
		\iff \Nn \models \varphi(a_0, \ldots, a_{n - 1})
	\]
	עבור נוסחות אטומיות וקשרים בינאריים הטענה כמובן טריוויאלית מהגדרה ולכן נניח שמתקיים,
	\[
		\varphi = \exists x\ \psi(x, x_1, \ldots, x_{n - 1})
	\]
	וכן שמתקיים $\Mm \models \varphi(a_1, \ldots, a_{n - 1})$.
	לכן,
	\[
		\exists b \in M,\ \Mm \models \psi(b, a_1, \ldots, a_{n - 1})
	\]
	ולכן $\Nn \models \psi(b, a_1, \ldots, a_{n - 1})$ וכן $\NN \models \varphi(a_1, \ldots, a_{n - 1})$. \\
	בכיוון השני נניח שמתקיים,
	\[
		\Nn \models \exists x\ \psi(x, a_1, \ldots, a_{n - 1})
	\]
	אבל אז מטרסקי־ווט נקבל שקיים $b \in M$ כך ש־$\Nn \models \psi(b, a_1, \ldots, a_{n - 1})$ ומהנחת האינדוקציה על $\psi$ נקבל,
	\[
		\Mm \models \psi(b, a_1, \ldots, a_{n - 1})
		\implies \Mm \models \varphi(a_1, \ldots, a_{n - 1})
	\]
	וסיימנו את מהלך האינדוקציה.
\end{proof}
\begin{corollary}
	נניח ש־$L = \{ = \}$ ונניח ש־$\Aa \subseteq \Bb$ מבנים אינסופיים בשפה $L$.
	אז $\Aa \prec \Bb$.
\end{corollary}
\begin{proof}
	נשתמש במבחן טרסקי־ווט (מעכשיו נכתוב גם TV).
	נניח ש־$a_0, \ldots, a_{n - 1} \in A$ וכן שמתקיים,
	\[
		\Bb \models \exists x\ \varphi(a_0, \ldots, a_{n - 1})
	\]
	יהי $b \in B$ שמעיד על כך, אם $b \in \{ a_0, \ldots, a_{n - 1} \}$ אז בכל מקרה סיימנו. \\
	נבחר $c \in A \setminus \{ a_0, \ldots, a_{n - 1} \}$ ונגדיר אוטומורפיזם של $\Bb$ על־ידי,
	\[
		f(z)
		= \begin{cases}
			c & z = b \\
			b & z = c \\
			z & \text{otherwise}
		\end{cases}
	\]
	לכן $f$ אוטומורפיזם ובפרט שיכון אלמנטרי ומתקיים $f(a_i) = a_i$.
	נסיק שמתקיים,
	\[
		\Bb \models \varphi(f(b), f(a_0), \ldots, f(a_{n - 1}))
	\]
	ולכן תנאי המבחן חלים.
\end{proof}
\begin{corollary}[לוונהיים־סקולם היורד]
	נניח ש־$\Aa$ הוא $L$־מבנה ו־$\kappa$ מונה כך ש־$\aleph_0 + |L| \le \kappa \le |A|$, אז קיים $\Bb \prec \Aa$ כך ש־$|B| = \kappa$.
\end{corollary}
\begin{proof}
	לכל נוסחה $\varphi(x_0, \ldots, x_n)$ נגדיר פונקציה $F_{\varphi} : A^n \to A$ על־ידי,
	\[
		F_{\varphi}(a_0, \ldots, a_n)
		= \begin{cases}
			b & \Aa \models \varphi(b, a_1, \ldots, a_n) \\
			c & \Aa \models \lnot \exists x\ \varphi(x, a_1, \ldots, a_n)
		\end{cases}
	\]
	עבור ערך שרירותי $c$.
	עתה, תהי $X \subseteq A$ כך ש־$|X| = \kappa$, נגדיר,
	\[
		X_0 = X,
		X_{n + 1} = \{ F_{\varphi}(a_1, \ldots, a_m) \mid a_1, \ldots, a_m \in X_n,\ \varphi \in \operatorname{form} \} \cup X_n
	\]
	לכל $n$, אז $|X_{n + 1}| = \kappa$ תמיד.
	נסמן $B = \bigcup_{n < \omega} X_n$, אז,
	\[
		\kappa \le |B| \le \kappa + \aleph_0 = \kappa
	\]
	מתקיים $\Bb \subseteq \Aa$ כי אם $F$ סימן פונקציה ו־$\bar{c} \in B^n$ אז $F(\bar{c}) \in B$ כי הוא העדות היחידה לנוסחה $F(\bar{c}) = x$.
	בהתאם $\Bb \subseteq \Aa$ מקיים את TV ישירות מהבניה.
	אם $b_1, \ldots, b_n \in B$ ו־$\varphi$ נוסחה אז יש $b_1, \ldots, b_n \in X_m$, העדות ל־TV תהיה ב־$B \supseteq X_{m + 1}$.
\end{proof}

\section{שיעור 2 --- 26.10.2025}
\subsection{לוונהיים־סקולם}
\begin{definition}[פונקציית סקולם]
	אם $\varphi(x_0, \ldots, x_{n - 1})$ נוסחה כלשהי,
	אז פונקציה $F_{\varphi} : N \to M$ כך ש־$\Mm \models \exists x\ \varphi(x, a_1, \ldots, a_{n - 1})$ אז $b = F_{\varphi}(a_1, \ldots, a_{n - 1})$ כך ש־$\Mm \models \varphi(b, a_1, \ldots, a_{n - 1})$.
\end{definition}
וננסח שוב את קריטריון טרסקי־ווט תוך שימוש בהגדרה זו.
\begin{theorem}[ניסוח שקול ללוונהיים־סקולם היורד]
	$F_{\varphi}(X^n) \subseteq X$ לכל $X \subseteq M$ ולכל $\varphi(x_0, \ldots, x_{n - 1})$ אז $X \prec \Mm$.
\end{theorem}
ותוך שימוש באפיון זה הוכחנו את משפט לוונהיים־סקולם היורד.
\begin{theorem}[לוונהיים־סקולם העולה]
	יהי $\Mm$ מודל אינסופי ו־$\kappa > |M|, |L|$, אז קיים $\Mm \prec \Nn$ מודל כך ש־$|N| = \kappa$.
\end{theorem}
נגדיר הגדרה שתשמש אותנו בהוכחת המשפט.
\begin{definition}[העשרה בקבועים]
	עבור מודל $\Mm$ ו־$A \subseteq M$ נסמן ב־$L_A$ את ההעשרה של $L$ על־ידי קבועים $\{ d_a \mid a \in A \}$ ואת $\Mm_A$ את ההעשרה של פירוש הקבועים כך ש־$d_a^{\Mm_A} = a$.
\end{definition}
\begin{notation}
	$\diag(\Mm) = \Th(\Mm_M)$.
\end{notation}
עתה נוכל לעבור להוכחה.
\begin{proof}
	נתחיל בלבנות $\tilde{\Nn}$ כך שיש שיכון אלמנטרי $j : \Mm \to \tilde{\Nn}$ וש־$|\tilde{N}| = \kappa$.
	נבחן את ההעשרה $L_M$ בקבועים הנוספים $\{ c_{\alpha} \mid \alpha < \kappa \}$ ואת התורה,
	\[
		T = \diag(M) \cup \{ c_{\alpha} \ne c_{\beta} \mid \alpha < \beta < \kappa \}
	\]
	מקומפקטיות ל־$T$ יש מודל.
	בנוסף מלוונהיים־סקולם היורד יש מודל כזה שעוצמתו היא $|L_M| + \kappa + \aleph_0 = \kappa$ ונסמנו $\tilde{\Nn}$.
	נגדיר $j(a) = d_a^{\tilde{\Nn}}$ ואז לפי הגדרת $\diag(\Mm)$ אם $\psi$ נוסחה ו־$\Mm \models \psi(a_0, \ldots, a_{n - 1}) \iff \psi(d_{a_0}, \ldots, d_{a_{n - 1}}) \in \diag(\Mm)$.
	וכל זה נכון אם ורק אם $\tilde{\Nn} \models \psi(d_{a_0}, \ldots, d_{a_{n - 1}}) \iff \tilde{\Nn} \models \psi(j(a_0), \ldots, j(a_{n - 1}))$.
	כעת נתקן את $\tilde{\Nn}$ כך ש־$\Mm \subseteq \Nn$ עבור $\Nn \cong \tilde{\Nn}$.
	קודם כל בלי הגבלת הכלליות $\tilde{N} \cap M = \emptyset$ ונגדיר $N = (\tilde{N} \setminus \rng j) \cup M$ ונגדיר את ההעתקה,
	\[
		f : \tilde{\Nn} \to \Nn,
		\quad
		f(x) = \begin{cases}
			x & x \notin \rng j \\
			j^{-1}(x) & x \in \rng j
		\end{cases}
	\]
	כלומר, הגדרנו את $\Nn$ כך שהיא תהיה איזומורפיזם.
\end{proof}
\begin{definition}[קטגוריות]
	יהי $\kappa$ מונה, תורה $T$ תיקרא $\kappa$־קטגורית אם יש מודל יחיד $\Nn \models T$ כך ש־$|N| = \kappa$ עד כדי איזומורפיזם.
\end{definition}
\begin{theorem}
	נניח ש־$T$ היא תורה בשפה $L$ ול־$T$ אין מודלים סופיים.
	אם בנוסף $T$ היא $\kappa$־קטגורית עבור $\kappa \ge |L|$ אז $T$ שלמה.
\end{theorem}
\begin{proof}
	נניח ש־$\varphi$ פסוק כך ש־$T \cup \{ \varphi \}$ עקבית, ונניח בשלילה שגם $T \cup \{ \lnot \varphi \}$ עקבית. \\
	אז מלוונהיים־סקולם העולה יש שני מודלים $\Mm_0, \Mm_1$ מעוצמה $|L| + \aleph_0 \le \kappa$ כך שמתקיים,
	\[
		\Mm_0 \models T \cup \{ \varphi \},
		\quad
		\Mm_1 \models T \cup \{ \lnot \varphi \}
	\]
	אבל $\Mm_0 \cong \Mm_1$ וזו סתירה.
\end{proof}
\begin{example}
	DLO, תורת הסדרים הקוויים הצפופים ללא נקודות קצה, בשפה $\{ < \}$.
	\[
		\forall x \forall y (x < y \to \exists z\ x < z < y),
		\quad
		\forall x \exists y\ (y < x) \land \exists z (x < z)
	\]
	יחד עם הפסוקים שמגדירים ש־$<$ הוא סדר קווי חד.
\end{example}
\begin{theorem}[קנטור]
	DLO היא $\aleph_0$־קטגורית. \\
	יתר על־כן, אם $\Mm, \Nn \models \operatorname{DLO}$ כך ש־$|M| = |N| = \aleph_0$ ומתקיים,
	\[
		\Mm \models a_0 < a_1 < \cdots < a_{n - 1},
		\quad
		\Nn \models b_0 < b_1 < \cdots < b_{n - 1}
	\]
	אז קיים איזומורפיזם $\sigma : \Mm \to \Nn$ המקיים $\sigma(a_i) = b_i$.
\end{theorem}
\begin{proof}
	נשתמש בהוכחת ההלוך ושוב (back and forth), נמנה את איברים $M$ ו־$N$,
	\[
		M = \{ a_i \mid i < \omega \},
		\quad
		N = \{ b_i \mid i < \omega \}
	\]
	ונבנה ברקורסיה על $\omega$ סדרת פונקציות $\sigma_i$ משמרות סדר.
	עבור $i = 0$ נגדיר $\sigma_0(a_i) = b_i$.
	נניח שבנינו את $\sigma_k$ ו־$k$ זוגי.
	נבחן את $j < \omega$ המינימלי כך ש־$a_j \notin \dom \sigma_k$.
	יש שלוש אפשרויות כאלה. \\
	האפשרות הראשונה היא שיש $d_0, d_1 \in \dom \sigma_k$ כך ש־$d_0 < a_j < d_1$ וזה הטווח המינימלי, כלומר $d_0 = \max\{ x \in \dom \sigma_k \mid a_j < x \}$.
	נבחן את $\sigma(d_0) < \sigma_k(d_1)$ ונבחר $e \in N$ שמקיים $\sigma_k(d_0) < e < \sigma_k(d_1)$.
	אז נגדיר $\sigma_{k + 1} = \sigma_k \cup \{ \langle a_j, e \rangle \}$.
	שתי האפשרויות האחרות הן ש־$a_j$ מעל או מתחת לכל $\dom \sigma_k$, ואז בהתאם נבחר נקודות מעבר לתחום זה, אשר קיימות מעצם חוסר קיום נקודות קצה. \\
	עבור $k$ אי־זוגי נבחן את $\sigma_k^{-1}$ וכמו במקרה הקודם נוסיף את $b_j$ עם $j$ מינימלי שאיננו ב־$\dom \sigma_k^{-1} = \rng \sigma_k$ באופן משמר סדר. \\
	נגדיר $\sigma = \bigcup_{k < \omega} \sigma_k$, זוהי פונקציה משמרת סדר חד־חד ערכית ועל.
\end{proof}

\subsection{הפרדה}
\begin{lemma}[הפרדה]\label{separation-lemma-1}
	נניח ש־$T_1, T_2$ תורות בשפה $L$.
	$\Sigma$ אוסף פסוקים ב־$L$ שסגור תחת גימום ואיווי ומכיל את $\perp, \top$ (כאשר ההכלה הזו חשובה רק למקרה הלא עקבי).
	אז התנאים הבאים שקולים:
	\begin{enumerate}
		\item יש $\varphi \in \Sigma$ כך ש־$T_1 \models \varphi, T_2 \models \lnot \varphi$
		\item לכל זוג מודלים $\Mm_1 \models T_1, \Mm \models T_2$ יש פסוק $\varphi \in \Sigma$ כך ש־$\Mm_1 \models \varphi, \Mm_2 \models \lnot \varphi$
	\end{enumerate}
\end{lemma}
\begin{proof}
	$1 \implies 2$ ברור, ולכן נניח את תנאי 2. \\
	נקבע את $\Mm_* = \Mm_1$, אז התורה,
	\[
		T_2 \cup \{ \varphi_{\Mm_*, \Mm_2} \mid \Mm_2 \models T_2 \}
	\]
	היא לא עקבית, אחרת אם $\Nn$ מקיים אותה אז $\Nn \models T_2$ אבל $\Nn \models \varphi_{\Mm_*, \Nn}$ וזו סתירה.
	לכן מקומפקטיות יש סדרה סופית של מבנים $\Mm_2^0, \ldots, \Mm_2^{n - 1} \models T_2$ כך שמתקיים,
	\[
		T_2 \cup \{ \varphi_{\Mm_*, \Mm_2^0, \ldots, \Mm_2^{n - 1}} \} \models \perp
	\]
	נסמן $\Mm_* \models \psi_{\Mm_*} = \bigwedge_{i < n} \varphi_{\Mm_*, \Mm_2^i} \in \Sigma$.
	כעת נבחן את $T_1 \cup \{ \lnot \psi_{\Mm_*} \mid \Mm_* \models T_1 \}$.
	היא לא עקבית ולכן $T_1 \models \lnot \bigwedge_{i < n} (\lnot \psi_{\Mm_*}) \equiv \bigvee_{i < n} \psi_{\Mm_*^i} \in \Sigma$.
	אבל $T_2 \models \lnot \bigvee_{i < n} \psi_{\Mm_*^1}$ כרצוי.
\end{proof}
נסתכל על זוג מבנים $\Mm \subseteq \Nn$, אז אם $\varphi$ פסוק מהצורה של $\forall x\ \psi$ עבור $\psi$ חסר כמתים, אז נכונותו ב־$\Nn$ תגרור את נכונותו ב־$\Mm$.
אנו רוצים להגדיר תכונה שגוררת שכל תת־מודל מקיים את התורה של המודל המקורי.
נראה שזהו למעשה המצב שבו זה קורה.
\begin{notation}
	נניח ש־$\Mm, \Nn$ מבנים ו־$\Delta$ קבוצת נוסחות.
	נסמן $f : \Mm \to_{\Delta} \Nn$ אם לכל נוסחה $\psi(x_0, \ldots, x_{n - 1}) \in \Delta$,
	\[
		\Mm \models \psi(a_0, \ldots, a_{n - 1})
		\implies \Nn \models \psi(f(a_0), \ldots, f(a_{n - 1}))
	\]
\end{notation}
\begin{lemma}\label{separation-lemma-2}
	תהי $\Delta$ קבוצת פסוקים סגורה תחת גימום, איווי הוספת כמת קיים והחלפת שמות משתנים.
	נניח ש־$\Mm$ מודל ו־$T$ תורה, אז התנאים הבאים שקולים:
	\begin{enumerate}
		\item לכל $\varphi \in \Delta \cap \Th(\Mm)$, $T \cup \{ \varphi \}$ עקבית
		\item יש מודל של $T$ ושיכון $f : \Mm \to_{\Delta} \Nn$
	\end{enumerate}
\end{lemma}
\begin{proof}
	$2 \implies 1$ טריוויאלי שכן $\Nn \models T \cup (\Th(\Mm) \cap \Delta)$, ולכן נוכיח את $1 \implies 2$. \\
	נבחן את $T \cup \{ \psi(d_{a_0}, \ldots, d_{a_{n - 1}}) \mid \psi \in \Delta, \Mm \models \psi(a_0, \ldots, a_{n - 1}) \}$ בשפה המועשרת.
	נניח בשלילה שהיא לא עקבית.
	אז $T \models \lnot \bigwedge \psi_i(d_{a_0}, \ldots, d_{a_{n - 1}})$ כאשר $\rho = \bigwedge \psi_i \in \Delta$.
	אז ממשפט הכללה על־ידי קבועים נסיק ש־$T \models \forall x_0 \cdots \forall x_{n - 1}\ \lnot \rho$, כלומר $T \models \lnot \exists x_0 \cdots \exists x_{n - 1}\ \rho(x_0, \ldots, x_{n - 1})$.
	אבל $\Mm \models \exists x_0 \cdots \exists x_{n - 1}\ \rho$ בסתירה ל־1.
\end{proof}
\begin{corollary} %\label{lemma-of-separation-3}
	יהיו $T_1, T_2$ תורות,
	אז התנאים הבאים שקולים:
	\begin{enumerate}
		\item יש פסוק מהצורה $\varphi = \forall x\ \psi$ כש־$\varphi$ חסר כמתים (פסוק גלובלי) כך ש־$T_1 \models \varphi, T_2 \models \lnot \varphi$
		\item אין מודל של $T_2$ שהוא תת־מודל של $T_1$
	\end{enumerate}
\end{corollary}
\begin{proof}
	$2 \implies 1$.
	נבחר $\Delta$ להיות פסוקים קיומיים, כלומר $\exists x\ \psi$ עבור $\psi$ חסרי כמתים (עד כדי שקילות).
	נראה שלכל מודל $\Mm_1 \models T_1, \Mm_2 \models T_2$ יש פסוק גלובלי שמפריד ביניהם.
	אחרת כל פסוק קיומי ש־$\Mm_2$ מספק עקבי עם $T_1$.
	לכן מהלמה הקודמת נקבל שיכון מ־$\Mm_2$ למודל של $T_1$ בסתירה.
	נגדיר את $\Sigma$ להיות הפסוקים ששקולים לפסוקים גלובליים, גם הם סגורים תחת גימום ואיווי, ונקבל פסוק מפריד כמבוקש.
\end{proof}
למעשה מצאנו אפיון סינטקטי שמאפיין את ההבדל האפשרי בין מבנים ותתי־מבנים.

\section{שיעור 3 --- 2.11.2025}
\subsection{משפט ווש}
נעסוק בבנייה חשובה מאוד בעולם המודלים.
\begin{definition}[מסנן]
	אוסף $\Ff \subseteq \Pp(X)$ של תתי־קבוצות של קבוצה $X$ יקרא מסנן אם מתקיימות התכונות:
	\begin{enumerate}
		\item $\emptyset \notin \Ff$
		\item אם $A \in \Ff$ ו־$A \subseteq B$ אז $B \in \Ff$
		\item אם $A, B \in \Ff$ אז גם $A \cap B \in \Ff$
	\end{enumerate}
\end{definition}
ההגדרה הזו באה לתאר לנו מהן קבוצות ''גדולות'', כלומר איך אנחנו יכולים לדבר באופן אבסטרקטי על המובן הגאומטרי שחלק מאוסף נחשב לגדול וחלק לא.
לכן נרצה להניח שאוסף ריק לא יכול להיות גדול, וכן סגירות ללקיחת קבוצות גדולות יותר וסגירות לחיתוך.
חשוב להסתכל על מסנן בתור אוסף של קבוצות שגדולות במובן של תורת המידה, כלומר אוסף שמכיל כמעט כל איבר.
\begin{example}
	$\Ff = \{ X \}$ עבור $X$, האוסף שבו רק הקבוצה בשלמותה תיחשב לקבוצה גדולה.
\end{example}
\begin{example}
	נניח ש־$\emptyset \ne x \subseteq X$, אז $\Ff_x = \{ y \subseteq X \mid x \subseteq y \}$ הוא מסנן, ואף נקרא המסנן הראשי.
\end{example}
\begin{example}
	נניח ש־$Y \subseteq \Pp(X)$ עם תכונת החיתוך הסופי, ונגדיר,
	\[
		\Hh = \{ y \subseteq X \mid x_1, \ldots, x_n \in X, \bigcap_{1 \le i \le n} x_i \subseteq y \}
	\]
	אף הוא מסנן.
\end{example}
נעבור להגדרה המשלימה והחשובה מאוד.
\begin{definition}[על־מסנן]
	תהי $X$ קבוצה ויהי $\Uu \subseteq \Pp(X)$ מסנן, אז הוא נקרא על־מסנן אם בנוסף לכל $x \subseteq X$ או ש־$x \in \Uu$ או ש־$X \setminus x \in \Uu$.
\end{definition}
זהו למעשה מסנן שמקיים את התכונה המהותית שכל קבוצה היא או גדולה, או קטנה במובן שהמשלים שלה הוא גדול.
\begin{definition}[מכפלה]
	תהי $\langle \Mm_i \mid i \in I \rangle$ סדרת מבנים בשפה $L$.
	נגדיר את המכפלה,
	\[
		\Nn = \prod_{i \in I} \Mm_i
	\]
	כך שמתקיים $N = \prod_{i \in I} M_i$, כלומר העולם מורכב מהמכפלה הקרטזית של העולמות של סדרת המבנים. \\
	לכל $R \in L$ יסמן יחס $n$־מקומי נגדיר,
	\[
		\langle f_0, \ldots, f_{n - 1} \rangle \in R^{\Nn}
		\iff \forall i \in I,\ \langle f_0(i), \ldots, f_{n - 1}(i) \rangle \in R^{\Mm_i}
	\]
	וכן לכל $F \in L$ סימן פונקציה $n$־מקומית, אז מתקיים,
	\[
		(F^{\Nn}(f_0, \ldots, f_{n - 1}))(i)
		= F^{\Mm_i}(f_0(i), \ldots, f_{n - 1}(i))
	\]
\end{definition}
נסיק אם כך שמכפלה היא מודל שמהווה בצורה ישירה מכפלה של מודלים אחרים.
אבל מבנה זה לא בהכרח משמר את התורה של המודלים המוכפלים, נראה דוגמה.
\begin{example}
	נניח ש־$F_0, F_1$ מודלים של שדות, ונניח גם שהשדות לא טריוויאליים. \\
	נגדיר את $F_0 \times F_1$, אז מודל זה הוא לא שדה, זאת שכן לאיבר $\langle 0_{F_0}, 1_{F_1} \rangle$ הוא שונה מאפס ואין לו הופכי.
\end{example}
המטרה שלנו היא למצוא דרך להכפיל שתשמר את המבנה והתורה באיזשהו אופן.
המטרה שלנו היא למצוא דרך ליצור מכפלה ככה שהצורה נשמרת אבל שאנחנו לא מורידים יותר מדי איברים, אלא כמות שמספיקה כדי לא לשבור את התורה.
ווש (Łoś) הצליח במשימה זו, זאת על־ידי שימוש במסננים.
\begin{definition}[יחס שקילות על מסנן]
	יהי $\Ff \subseteq \Pp(I)$ מסנן, ונניח ש־$\langle \Mm_i \mid i \in I \rangle$ סדרה של $L$־מבנים, ו־$\Nn$ מכפלתם. \\
	נאמר ש־$f \sim_{\Ff} g$ עבור $f, g \in N$ אם,
	\[
		\{ i \in I \mid f(i) = g(i) \} \in \Ff
	\]
\end{definition}
\begin{proposition}
	היחס $\sin_{\Ff}$ הוא יחס שקילות.
\end{proposition}
\begin{definition}[מכפלה מושרית מחלוקה]
	תהי $\langle \Mm_i \mid i \in I \rangle$ סדרת $L$־מבנים, ונגדיר את המודל $\Nn / \Ff$ כך שהעולם הוא $N / \sim_{\Ff}$. \\
	נגדיר גם שאם $R$ יחס $n$־מקומי, אז מתקיים,
	\[
		\langle [f_0], \ldots, [f_{n - 1}] \rangle \in R^{\Nn / \Ff}
		\iff \{ i \in I \mid \langle f_0(i), \ldots, f_{n - 1}(i) \rangle \in R^{\Mm_i} \} \in \Ff
	\]
	אם $F \in L$ סימן פונקציה $n$־מקומית, אז נגדיר,
	\[
		F^{\Nn / \Ff}([f_0], \ldots, [f_{n - 1}])
		= [F^{\Nn}(f_0, \ldots, f_{n - 1})]
	\]
\end{definition}
כלומר הפעם איחדנו חלק מהאיברים על־ידי הגדרה של שקילות עליהם, והשתמשנו במסנן כדי לייצג את החלוקה הזאת. אנחנו מדברים באיזשהו מובן על קבוצות האיברים הגדולים ומסתכלים על קבוצות אלה כאיברים שלנו.
לא ראינו שהגדרה זו בכלל תקפה, יכול להיות שהיא לא מוגדרת היטב.
\begin{proposition}
	$R^{\Nn / \Ff}, F^{\Nn / \Ff}$ מוגדרות היטב.
\end{proposition}
\begin{notation}
	אם $f, g \in N$ אז נסמן $e_{fg} = \{ i \in I \mid f(i) = g(i) \}$.
\end{notation}
\begin{exercise}
	הוכיחו את הטענה.
\end{exercise}
ראינו כי ההגדרה החדשה של מכפלה מרחיבה את ההגדרה הראשונה שלנו, וראינו גם שההגדרה הראשונה לא מצליחה לשמר את המבנה של המודל.
המסקנה שלנו היא שאם אנחנו רוצים לשמר את המבנה, אנחנו צריכים ללכת לכיוון ההפוך.
\begin{definition}[על־מכפלה וחזקה]
	תהיינה $\langle \Mm_i \mid i \in I \rangle$ סדרת של $L$־מבנים, ויהי על־מסנן $\Uu \subseteq \Pp(I)$, אז $\Nn = \prod_{i \in I} \Mm_i / \Uu$ נקרא על־מכפלה. \\
	אם $\Mm_i = \Mm_j$ לכל $i, j \in I$ אז נקרא ל־$\Nn$ על־חזקה.
\end{definition}
\begin{lemma}
	תהי $M_i$ סדרת מודלים ו־$\Uu$ על־מסנן.
	נניח ש־$f_0, \ldots, f_{n - 1} \in N$ ו־$t(x_0, \ldots, x_{n - 1})$ שם עצם מעל $L$.
	אז מתקיים,
	\[
		t^{\Nn / \Uu}([f_0], \ldots, [f_{n - 1}])
		= {[ t^{\Nn}(f_0, \ldots, f_{n - 1}) ]}_{\Uu}
	\]
\end{lemma}
\begin{proof}
	באינדוקציה על $t$.
	אם $t = x$ אז,
	\[
		t^{\Nn / \Uu}([f])
		= [f]
		= [t^{\Nn}(f)]
	\]
	אם $t = F^{\Nn / \Uu}(t_0, \ldots, t_{n - 1})$ אז מתקיים,
	\begin{align*}
		t^{\Nn / \Uu}([f_0], \ldots, [f_{n - 1}])
		& = F^{\Nn / \Uu}(t_0^{\Nn / \Uu}([f_0], \ldots, [f_{n - 1}]), \ldots, t_0^{\Nn / \Uu}([f_0], \ldots, [f_{n - 1}])) \\
		& = F^{\Nn / \Uu}([t_0^{\Nn}(f_0, \ldots, f_{n - 1})], \ldots, [t_{n - 1}^{\Nn}(f_0, \ldots, f_{n - 1})]) \\
		& = [F^{\Nn}(t_0^{\Nn}(f_0, \ldots, f_{n - 1}))] \\
		& = [t^{\Nn}(f_0, \ldots, f_{n - 1})]
	\end{align*}
	והשלמנו את מהלך האינדוקציה.
\end{proof}
\begin{theorem}[ווש]
	נניח ש־$\langle \Mm_i \mid i \in I \rangle$ סדרת $L$־מודלים ו־$\Uu \subseteq \Pp(I)$ על־מסנן. \\
	אז אם $\varphi(x_0, \ldots, x_{n - 1})$ אז,
	\[
		\Nn / \Uu \models \varphi([f_0], \ldots, [f_{n - 1}])
		\iff \{ i \in I \mid \Mm_i \models \varphi(f_0(i), \ldots, f_{n - 1}(i)) \} \in \Uu
	\]
\end{theorem}
\begin{proof}
	באינדוקציה על מבנה הנוסחה. \\
	נתחיל בנוסחה אטומית, $\varphi = R(t_0(x), \ldots, t_{n - 1}(x))$, אז מתקיים,
	\begin{align*}
		& \Nn / \Uu \models \varphi([f_0], \ldots, [f_{n - 1}]) \\
		\iff & (t_0^{\Nn / \Uu}([f_0], \ldots, [f_{n - 1}]), \ldots, t_{n - 1}^{\Nn / \Uu}([f_0], \ldots, [f_{n - 1}])) \in R^{\Nn / \Uu} \\
		\iff & ([t_0^{\Nn}(f_0, \ldots, f_{n - 1})], \ldots, [t_{n - 1}^{\Nn}(f_0, \ldots, f_{n - 1})]) \in R^{\Nn / \Uu} \\
		\iff & \{ i \in I \mid (t_0^{\Nn}(f_0, \ldots, f_{n - 1})(i), \ldots, t_{n - 1}^{\Nn}(f_0, \ldots, f_{n - 1})(i)) \in R^{\Mm_i} \} \in \Uu \\
		\iff & \{ i \in I \mid t^{\Mm_i}(f_0(i), \ldots, f_{n - 1}(i)) \in R^{\Mm_i} \} \in \Uu
	\end{align*}
	נניח שהטענה נכונה ל־$\varphi$ ונבדוק את $\lnot \varphi$,
	\begin{align*}
		& \Nn / \Uu \models \varphi([f_0], \ldots, [f_{n - 1}]) \\
		\iff & \{ i \in I \mid \Mm_i \models \varphi(f_0(i), \ldots, f_{n - 1}(i)) \} \in \Uu \\
		\iff & \{ i \in I \mid \Mm_i \models \lnot \varphi(f_0(i), \ldots, f_{n - 1}(i)) \} \notin \Uu
	\end{align*}
	נניח שהטענה נכונה ־$\varphi, \psi$, אז,
	\[
		\Nn / \Uu \models (\varphi \land \psi)([f_0], \ldots, [f_{n - 1}])
		\iff (\Nn / \Uu \models \varphi(\ldots)) \land (\Nn / \Uu \models \psi(\ldots))
	\]
	וזה נכון אם ורק אם $\{ i \in I \mid \varphi(\ldots) \} \in \Uu$ וגם עבור $\psi$, אבל $\Uu$ סגורה לחיתוך ולכן גם $\{ i \in I \mid \Mm_i \models (\varphi \land \psi)(\ldots) \} \in \Uu$. \\
	נעבור לחלק האחרון ונניח ש־$\psi(x_0, \ldots, x_{n - 1}) = \exists x_n\ \varphi(x_0, \ldots, x_n)$. \\
	נניח ש־$\Nn / \Uu \models \psi([f_0], \ldots, [f_{n - 1}])$ ולכן קיים $[g] \in N / \Uu$ כך ש־$\varphi^{\NN / \Uu}([f_0], \ldots, [f_{n - 1}], [g])$.
	אז מהנחת האינדוקציה,
	\[
		A = \{ i \in I \mid \Mm_i \models \varphi(f_0(i), \ldots, f_{n - 1}(i), g(i)) \} \in \Uu
	\]
	לכל $i \in A$ נקבל ש־$\Mm_i \models \exists v\ \varphi(f_0(i), \ldots, f_{n - 1}(i), v)$ ולכן גם,
	\[
		\{ i \in I \mid \Mm_i \models \psi(f_0(i), \ldots, f_{n - 1}(i)) \} \in \Uu
	\]
	וסיימנו את הכיוון הראשון. \\
	נניח בכיוון ההפוך ש־$B = \{ i \in I \mid \Mm_i \models \psi([f_0], \ldots, [f_{n - 1}]) \} \in \Uu$.
	לכל $i \in B$ נבחר $g_i \in \Mm_i$ כך שמתקיים,
	\[
		\Mm_i \models \psi(f_0(i), \ldots, f_{n - 1}(i), g_i)
	\]
	עבור $i \in I \setminus B$ נבחר $b_i$ שרירותי.
	נגדיר את הפונקציה $g(i) = g_i$ לכל $i \in I$ ולכן $g \in \Nn$, אז מהנחת האינדוקציה,
	\[
		\Nn / \Uu \models \varphi([f_0], \ldots, [f_{n - 1}], [g])
		\implies \Nn / \Uu \models \psi([f_0], \ldots, [f_{n - 1}])
	\]
	והטענה נובעת.
\end{proof}
\begin{theorem}[הקומפקטיות]
	אם $T$ תורה ספיקה סופית אז היא ספיקה.
\end{theorem}
\begin{proof}
	נסמן $I = \{ S \subseteq T \mid |S| < \omega \}$.
	לכל $S \in I$ נגדיר את המודל $\Mm_S$, קיים כזה מהספיקות הסופית.
	לכל $t \in I$ נסמן $Y_t = \{ w \in I \mid w \supseteq t \}$.
	לאוסף $\{ X_s \mid s \in I \}$ יש את תכונת החיתוך הסופי.
	יהי $\Uu$ על־מסנן מעל $I$ כך ש־$X_S \in \Uu$ לכל $S \in I$. \\
	נגדיר את $\Nn = \prod_{S \in I} \Mm_S / \Uu$ ונטען ש־$\Nn \models T$. \\
	יהי $\varphi \in T$ אז $X_{\{ \varphi \}} \in \Uu$ ולכן $X_{\{ \varphi \}} \subseteq \{ t \in I \mid M_t \models \varphi \} \in \Uu$.
	ממשפט ווש $\Nn \models \varphi$.
\end{proof}
\begin{corollary}
	יהי $\kappa$ מונה אינסופי ויהי $\Aa$ מודל. נסמן $\Aa_i = \Aa$ לכל $i \in \kappa$.
	יהי $\Uu$ על־מסנן מעל $\kappa$, ויהי $\iota : \Aa \to \Aa^{\kappa} / \Uu$ על־ידי $\iota(a) = [c_a]$. \\
	אז $\iota$ שיכון אלמנטרי.
\end{corollary}
\begin{proof}
	עבור נוסחה $\varphi$ מתקיים,d
	\[
		\Aa^{\kappa} / \Uu \models \varphi(\iota(a_0), \ldots, \iota(a_{n - 1}))
		\iff \{ i \in I \mid \Aa_i \models \varphi(a_0, \ldots, a_{n - 1}) \} \in \Uu
		\iff \Aa \models \varphi(a_0, \ldots, a_{n - 1})
	\]
\end{proof}

\section{שיעור 4 --- 9.11.2025}
\subsection{חילוץ כמתים}
\begin{definition}[תורה מחלצת כמתים]
	תהי $T$ תורה בשפה $L$, נאמר ש־$T$ מחלצת כמתים אם לכל נוסחה $\varphi(x_0, \ldots, x_{n - 1})$ קיימת נוסחה חסרת כמתים $\psi(x_0, \ldots, x_{n - 1})$,
	כך ש־$T \models \forall x_0 \ldots \forall x_{n - 1}\ (\varphi \leftrightarrow \psi)$.
\end{definition}
\begin{remark}
	יתכן שנגיע למצב שסתירה או טאוטולוגיה שקולות לפסוק חסר כמתים, אבל לא בהכרח השפה עשירה מספיק כדי לדבר על הפסוקים הללו.
	בהתאם החל מעכשיו נניח ש־$\perp$ חסרת כמתים, ולעשה כאיווי ריק של נוסחות אטומיות.
\end{remark}
\begin{remark}
	נשים לב שאם בשפה אין קבועים אז כשנפעיל את הגדרה על פסוק $\varphi$ נקבל ש־$\psi \in \{ \perp, \lnot \perp \}$.
\end{remark}
\begin{example}
	נניח ש־$T = \operatorname{DLO}$, תורת הסדרים הקוויים הצפופים ללא נקודות קצה.
	$T$ מחלצת כמתים ואין לה קבועים ולכן היא שלמה. \\
	תהי נוסחה $\varphi(x_0, \ldots, x_{n - 1})$, ונבחן את $\Sigma$ קבוצת הנוסחות מהצורה,
	\[
		\bigwedge_{i, j} {(x_i = x_j)}^{\varepsilon_{ij}}
		\land \bigwedge_{i, j} {(x_i \le x_j)}^{\varepsilon_{ij}}
	\]
	כאשר $\varepsilon_{ij}$ הם הנוסחה או שלילתה, נבחין כי האוסף הזה הוא סופי.
	נגדיר גם את $\Sigma_\varphi \subseteq \Sigma$ תת־האוסף כך שמתקיים $\psi \in \Sigma_\varphi \iff T \models \forall \bar{x}\ (\psi \to \varphi)$.
	אז מתקיים $T \models \forall \bar{x}\ (\bigvee \Sigma_\varphi \to \varphi)$ ונרצה לבדוק את הכיוון ההפוך.
	כלומר עלינו להראות שאם $\varphi(a_0, \ldots, a_{n - 1})$ מתקיים אז יש $\psi \in \Sigma_\varphi$ עבורו $\psi(a_0, \ldots, a_{n - 1})$ נכון.
	נשים לב כי כל זוג נוסחות שונות מ־$\Sigma$ סותרות זו את זו ולכן ל־$(a_0, \ldots, a_{n - 1})$ יש $\psi \in \Sigma$ יחיד כך ש־$\psi(a_0, \ldots, a_{n - 1})$ נכון.
	נניח בשלילה ש־$\psi \notin \Sigma_\varphi$ ושהוא בן־מניה.
	בלי הגבלת הכלליות אנו דנים במודל בו קיימים $b_0, \ldots, b_{n - 1}$ כך שמתקיים $\psi(b_0, \ldots, b_{n - 1})$ אבל $\lnot \varphi(a_0, \ldots, a_{n - 1})$ אבל קיימת $\sigma : a_i \mapsto b_i$ כסתירה.
\end{example}
\begin{remark}
	חילוץ כמתים תלוי בבחירת השפה $L$.
	לדוגמה אם $L$ שפה כלשהי ונגדיר את,
	\[
		\tilde{L} = L \cup \{ R_\varphi(x_0, \ldots, x_{n - 1}) \mid \varphi(x_0, \ldots, x_{n - 1}) \text{ is a formula} \}
	\]
	(הרחבת מורלי) ונגדיר את התורה,
	\[
		\tilde{T} = T \cup \{ \forall \bar{x}\ \varphi \leftrightarrow R_\varphi \mid \varphi \in \form_L \}
	\]
	אז נקבל תורה מחלצת כמתים.
\end{remark}
\begin{definition}[נוסחת קיים פרימיטיבית]
	נוסחת $\exists$ פרימיטיבית היא נוסחה מהצורה $\exists x \bigwedge_{i < n} \psi_i^{\varepsilon_i}$ כאשר $\psi_i$ אטומית.
\end{definition}
\begin{lemma}
	$T$ מחלצת כמתים אם ורק אם לכל נוסחת $\exists$ פרימיטיבית $\varphi$ יש נוסחה חסרת כמתים $\psi$ כך שמתקיים,
	\[
		T \models \forall \bar{x}\ (\varphi \leftrightarrow \psi)
	\]
\end{lemma}
\begin{proof}
	נוכיח באינדוקציה על מבנה הנוסחה. לנוסחות אטומיות הטענה טריוויאלית. גם להוספת שלילה הטענה נובעת ישירות, וכך גם לגימום. \\
	נבחן את המקרה של הוספת כמת, כלומר $\exists x \varphi$.
	לפי הנחת האינדוקציה $\varphi$ שקולה לנוסחה $\psi$ חסרת כמתים.
	אז $\psi$ שקולה לאיווי סופי של נוסחות מהצורה $\bigwedge \psi_i^{\varepsilon_i}$.
	ואז מתקבל,
	\[
		\exists x\ \bigvee_{i < m} \rho_i
		\equiv \bigvee_{i < m} \exists x\ \rho_i
	\]
	ולכן $\exists x\ \psi$ שקולה לאיווי של נוסחות $\exists$ פרימיטיבית.
\end{proof}
עתה נוכל לעבור למבחן כללי לחילוץ כמתים.
\begin{notation}
	יהי $\Mm$ מבנה של $L$ ויהי $A \subseteq M$, אז נסמן $\langle A \rangle \subseteq \Mm$ תת־המבנה הנוצר על־ידי $A$.
	במידה שאין קבועים ו־$A = \emptyset$ אז נגדיר $\langle \emptyset \rangle = \emptyset$, למרות שהו לא תת־מבנה.
\end{notation}
\begin{theorem}
	התנאים הבאים שקולים,
	\begin{enumerate}
		\item $T$ מחלצת כמתים
		\item לכל זוג מודלים $\Mm, \Nn \models T$ ו־$\langle A \rangle$ תת־מבנה נוצר סופית משותף (כולל $A = \emptyset$) ולכל פסוק קיים פרימיטיבי $\varphi$ ב־$L(\langle A \rangle)$, \\
			מתקיים $\Mm_A \models \varphi \iff \Nn_A \models \varphi$ (כלומר העשרת המבנים על־ידי קבועים ל־$A$).
	\end{enumerate}
\end{theorem}
\begin{proof}
	$1 \implies 2$:
	אם $\varphi$ פסוק $\exists$ פרימיטיבי אז $\varphi$ הוא מהצורה $\tilde{\varphi}(d_{a_0}, \ldots, d_{a_{n - 1}})$ עבור $\tilde{\varphi} \in \form_L$.
	עם המשתנים $x_0, \ldots, x_{n - 1}$, הנחנו ש־$T$ מחלצת כמתים ולכן $\tilde{\varphi}$ שקולה לנוסחה חסרת כמתים $\tilde{\psi}$ כך ש־$\tilde{\psi} \in \form_L$. \\
	אז נובע ש־$\varphi$ שקולה ל־$\tilde{\psi}(d_{a_0}, \ldots, d_{a_{n - 1}})$, אז,
	\[
		\Mm_A \models \tilde{\psi}(d_{a_0}, \ldots, d_{a_{n - 1}})
		\iff \langle A \rangle \models \tilde{\psi}(a_0, \ldots, a_{n - 1})
		\iff \Nn_A \models \tilde{\psi}(d_{a_0}, \ldots, d_{a_{n - 1}})
	\]
	ומצאנו שהטענה חלה.

	$2 \implies 1$:
	יהי פסוק קיים פרימיטיבי $\varphi$ ונבחן את התורות נבחן את $T_1 = T \cup \{ \varphi \}$ ו־$T_2 = T \cup \{ \lnot \varphi \}$.
	אם נמצא פסוק חסר כמתים ב־$L(A)$, $\psi$, כך שמתקיים $T_1 \models \psi$ וכן $T_2 \models \lnot \psi$ אז סיימנו.
	\[
		T_1 \models \psi \iff T \models \varphi \to \psi
	\]
	בפסוקים $\varphi, \psi$ יש קבועים מתוך $A$ ואנו נרצה להראות ש־$T \models \forall \bar{x}\ (\tilde{\varphi} \to \tilde{\psi})$.
	זוהי הכללה על־ידי קבועים שתעבוד כאשר הקבועים אינם בשפה. \\
	באופן דומה,
	\[
		T_2 \models \lnot \psi
		\iff T \models \lnot \varphi \to \lnot \psi
		\iff T \models \psi \to \varphi
	\]
	לכן נרצה להראות הוא שלכל $\Mm \models T_1$ ו־$\Nn \models T_2$ יש פסוק חסר כמתים $\psi$ כך ש־$\Mm \models \psi$ ו־$\Nn \models \lnot \psi$.
	נניח ש־$c_0, \ldots, c_{n - 1}$ קבועים חדשים שנציב במקום המשתנים של $\varphi$ (ובהמשך נשתמש בהם ב־$A$). \\
	אם בשלילה אכן אין פסוק $\psi$ חסר כמתים בשפה $L(c_0, \ldots, c_{n - 1})$ המפריד בין $\Mm$ ל־$\Nn$ אז מתקיים,
	\[
		\langle A \rangle = \langle c_0^\Mm, \ldots, c_{n - 1}^\Mm \rangle
		\cong \langle c_0^\Nn, \ldots, c_{n - 1}^\Nn \rangle
	\]
	נבחין כי האינדוקציה על ידי רקורסיה של שמות עצם ב־$L(\{c_0, \ldots, c_{n - 1}\})$ כאשר בכל שלב הפונקציה אכן מוגדרת היטב וחד־חד ערכית בזכות הסכמה בין $\Mm$ ו־$\Nn$ על נוסחות חסרות כמתים בשפה המועשרת.
	לכן בלי הגבלת הכלליות $A \subseteq N$ ונוכל להניח את הנחות המשפט.
	לכן $\Mm_A \models \varphi \iff \Nn_A \models \varphi$ בסתירה להגדרת $T_1, T_2$.
	נובע שבהכרח יש הפרדה על־ידי $\psi \in \Sigma$ מלמה\ \ref{separation-lemma-1} ונקבל ש־$T_1$ ו־$T_2$ מופרדות על־ידי פסוק מ־$\Sigma$.
	במקרים בהם יש ל־$\varphi$ משתנים חופשיים או שיש ל־$L$ קבועים, ובמקרה שנותר $\varphi$ פסוק ב־$L$ ול־$L$ אין קבועים.
	במקרה זה נפעיל את ההנחה ל־$A = \emptyset$ ונקבל ש־$\varphi \in T$ או $\lnot \varphi \in T$ ולכן $T \models \varphi \leftrightarrow \perp$ או $T \models \varphi \leftrightarrow (\lnot \perp)$.
\end{proof}
נעבור לשימוש במשפט.
\begin{definition}
	$\operatorname{ACF}$ היא התורה בשפה $L = \{0, 1, +, \cdot\}$ של שדות סגורים אלגברית.
	היא מורכבת מאקסיומות השדה, אקסיומת השדה הסגור אלגברית,
	\[
		\forall a_0 \ldots \forall a_n (a_n \ne 0 \to \exists x\ a_n x^n + a_{n - 1} x^{n - 1} + \cdots + a_0 = 0)
	\]
	עבור מציין $p$ נוסיף את האקסיומה $c_p = \overbrace{1 + \cdots + 1}^{p \text{ times}} = 0$ ועבור מציין 0 נוסיף את $\{ \lnot c_p \mid p \text{ is prime}\}$. \\
	נסמן ב־$\operatorname{ACF}_p$ את התורה הנוצרת עבור מציין $p$.
\end{definition}
\begin{theorem}
	$\operatorname{ACF}$ מחלצת כמתים.
\end{theorem}
\begin{proof}
	נוכיח שאם $\Mm, \Nn \models \operatorname{ACF}$ ו־$\Aa \subseteq \Mm, \Nn$ נוצר סופית ו־$\varphi$ פסוק $\exists$ פרימיטיבי ב־$L(A)$ אז $\Mm_A \models \varphi \iff \Nn_A \models \varphi$.
	נשים לב שיש תת־שדה $F_A \subseteq \Mm$ ואיזומורפיזם $f : F_A \to \tilde{F}_A$ כך ש־$\tilde{F}_A \subseteq \Nn$ וכן $f \restriction A = \id_A$.
	איברי $F_A$ הם מהצורה $\frac{p(a_0, \ldots, a_{n - 1})}{q(a_0, \ldots, a_{n - 1})}$ כאשר $p, q$ פולינומים ממעלה $n$ עם מקדמים שלמים.
	כעת נגדיר את $f$ על־ידי,
	\[
		{\left(\frac{p(a_0, \ldots, a_{n - 1})}{q(a_0, \ldots, a_{n - 1})}\right)}^\Mm
		\mapsto {\left(\frac{p(a_0, \ldots, a_{n - 1})}{q(a_0, \ldots, a_{n - 1})}\right)}^\Nn
	\]
	$f$ מוגדרת היטב היא שניתן לחשב פורמלית סכום של פונקציות רציונליות והתאפסות של המכנה $q$ שקולה לשוויון של שני פולינומים ב־$\Aa$.
	ידוע ש־$\Aa$ תת־מבנה משותף ל־$\Mm$ ול־$\Nn$ החישוב הוא זהה ולכן $f$ היא אכן איזומורפיזם.
	בלי הגבלת הכלליות נניח שגם $\Aa$ שדה.
	נסיק ש־$\varphi$ היא מהצורה $\exists x\ \bigwedge_{i < n} (p_i(x) = 0) \land \bigwedge_{i < m} (q_i(x) \ne 0)$, שכן אחרת נוכל להעביר אגפים.
	נניח ש־$n > 0$ ונניח ש־$\Mm \models \varphi$ וש־$b \in M$ מעיד על כך.
	נסמן את $m(x)$ הפולינום המינימלי של $b$ ב־$A[x]$, אז לכל $i < n$ מתקיים $m \mid p_i$.
	בנוסף $m \nmid \prod_{i < n} q_i$, זאת שכן $m \nmid q_i$ לכל $i < n$ והוא אי־פריק.
	ב־$\Nn$ יש שורש ל־$m$, נסמן אותו ב־$\tilde{b}$, איבר זה לא מאפס את $\prod q_i$, אחרת הפולינום המינימלי של $\tilde{b}$, $\tilde{m}$, יחלק את $m$ וגם את $\prod q_i$ ולכן בהכרח יהיה שונה מ־$m$ בסתירה לאי־פריקות $m$. \\
	אם $n = 0$ אז נשתמש בכך שיש אינסוף איברים בשדה סגור אלגברית ורק מספר סופי שלהם מאפס את $\prod q_i$.
\end{proof}
\begin{remark}
	הטיעון למעשה מניב אלגוריתם להמרת נוסחת $\exists$ פרימיטיבית לנוסחה חסרת כמתים.
\end{remark}
\begin{corollary}
	נניח ש־$\FF$ שדה סגור אלגברית ונניח ש־$X \subseteq X$ תת־קבוצה גדירה עם פרמטרים, כלומר $X = \{ x \in \FF \mid \FF \models \varphi(x) \}$ עבור נוסחה $\varphi \in \form_{L_{\operatorname{ACF}}(\FF)}$.
	אז במקרה זה $X$ סופית או שמשלימתה סופית.
\end{corollary}
עתה נרצה לעבור לדבר על ממשיים במטרה להראות שגם שם אפשר לחלץ כמתים.
\begin{definition}[תורת השדות הסגורים ממשית]
	RCF היא תורה מעל $L = \{0, 1, +, \cdot, \le\}$, תורת השדות הסגורים ממשית היא התורה של שדה סדור ובנוסף,
	\begin{enumerate}
		\item משפט ערך הביניים לפולינומים: אם $f$ פולינום ו־$f(a) \cdot f(b) \le 0$ אז קיים $a \le c \le b$ כך ש־$f(c) = 0$
		\item משפט רול לפולינומים:
			אם $f$ פולינום ו־$f(a) = f(b)$ ($a < b$) אז קיים $a < b < c$ כך ש־$f'(c) = 0$,
			כאשר $f'$ היא הנגזרת הפורמלית של $f$
	\end{enumerate}
	אקסיומות השדה הסדור הן:
	\begin{enumerate}
		\item אם $a \le b$ אז $a + c \le b + c$
		\item אם $0 \le a, b$ אז $0 \le a \cdot b$
	\end{enumerate}
	בנוסף לאקסיומות השדה.
\end{definition}
\begin{remark}
	בספרות מקובלת ההגדרה השקולה ששדה סגור ממשית הוא שדה סדור בו לכל איבר חיובי יש שורש ריבועי ולכל פולינום ממעלה אי־זוגית יש שורש.
\end{remark}
\begin{theorem}
	$\operatorname{RCF}$ מחלצת כמתים.
\end{theorem}
\begin{proof}
	כמו במקרה הקודם נבחר $\Aa \subseteq \Mm, \Nn$ תת־שדה של $\Mm, \Nn \models \operatorname{RCF}$, ותהי $\varphi$ נוסחה $\exists$ פרימיטיבית.
	אז $\varphi$ מהצורה $\exists x\ \bigwedge \psi_i^{\varepsilon_i}$ עבור $\psi_i$ אטומיות.
	אז $\psi_i$ מהצורה $p_i(x) = 0$ או $\ne 0$ או $\ge 0$.
	בנוסף $p_i(x) \ne 0$ שקול ל־$p_i(x) > 0 \lor p_i(x) < 0$ ולכן ניתן להציג את $\varphi$ כך ש־$\psi_i$ הוא $p_i(x) = 0$ או $p_i(x) > 0$.
\end{proof}

\section{שיעור 5 --- 16.11.2025}
\subsection{שדות סגורים ממשית}
מטרתנו היא הוכחת המשפט בו סיימנו את השיעור הקודם.
\begin{notation}
	עבור $a \in F$ איבר בשדה סדור, נסמן $\operatorname{sgn}(a) \in \{0, -1, 1\}$ להיות $0$ אם $a = 0$, וכן $1$ אם $a > 0$ ובשאר המקרים $-1$.
\end{notation}
\begin{proposition}
	נניח ש־$f = \sum_{i = 0}^n a_i x^i$ פולינום.
	אז יש $A_0$ כך שלכל $A_0 < x$ מתקיים $\operatorname{sgn}(f(x)) = \operatorname{sgn}(a_n)$ ועבור $x < -A_0$ מתקיים $\operatorname{sgn}(f(x)) = {(-1)}^n \operatorname{sgn}(a_n)$.	
\end{proposition}
כלומר החל ממרחק מסוים מהראשית הסימן של פולינום נקבע רק על־ידי המונום המוביל שלו.
\begin{proof}
	נבחר $A_0 > \sum_{i = 0}^n \frac{|a_i|}{|a_n|} \cdot 2$ ונניח ש־$a_n > 0$.
	אז במקרה זה,
	\[
		a_n x^n + a_{n - 1} x^{n - 1} + \cdots
		> a_n x^{n - 1} \cdot \left(\sum_{i = 0}^n \frac{|a_i|}{|a_n|} \cdot 2\right) + a_{n - 1} x^{n - 1} + \cdots
		> a_n x^{n - 1} \cdot \left(\sum_{i = 0}^n \frac{|a_i|}{|a_n|}\right) + \sum_{i = 0}^{n - 1} (|a_i| + a_i) x^i
		\ge A_0
		> 0
	\]
	הצד השני זהה.
\end{proof}
\begin{lemma}
	נניח ש־$f \in F[x]$ פולינום בשדה סגור ממשית, ונניח ש־$a < b$ ושלכל $c \in (a, b)$ אז $f'(c) \ne 0$.
	אז אם $f(a) \cdot f(b) > 0$ אז הסימן של $f(c)$ קבוע לכל $c \in (a, b)$ ושווה לאחד הסימנים של $f(a), f(b)$.
	במקרה זה גם $f$ מונוטונית ממש ב־$[a, b]$.
	אם $f(a) \cdot f(b) < 0$ אז לכל סימן $s \in \{-1, 0, 1\}$ קיים $c \in (a, b)$ כך ש־$\operatorname{sgn}(f(c)) = s$.
\end{lemma}
כדי להוכיח את הטענה נראה קודם את משפט לגרנז'.
\begin{theorem}
	אם $a < b$ אז יש $c \in (a, b)$ כך ש־$f'(c) = \frac{f(b) - f(a)}{b - a}$.
\end{theorem}
\begin{proof}
	נגדיר $g(x) = \frac{f(b) - f(a)}{b - a}(x - a) - (f(x) - f(a))$.
	אז מתקיים $g(a) = g(b) = 0$ ולכן קיים $c \in (a, b)$ כך ש־$g'(c) = 0$.
	אבל $g'(c) = \frac{f(b) - f(a)}{b - a} - f'(c) = 0$ מהגדרת הנגזרת הפורמלית.
\end{proof}
נעבור להוכחת הלמה.
\begin{proof}
	הסימן של $f'(x)$ קבוע ל־$x \in (a, b)$ אחרת מערך הביניים הייתה נקודת איפוס.
	אם הסימן חיובי אז לכל $c < d$ בקטע,
	\[
		0 < \frac{f(d) - f(c)}{d - c}
	\]
	ולכן $f(d) > f(c)$ והטענה דומה בכיוון ההפוך.

	נניח ש־$f(a), f(b) > 0$, אז ממונוטוניות לכל $c \in (a, b)$ נקבל $0 \le f(a) \le f(c)$ ולכן לא יכולה להיות התאפסות.

	ההוכחה לחלק האחרון של הלמה דומה.
\end{proof}
נעבור להוכחת המשפט.
\begin{proof}
	נניח ש־$K, L$ שדות סגורים ממשית ונניח ש־$F \subseteq K, L$ תת־שדה משותף.
	תהי $\varphi$ נוסחה $\exists$ פרימיטיבית ב־$L_F$.
	אז נסמן,
	\[
		\varphi = \exists x\ \left(\bigwedge_{i = 0}^{m - 1} f_i(x) = 0\right) \land \left(\bigwedge_{j = 0}^{n - 1} g_j > 0\right)
	\]
	בלי הגבלת הכלליות.

	נוכיח באינדוקציה את הטענה:
	נניח ש־$F$ שדה סדור כך ש־$F \subseteq K, L$ סגורים ממשית.
	נניח ש־$f_0, \ldots, f_n \in F[x]$ ואיברים $x_0 < \cdots < x_m$ איברים ב־$K$, אז קיימים $y_1 < \cdots < y_m$ ב־$L$ כך שלכל $0 \le i \le n$ ו־$0 \le j \le m$ מתקיים,
	\[
		\operatorname{sgn}_K(f_i(x_j))
		\operatorname{sgn}_L(f_i(x_j))
	\]
	המקרה ש־$m = 0$ מוכיח את חילוץ הכמתים. \\
	נוכיח את הטענה באינדוקציה על $d$ הדרגה המקסימלית של $f_1, \ldots, f_n$ ו־$\delta$ מספר הפולינומים בעלי דרגה $d$ באותה רשימה. \\
	עבור $d = 0$ הפולינומים קבועים והטענה טריוויאלית.
	נניח עתה ש־$d \ge 1$ וכן שהנחת האינדוקציה מתקיימת עבור $\delta - 1$.
	המקרה ש־$\delta = 1$ טענת האינדוקציה מתקיימת ל־$d' < d$.
	נניח שנתונה לנו הרשימה $f_0, \ldots, f_n$ ונניח בלי הגבלת הכלליות ש־$\deg f_0 = d$ וכן ש־$f_n = 0$, וכן,
	\[
		\forall i\ f_i' \in \{ f_0, \ldots, f_n \}
	\]
	ואף ש־$f_0 \mod f_i$ שייכת לרשימה.
	לבסוף גם נניח ש־$f_i \ne f_j$ לכל $i \ne j$.

	נבחין כי אם הלמה מתקיימת ל־$\langle g_i \mid i < n \rangle$ וניקח את $x_0 < \cdots < x_m$ להיות כל השורשים של $g_* = \prod_{g_i \ne 0} g_i$ ב־$K$ אז $y_0 < \cdots < y_m$ הם על השורשים של $g_*$ ב־$L$. \\
	נניח אחרת, ש־$y$ שורש נוסף ב־$L$ ונפעיל את הלמה מ־$L$ ל־$K$, אז $y_0 < \cdots y_j < y < y_{j + 1} < \cdots < y_m$ ונקבל ש־$x_0' < \cdots < x_{m + 1}'$ הם שורשי $g_*$ בסתירה.

	נניח שהנחת האינדוקציה חלה עבור $(f_1, \ldots, f_n)$ ויהיו $x_0 < \cdots < x_m$, אז בלי הגבלת הכלליות, רשימה זו מכסה את שורשי $f_* = \prod_{1 \le i < n} f_i$,
	וכן $x_0$ קטן מספיק כך שלכל $0 \le i$ הוא ${(-1)}^{\deg f_i}$ כפול סימן המקדם המוביל.
	נניח שגם $x_m$ גדול מספיק כך ש־$\operatorname{sgn}(f_i(x_m))$ סימן המקדם המוביל של $f_i$ לכל $i$. \\
	נבחן את האוסף $\{ x_i \mid \forall 0 \le j < n,\ f_j(x_i) \ne 0 \} \iff f_*(x_i) \ne 0$.
	נסמן ב־$N$ את גודל האוסף הזה, אז $N \ge 2$.
	אם $N = 2$, אז מהנחת האינדוקציה עבור $(f_1, \ldots, f_n)$ נתאים להם $y_1 < \cdots < y_{m - 1}$ שהם כל שורשי $f_*$ ב־$L$.
	נבחר $y_0$ מאוד קטן ו־$y_m$ מספיק גדול שיתאימו ל־$x_0, x_m$ בכל סימני הפולינומים.
	עבור $0 < j < m$ יש $0 \le i < n$ כך ש־$f_i(x_j) = 0$,
	\[
		f_0(x_j) = \overbrace{f_i(x_j) g(x_j)}^{= 0} + f_{i'}(x_j)
	.\]
	ולכן $\operatorname{sgn}^K f_0(x_j) = \operatorname{sgn}^K f_{i'}(x_j) = \operatorname{sgn}^L f_{i'}(y_j) = \operatorname{sgn}^L f_0(y_j)$.

	נעשה אינדוקציה פנימית על $N$.
	נניח ש־$x_j$ שאיננו $x_0$ או $x_m$ ואיננו שורש של $f_*$.
	לכל $0 \le i < n$ לא יתכן ש־$f_i(x_{j - 1}) = f_i(x_{j + 1}) = 0$, יתר על כן $f$ מונוטונית ממש בקטע $(x_{j - 1}, x_{j + 1})$.
	מהאינדוקציה על $N$ יש $y_0 < \cdots < y_{j - 1}$ עם סימנים מתואמים.
	נסתכל על הנקודות $y \in (y_{j - 1}, y_{j + 1})$.
	אם $i \ne 0$ אז $\operatorname{sgn} f_i(y)$ קבוע ושווה לסימן השונה מאפס של אחת הקצוות ואותו דבר קורה ל־$x_j$.
	ל־$i = 0$ יתכן כי מוחלף סימן באמצע.
	אם אכן $f_0(y_{j - 1}) \cdot f_0(y_{j + 1}) < 0$ אז לכל סימן $s$ יש $y_{j - 1} < y < y_{j + 1}$ כך ש־$\operatorname{sgn} f_0(y) = s$ בפרט עבור $\operatorname{sgn} f_0(x_i)$.
	אחרת הסימן קבוע וכל $y$ יעבוד.
\end{proof}
\begin{corollary}
	RCF תורה שלמה, שכן $\QQ$ מבנה משותף.
	למעשה התורה אפילו כריעה, אבל בסיבוכיות גבוהה מאוד.
\end{corollary}
\begin{remark}
	נניח ש־$K$ RCF, אז כל תת־קבוצה של $K$ גדירה אם ורק אם היא איחוד סופי של קטעים (לא בהכרח חסומים) וקבוצה סופית.
	תכונה זו נקראת O־מינימליות.
\end{remark}

\subsection{טיפוסים}
\begin{definition}[טיפוס]
	תהי $T$ תורה, $p \in S_n(T)$ הוא אוסף של נוסחות עם משתנים חופשיים $\varphi(x_0, \ldots, x_{n - 1})$ כך שהתורה,
	\[
		T \subseteq \{ \varphi(c_0, \ldots, c_{n - 1}) \mid \varphi \in p \}
	\]
	כאשר $c_0, \ldots, c_{n - 1}$ קבועים חדשים, היא תורה שלמה ועקבית.
	נקרא ל־$p$ כזה טיפוס שלם עם $n$ משתנים חופשיים. \\
	$p$ יקרא טיפוס חלקי אם $T \cup \{ \varphi(c_0, \ldots, c_{n - 1}) \mid \varphi \in p \}$ היא עקבית.
\end{definition}
\begin{remark}
	כל טיפוס חלקי ניתן להרחבה לטיפוס מלא.
\end{remark}
\begin{example}
	$S_0(T)$ הוא כל ההשלמות של $T$.
	$S_1(T)$ טיפוסים מעל $T$.
	בתורה $T = \Th(\QQ)$ אין טיפוסים, אבל ב־$\Th(\QQ_\QQ) = \diag(\QQ)$ יש $2^{\aleph_0}$ טיפוסים.
	טיפוס $p$ בתורה של $\QQ$ עם פרמטרים מ־$\QQ$ הוא מהצורה,
	\[
		\{ x < d_q \mid q \in H \} \cup \{ d_r < x \mid r \in L \}
	\]
	או שהוא מהצורה $x = d_q$.
\end{example}
\begin{example}
	נבחן את שדה ACF, לדוגמה על $\FF = \overline{\QQ}$ ונבחן את הטיפוסים ב־$S_1$ ב־$T = \diag(\FF)$.
	אז הטיפוסים הם המקרים $x = d_a$ או $P(x) \ne 0$ לכל $P \in \FF[x]$.
	נוכל גם לבחון את הטיפוסים מעל $T = \operatorname{ACF}$, במקרה זה או ש־$Q(x) = 0$, או הטיפוס שאומר ש־$x$ איננו אלגברי, כלומר שלכל $Q \in \FF[x]$ גדיר מתקיים $Q(x) \ne 0$.
\end{example}
\begin{definition}[מימוש והשמטת טיפוסים]
	נניח ש־$\Mm \models T$ ו־$p \in S_n(T)$, נאמר ש־$\Mm$ מממש את $p$ אם קיים $a \in M$ כך ש־$\Mm \models \varphi(a)$ לכל $\varphi \in p$, אחרת נאמר ש־$\Mm$ משמיט את $p$.
\end{definition}
\begin{remark}
	נאמר ש־$p$ טיפוס עם פרמטרים מ־$A \subseteq M$ כאשר $p$ טיפוס בשפה המועשרת על־ידי $A$ ביחס ל־$\Th(\Mm_A)$.
\end{remark}
\begin{definition}[נוסחה מבודדת]
	נאמר שנוסחה $\varphi(x)$ מבודדת את הטיפוס $p$ אם מתקיים $T \models \forall x\ (\varphi(x) \to  \psi(x))$ לכל $\psi \in p$,
	ובנוסף $T \cup \{ \exists x\ \varphi \}$ עקבית. \\
	טיפוס $p$ יקרא מבודד אם יש נוסחה שמבודדת אותו.
\end{definition}
\begin{remark}
	אם $T$ שלמה אז לכל $\Mm \models T$ כל טיפוס מבודד מתממש.
\end{remark}
\begin{theorem}[השמטת טיפוסים]
	תהי $T$ תורה שלמה ועקבית בשפה בת־מניה ו־$p \in S_1(T)$ טיפוס לא מבודד אז יש מודל $\Mm \models T$ שמשמיט את $p$. \\
	יתר על־כן, גם אם $\langle p_n \mid n < \omega \rangle$ סדרת טיפוסים לא מבודדים, אז יש מודל של $T$ שמשמיט את כולם.
\end{theorem}
\begin{proof}
	נתחיל מהעשרת השפה $L$ על־ידי אינסוף קבועים הנקין, כלומר הקבועים $c_{\varphi}$ לכל $\varphi$ נוסחה.
	תהי $T_H$ הרחבה של $T$ יחד עם $\exists x\ \varphi \to \varphi(c_\varphi)$.
	ונרחיב בעדינות את $T_H$ לתורה שלמה כך שלכל קבוע $d$ ולכל $p_n$ יהיה $\psi \in p_n$ כך ש־$\lnot \psi(d)$ מתקיים.

	תהי $\langle \langle d_n, p_{k_n} \rangle \mid n < \omega \rangle$ מניה של כל הזוגות של קבועים וטיפוס מהרשימה.
	בשלב ה־$n$ נתונה לנו תורה $T_n$, כאשר $T_0 = T_H$.
	נטען כי יש $\psi \in p_{k_n}$ כך ש־$T \cup \{ \lnot \psi(d_n) \}$ עקבית.
	אחרת יש $\varphi_0, \ldots, \varphi_{n - 1} \in T_n$ כך שמתקיים,
	\[
		T_H \cup \{ \varphi_0, \ldots, \varphi_{n - 1} \} \models \psi(d_n)
	\]
	לכל $\psi \in p_{k_n}$, כלומר,
	\[
		T_H \models \bigwedge \varphi_i \to \psi(d_n)
	\]
	לכל $\psi \in p_{k_n}$.
	אז יש פסוק $\varphi$ כך ש־$T \models \forall x\ (\varphi(x) \to \psi(x))$ לכל $\psi \in p_{k_n}$ בסתירה.
\end{proof}

\section{שיעור 6 --- 23.11.2025}
\subsection{שלמות מודלית}
נשים לב להערה הבאה.
\begin{remark}
	נניח ש־$T$ מחלצת כמתים ו־$\Mm, \Nn \models T$, אז אם $\Mm \subseteq \Nn$ אז גם $\Mm \prec \Nn$.
\end{remark}
נרצה אם כך לבחון את המקרה הזה ולהבינו.
\begin{definition}[שלמות מודלית]
	$T$ שלמה מודלית אם לכל $\Mm, \Nn \models T$ אם $\Mm \subseteq \Nn$ אז $\Mm \prec \Nn$.
\end{definition}
ועתה נוכל לנסות לאפיין מקרה זה.
\begin{definition}[עמיתה מודלית]
	נניח ש־$T$ ו־$T^*$ תורות מעל השפה $L$.
	נאמר ש־$T^*$ היא עמיתה מודלית של $T$ אם מתקיימים התנאים הבאים,
	\begin{enumerate}
		\item כל מודל של $T$ הוא תת־מבנה של מודל של $T^*$
		\item כל מודל של $T^*$ הוא תת־מודל של מודל של $T$
		\item $T^*$ שלמה מודלית
	\end{enumerate}
\end{definition}
\begin{example}
	אם $L$ שפת תורת החוגים ו־$T$ תורת החוגים הקומוטטיביים בלי מחלקי אפס, או אפשר לבחור את תורת השדות, אז נוכל לקחת את $T^*$ להיות ACF\@.
\end{example}
\begin{example}
	בשפת תורת הגרפים ותורת הגרפים אז $T^*$ תהיה תורת הגרפים המקריים, זו המקיימת,
	\[
		\forall x_0 \cdots \forall x_{n - 1} \forall y_0 \cdots \forall y_{n - 1} \exists z\ (\bigwedge_{i < j} x_i \ne y_i) \to \bigwedge E(x_i, z) \land \bigwedge \lnot E(y_j, z)
	\]
\end{example}
\begin{example}
	עבור $T$ תורת החבורות האבליות ללא פיתול, אז $T^*$ תהיה תורת החבורות האבליות חלוקה ללא פיתול.
\end{example}
נבחין כי במקרה יש חילוץ כמתים בכל הדוגמות, זהו לא באמת מקרה.
\begin{definition}[השלמה מודלית]
	במידה ש־$T^*$ מחלצת כמתים נאמר שהיא השלמה מודלית של $T$.
\end{definition}
ניזכר בלמה\ \ref{separation-lemma-2}, ונסמן,
\begin{notation}
	אם $T$ תורה אז נסמן $T_\forall = \{ \varphi \in \sent \mid \varphi = \forall x_0 \cdots \forall x_{n - 1} \psi, \psi \text{ is quantifier-free}, T \models \varphi \}$ קבוצת הפסוקים הכוללים ב־$T$.
\end{notation}
נעבור ללמה שתשמש אותנו.
\begin{lemma}
	נניח ש־$T_1, T_2$ תורות בשפה $L$, אז התנאים הבאים שקולים:
	\begin{enumerate}
		\item יש פסוק כולל $\varphi$ כך ש־$T_1 \models \varphi$ ו־$T_2 \cup \{ \lnot \varphi \}$ עקבית
		\item יש מודל של $T_2$ שלא ניתן לשכן במודל של $T_1$
	\end{enumerate}
\end{lemma}
\begin{proof}
	$1 \implies 2$:
	ברור, אם $\Mm \models T_2 \cup \{ \lnot \varphi \}$ אז לא ניתן לשכנו ל־$\Nn$ שמקיים את $T_1$,
	אחרת הוא יקיים את $\varphi$ ובפרט אם $c_0, \ldots, c_{n - 1}$ מעידים על $\lnot \varphi$ ב־$\Mm$ אז הם יעידו על $\lnot \varphi$ גם ב־$\Nn$.

	$2 \implies 1$:
	נניח את שלילת התנאי הראשון.
	לכל פסוק כולל $\varphi$ כך ש־$T_1 \models \varphi$ מתקיים ש־$T_2 \models \varphi$.
	נניח ש־$\Nn \models T_2$ מודל, ונניח ש־$\Mm \models \psi$ נוסחת קיים.
	אם $T_1 \cup \{ \psi \}$ לא עקבית אז $T_1 \models \lnot \psi$ ולכן $T_2 \models \lnot \psi$ בסתירה.
\end{proof}
המשמעות היא ששאלת קיום השיכון ניתנת לתרגום לשאלה על קבוצת הפסוקים הכוללים שהיא מוכיחה.
\begin{corollary}
	כל מודל של $T_\forall$ ניתן לשיכון במודל של $T$.
\end{corollary}
\begin{proof}
	נבחר $T_1 = T$ ו־$T_2 = T_\forall$ ונשתמש בלמה.
\end{proof}
\begin{definition}
	נניח ש־$\Mm \subseteq \Nn$ בשפה $L$, נאמר ש־$\Mm$ סגורה קיומית ביחס ל־$\Nn$,
	אם לכל נוסחה מהצורה $\exists x_0 \cdots \exists x_{n - 1}\ \psi$ עבור $\psi \in \form_{L(M)}$ חסרת כמתים,
	אז אם $\Nn \models \varphi$ אז גם $\Mm \models \varphi$.
\end{definition}
\begin{theorem}
	התנאים הבאים שקולים עבור תורה $T$:
	\begin{enumerate}
		\item $T$ שלמה מודלית
		\item $T$ סגורה קיומית, בין מודלים של $T_\forall$
		\item כל שיכון בין מודלים של $T$ משמר נוסחות כוללות
		\item כל נוסחה כוללת שקולה (ביחס ל־$T$) לנוסחת קיים
		\item כל נוסחה שקולה (ביחס ל־$T$) לנוסחת קיים
	\end{enumerate}
\end{theorem}
\begin{proof}
	$1 \implies 2$:
	נניח ש־$T$ שלמה מודלית ונניח ש־$\Mm \models T, \Nn \models T_\forall$, אז יש מודל $\Mm^* \models T$ כך ש־$\Nn \subseteq \Mm^*$.
	נובע ש־$\Mm \subseteq \Mm^*$ ולכן $\Mm \prec \Mm^*$.
	נניח ש־$\Nn \models \exists x_0 \cdots \exists x_{n - 1}\ \psi$ עבור נוסחה חסרת כמתים, כל שיכון הוא שיכון $\exists$ ולכן,
	\[
		\Mm^* \exists x_0 \cdots \exists x_{n - 1}\ \psi
		\implies \Mm \exists x_0 \cdots \exists x_{n - 1}\ \psi
	\]

	$2 \implies 3$:
	יהי שיכון $f : \Mm \to \Nn$ מודלים של $T$.
	אם יש נוסחה כוללת עם פרמטרים ב־$M$ שמתקיימת ב־$\Mm$, בלי הגבלת הכלליות $f = \id$, אם היא לא מתקיימת ב־$\Nn$ אז שלילתה, שהיא נוסחת קיים, מתקיימת ב־$\Nn$ ומההנחה שלנו אותה נוסחה תתקיים ב־$\Mm$.

	$3 \implies 4$:
	נניח ש־$\varphi(x_0, \ldots, x_{n - 1})$ היא נוסחה כוללת.
	נבחן את התורות $T \cup \{ \varphi(c_0, \ldots, c_{n - 1}) \}, T \cup \{ \lnot \varphi(c_0, \ldots, c_{n - 1}) \}$.
	נשתמש בלמה\ \ref{separation-lemma-2}, כל שיכון הוא שיכון $\forall$, אז לכל מודל של $T \cup \{ \varphi(c_0, \ldots, c_{n - 1}) \}$ יש פסוק קיים $\psi_M$,
	עבורו $T \cup \{ \lnot \varphi(c_0, \ldots, c_{n - 1}) \}$ ומקומפקטיות ניתן למצוא $\psi$ יחיד.
	אז נובע ש־$T \cup \{ \varphi(c_0, \ldots, c_{n - 1}) \} \models \psi$ וגם $T \cup \{ \lnot \varphi(c_0, \ldots, c_{n - 1}) \} \models \lnot \psi$.
	מתקיים בהתאם גם $T \models \varphi \leftrightarrow \psi$ וכן $T \models \forall z_0 \cdots \forall z_{n - 1}\ (\varphi(z_0, \ldots, z_{n - 1}) \leftrightarrow \psi(z_0, \ldots, z_{n - 1}))$.

	$4 \implies 5$:
	באינדוקציה על מבנה הנוסחה תוך שימוש בכך שאם $\varphi$ נוסחת קיים אז גם $\exists x\ \varphi$ נוסחת קיים.

	$5 \implies 1$:
	נניח ש־$\Mm \subseteq \Nn$ מודלים של $T$.
	אז,
	\[
		\Mm \models \varphi(a_0, \ldots, a_{n - 1}) \iff \Mm \models \exists z_0 \cdots \exists z_{n - 1}\ \psi(z_0, \ldots, z_{n - 1}, a_0, \ldots, a_{n - 1})
	\]
	אז נובע שגם $\Nn \models \exists z_0 \cdots \exists z_{n - 1}\ \psi(z_0, \ldots, z_{n - 1}, a_0, \ldots, a_{n - 1})$,
	אז גם $\Nn \models \varphi(a_0, \ldots, a_{n - 1})$.
	נסיק ש־$\diag(\Mm) \subseteq \Th(\Nn_M)$.
\end{proof}
\begin{lemma}
	התנאים הבאים שקולים עבור $T$,
	\begin{enumerate}
		\item $T$ שלמה מודלית
		\item $T$ היא התורה של אוסף המודלים של $T_\forall$ סגורה קיומית ביחס ל־$T_\forall$
	\end{enumerate}
\end{lemma}
\begin{proof}
	נניח כי $\Mm \models T_\forall$ הסגור קיומית ביחס למודלים של $T_\forall$.
	נבחן את $\Mm \subseteq \Nn \models T$ מודל כלשהו, ונרצה להשתמש במבחן טרסקי־ווט כדי להראות ש־$\Mm \prec \Nn$.
	נניח ש־$\Nn \models \exists x\ \psi(x, a_0, \ldots, a_{n - 1})$, עלינו להראות כי יש עדות לכך על־ידי איבר של $\Mm$.
	קיימת נוסחה $\rho$ כך שהיא נוסחת קיים וגם מתקיים,
	\[
		\Nn \models \forall z\ \psi(z, \ldots) \leftrightarrow \rho(z, \ldots)
	\]
	אבל $\Mm$ סגור קיומית ולכן,
	\[
		\Nn \models \exists x\ \rho(x, a_0, \ldots, a_{n - 1})
		\implies \Mm \models \exists x\ \rho(x, a_0, \ldots, a_{n - 1})
	\]
	נבחר את $b \in M$ להעיד על כך ולכן,
	\[
		\Mm \models \rho(b, a_0, \ldots, a_{n - 1})
		\implies \Nn \models \rho(b, a_0, \ldots, a_{n - 1})
	\]
	ולכן $\Nn \models \psi(b, a_0, \ldots, a_{n - 1})$ כרצוי.

	בכיוון ההפוך כך מודל של $T$ סגור קיומית ביחס למודלים של $T_\forall$ ולכן מהמשפט הקודם $T$ שלמה מודלית.
\end{proof}
\begin{corollary}
	אם תורה $T_0$ מכילה רק פסוקים כוללים, אז העמיתה המודלית שלה קיימת ויחידה.
\end{corollary}

\subsection{חזרה לטיפוסים}
הגדרנו טיפוסים כקבוצות של נוסחות עקביות ושלמות במשתנים חופשיים $x_0, \ldots, x_{n - 1}$.
טיפוס מעל תורה $T$ הוא טיפוס שמכיל את $T$.
אם נסיר את דרישת השלמות נקבל טיפוס חלקי.
טיפוס מבודד אם יש נוסחה $\psi$ כך ש־$\forall \varphi \in p,\ T \models \forall x_0 \cdots \forall x_{n - 1}\ \varphi \to \varphi$ כאשר $T \cup \{ \exists \bar{x}\ \psi \}$ עקבית.

נניח ש־$L$ שפה בת־מניה, נעשיר את $L$ על־ידי $\aleph_0$ קבועים חדשים ונסמן את השפה ב־$\tilde{L}$.
נניח ש־$T$ תורה עקבית ב־$L$, נגדיר טופולוגיה על האוסף $\Tt = \{ \tilde{T} \mid T \subseteq \tilde{T}, \tilde{T} \text{ is consistent and complete} \}$ על־ידי בסיס הפתוחות $U_\varphi$ כאשר $\varphi \in \sent_{\tilde{L}}$,
\[
	U_\varphi = \{ \tilde{T} \in \Tt \mid \varphi \in \tilde{T} \}
\]
\begin{proposition}
	$\Tt$ האוסדורף קומפקטי.
\end{proposition}
\begin{proof}
	נניח ש־$C = \{ U_{\varphi_i} \mid i \in I \}$ כיסוי של $\Tt$, כלומר לכל $\tilde{T} \in \Tt$ יש $i$ כך ש־$\tilde{T} \models \varphi_i$.
	אם אין תת־כיסוי סופי אז לכל $I_0 \subseteq I$ סופית, $T \cup \{ \lnot \varphi \mid i \in I_0 \}$ עקבית.
	מקומפקטיות נובע ש־$T \cup \{ \lnot \varphi_i \mid i \in I \}$ עקבית בסתירה, וזו סתירה לכך ש־$C$ כיסוי, ובהתאם $\Tt$ קומפקטי.

	נניח ש־$S_0, S_1 \in \Tt$ שונות, אז קיים $\varphi \in S_0$ כך ש־$\lnot \varphi \in S_1$ ולכן $S_0 \in U_\varphi$ וכן $S_1 \in U_{\lnot \varphi}$.
\end{proof}
ניזכר במשפט הבא מטופולוגיה,
\begin{theorem}[משפט הקטגוריה של בייר]
	נניח ש־$X$ מרחב האוסדורף קומפקטי ונניח כי $D_n \subseteq X$ צפופה ופתוחה ל־$n < \omega$, אז $\bigcap_{n < \omega} D_n \ne \emptyset$.
\end{theorem}
\begin{corollary}
	נניח ש־$\langle p_n \mid n < \omega \rangle$ סדרת טיפוסים חלקיים ולא מבודדים עם משתנים חופשיים $x_0, \ldots, x_{n - 1}$ מעל $T$.
	אז יש מודל $\Mm \models T$ שמשמיט את $p_n$ לכל $n$.
\end{corollary}
\begin{proof}
	נרצה להגדיר קבוצות פתוחות צפופות, לכל נוסחה $\psi(x)$ ב־$\tilde{L}$, נגדיר,
	\[
		E_\psi
		= \bigcup_{n < \omega} U_{(\exists x\ \psi \to \psi(c_n))}
	\]
	כאשר $c_n$ קבועים חדשים ב־$\tilde{L}$ שלא מופיעים ב־$\psi$.
	נראה ש־$E_{\psi}$ צפופה.
	תהי $U_\varphi$ פתוחה ולא ריקה, אז $U_\varphi \cap E_{\psi}$ היא קבוצת כל התורות שמכילות את $\varphi$ ומכילות פסוק מהצורה $\exists x\ \psi \to \psi(c_n)$ ל־$n$ כלשהו.
	אם $T \cup \{ \varphi \} \models \lnot (\exists x\ \psi \to \psi(c_n))$ לכל $n$ אז נבחר $c_n$ שלא מופיע ב־$\varphi$, ולכן,
	\[
		T \cup \{ \varphi \} \models \forall y\ (\lnot \exists x\ \psi \to \psi(y))
		\equiv \forall y\ (\exists x\ \psi \land \lnot \psi(y))
	\]
	נובע ש־$T \cup \{ \varphi \}$ לא עקבית, כלומר $U_\varphi = \emptyset$.
	עבור $k < \omega$ וקבועים $c_{i_0}, \ldots, c_{i_{m_k - 1}}$ נגדיר,
	\[
		D
		= D_{km, c_{1_0}, \ldots, c_{i_{m_k - 1}}}
		= \bigcup_{\psi \in p_k} U_{\lnot (c_{i_0}, \ldots, c_{i_{m_k - 1}})}
	\]
	נראה ש־$D$ צפופה.
	נניח ש־$U_\varphi$ קבוצה פתוחה ולא ריקה ונניח ש־$U_\varphi \cap D$ ריקה.
	אז לכל $\psi \in p_n$ מתקיים $T \cup \{ \varphi \} \models \psi(c_{i_0}, \ldots, c_{i_{m_k - 1}})$ ולכן גם,
	\[
		T \models \varphi \to \psi(c_{i_0}, \ldots, c_{i_{m_k - 1}})
	\]
	נניח שהקבועים המופיעים ב־$\varphi$ (מתוך $\tilde{L} \setminus L$) הם $c_{i_0}, \ldots, c_{i_{m_k - 1}}, d_0, \ldots, d_{r - 1}$,
	\[
		T \models \overline{\varphi}(d_0, \ldots, d_{r - 1}, c_{i_0}, \ldots, c_{i_{m_k - 1}}) \to \psi(c_{i_0}, \ldots, c_{i_{m_k - 1}})
	\]
	כך שמתקיים,
	\[
		T \models \forall x_0 \cdots \forall x_{n - 1}\ ((\exists y_0 \cdots \exists y_{n - 1}\ \overline{\varphi}(y_0, \ldots, y_{r - 1}, x_0, \ldots, x_{m_k - 1})) \to \psi(x_0, \ldots, x_{m_k - 1}))
	\]
	וכן הטיפוס $p_k$ מבודד על־ידי הנוסח $\exists y_0 \cdots \exists y_{r - 1}\ \overline{\varphi}$.
	בהתאם,
	\[
		E_n = \{ E_\psi \mid \psi \in \form_{\tilde{L}} \} \cup \{ D_{k, c_{i_0}, \ldots, c_{i_{m_k - 1}}} \mid \{ i_0, \ldots, i_{m_k - 1} \} \in {[w]}^{\le m_k} \}
	\]
	ולכן $\exists \tilde{T} \in \cap E_{\psi} \cap \bigcap D_{k, i_0, \ldots, i_{m_k - 1}}$.
\end{proof}
עבור $n$ טבעי נגדיר טופולוגיה על $S_n(T)$ באותו אופן, אבל בשפה $L \cup \{ c_0, \ldots, c_{n - 1} \}$ קבועים חדשים.
כלומר,
\[
	U_\varphi = \{ p \in S_n(T) \mid \varphi_{x_0, \ldots, x_{n - 1}}^{c_0, \ldots, c_{n - 1}} \in p \}
\]
טיפוס כך ש־$\{ p \}$ הוא טיפוס מבודד, ובהתאם המרחב שהגדרנו הוא דיסקרטי אם כל הטיפוסים מבודדים.

\section{שיעור 7 --- 30.11.2025}
\subsection{מרחב הטיפוסים}
ניזכר ש־$S_n(T)$ הוא מרחב טופולוגי האוסדורף קומפקטי.
\begin{corollary}
	אם כל טיפוס ב־$S_n(T)$ מבודד אז $|S_n(T)|$ סופי.
\end{corollary}
נניח ש־$T$ שלמה. אם $p$ טיפוס מבודד על־ידי $\psi$ אז $T \models \exists \bar{x}\ \psi(\bar{x})$ ולכן בכל מודל של $T$ נקבל ש־$p$ מתממש.
\begin{definition}[רוויה]
	מבנה $\Mm \models T$ נקרא $\omega$־רווי ($\omega$-saturated) אם לכל $A \subseteq M$ סופית ולכל $p \in S_1(T(A))$ מתממש ב־$\Mm$. \\
	מבנה $\Mm$ בן־מניה נקרא רווי אם הוא $\omega$־רווי.
\end{definition}
\begin{example}
	נבחר את $\Mm = \langle \NN, +, \cdot, 0, 1 \rangle$ אז $S_1(\emptyset)$ היא מעוצמת הרצף.
	אם $P$ קבוצת הראשוניים אז לכל $X \subseteq P$ היה טיפוס חלקי $P_X(x) = \{ \exists y\ (y \cdot \underline{p} = x) \mid p \in X \} \cup \{ \lnot \exists y\ (y \cdot \underline{p} = x) \mid p \notin X \}$.
\end{example}
\begin{example}
	הפעם נגדיר את $\Mm = \langle \QQ, < \rangle$, מודל זה יהיה רווי.
	נניח ש־$A \subseteq \QQ$ סופית, אם $\psi \in \form_{L(A)}$ עם משתנה חופשי $x$,
	אז מחילוץ כמתים $\psi$ שקולה לנוסחה חסרת כמתים,
	\[
		\psi = \bigvee_{i < m} \bigwedge_{j < n_i} \rho_{ij}
	\]
	עבור $\rho_{ij}$ אטומיות או שלילתן.
	אם $p$ טיפוס אז בהכרח הוא גימום מהצורה,
	\[
		\bigwedge {(a_i < x)}^{\varepsilon_i^0}
		\land \bigwedge {(a_i = x)}^{\varepsilon_i^1}
		\land \bigwedge {(x < a_i)}^{\varepsilon_i^2}
	\]
	ולכן $p$ מבודד על־ידי נוסח מהצורה $x = a_i$ או $a_i < x < a_j$ או $a_i < x$ עבור $a_i$ מקסימלי או באופן דומה עבור מינימלי.
	כלומר מצאנו שיש כמות סופית של טיפוסים, כלומר $|S_1(A)| < \omega$ ולמעשה יש מספר סופי של נוסחות שאינו תלוי ב־$A$ שהצבת $A$ בהן מניבה את הנוסחה המבודדת.
\end{example}
\begin{theorem}[איזומורפיזם מודלים רוויים בני־מניה]
	נניח ש־$\Mm, \Nn \models T$ מודלים בני־מניה רווים ו־$T$ שלמה.
	אז $\Mm \cong \Nn$ ואף אם $A \subseteq M$ ו־$B \subseteq N$ סופיות ו־$f : A \to B$ שיכון אלמנטרי חלקי
	(כלומר לכל נוסחה $\psi(x_0, \ldots, x_{n - 1})$ ו־$a_0, \ldots, a_{n - 1} \in A$ מתקיים $\psi^\Mm(a_0, \ldots, a_{n - 1}) = \psi^\Nn(f(a_0), \ldots, f(a_{n - 1}))$) יש הרחבה לאיזומורפיזם של המבנים.
\end{theorem}
משפט זה מזכיר מאוד את משפט קנטור והוכחתו מאוד דומה.
\begin{proof}
	החלק הנוסף גורר את הטענה כי $f = \emptyset$ היא שיכון אלמנטרי חלקי.
	נניח ש־$f : A \to B$ שיכון אלמנטרי חלקי ו־$a \in M$, אז יש $f \subseteq f'$ כך ש־$\dom f' = B \cup \{ a \}$ ולכל $b \in N$ יש $f \subseteq f'$ כך ש־$\rng f' = B \cup \{ b \}$.
	בלי הגבלת הכלליות נניח גם ש־$|A| = |B|$.

	נבחן את $S_1(A) \ni p = tp(a / A) = \{ \varphi(x) \mid \Mm \models \varphi(a), \varphi \in \form_{L_A} \}$
	ואת $q = f_x(p) = \{ \psi \mid \psi = \varphi \in p, \psi = \varphi_{d_{f(x)}}^{d_x}, \forall x \in A \}$.
	אז $q$ עקבי שכן אם $\psi \in q$ אז $\Nn \models \exists x\ \psi$ כיוון ש־$f$ אלמנטרית ($q$ סגור לגימום).
	לכל נוסחה ב־$L_B$, $\psi$, או ש־$\psi \in q$ או ש־$\lnot \psi \in q$ כל עוד המשתנה החופשי הוא $x$.
	אז קיים $b \in B$ שמממש את $q$ כי $\Nn$ רווי, כעת $f' = f \cup \{ \langle a, b \rangle \}$ אלמנטרית חלקית.
\end{proof}
\begin{corollary}
	אם יש לתורה שלמה $T$ מודל בן־מניה רווי, אז הוא יחיד עד כדי איזומורפיזם.
\end{corollary}
\begin{theorem}[Ryll-Nardzewzki]
	תהי $T$ תורה שלמה בשפה בת־מניה ללא מודלים סופיים, אז התנאים הבאים שקולים:
	\begin{enumerate}
		\item $T$ היא $\aleph_0$־קטגורית
		\item כל טיפוס $p \in S_n(T)$ מבודד
		\item $S_n(T)$ היא סופית
		\item לכל $n < \omega$ יש מספר סופי של נוסחות במשתנים חופשיים $x_0, \ldots, x_{n - 1}$ עד כדי שקילות ב־$T$
		\item כל מודל בן־מניה של $T$ הוא רווי
	\end{enumerate}
\end{theorem}
\begin{proof}
	ראינו כי $2 \iff 3$.

	$1 \implies 2$,
	נניח בשלילה שיש טיפוס $p$ לא מבודד, אז $p$ טיפוס ולכן עקבי קיים מודל של $T$ שמממש אותו, אבל ממשפט השמטת טיפוסים יש מודל של $T$ שמשמיט אותו, שניהם בני־מניה.
	הם כמובן לא איזומורפיים בסתירה.

	$2 + 3 \implies 4$,
	נניח ש־$p_0, \ldots, p_{m - 1}$ הם הטיפוסים ב־$S_n(T)$ ונניח ש־$\psi_0, \ldots, \psi_{m - 1}$ נוסחות מבודדות.
	נניח ש־$\psi$ נוסחה כלשהי ב־$n$ משתנים.
	אם $\psi$ לא עקבית אז סיימנו ולכן נניח אחרת, כלומר $\{ \psi \}$ מתרחב לטיפוס שלם ונבחן את $I = \{ i \mid \psi \in \pi_i \}$.
	אז נקבל ש־$\forall \bar{x}\ (\bigvee_{i \in I} \psi_i \to \psi)$. \\
	בכיוון ההפוך אם $\Nn \models T$ ו־$a_0, \ldots, a_{n - 1}$ מספקים את $\psi$ אז $p_j = tp^\Nn(a_0, \ldots, a_{n - 1})$ אבל $\psi_j \in p_j$ ולכן $\Nn \models \psi_j(a_0, \ldots, a_{n - 1})$.
	נקבל ש־$\Nn \models (\psi \to \bigvee_{i \in I} \psi_i)(\bar{a})$ אבל $\bar{a}$ שרירותית ולכן,
	\[
		\Nn \models \forall \bar{x}\ (\psi \to \bigvee_{i \in I} \psi_i)
	\]
	ולכן $T \models \forall \bar{x} (\psi \to \bigvee_{i \in I} \psi_i)$.

	$4 \implies 3$,
	נניח ש־$p_0, \ldots, p_{m - 1}$ נציגי מחלקות של נוסחות ב־$x_0, \ldots, x_{n - 1}$.
	טיפוס $p$ הוא איחוד של מחלקות שקילות ולכן יש לכל היותר $2^m$ טיפוסים.

	$2 \implies 5$,
	נראה שלכל טיפוס ב־$S_1(A)$ כאשר $A \subseteq \Mm \models T$ סופית, מבודד.
	טיפוס ב־$S_1(A)$ הוא מהצורה,
	\[
		p = \{ \varphi(x_0, a_1, \ldots, a_n) \mid \varphi(x_0, \ldots, x_n) \}
	\]
	נטען כי $q = \{ \varphi(x_0, \ldots, x_n) \mid \varphi(x_0, a_1, \ldots, a_n) \in p \}$ טיפוס.
	$q$ סגור לגימום כי $p$ סגור לגימום.
	לכל נוסחה $\varphi$ מתקיים $\varphi \in q$ או $\lnot \varphi \in q$, שכן $p$ מקיים טענה זו.
	לכל $\varphi \in q$ מתקיים גם $T \models \exists x_0 \exists \bar{x}\ \varphi(x_0, \bar{x})$ שכן זהו המצב ב־$\Mm$. \\
	$q$ מבודד, ונניח ש־$\psi$ מבודדת את $q$.
	\[
		T \models \forall \bar{x} \forall x_0\ (\psi(x_0, \bar{x}) \to \varphi(x_0, \bar{x}))
	\]
	לכל $\varphi \in q$ ו־$\psi$ עקבית.
	\[
		T \models \exists \bar{x} \exists x_0 \psi(x_0, \bar{x})
	\]
	ויתר־על־כן,
	\[
		T \models \forall \bar{x} \exists x_0\ \psi(x_0, \bar{x})
	\]
	אם נוסחה $\varphi(x_0, a_1, \ldots, a_n) \in p$ אז $\Mm_A \models \exists x_0\ \varphi(x_0, a_1, \ldots, a_n)$ ולכן,
	\[
		\exists x_0\ \varphi(x_0, \ldots, x_n) \in tp^\Mm(a_1, \ldots, a_n)
	\]
	זאת שכן $\varphi(x_0, \ldots, x_n) \in q$ גורר $\exists x_0\ \varphi(x_0, \ldots, x_n) \in q$.
	במקרה שלנו נקבל נוסחה ב־$tp(a_1, \ldots, a_n)$.
	כל $\varphi$ בטיפוס $q$ כנביעה מ־$\psi$ ו־$\psi$ עקבית, אז נובע ש־$\psi \in q$ ולכן $\exists x_0\ \psi \in tp(\bar{a})$. \\
	נרצה להראות ש־$\psi(x_0, a_1, \ldots, a_n)$ עקבית.
	זה נכון שכן $\exists x_0\ \psi(x_0, \ldots, x_n)$ שייכת לטיפוס של $\bar{a}$ ולכן $p$ מממומש. \\
	כהערה נאמר שהראינו שאם כל טיפוס ב־$S_n(T)$ מבודד ו־$|A| = n - 1$ אז כל טיפוס ב־$S_n(A)$ מבודד.

	$5 \implies 1$,
	נובע מהמשפט שכל שני מודלים בני־מניה רוויים איזומורפיים.
\end{proof}
\begin{definition}[גדירות]
	יהי $\Mm$ מודל ו־$A \subseteq M$.
	קבוצה $D \subseteq M^n$ נקראת $A$־גדירה אם קיימת $\varphi \in \form_{L(A)}$ כך שמתקיים,
	\[
		D = \{ (b_0, \ldots, b_{n - 1}) \in M^n \mid \Mm_A \models \varphi(b_0, \ldots, b_{n - 1}) \}
	\]
	$D$ היא $0$־גדירה אם היא $\emptyset$־גדירה.
\end{definition}
\begin{definition}[אינווריאנטיות]
	קבוצה $D$ היא $G$־אינווריאנטית אם לכל $g \in G$ ולכל $(b_0, \ldots, b_{n - 1}) \in D$ מתקיים $(g b_0, \ldots, g b_{n - 1}) \in D$.
\end{definition}
\begin{proposition}
	תהי $T$ תורה שלמה ללא מודלים סופיים מעל שפה בת־מניה.
	אז התנאים הבאים שקולים:
	\begin{enumerate}
		\item $T$ היא $\aleph_0$־קטגורית
		\item לכל $n < \omega$ יש מודל בן־מניה $\Mm \models T$ בו כל $D \subseteq M^n$ שהיא אינווריאנטית תחת $\aut(\Mm)$ היא $A$־גדירה ל־$A \subseteq M$ סופית
		\item לכל $n < \omega$ יש מודל בן־מניה $\Mm \models T$ בו כל קבוצה $D \subseteq M^n$ כך שהיא $\aut(\Mm)$־אינווריאנטית היא גם $0$־גדירה
	\end{enumerate}
\end{proposition}
\begin{proof}
	$1 \implies 3$,
	נניח ש־$\bar{a} \in D$ ויהי $p \in tp(\bar{a})$.
	אם $\bar{b} \in M^n$ ו־$p = tp(\bar{b})$ אז העתקה ששולחת את $a_i$ ל־$b_i$ היא שיכון אלמנטרי חלקי, אז היא מתרחבת לאוטומורפיזם.
	אז יש $g \in \aut(\Mm)$ כך ש־$\bar{b} = g(\bar{a})$ ולכן $\bar{b} \in D$.
	לכן $D = \bigcup_{p \in \{ p_0, \ldots, p_{n - 1} \}} \{ \bar{a} \mid tp(\bar{a}) = p_i \}$ אבל כל אחד מהטיפוסים הללו מבודדים על־ידי $\psi_i$ ולכן,
	\[
		D = \left\{ \bar{a} \mid \Mm \models \bigvee_i \mid \psi_i(\bar{a}) \right\}
	\]

	$3 \implies 2$ טריוויאלי ולכן נעבור ל־$3 \implies 1$.
	נבחין כי כל גדירות היא אוטומטית על־ידי מספר סופי של פרמטרים.
	נבחר $\Mm$ שמקיים את ההנחה. לכל טיפוס $p \in S_n(T)$ נגדיר,
	\[
		D_p = \{ \bar{a} \in M^n \mid \Mm \models p(\bar{a}) \}
	\]
	נטען כי $P = \{ p \in S_n(T) \mid D_p \ne \emptyset \}$ סופית.
	אם $X \subseteq P$ אז $\bigcup_{p \in X} D_p$ היא $\aut(\Mm)$־אינווריאנטית.
	אם $|P| \ge \aleph_0$ אז מתקבלות באופן הזה לפחות $2^{\aleph_0}$ קבוצות אינווריאנטיות שונות.
	זה בלתי־אפשרי שכן יש מספר בן־מניה של הגדרות אפשריות. \\
	בהתאם קיבלנו מודל $\Mm \models T$ שבו יש מספר סופי של טיפוסים ממומשים $\{ p_0, \ldots, p_{n - 1} \}$ כל אחד מהם מבודד על־ידי נוסחה $\psi_i(\bar{x}, \bar{a})$.
	הסיבה לכך היא ש־$D_p$ היא גדירה ולכן ישנה נוסחה $\psi \in \form_{L(A)}$ כך ש־$D_p = \{ \bar{b} \mid \Mm \models \psi(\bar{b}) \}$,
	\[
		\Mm \models \forall \bar{x}\ (\psi_i(\bar{x}, \bar{a}) \to \varphi(\bar{x}))
	\]
	לכל $\varphi \in p$.
	\[
		\rho_{\varphi}
		= \exists \bar{y}\ (( \forall \bar{x}\ \bigvee_{i < n} \psi_i(\bar{x}, \bar{y}))
		\land \bigwedge_{i < n} (\forall \bar{x}\ \psi_i(\bar{x}, \bar{y}) \to \varphi(\bar{x})))
	\]
	בתוספת הטענה שהחלוקה זרה, לכל $i < n$ $T \models \rho_{\varphi}$. \\
	נניח ש־$\Nn \models T$ וכן ש־$q$ טיפוס אחר ש־$\Nn$ מממש.
	אז קיימת נוסחה $\varphi_q \in q$ כך ש־$\lnot \varphi_q \in p_i$ לכל $i < n$.
	נניח ש־$\bar{b}$ מממש את $q$.
	נסתכל על הפסוק $\rho$ ב־$\Nn$, יש $\bar{a}^N$ עבור $\bar{y}$.
	נציב ב־$\bar{x}$ את $\bar{b}$, יש $i$ עבורו $\psi_i(\bar{a}^N, \bar{b})$ מתקיים ולכן $\lnot \varphi_i(\bar{b})$ בסתירה.
	נסיק שאין טיפוסים נוספים ולכן $S_n(T)$ סופית.
\end{proof}
\begin{notation}
	נאמר כי $\Mm$ הוא $\aleph_0$־קטגורי אם $\Th(\Mm)$ היא $\aleph_0$־קטגורית.
\end{notation}
\begin{corollary}
	אם $\Mm$ הוא $\aleph_0$־קטגורי ו־$A \subseteq M$ סופית אז גם $\Mm_A$ הוא $\aleph_0$־קטגורי.
\end{corollary}
\begin{proof}
	אם $\Mm$ אכן $\aleph_0$־קטגורי ו־$n < \omega$, $|A| = m$ אז יש מספר סופי של נוסחות עד־כדי שקילות בנות $n + m$ משתנים.
	נובע שיש מספר סופי של נוסחות ב־$L(A)$ בנות $n$ משתנים עד־כדי שקילות.

	בכיוון ההפוך אם $\Mm_A$ הוא $\aleph_0$־קטגורי.
	אז מהמשפט הקודם כל קבוצה $\aut(\Mm)$־אינווריאנטית היא גדירה עם פרמטרים ב־$A$ במודל $\Mm$.
	בפרט $\aut(\Mm_A)$־אינווריאנטית ולכן $\Th(\Mm)$ היא $\aleph_0$־קטגורית.
\end{proof}
\begin{theorem}[שני המודלים של ווט]\label{theorem-two-models-of-vaught}
	נניח ש־$T$ תורה שלמה בשפה בת־מניה.
	אז לא יתכן של־$T$ יש בדיוק שני מודלים בני־מניה עד־כדי איזומורפיזם.
\end{theorem}
\begin{proof}
	אם יש $n$ עבורו $S_n(T)$ לא בן־מניה אז יש מספר לא בן־מניה של מודלים שונים.
	לכן $|S_n(T)| \le \aleph_0$, במקרה זה יש מודל רווי.
	התורה לא $\aleph_0$־קטגורית אז יש טיפוס $p$ לא מבודד.
	לכן יש מודל $\Mm_0$ שמשמיט את $p$ ומודל $\Mm_1$ שמממש את $p$ על־ידי $\bar{a}$.
	אם בהכרח $\Mm_1$ רווי אז $\Th({(\Mm_1)}_{\bar{a}})$ היא $\aleph_0$־קטגורית (כי כל מודל רווי).
	אבל אז $\Th(\Mm_1)$ היא $\aleph_0$־קטגורית בסתירה להנחה. לכן המודל הרווי שונה משניהם.
\end{proof}

\section{שיעור 8 --- 7.12.2025}
\subsection{שני המודלים}
נמשיך לדבר על משפט שני המודלים של ווט.
נניח שהשפה שלנו היא בת־מניה בחלק זה.
\begin{definition}[תורה קטנה]
	תורה $T$ תיקרא קטנה אם לכל $n < \omega$ מתקיים $|S_n(T)| \le \omega$.
\end{definition}
\begin{remark}
	אם $T$ איננה קטנה אז יש ל־$T$ מספר לא בן־מניה של מודלים בני־מניה, כאשר $T$ שלמה עקבית ובעלת מודל אינסופי.
\end{remark}
\begin{proof}
	נניח ש־$\aleph_1 \le |S_n(T)|$. לכל $p \in S_n(T)$ נתאים מודל $\Mm_p$ בן־מניה שמממש את $p$.
	לכל $\Mm_p$ נסמן $A_p = \{ q \in S_n(T) \mid \Mm \text{ realizes } q \}$, ולכן $A_p = \{ tp^{\Mm_p}(\bar{b}) \mid \bar{b} \in \Mm_p^n \}$.
	אם $\Mm_p \cong \Mm_q$ אז גם $A_p = A_q$ וכן,
	\[
		\bigcup \{ A_p \mid {[ \Mm_p ]}_{\cong} \} = S_n(T)
	\]
	ולכן מספר מחלקות השקילות הוא לא בן־מניה.
	בהינתן $p$ מדוע יש מודל של $T$ בן־מניה המממש את $p$: נעשיר את השפה $L$ בסימן קבוע ונסתכל על $T \cup p(c)$ העקבית.
\end{proof}
\begin{proposition}
	נניח ש־$T$ קטנה שלמה ועקבית ללא מודלים סופיים, אז יש מודל רווי ובן־מניה ל־$T$.
\end{proposition}
\begin{proof}
	יהי $\Mm_0 \models T$ בן־מניה.
	לכל טיפוס $p \in \bigcup_{n < \omega} S_n(T)$ נוסיף סדרת קבועים $\bar{a}_p$ ונבחן את התורה $\diag(\Mm_0) \cup \{ p(\bar{a}_p) \}$.
	\[
		\exists \bar{x}_0\ \varphi(\bar{x}_0) \land \exists \bar{x}_1\ \varphi_1(\bar{x}_1), \ldots \in T
	\]
	מתקיים ב־$\Mm$ לכן התורה עקבית.
	יהי $\Mm_1$ שמקיים תורה זו, אז $\Mm_0 \prec \Mm_1$ ונבחן את $\bigcup_{|A| < \aleph_0, A \subseteq \Mm_1} S_n(A)$, אוסף בן־מניה.
	אז קיים $\Mm_2$ שמממש את כולם ומרחיב אלמנטרית את $\Mm_1$, נחזור על כך $\omega$ פעמים ונסמן $\Mm_{\omega} = \bigcup \Mm_n$.
	אז $\Mm_{\omega}$ מודל רווי.
	נשים לב ש־$S_n(A)$ בן־מניה עבור $A \subseteq M$ סופית ולכן $S_n(A) \hookrightarrow S_{n + |A|}(T)$ על־ידי התאמת טיפוסים מהצורה $p = \{ \varphi(a_0, \ldots, a_{k - 1}, x_0, \ldots, x_{n - 1}) \mid \varphi \in \form \}$,
	על־ידי $q = \{ \varphi(y_0, \ldots, y_{k - 1}, x_0, \ldots, x_{n - 1}) \mid \varphi \}$.
	$q$ הוא טיפוס שכן הוא סגור לגימום ושלם, ו־$q$ עקבי שכן אם $\varphi \in q$ אז,
	\[
		\Mm \models \exists \bar{x}\ \varphi(a_0, \ldots, a_{k - 1}, \bar{x})
		\implies \Mm \models \exists \bar{y} \exists \bar{x}\ \varphi(\bar{y}, \bar{x})
		\in T
	\]
\end{proof}
\begin{remark}
	למעשה $T$ קטנה אם ורק אם יש מודל רווי בן־מניה.
\end{remark}
נחזור למשפט\ \ref{theorem-two-models-of-vaught}.
\begin{proof}
	אם $T$ לא קטנה אז יש לפחות $\aleph_1$ טיפוסי איזומורפיזם, בפרט יש לפחות $3$.
	אחרת קיים מודל רווי $\Mm_0$.
	נניח ש־$p_0 \in S_n(T)$, ממשפט השמטת הטיפוסים $\Mm_1$ משמיט את $p$.
	קיים $\bar{a} \in \Mm_0^n$ כך ש־$\Mm_0 \models p(\bar{a})$ ונסמן $\Th({(\Mm_0)}_{\bar{a}}) = T_{\bar{a}}$ אז $T_{\bar{a}}$ לא $\aleph_0$־קטגורית ולכן קיים $q \in S_m(T_{\bar{a}})$ לא מבודד.
	$\Mm_0$ מממש את $q$ עם $\bar{b}$ ולכן $T_{\bar{a} \bar{b}}$ לא $\aleph_0$־קטגורית.
	נניח ש־$r$ טיפוס לא מבודד נוסף, אז קיים $\Mm_3$ שמשמיט את $r$ ומממש את $p, q$.
\end{proof}
\begin{remark}
	במהות החלוקה היא שאם $\Mm_0$ רווי אז $\Mm_0 \models p(a)$ ו־$\Mm_1$ משמיט את $p$.
	אז יש טיפוס כך ש־$\Mm \models q(b)$ וכן קיים $\Mm_2$ משמיט את $q$, ונניח ש־$q \in S(a)$ אז קיים $\Mm_3$ שמשמיט את $r$.
\end{remark}
\begin{example}
	נבחר את $T = \operatorname{DLO} \cup \{ c_n < c_{n + 1} \mid n < \omega \}$.
	ראינו ש־$T$ מחלצת כמתים ושלמה וכן ש־$p(x) \supseteq \{ c_n < x \mid n < \omega \}$.
	במקרה זה לדוגמה $M_1 = \QQ_{< 0}$ כאשר $c_n = -\frac{1}{n}$ ונגדיר $a = 0$ כלשהו.
	אז $q = \{ c_n < x \mid n < \omega \} \cup \{ x < a \}$.
	במקרה זה נקבל $\Mm_2$ מודל מעל $\QQ$ ובהתאם $\Mm_0 = \QQ \setminus \{ 0 \}$.
	מדוע $\Mm_0$ רווי? מחילוץ הכמתים.
	$\Mm_3$ מקיים של־$c_n$ יש גבול (יש מינימום לחסמים עליונים של $\{ c_n^\Mm \mid n < \omega \}$).
\end{example}
\begin{definition}[מודל אטומי וראשוני]
	מודל $\Mm$ הוא אטומי אם לכל $\bar{a} \in M^n$ כך ש־$tp^\Mm(\bar{a})$ מבודד.
	מודל $\Mm$ לתורה $T$ יקרא ראשוני אם לכל $\Nn \models T$ יש שיכון אלמנטרי $j : \Mm \to \Nn$.
\end{definition}
\begin{example}
	המודל הסטנדרטי של האריתמטיקה הוא ראשוני.
\end{example}
\begin{remark}
	נשים לב שבאופן טריוויאלי אם $T$ היא $\aleph_0$־קטגורית אז המודל היחיד שלה הוא אטומי וראשוני.
\end{remark}
\begin{proof}
	אטומי ממשפטים שכבר מצאנו על שקילות ל־$\aleph_0$־קטגוריות. \\
	$\Mm_0 \prec \Nn \models T$ כש־$\Mm_0$ בן־מניה אז $\Mm_0 \cong \Mm$ ומשרשור נקבל את $j$.
\end{proof}
\begin{proposition}
	אם $\Mm$ אטומי ובן־מניה לתורה שלמה אז $\Mm$ ראשוני.
\end{proposition}
\begin{proof}
	נמנה את איברי $\Mm$ על־ידי $M = \{ a_n \mid n < \omega \}$, ויהי $\Nn \models T$.
	נבנה באינדוקציה $f_n : \{ a_0, \ldots, a_{n - 1} \} \to N$ שיכון אלמנטרי חלקי באופן הבא:
	עבור $n = 0$ אז $\Nn \equiv \Mm$ וסיימנו.
	נניח כי בנינו את $f_n$ ונבחן את הטיפוס $tp(a_0, \ldots, a_{n - 1}) = p_n$, אז $p_n$ מבודד על־ידי $\psi_n$ כלשהו.
	הנוסחה $\exists x_{n - 1}\ \psi_n(x_0, \ldots, x_{n - 2}, x_{n - 1})$ שייכת ל־$p_{n - 1}$ שכן $\Mm \models \exists x_{n - 1}\ \psi_n(a_0, \ldots, a_{n - 2}, x_{n - 1})$.
	לכן $T \models \forall x_0 \cdots \forall x_{n - 0}\ \psi_{n - 1}(x_0, \ldots, x_{n - 2}) \to \exists x_{n - 1}\ \psi_n(x_0, \ldots, x_{n - 1})$.
	אז נובע ש־$\Nn \models \psi_{n - 1}(f_{n - 1}(a_0), \ldots, f_{n - 1}(a_{n - 2}), x_{n - 1})$.
	יהי $b$ המעיד על כך ונגדיר $f_n = f_{n - 1} \cup \{ \langle a_{n - 1}, b \rangle \}$.
	אכן $\psi_n \in tp(f_{n - 1}(a_0), \ldots, f_{n - 1}(a_{n - 2}), b) = \bar{p}_n$, אבל לכל נוסחה $\varphi \in p_n$ מתקיים $T \models \forall \bar{x}\ (\psi_n(\bar{x}) \to \varphi(\bar{x}))$,
	נובע ש־$p_n \subseteq \bar{p}_n$ ולכן שווה לו ונסיק ש־$j = \bigcup f_n$ ולכן $j : \Mm \to \Nn$ שיכון אלמנטרי.
\end{proof}
\begin{example}
	נסתכל על $T = \operatorname{ACF}_0$ אז $\bar{\QQ}$ מודל אטומי.
	אז לפחות ל־$a \in \bar{\QQ}$ נקבל ש־$tp(a)$ מבודד על־ידי נוסחה מהצורה $p(x) = 0$ כאשר $p$ הפולינום המינימלי.
\end{example}
\begin{corollary}
	נניח ש־$\Mm, \Nn$ מודלים בני־מניה לתורה $T$ שלמה אז $\Mm \cong \Nn$.
\end{corollary}
\begin{proof}
	כמו קודם אבל הפעם עם back and forth.
\end{proof}
\begin{corollary}
	אם $\Mm$ מודל ראשוני של $T$ אז $\Mm$ אטומי ובן־מניה.
\end{corollary}
\begin{proof}
	אם יש $\bar{a} \in M^n$ כך ש־$p_n = tp(\bar{a})$ לא מבודד אז יש מודל של $T$ שמשמיט אותו ולא יתכן שיש שיכון אלמנטרי מ־$\Mm$ לאותו מודל.
\end{proof}
\begin{corollary}
	מודל בן־מניה הוא אטומי אם ורק אם הוא ראשוני.
\end{corollary}
\begin{example}
	נניח ש־$L = \{ B_n \mid n < \omega \}$ עבור $B_n$ יחסים חד־מקומיים יחד עם התורה,
	\[
		\left\{ \bigwedge_{n \in Z} B_n(x) \land \bigwedge_{n \in Y} \lnot B_n(x) \mid Z, Y \subseteq \omega \text{ disjoint and finite} \right\}
	\]
	הוכחנו ש־$T$ היא שלמה.
	לכל $a$ במודל של $T$ נקבל $tp(a) = \{ B_n(a) \mid n \in X \} \cup \{ \lnot B_m \mid m \notin X \}$ כאשר $X \subseteq \omega$ מחילוץ כמתים שהוכחנו, והוא לא מבודד.
	נסיק בהתאם שאין מודל אטומי.
\end{example}
\begin{theorem}[שקילות לקיום מודל ראשוני]
	בשפה בת־מניה, ל־$T$ שלמה יש מודל ראשוני אם ורק אם לכל $n$ אוסף הטיפוסים המבודדים צפוף ב־$S_n(T)$.
\end{theorem}
\begin{proof}
	נניח ש־$\Mm$ ראשוני ונניח ש־$\varphi$ נוסחה כך ש־$U_\varphi = \{ q \in S_n(T) \mid \varphi \in q \} \ne \emptyset$,
	אז יש מודל $\Nn \models T$ כך ש־$\Nn \models q(c)$ לאיזשהו $q \in U_\varphi$.
	טענה זו נכונה אם ורק אם $\exists \bar{x}\ \varphi(x) \in T \iff \Nn \models \exists \bar{x}\ \varphi(\bar{x})$.
	נובע ש־$\Mm \models \varphi(\bar{x})$ ונסמן את העד ב־$\bar{a}$.
	$tp(\bar{a})$ מבודד שכן $\Mm$ אטומי ולכן $\varphi \in tp(\bar{a})$, כלומר $tp(\bar{a}) \in U_\varphi$.

	בכיוון ההפוך נניח שלכל $n$ הטיפוסים המבודדים צפופים ב־$S_n(T)$.
	$p \in S_n(T)$ מבודד אם ורק אם יש נוסחה $\psi(x_0, \ldots, x_{n - 1})$ כך ש־$\psi \in p$ ולכל $\theta \in p$ מתקיים $T \models \forall \bar{x}\ (\psi \to \theta)$.
	כלומר $\psi$ מבודדת טיפוס אם לכל נוסחה $\theta(x_0, \ldots, x_{n - 1})$ מתקיים $\forall \bar{x}\ (\psi \to \theta)$ או $\forall \bar{x}\ (\psi \to (\lnot \theta))$.
	נאמר כי $\psi$ \textbf{שלמה} אם $\{ \theta \mid \forall \bar{x}\ (\psi \to \theta) \} \in S_n(T)$,
	ונסתכל על הטיפוס החלקי $p_n = \{ \lnot \psi \mid \psi(x_0, \ldots, x_{n - 1}) \text{ is complete} \}$.
	בלי הגבלת הכלליות $p_n$ עקבית (שכן מטרתנו להשמיט כל $p_n$) ונטען ש־$p_n$ לא מבודדת.
	אחרת נניח ש־$\varphi$ נוסחה עקבית המבודדת את $p_n$.
	$\varphi$ עקבית ולכן $U_\varphi \ne 0$ אז מההנחה יש טיפוס שלם מבודד ב־$U_\varphi$.
	נאמר ש־$\psi$ מבודדת אותו, אז לכל $\lnot \tilde{\psi} \in p_n$ מתקיים,
	\[
		T \models \forall \bar{x}\ (\varphi \to (\lnot \tilde{\psi}))
	\]
	בפרט ל־$\tilde{\psi} = \psi$ ולכן, $T \models \forall \bar{x}\ (\varphi \to (\lnot \psi))$.
	אך מצד שני $T \models \exists \bar{x}\ (\psi \land \varphi)$ וזו סתירה. \\
	לכן קיבלנו שכל $p_n$ לא מבודד אז יש מודל $\Mm$ של $T$ שמשמיט את $p_n$ לכל $n$ (משפט השמטת טיפוסים המורכב),
	כלומר לכל $\bar{a} \in M^n$ בהכרח $tp(\bar{a})$ מבודד, שכן יש $\psi$ שלמה ששייכת אליו ולכן מבודדת אותו.
	לכן $\Mm$ מודל בן־מניה ואטומי.
\end{proof}

\subsection{גבולות פרייסה}
תורות $\aleph_0$־קטגוריות עם חילוץ כמתים, הומוגניות וומהותית נכונות מתוך תת־מבנים סופיים.
טענה זו שקולה לתכונת הומוגניות של הרחבת איזומורפיזם, האוסף $\operatorname{Age}(\Mm)$ של תת־מודלים סופיים של $\Mm$ עד כדי איזומורפיזם יקיים תכונות על הדיאגרמה של $A, B \in \operatorname{Age}$.

\listoftheorems[title=הגדרות ומשפטים,ignoreall,show={theorem,definition},swapnumber,onlynamed={proposition,lemma}]

\end{document}
