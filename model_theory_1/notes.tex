\input{../article_base.tex}
\title{תורת המודלים 1 --- סיכום}
\setcounter{secnumdepth}{2}

\usepackage{fancyhdr}
\pagestyle{fancy}
\renewcommand{\headrulewidth}{0pt}

\begin{document}
\maketitle
\maketitleprint[purple]

\tableofcontents

\section{שיעור 1 --- 19.10.2025}
\subsection{רקע}
שעת קבלה בימי ראשון בשתיים־עשרה.
יהיו כשישה תרגילים ומטלה מסכמת.
תורת המודלים היא תחום בלוגיקה שעוסקת בניתוח של תורות ושל מודלים המתקבלים מהם.
נראה דוגמה למשפט שנובע מתחום זה.
\begin{example}
	משפט אקס־גרוטנדיק, הגורס כי אם פונקציה $f : \CC^n \to \CC^n$ כך שכל קורדינטה שלה היא פולינום ב־$n$ משתנים.
	נניח ש־$f$ חד־חד ערכית, אז $f$ היא גם על.
\end{example}
זהו משפט מוזר מאוד ומפתיע.
הדרך להוכיח אותו היא כזו,
נניח שיש לנו סדרה של פולינומים כך שהם חד־חד ערכיים ולא על, אז הכישלון שנקבל הוא על־ידי פסוק מסדר ראשון בשפת תורת החוגים $\varphi$ כך ש־$\CC \models \varphi$.
\[
	\exists a_0, \ldots, \exists a_{N}\ \forall \bar{x} \forall \bar{y}\ (a_0 x_0 \cdots = a_0 y_0 \cdots) \to \bar{x} = \bar{y}
	\land \exists \bar{z} \forall \bar{x} \lnot \bigwedge_{i < N} a_0 \bar{x} = z_i
\]
נבחין כי מתקיימת העובדה שנוכיח בהמשך,
\begin{remark}
	התורה של שדה סגור אלגברית ממציין נתון היא שלמה.
	בפרט כל שדה סגור אלגברית ממציין 0 מקיים את $\varphi$.
\end{remark}
מההערה ושלמות נסיק שכל שדה מספיק סגור אלגברית ממימד מספיק גדול מקיים את $\varphi$.
בפרט ל־$p$ ראשוני מספיק גדול $\overline{\FF}_p \models \varphi$.
נסתכל על מקדמים של הפולינום הבעייתי $a_0, \ldots, a_{N}$ ונקבל שהם שייכים ל־$\FF_p[a_0, \ldots, a_{N}] = \tilde{\FF}$ שדה סופי כלשהו.
נניח ש־$z_{0}, \ldots, z_{n - 1}$ מעידה על הפולינומים האלו, אז,
\[
	\tilde{\FF}[z_0, \ldots, z_{n - 1}] = \tilde{\tilde{\FF}} \subseteq \overline{\FF}_p 
\]
אז $f \restriction \tilde{\tilde{\FF}}$ חד־חד ערכית ולכן על ולכן $\bar{z}$ מתקבל כסתירה. \\
הרעיון המגניב הוא שהצלחנו למצוא טענה מאוד מורכבת על־ידי שימוש במודלים שונים מאותו עולם.

בקורס עצמו אנחנו נוכיח טענות בעולם של תורת המודלים, המשפטים המרכזיים הם:
\begin{itemize}
	\item משפט Vaught: תהי $T$ תורה בת־מניה שלמה, אז לא יתכן של־$T$ יש בדיוק שני מודלים לא איזומורפיים בני־מניה עד כדי איזומורפיזם
	\item משפט מורלי (Morley): יהי $\kappa$ מונה לא בן־מניה, $T$ תורה מעל שפה בת־מניה, אז $T$ היא $\aleph_1$־קטגורית אם ורק אם $T$ היא $\kappa$־קטגורית
\end{itemize}

\subsection{תזכורת למושגים והגדרות}
\begin{definition}[שפה]
	אוסף של סימני קבועים יחסים ופונקציות.
\end{definition}
\begin{definition}[שמות עצם]
	שמות עצם הם אובייקט סינטקטי שמורכב מסימני פונקציה קבועים ומשתנים.
\end{definition}
\begin{definition}[משתנה חופשי]
	משתנים חופשיים, נסמן $\varphi(x_0, \ldots, x_{n - 1})$ כאשר המשתנים $x_0, \ldots, x_{n - 1}$ חופשיים ב־$\varphi$. \\
	נוכל גם לדבר על המשתנים החופשיים של פסוק, ונסמן באופן דומה $t(x_0, \ldots, x_{n - 1})$.
\end{definition}
\begin{definition}[פסוק]
	פסוק הוא נוסחה ללא משתנים חופשיים.
\end{definition}
\begin{definition}[השמה]
	בהינתן נוסחה $\varphi(x_0, \ldots, x_{n - 1})$ ומבנה $\Aa$, $a_0, \ldots, a_{n - 1} \in A$,
	אז $\Aa \models \varphi(a_0, \ldots, a_{n - 1})$ בהתאם להגדרת האמת והחישוב הרקורסיבית שראינו בקורסים קודמים.
\end{definition}
\begin{definition}[הומומורפיזם של מבנים]
	בהינתן שני מבנים $\Aa, \Bb$ בשפה $L$, אז נסמן פונקציה $f : \Aa \to \Bb$ כפונקציה בין העולמות כך שהיא הומומורפיזם,
	כלומר היא מכבדת פונקציות קבועים ויחסים במובן הבא,
	\[
		\bar{a} \in R^\Aa
		\implies f(\bar{a}) \in R^\Bb
	\]
	שיכון הוא מקרה בו גם הכיוון השני מתקיים. \\
	איזומורפיזם הוא שיכון שהוא גם על. \\
	אוטומורפיזם הוא איזומורפיזם בין מבנה לעצמו.
\end{definition}
\begin{definition}[תת־מבנה]
	נסמן תת־מבנה של מבנים על־ידי $\Aa \subseteq \Bb$ אם $\id : \Aa \to \Bb$ שיכון.
	בפרט הקבוצה $A$ סגורה תחת הפונקציות של $B$ ומכילה את כל הקבועים.
\end{definition}
\begin{theorem}[משפט הקומפקטיות]
	נניח ש־$\Sigma$ קבוצת פסוקים בשפה $L$ כך שלכל $\Sigma_0 \subseteq \Sigma$ סופית היא ספיקה, אז $\Sigma$ ספיקה.
\end{theorem}
\begin{definition}[תורה]
	תורה היא קבוצת פסוקים סגורה למסקנות.
	תורה $T$ היא עקבית אם $\perp \notin T$, ממשפט השלמות הגדרה זו שקולה לקיום מודל ל־$T$. \\
	תורה $T$ היא שלמה אם לכל פסוק $\varphi$ מתקיים $\varphi \in T$ או $\lnot \varphi \in T$.
\end{definition}
לדוגמה אם $\Aa$ מבנה, אז $\operatorname{Th}(\Aa)$ שלמה.
\begin{definition}[שקילות]
	$\Aa \equiv \Bb$ אם $\operatorname{Th}(\Aa) = \operatorname{Th}(\Bb)$ ו־$\Aa \cong \Bb$ אם יש איזומורפיזם.
	מתקיים $\Aa \cong \Bb \implies \Aa \equiv \Bb$.
\end{definition}
\begin{definition}
	$f : \Aa \to \Bb$ נקראת שיכון אלמנטרי אם לכל נוסחה $\varphi(x_0, \ldots, x_{n - 1})$ ו־$a_0, \ldots, a_{n - 1} \in A$ אז,
	\[
		\Aa \models \varphi(a_0, \ldots, a_{n - 1})
		\iff \Bb \models \varphi(f(a_0), \ldots, f(a_{n - 1}))
	\]
	אם $f = \id$ אז נגיד ש־$\Aa \prec \Bb$ תת־מודל אלמנטרי.
\end{definition}
\begin{remark}
	נניח ש־$\langle \Aa_n \mid n < \omega \rangle$ שרשרת מבנים כך ש־$\Aa_n \subseteq \Aa_{n + 1}$,
	אז יש דרך אחת להגדיר את איחוד המבנים $\Aa_{\omega} = \bigcup_{n < \omega} \Aa_n$ כך ש־$\Aa_n \subseteq \Aa_{\omega}$.
	נעיר כי גם אם נוסיף את ההנחה ש־$\Aa_n \equiv \Aa_{n + 1}$ לא בהכרח נקבל שגם $\Aa_{\omega} \equiv \Aa_n$. \\
	לדוגמה עבור $L = \{ \le \}$ ו־$\Aa_n = \{ z \in \ZZ \mid -n \le z \}$ אז $\Aa_{\omega} = \ZZ$ אבל התורות אכן שונות.
\end{remark}
\begin{definition}[קטגוריות]
	נאמר שתורה $T$ היא $\kappa$־קטגורית אם לכל $\Aa, \Bb \models T$ אז מתקיים,
	\[
		|A| = |B|
		\implies \Aa \cong \Bb
	\]
\end{definition}
\begin{remark}
	סודר $\alpha$ נקרא מונה אם לא קיים $\beta < \alpha$ ופונקציה $f : \beta \to \alpha$ על. \\
	לכל מונה שונה מ־0 קיים מונה גדול יותר ומינימלי המסומן $\kappa^+$ ומכונה המונה העוקב של $\kappa$. \\
	נסמן ${(\aleph_0)}^+ = \aleph_0$.
\end{remark}
\begin{theorem}
	נניח ש־$\langle \Aa_n \mid n < \omega \rangle$ כך ש־$\Aa_n \prec \Aa_{n + 1}$ אז $\Aa_n \prec \Aa_{\omega}$.
\end{theorem}
\begin{proof}
	קודם כל נשים לב לעובדה השימושית הבאה, אם $\Mm \prec \Nn \prec \Kk$ אז $\Mm \prec \Kk$.
	נובע שלכל $n < m$ מתקיים $\Aa_n \prec \Aa_m$.
	נוכיח את הטענה באינדוקציה על מבנה הנוסחה, לכל $n < \omega$ ולכל $a_0, \ldots, a_{m - 1} \in \Aa_n$ מתקיים,
	\[
		\Aa_n \models \psi(a_0, \ldots, a_{m - 1})
		\Aa_{\omega} \models \psi(a_0, \ldots, a_{m - 1})
	\]
	עבור $\psi$ אטומית הטענה נובעת מכך שאלו הם תתי־מבנים.
	אם הטענה נכונה עבור $\psi$ היא נכונה גם עבור שלילה וכך גם לקשרים הבינאריים. \\
	נניח ש־$\varphi = \exists x_0\ \psi$ כאשר $\varphi = \varphi(x_1, \ldots, x_{m - 1})$.
	אם $\Aa_n \models \varphi(a_1, \ldots, a_{m - 1})$ אז $\Aa_n \models \exists x_0\ \psi(x_0, \ldots, a_{m - 1})$ ולכם יש $a_0 \in \Aa_n$ כך שמתקיים $\Aa_n \models \psi(a_0, \ldots, a_{m - 1})$.
	מהנחת האינדוקציה נקבל שגם $\Aa_{\omega} \models \psi(a_0, \ldots, a_{m - 1})$ ולכן $\Aa_{\omega} \models \exists x_0\ \psi(x_0, a_0, \ldots, a_{m - 1})$. \\
	בכיוון השני נניח ש־$\Aa_{\omega} \models \exists x_0\ \psi(x_0, a_1, \ldots, a_{m - 1})$.
	לכן קיים $b \in A_{\omega}$ כך שמתקיים $\Aa_{\psi} \models \psi(b, a_1, \ldots, a_{m - 1})$ ובהתאם קיים $k < \omega$ כלשהו כך ש־$n \le k$ ומתקיים $b \in A_k$.
	מהנחת האינדוקציה $\Aa_{\omega} \models \psi(b, a_1, \ldots, a_{m - 1})$ ולכן מאינדוקציה $\Aa_k \models \psi(b, a_1, \ldots, a_{m - 1})$ ולבסוף גם,
	\[
		\Aa_n \prec \Aa_k \exists x_0\ \psi(x_0, a_1, \ldots, a_{m - 1})
	\]
	ונסיק שמתקיים גם,
	\[
		\Aa_n \models \exists x_0\ \psi(x_0, a_1, \ldots, a_{m - 1})
	\]
	כפי שרצינו.
\end{proof}
\begin{theorem}[מבחן טרסקי־ווט]
	נניח ש־$\Mm \subseteq \Nn$ תת־מבנה כך שלכל נוסחה $\varphi(x, x_0, \ldots, x_{n - 1})$ ופרמטרים $a_0, \ldots, a_{n - 1} \in M$ כך שמתקיים,
	\[
		\Nn \models \exists x\ \varphi(x, a_0, \ldots, a_{n - 1})
		\implies \exists b \in M,\ \Nn \models \varphi(b, a_0, \ldots, a_{n - 1})
	\]
	אם ורק אם תקיים $\Mm \prec \Nn$.
\end{theorem}
\begin{proof}
	אם $\Mm \prec \Nn$ ומתקיים,
	\[
		\Nn \models \exists x\ \varphi(x, a_0, \ldots, a_{n - 1})
		\implies \Mm \models \exists x\ \varphi(x, a_0, \ldots, a_{n - 1})
	\]
	ולכ קיים $b \in M$ כך שמתקיים $\varphi^\Mm(b, a_0, \ldots, a_{n - 1})$ ולכן בהכרח גם $\Nn \models \varphi(b, a_0, \ldots, a_{n - 1})$.

	נעבור לכיוון השני, ושוב נוכיח באמצעות אינדוקציה על מבנה הנוסחה $\varphi(x_0, \ldots, x_{n - 1})$, שכן $a_0, \ldots, a_{n - 1} \in M$ אז,
	\[
		\Mm \models \varphi(a_0, \ldots, a_{n - 1})
		\iff \Nn \models \varphi(a_0, \ldots, a_{n - 1})
	\]
	עבור נוסחות אטומיות וקשרים בינאריים הטענה כמובן טריוויאלית מהגדרה ולכן נניח שמתקיים,
	\[
		\varphi = \exists x\ \psi(x, x_1, \ldots, x_{n - 1})
	\]
	וכן שמתקיים $\Mm \models \varphi(a_1, \ldots, a_{n - 1})$.
	לכן,
	\[
		\exists b \in M,\ \Mm \models \psi(b, a_1, \ldots, a_{n - 1})
	\]
	ולכן $\Nn \models \psi(b, a_1, \ldots, a_{n - 1})$ וכן $\NN \models \varphi(a_1, \ldots, a_{n - 1})$. \\
	בכיוון השני נניח שמתקיים,
	\[
		\Nn \models \exists x\ \psi(x, a_1, \ldots, a_{n - 1})
	\]
	אבל אז מטרסקי־ווט נקבל שקיים $b \in M$ כך ש־$\Nn \models \psi(b, a_1, \ldots, a_{n - 1})$ ומהנחת האינדוקציה על $\psi$ נקבל,
	\[
		\Mm \models \psi(b, a_1, \ldots, a_{n - 1})
		\implies \Mm \models \varphi(a_1, \ldots, a_{n - 1})
	\]
	וסיימנו את מהלך האינדוקציה.
\end{proof}
\begin{corollary}
	נניח ש־$L = \{ = \}$ ונניח ש־$\Aa \subseteq \Bb$ מבנים אינסופיים בשפה $L$.
	אז $\Aa \prec \Bb$.
\end{corollary}
\begin{proof}
	נשתמש במבחן טרסקי־ווט (מעכשיו נכתוב גם TV).
	נניח ש־$a_0, \ldots, a_{n - 1} \in A$ וכן שמתקיים,
	\[
		\Bb \models \exists x\ \varphi(a_0, \ldots, a_{n - 1})
	\]
	יהי $b \in B$ שמעיד על כך, אם $b \in \{ a_0, \ldots, a_{n - 1} \}$ אז בכל מקרה סיימנו. \\
	נבחר $c \in A \setminus \{ a_0, \ldots, a_{n - 1} \}$ ונגדיר אוטומורפיזם של $\Bb$ על־ידי,
	\[
		f(z)
		= \begin{cases}
			c & z = b \\
			b & z = c \\
			z & \text{otherwise}
		\end{cases}
	\]
	לכן $f$ אוטומורפיזם ובפרט שיכון אלמנטרי ומתקיים $f(a_i) = a_i$.
	נסיק שמתקיים,
	\[
		\Bb \models \varphi(f(b), f(a_0), \ldots, f(a_{n - 1}))
	\]
	ולכן תנאי המבחן חלים.
\end{proof}
\begin{corollary}[לוונהיים־סקולם היורד]
	נניח ש־$\Aa$ הוא $L$־מבנה ו־$\kappa$ מונה כך ש־$\aleph_0 + |L| \le \kappa \le |A|$, אז קיים $\Bb \prec \Aa$ כך ש־$|B| = \kappa$.
\end{corollary}
\begin{proof}
	לכל נוסחה $\varphi(x_0, \ldots, x_n)$ נגדיר פונקציה $F_{\varphi} : A^n \to A$ על־ידי,
	\[
		F_{\varphi}(a_0, \ldots, a_n)
		= \begin{cases}
			b & \Aa \models \varphi(b, a_1, \ldots, a_n) \\
			c & \Aa \models \lnot \exists x\ \varphi(x, a_1, \ldots, a_n)
		\end{cases}
	\]
	עבור ערך שרירותי $c$.
	עתה, תהי $X \subseteq A$ כך ש־$|X| = \kappa$, נגדיר,
	\[
		X_0 = X,
		X_{n + 1} = \{ F_{\varphi}(a_1, \ldots, a_m) \mid a_1, \ldots, a_m \in X_n,\ \varphi \in \operatorname{form} \} \cup X_n
	\]
	לכל $n$, אז $|X_{n + 1}| = \kappa$ תמיד.
	נסמן $B = \bigcup_{n < \omega} X_n$, אז,
	\[
		\kappa \le |B| \le \kappa + \aleph_0 = \kappa
	\]
	מתקיים $\Bb \subseteq \Aa$ כי אם $F$ סימן פונקציה ו־$\bar{c} \in B^n$ אז $F(\bar{c}) \in B$ כי הוא העדות היחידה לנוסחה $F(\bar{c}) = x$.
	בהתאם $\Bb \subseteq \Aa$ מקיים את TV ישירות מהבניה.
	אם $b_1, \ldots, b_n \in B$ ו־$\varphi$ נוסחה אז יש $b_1, \ldots, b_n \in X_m$, העדות ל־TV תהיה ב־$B \supseteq X_{m + 1}$.
\end{proof}

\listoftheorems[title=הגדרות ומשפטים,ignoreall,show={theorem,definition},swapnumber,onlynamed={proposition,lemma}]

\end{document}
