\input{../article_base.tex}
\title{פתרון מטלה 9 --- חישוביות וקוגניציה, 6119}

\usepackage{pgfplots, tikz}
\usetikzlibrary{math}
\graphicspath{{.}{../images/}}
\pgfplotsset{compat=1.18}


\begin{document}
\maketitle
\maketitleprint[beige]

\question{}
\subquestion{}
נמצא נקודת שיווי משקל נאש עבור שתי נהגות שצריכות לבחור כבישים כאשר מטריצת הערך היא,
\[
	M
	= \begin{pmatrix}
		A A & A B \\
		B A & B B
	\end{pmatrix}
	= \begin{pmatrix}
		(13, 14) & (11, 11) \\
		(12, 12) & (14, 13)
	\end{pmatrix}
\]
\begin{solution}
	נבחין כי אין לנהגת א' או לנהגת ב' בחירה שתמיד תטיב איתן ולכן אין פתרון יחיד למערכת.
	אם האסטרטגיה היא $M_{A A}$ אז לנהגת א' משתלם להחליף ל־$M_{B A}$, אם וזהו שיווי משקל.
	באופן דומה עבור האסטרטגיה $M_{B B}$ אז לשתי הנהגות משתלם להחליף ונקבל שגם $M_{B A}$ נקודת שיווי משקל.
\end{solution}

\subquestion{}
נניח שיש $4000$ נהגות ושני כבישים, ונניח גם שהעלות לנהוג בכל כביש מיוצגת על־ידי,
\[
	T_A(N_A)
	= \frac{N_A}{100} + 45,
	\qquad
	T_B(N_B)
	= \frac{N_B}{100} + C
\]
עבור $T_A, T_B$ זמן הנסיעה בכביש ו־$N_A, N_B$ מספר הנהגות בכל כביש, ו־$C$ פרמטר.

\subsubsection{i}
נבדוק עבור איזה $C$ קיימת נקודת שיווי משקל נאש מעורבת והומוגנית בה כל אחת מהנהגות בוחרת בכביש $A$ בסיכוי $p = \frac{3}{4}$, ומה יהיה זמן הנסיעה הממוצע.
\begin{solution}
	נחשב את תוחלת זמן ההמתנה,
	\[
		\EE(T)
		= \EE(T_A(N_A) + T_B(N_B))
	.\]
	ונתון כי $N_A + N_B = 4000$ וכן $N_A \sim \operatorname{Bin}(N, p) = \operatorname{Bin}(4000, \frac{3}{4})$.
	מפיתוח הנוסחה הראשונה ושימוש בנוסחת תוחלת ברנולי נקבל,
	\[
		\EE(T)
		= \frac{1}{100} \EE(N_A) + 45 + \frac{1}{100} \EE(N_B) + C
		= \frac{1}{100} (\EE(N_A) + \EE(N - N_A)) + 45 + C
		= \frac{1}{100} \cdot 40 + 45 + C
		= 85 + C
	.\]
	באופן דומה נקבל שנהגת מסוימת תקבל,
	\[
		\EE(T_A)
		= \frac{1}{100} N \cdot p + 45,
		= 40 \cdot \frac{3}{4} + 45
		= 75
		\qquad
		\EE(T_B)
		= 40 \cdot \frac{1}{4} + C
		= 10 + C
	.\]
	ולכן יש שיווי משקל אם ורק אם $10 + C = 75 \iff C = 65$.
\end{solution}

\subsubsection{ii}
נתאר את הקשר בין ממצאי הניסוי לבין ממצאי ניסוי חוק ההתאמה.
\begin{solution}
	לא ברור לי מה הוא חוק ההתאמה.
\end{solution}

\subquestion{}
נניח ש־$C = 65$ ונניח שתמיד 3000 נהגות בוחרות ב־A ו־1000 נהגות בוחרות B, נבדוק אם אסטרטגיה זו היא נקודת שיווי משקל של נאש.
\begin{solution}
	במקרה זה אין לנהגות שום בחירה ולכן זהו שיווי משקל באופן ריק.
	אם נניח שהן יכולות לשנות את שמן ובהתאם לשנות את הכביש שהן נוהגות בו תמיד, אז נקבל שכרגע התוחלות זהות ואם הן תעבורנה כביש אז הוא יהיה איטי יותר, ולכן זוהי אכן נקודת שיווי משקל.
\end{solution}

\subquestion{}
נניח ש־$C = 65$ והבחירה נתונה בידי הנהגות.
נניח שסכימת הבחירות נתונה על־ידי הגרף,
\[
	G = ((\text{begin}, \text{end}, A, B), \{ (\text{begin}, A, \frac{N_A}{100}), (\text{begin}, B, 65), (A, B, 0), (A, \text{end}, 45), (B, \text{end}, \frac{N_B}{100}) \})
.\]

\subsubsection{i}
נמצא את נקודת שיווי משקל של נאש במצב החדש.
\begin{solution}
	עלינו למצוא את התוחלת של בחירת כביש A (בהסתברות $p$) ואת בחירת קיצור הדרך בהסתברות $q$.
	נתאר את התוחלת של כל אחד מהמצבים,
	\[
		\EE(T_B)
		= \EE(\frac{N_B + N_{A, 1}}{100} + 65)
		= \frac{1}{100} (1 - p) N + 65
		= 40 (qp + (1 - p)) + 65
	.\]
	נעבור לכביש A עם הבחירה להשתמש בקיצור הדרך,
	\[
		\EE(T_{A, 1})
		= \frac{1}{100} \EE(N_A) + 0 + \frac{1}{100} \EE(N_B)
		= \frac{1}{100} N p \cdot q + 0 + \frac{1}{100} N ((1 - p) + pq)
		= 40 p q + 40 (1 - p) + 40 p q
	.\]
	ולבסוף הבחירה לא לעבור מכביש A,
	\[
		\EE(T_{A, 2})
		= \frac{1}{100} \EE(N_A) + 45
		= 40 p + 45
	.\]
	אנו רוצים למצוא מתי אף אפשרות לא עדיפה, כלומר מתי התוחלת שווה בשלושת המקרים.
	\[
		80 p q + 40 (1 - p) = 40 p + 45
		\iff 80 p q = 80 p + 5
		\iff q = 1 + \frac{1}{16 p}
	.\]
	ולכן,
	\[
		40 p + 45 = 40 (q p + (1 - p)) + 65
		\iff 40 p = 40 p q + 40 (1 - p) + 20
		\iff p = \frac{62.5}{80} = 0.78125
	.\]
	ובהתאם גם $q = \frac{1}{16 p} = 0.08$.
\end{solution}

\subsection{ii}
נבדוק האם כדאי לבנות את קיצור הדרך.
\begin{solution}
	התשובה היא שלא, בהתאם לתוצאות תת־הסעיף הקודם.
\end{solution}

\question{}


\end{document}
