\input{../article_base.tex}
\title{פתרון מטלה 7 --- חישוביות וקוגניציה, 6119}

\usepackage{pgfplots, tikz}
\usetikzlibrary{math}
\graphicspath{{.}{../images/}}
\pgfplotsset{compat=1.18}

% chktex-file 17
% chktex-file 9

\begin{document}
\maketitle
\maketitleprint[beige]

\question{}
לפוני מקבל הצעה לשלם מטבע אחד כדי לקבל שניים ביחידת הזמן הבאה.

\subquestion{}
נניח שפונקציית ההיוון של פלוני היא $D(\tau) = \delta^{\tau}$ עבור $\delta \in (0, 1)$. \\
נמצא את הערכים של $\delta$ עבורם משתלם לפלוני לקבל את ההצעה.
\begin{solution}
	נבחין כי הרווח הצפוי המחושב של פלוני הוא $-1 \cdot D(0) + 2 \cdot D(1) = -1 + 2 \delta$, וכמובן פלוני ירצה להשקיע אם ורק אם הרווח הוא חיובי, לכן $-1 + 2 \delta > 0$.
	כלומר אם $\delta > \frac{1}{2}$ אז לפלוני משתלם להשקיע.
\end{solution}

\subquestion{}
נניח שפונקציית ההיוון מוגדרת על־ידי,
\[
	D(\tau)
	= \begin{cases}
		1 & \tau = 0 \\
		\beta \delta^{\tau} & \tau > 0
	\end{cases}
\]
עבור $\delta = \frac{4}{5}, \beta = \frac{3}{4}$. \\
נחשב את הערך של קבלת ההצעה בהווה וביחידת הזמן הבאה.
\begin{solution}
	הנוסחה לערך שבהשקעה עכשיו היא $-D(0) + 2 D(1)$, ועבור השקעה ביחידת הזמן הבאה $-D(1) + 2D(2)$, נציב,
	\[
		-D(0) + 2 D(1)
		= -1 + 2 \cdot \frac{3}{4} \cdot \frac{4}{5}
		= -1 + \frac{6}{5}
		= \frac{1}{5}
	\]
	וכן,
	\[
		-D(1) + 2 D(2)
		= - \frac{3}{4} \cdot \frac{4}{5} + 2 \cdot \frac{3}{4} \cdot \frac{16}{25}
		= -\frac{15}{25} + \frac{24}{25}
		= \frac{9}{25}
	\]
	כלומר הגמול של פלוני הוא $\frac{7}{3}$ ו־$\frac{9}{25}$ בהתאמה.
\end{solution}

\subquestion{}
נניח שניתן לממש את ההצעה בכל יחידת זמן ושהיא ניתנת למימוש רק פעם אחת,
נבדוק מה הרווח של מימוש בזמן נתון.
\begin{solution}
	נניח שפלוני מממש את ההצעה בזמן $t \in \NN$, אז נקבל שהרווח הוא,
	\[
		-D(t) + 2D(t + 1)
		= \beta (-\delta^t + 2 \delta^{t + 1})
		= \beta \delta^t (-1 + 2 \cdot \delta)
		= \frac{3}{4} \cdot \frac{4^t}{5^t} \cdot \frac{3}{5}
	\]
	כלומר יוצא שלדחות את ההשקעה יניב רווח קטן יותר, אך שדחייה ביחידת זמן אחת תניב רווח גדול יותר.
	בהתאם נקבל את הפרדוקס שתמיד משתלם לדחות את ההחלטה ליחידת הזמן הבאה.
\end{solution}

\subquestion{}
נניח עתה שניתן לממש את ההצעה רק בחמש יחידות הזמן הקרובות, ונקבע מתי הכי משתלם לבצעה.
\begin{solution}
	נבחין כי מהסעיף הקודם נובע שהרווח בעוד חמש יחידות זמן הוא כ־$0.14$, כלומר קטן מלקחת את ההצעה בהווה.
	בהתאם על פלוני לחכות ארבע יחידות זמן ולבסוף להשקיע ביחידת הזמן החמישית, ובכך הרווח הוא הגדול ביותר בהינתן שיש דחיינות.
\end{solution}

\subquestion{}
בהנחה שוב שניתן לקבל את ההצעה בכל יחידת זמן, נמצא פתרון עקבי בזמן שפותר את הקושי העולה בסעיף ג'.
\begin{solution}
	נמצא אסטרטגיה מקרית שתהווה פתרון עקבי בזמן עבור פלוני.
	נניח שפלוני בוחר בכל יחידת זמן אם להשקיע לפי הערך $p$, כלומר אם $X$ משתנה מקרי המציג אם פלוני בחר לקבל את ההצעה,
	\[
		\PP(X = n) = {(1 - p)}^{n - 1} \cdot p
	\]
	ובהתאם נקבל שהרווח של פלוני יהיה $u(t) = \PP(X = t) (-D(t) + 2 D(t + 1))$.
	נציב ונקבל $u(0) = -1 + 2 \beta \delta$ בהתאם לסעיף ב'.
	נחשב את התוחלת בהנחה שפלוני לא קיבל את ההצעה בהווה,
	\[
		\EE(u(n) \mid n > 0)
		= \sum_{n = 1}^\infty {(1 - p)}^{n - 1} p \cdot \beta \delta^n (-1 + 2 \delta)
		= p \beta (-1 + 2 \delta) \sum_{n = 1}^\infty {(1 - p)}^{n - 1} \delta^n
		= \frac{p \beta \delta (-1 + 2 \delta)}{1 - \delta (1 - p)}
	\]
	כאשר השתמשנו בטור סדרה גאומטרית בחישוב, עתה נשווה,
	\[
		-1 + 2 \beta \delta
		= \frac{p \beta \delta (-1 + 2 \delta)}{1 - \delta (1 - p)}
		\implies \frac{-1 + 2 \beta \delta}{\beta \delta (-1 + 2 \delta)} = \frac{p}{1 - \delta + \delta p}
		\implies \frac{-1 + 2 \beta \delta}{\beta \delta (-1 + 2 \delta)} (1 - \delta) = p \left(1 - \frac{-1 + 2 \beta \delta}{\beta \delta (-1 + 2 \delta)} \delta\right)
	\]
	ונסיק,
	\[
		p
		= \frac{\frac{-1 + 2 \beta \delta}{\beta \delta (-1 + 2 \delta)} (1 - \delta)}{1 - \frac{-1 + 2 \beta \delta}{\beta \delta (-1 + 2 \delta)} \delta}
		= \frac{(-1 + 2 \beta \delta) (1 - \delta)}{\beta \delta (-1 + 2 \delta) - (-1 + 2 \beta \delta) \delta}
	\]
	ובהצבה נקבל $p = \frac{1}{5}$.
\end{solution}

\question{}
סטודנט צריך לבצע מטלה ולהגישה, כאשר הוא מקבל רווח שלילי ביום ההגשה של $-C$ וביום למחרת $D$ על ההצלחה בהגשתה.
פונקציית ההיוון שלו היא $D(t) = \frac{1}{1 + k t}$ עבור $C, V, k$ פרמטרים.

\subquestion{}
נכתוב קוד המחשב את הרווח אחרי מספר ימים נתון.

\subquestion{}
נריץ את המבחן למשך 30 ימים עם $C = 60, V = 100, k = 0.5$.
\begin{solution}
	במקרה זה הסטודנט לא כותב את המטלה באף יום, אלא ממשיך לדחות אותה.
\end{solution}

\subquestion{}
נגדיר סוכן מקרי ונחשב את המדיניות המקרית שבה תהיה עקביות בזמן.
נעשה זאת על־ידי חישוב התוחלת של ביצוע המטלה ואי־ביצועה.

\subsubsection{i}
נחשב את תוחלת הגמול מביצוע המטלה מיד, ואת תוחלת אי־ביצוע המטלה כעת.
\begin{solution}
	אם הסטודנט מבצע את המטלה בכל יום בהסתברות של $p$, אז נקבל,
	\[
		\EE(u \mid t = 0)
		= p u(0)
		= p (-C D(0) + V D(1))
	\]
	וכן,
	\[
		\EE(u \mid u(0) = 0)
		= p u(1) + (1 - p) p u(2) + \cdots
		= \sum_{n = 1}^\infty {(1 - p)}^{n - 1} p u(n)
	\]
\end{solution}

\subsubsection{ii}
נכתוב קוד לשערוך $p$ באופן נומרי על־ידי חישוב $1000$ האיברים הראשונים בתחום וקפיצות $\eta = 0.001$.
\begin{solution}
	הרצה של הפונקציה מניבה את התוצאה $p = 0.595$.
\end{solution}

\subquestion{}
נשתמש ב־$p$ שמצאנו בסעיף הקודם כדי להריץ $P = 10000$ סוכנים מקריים המריצים את החישוב של האם לבצע את המטלה או לא, וניצור היסטוגרמה של התוצאה.
\begin{solution}
	נציג את ההיסטוגרמה עבור ההרצות.
	\begin{center}
		\includegraphics[width=9cm]{./bin/out7_1}
	\end{center}
\end{solution}

\subquestion{}
נחקור את הפרמטר $k$ על־ידי חישוב $p$ לכל $k \in [0.05, 1] / 0.05$.
\begin{solution}
	נחשב את הערך של $p(k)$:
	\begin{center}
		\includegraphics[width=9cm]{./bin/out7_2}
	\end{center}
	נבחין כי הגרף מתחלק לכארבעה חלקים שונים.
	כאשר $k \in [0.05, 0.2]$ ערך $p$ יורד, כלומר ככל שהסטודנט אימפולסיבי במידה זעירה אז הוא ידחה את המטלה.
	לאחר מכן בתחום $k \in [0.2, 0.7]$ אנו מקבלים עליה מדורגת ב־$p$, כלומר בתחום זה הסטודנט מספיק אימפולסיבי כדי לסיים את המטלה כמה שיותר מהר ואז לסיים אותה.
	בתחום הבא כאשר $k$ הוא בערך $0.7$ מתקבלת ירידה חדה ולאחריה $p$ קרוב מאוד לאפס באופן עקבי.
	נסביר את הירידה החדה בכך שהסטודנט כל־כך אימפולסיבי שהוא פשוט לא יעשה את המטלה בעתיד, ובהתאם הסיכוי שהוא יעשה אותה עכשיו לאחר איזון יורד למספר זניח.
\end{solution}

\end{document}
