\input{../article_base.tex}
\title{פתרון מטלה 5 --- חישוביות וקוגניציה, 6119}

\usepackage{pgfplots, tikz}
\usetikzlibrary{math}
\graphicspath{{.}{../images/}}
\pgfplotsset{compat=1.18}

% chktex-file 17
% chktex-file 9

\begin{document}
\maketitle
\maketitleprint[beige]

\section{שאלת הכנה}
\subquestion{}
נבדוק מה נכון לומר על מקורות מקריים בבעיות למידת חיזוק.
\begin{solution}
	הפלט של המערכת חייב להיות פונקציה דטרמניסטית של הקלט (תשובה ב'), אחרת לא נוכל לבצע הליך למידה לאחר סף מסוים (כתלות בשונות).
	פונקציית הגמול יכולה להיות דטרמניסטית (תשובה ג'), שהרי נוכל להגדירה בעצמנו.
	תשובה ה', שכן אחרת לא נוכל לנצל את כוחו האמתי של מנגנון למידת החיזוק.
\end{solution}

\subquestion{}
נניח שמערכת לומדת בעזת למידת חיזוק, קיבלה קלט והוציאה פלט מקרי בהינתן הקלט והפרמטר $w$ וקיבלה גמול שלילי.
נבדוק מה אפשרי בהינתן מצב זה.
\begin{solution}
	ערך הזכאות יכול להיות חיובי או שלילי, אך בהנחה שאנו מחשבים באלגוריתם reinforce נקבל שבהתאמה כלל הלמידה יהיה $w$ יהיה שלילי או חיובי (תשובות ב' וג').
\end{solution}

\question{}
נדון בחישוב שמבצע טורף במהלך ניסיון לחזות את מיקום טרפו.
נסמן את מיקום הטרף $y$ ומחשבת הטורף $\hat{y}$, נניח כי שניהם $\in \RR$.
נניח ש־$\EE(y) = m, \var(y) = s^2$ עבור ערכים קבועים.
נניח ש־$\hat{y}$ משתנה מקרי נורמלי ו־$\hat{y} \sim N(\mu, \sigma^2)$ כאשר $\mu, \sigma$ נלמדים באמצעות אלגוריתם reinforce.
לכל ניסוי הטורף מקבל את התגמול $r = - {(\hat{y} - y)}^2$.

\subquestion{}
נגדיר במצב הנתון מה היא המערכת הלומדת, מה הפרמטרים הפנימיים, מה הקלט הפלט והגמול.
נבין גם מה מפת התלויות במקרה הנתון.
\begin{solution}
	TODO
\end{solution}

\end{document}
