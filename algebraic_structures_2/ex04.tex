\input{../article_base.tex}
\title{פתרון מטלה 04 --- מבנים אלגבריים (2), 80446}

\begin{document}
\maketitle
\maketitleprint[red]

\question{}
תהי $L / F$ הרחבת שדות ויהיו $g, h \in F[x]$.
נראה ש־$\gcd_{L[x]}(g, h) = \gcd_{F[x]}(g, h)$, כלומר נראה ש־$\gcd$ נשמר תחת הרחבת שדות.
\begin{proof}
	נניח ש־$f = \gcd_{F[x]}(g, h)$.
	נניח גם ש־$\overline{f} = \gcd_{L[x]}(g, h)$.
	כל פולינום $l \in F[x]$ הוא בפרט פולינום גם ב־$L[x]$ ולכן אם $l \mid g, h$ אז $\overline{f} \mid l$, אז בפרט גם $\overline{f} \mid f$ במקרה של $l = f$.

	המעלה של $\overline{f}$ קטנה משל $f$, אחרת ניקח את החיסור שלהם כשהם מתוקנים ונקבל פולינום קטן יותר שמחלק את $g, h$.
	נניח ש־$f = q \overline{f} + r$.
	אם $q, r \in F[x]$ אז $\overline{f} \in F[x]$ ונסיק ש־$f = \overline{f}$.
	נניח אם כך אחרת, אילו $q \in F[x]$ אבל $r \notin F[x]$ אז נקבל ש־$r = q \overline{f} - f \in F[x]$ וקיבלנו סתירה.
	נניח ש־$q \notin F[x]$.

	יהי $\alpha x^i$ מונום המרכיב את $q$ המעיד על $l \notin F[x]$, כלומר $\alpha \in L \setminus F$, ונניח גם כי זהו המונום הגדול ביותר המעיד על־כך, דהינו אם קיים $\beta x^j$ כך שגם $\beta \notin F$ ו־$j > i$, אז נבחר אותם במקום.
	לכל $a \in F$, אנו יודעים ש־$a + \alpha, a \cdot \alpha \notin F$, כהיסק מסגירות הכפל לפעולות אלה.
	נניח בלי הגבלת הכלליות ש־$f, \overline{f}$ מתוקנים שניהם.
	נבחין כי הדרגה של $r$ קטנה מזו של $q \cdot \overline{f}$, וכן נבחין ש־$\alpha x^i x^n$ עבור $n = \deg \overline{f}$ הוא מונום של $q \overline{f}$, ומהגדרת $r$ אין לו איברים מסדר $i + n$.
	נקבל אם כך ש־$q \cdot \overline{f} \notin F[x]$ וגם ש־$q \overline{f} + r \notin F[x]$, אבל זו סתירה כי $q \overline{f} + r = f \in F[x]$.
	
	קיבלנו ש־$q, r \in F[x]$ ולכן בפרט $f = \overline{f}$.
\end{proof}

\question{}
יהי $F$ שדה.

\subquestion{}
יהי $c \in F$ ויהיו $g, h \in F[x]$.

\subsubsection{i}
נראה ש־${(g + h)}' = g' + h'$.
\begin{proof}
	נניח ש־$g = \sum_{i = 0}^n \alpha_i x^i$ וכן ש־$h = \sum_{i = 0}^n \beta_i x^i$, עבור $\alpha_i, \beta_i \in F$, $n \in \NN$.
	נבחין כי אם דרגות הפולינומים לא שוות, אז $\alpha_i = 0$ או $\beta_i = 0$ החל מ־$i$ כלשהו, אך אין משמעות לעובדה זו בחישוב.
	עתה נבדוק את הזהות,
	\[
		(g + h)'
		= \sum_{i = 1}^n i (\alpha_i + \beta_i) x^{i - 1}
		= \sum_{i = 1}^n i \alpha_i x^{i - 1} + \sum_{i = 1}^n i \beta_i x^{i - 1}
		= g' + h'
	\]
	ולכן הזהות אכן חלה.
\end{proof}

\subsubsection{ii}
נראה ש־$(c \cdot g)' = c \cdot g'$.
\begin{proof}
	נבדוק,
	\[
		(c \cdot g)'
		= \left( c \cdot \sum_{i = 0}^n \alpha_i x^i \right)'
		= \left( \sum_{i = 0}^n c \alpha_i x^i \right)'
		= \sum_{i = 1}^n i c \alpha_i x^{i - 1}
		= c \sum_{i = 1}^n i \alpha_i x^{i - 1}
		= c \cdot g'
	\]
	וקיבלנו שאכן הזהות מתקיימת.
\end{proof}

\subsubsection{iii}
נראה שמתקיים, $(g \cdot h)' = g' \cdot h + g \cdot h'$.
\begin{proof}
	\begin{align*}
		(g \cdot h)'
		& = \left( g \cdot \sum_{i = 0}^n \beta_i x^i \right)' \\
		& = \left( \sum_{i = 0}^n g \cdot \beta_i x^i \right)' \\
		& = \sum_{i = 0}^n \beta_i (g \cdot x^i)' \\
		& = \sum_{i = 0}^n \beta_i \sum_{j = 1}^n \alpha_j (i + j) x^{i + j - 1} \\
		& = \sum_{i = 0}^n \beta_i \left( i x^{i - 1} \sum_{j = 1}^n \alpha_j x^j + x^i \sum_{j = 1}^n \alpha_j j x^{j - 1} \right) \\
		& = \sum_{i = 0}^n \beta_i \left( i x^{i - 1} g + x^i g' \right) \\
		& = g \sum_{i = 0}^n \beta_i i x^{i - 1} + g' \sum_{i = 0}^n \beta_i x^i \\
		& = g h' + g' h
	\end{align*}
\end{proof}

\subquestion{}
נוכיח את המקרה הפרטי לכלל לופיטל,
אם $a \in F$ שורש של $g \in F[x]$ כך ש־$g(x) = h(x) \cdot (x - a)$, אז $g'(a) = h(a)$.
\begin{proof}
	מתקיים,
	\[
		g'(x)
		= h'(x) (x - a) + h(x)
	\]
	מהזהויות שמצאנו בסעיף הקודם.
	נציב ונקבל,
	\[
		g'(a)
		= h'(a) (a - a) + h(a)
		= h(a)
	\]
	ומצאנו כי אכן מתקיים השוויון שרצינו להראות.
\end{proof}

\question{}
בכל סעיף נגדיר פולינום ונבדוק אם הוא ספרבילי מעל $\QQ$, נמצא שורשים מריבוי גדול מאחד.

\subquestion{}
נגדיר $f(x) = x^3 - 3x + 2$.
\begin{solution}
	נבחין כי $f(x) = {(x - 1)}^2 (x + 2)$ מחלוקת פולינומים והעובדה ש־$f(1) = 0$.
	לכן נסיק שהפולינום הוא לא ספרבילי וש־$1$ שורש כפול.
\end{solution}

\subquestion{}
נגדיר $f(x) = x^3 - 7x + 6$.
\begin{solution}
	עלינו לבדוק את $x = \pm 1, \pm 2, \pm 3$,
	נקבל מחישוב ש־$f(2) = 0$, ומחילוק פולינומים נקבל,
	\[
		f(x)
		= (x - 1) (x - 2) (x + 3)
	\]
	ולכן הפולינום הוא ספרבילי.
\end{solution}

\subquestion{}
נגדיר $f(x) = x^4 - 4x^3 + 6x^2 - 4x + 1$.
\begin{solution}
	הפעם עלינו לבדוק את $x = \pm 1$ בלבד, ונקבל $f(1) = 0$.
	מחלוקת פולינומים נקבל $f(x) = (x - 1)(x^3 - 3x^2 + 3x - 1) = {(x - 1)}^4$.
	נסיק אם כך שהפולינום לא ספרבילי והריבוי של $x = 1$ הוא $4$.
\end{solution}

\question{}
בכל סעיף נגדיר הרחבת שדות, ונבדוק אם היא נורמלית.

\subquestion{}
נבחן את $L = \QQ(\sqrt{2}, \sqrt{3}) / \QQ$.
\begin{solution}
	יהי $\sigma \in \aut_{\QQ}(\overline{\QQ})$, אנו רוצים להראות ש־$\sigma(L) = L$.
	לכל $q \in \QQ$ מהגדרה $\sigma(q) = q \in L$.
	עבור $q = \sqrt{2}$ נבחין כי $\sigma(q) \in \{ \pm q \} \subseteq L$ וכך גם עבור $q = \sqrt{3}$, ונסיק כי לכל $q \in L$ גם $\sigma(q) \in L$.
	קיבלנו אם כך ש־$\sigma$ משמר את $L$, ולכן ממשפט השקילות לנורמליות נסיק ש־$L / \QQ$ הרחבת שדות נורמלית.

	נוכל להשתמש גם בתנאי השלישי של המשפט, יהי $\alpha \in L$ ונרצה להראות ש־$f_{\QQ, \alpha}$ מתפצל לחלוטין ב־$L$.
	עבור $\alpha \in \QQ$ נקבל גורם לינארי, ולכן הפיצול טריוויאלי.
	עבור $\alpha = \sqrt{2}$ נקבל את $f_{\QQ, \alpha}(x) = x^2 + 2$, ופולינום זה אכן מפצל לחלוטין ל־$(x + \sqrt{2})(x - \sqrt{2})$.
	באופן דומה נסיק שהטענה נכונה עבור $q = \sqrt{3}$, ועבור צירופים לינאריים שלהם, ונקבל משקילות שאכן ההרחבה נורמלית.
\end{solution}

\subquestion{}
נבדוק את $L = \QQ(\sqrt[3]{2}) / \QQ$.
\begin{solution}
	נגדיר $\omega = e^{\frac{2\pi i}{3}} = -\frac{1}{2} + \frac{\sqrt{3}}{2} i$, נבחין כי $\omega^3 = 1$.
	עתה נגדיר את $\sigma : L \hookrightarrow \overline{\QQ}$ כ־$\QQ$־שיכון כך ש־$\sigma(\sqrt[3]{2}) = \omega \sqrt[3]{2}$.
	נבחין כי $f_{\QQ, \sqrt[3]{2}}(\omega \sqrt[3]{2}) = 0$ ולכן $\omega \in \overline{\QQ}$ ויש הצדקה להגדרה זו. \\
	עלינו להראות שזהו אכן $\QQ$־שיכון.
	עבור $q \in \QQ$, מוגדר ש־$\sigma(q) = q$, וכמו־כן $\sigma(\sqrt[3]{2}) = \omega \sqrt[3]{2}$ ו־$\sigma(\sqrt[3]{2^2}) = \omega^2 \sqrt[3]{2^2}$,
	נסיק שאכן $\sigma$ שיכון כפי שרצינו, ונבחין גם ש־$\sigma(L) = \QQ(\omega \sqrt[3]{2})$. \\
	נגדיר עתה שיכון נוסף, הוא $\id$ עצמו, הוא בבירור שיכון של $L \hookrightarrow \overline{\QQ}$, אבל $\id(L) = L \ne \QQ(\omega \sqrt[3]{2})$.
	נסיק אם כך שלא מתקיימת ההגדרה של נורמליות.
\end{solution}

\subquestion{}
נבדוק את $L = \QQ(\sqrt[3]{2}, \omega \sqrt[3]{2}) / \QQ(\omega) = K$, עבור $\omega$ שהוגדר בסעיף הקודם.
\begin{solution}
	יהי $\sigma \in \aut_K(\overline{L})$ ונרצה להראות ש־$\sigma(L) = L$.
	עבור $q \in \QQ$ ועבור $\omega$ הטענה נובעת מהגדרת $\aut_K$.
	נבחין כי גם $\sigma(\sqrt[3]{2} \omega) = \sigma(\sqrt[3]{2}) \cdot \sigma(\omega) = \sigma(\sqrt[3]{2}) \cdot \omega$ ולכן רק $\sigma(\sqrt[3]{2})$ קובע האם $\sigma(L) = L$.
	נבחין כי $\sigma(\sqrt[3]{2}) \in \{ \sqrt[3]{2}, \omega \sqrt[3]{2}, \omega^2 \sqrt[3]{2} \} \subseteq L$, ולכן נוכל להסיק שאכן $\sigma(L) = L$.
	נסיק ממשפט השקילות לנורמליות שאכן $L / K$ הרחבת שדות נורמלית.
\end{solution}

\question{}
יהיו $a, b, c \in \QQ$ כך שלא $a = b = c = 0$.
נמצא נוסחה מפורשת ל־$x, y, z \in \QQ$ כך שמתקיים,
\[
	{(a + b \cdot \sqrt[3]{5} + c \cdot \sqrt[3]{5^2})}^{-1}
	= x + y \cdot \sqrt[3]{5} + z \cdot \sqrt[3]{5^2}
\]
\begin{solution}
	נגדיר את $\omega$ של השאלה הקודמת, שורש היחידה הפרימיטיבי מסדר 3.
	נגדיר,
	\[
		S
		= (a + b \cdot \omega \sqrt[3]{5} + c \cdot \omega^2 \sqrt[3]{5^2}) \cdot (a + b \cdot \omega^2 \sqrt[3]{5} + c \cdot \omega \sqrt[3]{5^2})
	\]
	נבחין כי $\omega^2 = \overline{\omega}$, וכן $\omega + \omega^2 = 2 \re \omega = -1$, נובע,
	\begin{align*}
		S & = a^2 + b^2 \cdot \sqrt[3]{5^2} + 5 c^2 \sqrt[3]{5}
		+ a b \cdot \omega^2 \sqrt[3]{5} + a c \cdot \omega \sqrt[3]{5^2}
		+ a b \cdot \omega \sqrt[3]{5} + 5 b c \cdot \omega^2
		+ a c \cdot \omega^2 \sqrt[3]{5^2} + 5 b c \cdot \omega \\
		& = a^2 + b^2 \cdot \sqrt[3]{5^2} + 5 c^2 \sqrt[3]{5}
		- a b \sqrt[3]{5} - a c \sqrt[3]{5^2} - 5 b c
	\end{align*}
	ולכן נסיק $S \in \QQ(\sqrt[3]{5})$.
	נגדיר גם $\alpha = (a + b \sqrt[3]{5} + c \sqrt[3]{5^2})$, ומחישוב ישיר מתקיים,
	\begin{align*}
		S \cdot \alpha
		= a^3 + 5 b^3 + 25 c^3 - 15 a b c
	\end{align*}
	דהינו $S \cdot \alpha \in \QQ$, ולכן נוכל לבחור $\frac{S}{S \cdot \alpha}$ ונקבל,
	\[
		\frac{S}{S \cdot \alpha}
		\cdot (a + b \cdot \sqrt[3]{5} + c \cdot \sqrt[3]{5^2})
		= 1
		\iff
		\frac{S}{S \cdot \alpha}
		= {(a + b \cdot \sqrt[3]{5} + c \cdot \sqrt[3]{5^2})}^{-1}
	\]
	ונוכל להסיק,
	\[
		x = \frac{1}{a^3 + 5 b^3 + 25 c^3 - 15 a b c} (a^2 - 5 b c),
		y = \frac{1}{a^3 + 5 b^3 + 25 c^3 - 15 a b c} (5 c^2 - a b),
		z = \frac{1}{a^3 + 5 b^3 + 25 c^3 - 15 a b c} (b^2 - a c)
	\]
\end{solution}

\end{document}
