\input{../article_base.tex}
\title{פתרון מטלה 08 --- מבנים אלגבריים (2), 80446}

\begin{document}
\maketitle
\maketitleprint[red]

\question{}
יהי שדה $K$ כך ש־$p\text{-rank}$ של השדה הוא 1, כלומר $[K : K^p] = p^1$.

\subquestion{}
נראה שלכל $n \in \NN$ יש ל־$K$ בדיוק הרחבה אחת $L / K$ שהיא בלתי־ספרבילית לחלוטין מדרגה $p^n$,
וש־$L = K(a^{1 / p^n})$ עבור כל $a \in K \setminus K^p$.
\begin{proof}
	מתכונות $p \text{-rank}$ מתקיים,
	\[
		[K^{1 / p^n} : K]
		=
		[K^{1 / p^n} : K^{1 / p^{n - 1}}]
		\cdots
		[K^{1 / p} : K]
		= {[K : K^p]}^n
		= p^n
	\]
	ולכן נוכל להגדיר $L = K^{1 / p^n}$ ולקבל הרחבה מסדר $p^n$ לש $K$.
	נראה כי היא בלתי בלתי־ספרבילית לחלוטין.
	ממשפט מההרצאה מספיק שנוכיח $L \subseteq K^{1 / p^\infty}$, אבל זה ידוע מהגדרת $K^{1 / p^\infty}$ ולכן $L$ הרחבה בלתי־ספרבילית לחלוטין.

	נניח ש־$L_0 / K$ הרחבה בלתי־ספרבילית לחלוטין מסדר $p^n$, נראה ש־$L_0 = L$. \\
	בלתי־ספרביליות לחלוטין שקולה ל־$L_0 \subseteq K^{1 / p^\infty}$, ו־$[L_0 : K] = p^n$ מאלץ $L_0 \subseteq K^{1 / p^n} \cap K^{1 / p^\infty}$ ולכן $L_0 = L$ בלבד.

	נניח ש־$a \in K \setminus K^p$ איבר שאיננו שורש $p$.
	אז $[K(a^{1 / p^n}) : K] = p^n$ בלבד, ולכן מיחידות $K(a^{1 / p^n}) = L$.
\end{proof}

\subquestion{}
נוכיח שלכל הרחבה סופית $L / K$ יש שדה ביניים $L / L_i / K$ כך ש־$L_i / K$ בלתי־ספרבילית לחלוטין ו־$L / L_i$ ספרבילית.
\begin{proof}
	אנו יודעים כי $[L : K] = p^m$ עבור $m \in \NN$, וכן נבחר $n \le m$ המספר הגדול ביותר כך ש־$L / K^{1 / p^n} / K$, ונסמן $L_i = K^{1 / p^n}$.
	מהסעיף הקודם ברור כי $L_i / K$ בלתי־ספרבילית לחלוטין, ולכן נותר לבדוק את $L / L_i$.
	ישירות מהגדרת $L_i$ אנו יודעים כי אין $\alpha \in L_i$ כך ש־$\alpha$ בלתי־ספרבילי, אחרת מלכתחילה $\alpha \in L_i$, ולכן ${[L : L_i]}_i = 1$ ונובע ש־$L / L_i$ ספרבילי.
\end{proof}

\question{}
נמצא שדה ביניים של ההרחבה $\QQ(\xi_7) / \QQ$ מדרגה $2$ מעל $\QQ$.

\subquestion{}
נמצא תת־חבורה מאינדקס $2$ של $\gal(\QQ(\xi_7) / \QQ) \simeq \ZZ / 6\ZZ$,
ונמצא במפורש אוטומורפיזם $\sigma$ שיוצר אותה.
\begin{solution}
	נגדיר $H = \langle 2 \rangle \le \ZZ / 6\ZZ$, זוהי תת־חבורה מאינדקס $2$.
	היא מתאימה לאוטומורפיזם $\sigma \in \gal(\QQ(\xi_7) / \QQ)$ המוגדר על־ידי,
	\[
		\sigma(\xi_7)
		= \xi_7^2
	\]
\end{solution}

\subquestion{}
נחשב את $z = \sum_{h \in H} h(\xi_7)$ ונראה ש־$h(z) = z$ לכל $h \in H$.
\begin{proof}
	אנו יודעים ש־$\xi_7 \mapsto \xi_7^n$ ל־$n \in {(\ZZ / 7\ZZ)}^\times$ ולכן,
	\[
		z
		= \sum_{h \in H} h(\xi_7)
		= \sum_{n = 1}^6 \xi_7^n
		= -1 + \sum_{n = 0}^6 \xi_7^n
		= -1 + \frac{\xi_7^7 - 1}{\xi_7 - 1}
		= -1
	\]
	בפרט $z \in \QQ$ ו־$h(z) = z$ לכל $h \in \gal(\QQ(\xi_7) / \QQ)$.

	TODO
\end{proof}

\subquestion{}
נסיק ש־$z \in {\QQ(\xi_7)}^H$ וש־$[\QQ(z) : \QQ] \le 2$.
\begin{proof}
	TODO
\end{proof}

\question{}
יהי $K$ שדה, ו־$f \in K[x]$ אי־פריק וספרבילי.
נניח גם כי $L$ שדה פיצול של $f$ מעל $K$.

\subquestion{}
נראה שאם כל שדה ביניים $L / E / K$ הוא נורמלי אז כל שורש של $f$ יוצר את $L$ מעל $K$.
\begin{proof}
	יהי $\alpha$ שורש של $f$, אז $L / K(\alpha) / K$ הרחבה נורמלית, ו־$f_{\alpha, K}$ מתפצל לחלוטין בשדה זה.
	אבל $f = f_{\alpha, K}$ ישירות מהגדרה, כלומר $K(\alpha) = L$ בלבד.
\end{proof}

\subquestion{}
נסיק שאם $\gal(L / K)$ אבלית אז כל שורש של $f$ יוצר את $L$ מעל $K$.
\begin{proof}
	בחבורה אבלית כל תת־חבורה היא תת־חבורה נורמלית של החבורה, לכן אם $N \le \gal(L / K)$ אז $N \trianglelefteq \gal(L / K)$ ו־$L^N / K$ הרחבה נורמלית של $K$.
	מהסעיף הקודם נסיק ישירות שכל שורש של $f$ יוצר את $L$ מעל $K$.
\end{proof}

\question{}
נסמן $f(x) = x^4 - 7x^2 + 7 \in \QQ[x]$, ונניח כי $L / \QQ$ שדה פיצול של $f$.

\subquestion{}
נסמן,
\[
	\beta_1
	= \frac{7 + \sqrt{21}}{2},
	\quad
	\beta_2
	= \frac{7 - \sqrt{21}}{2}
\]
השורשים של $y^2 - 7y + 7 = 0$.

\subquestion{}
נראה ש־$\QQ(\beta_1, \beta_2) = \QQ(\beta_1)$ וש־$[\QQ(\sqrt{\beta_1}, \sqrt{\beta_2}) : \QQ(\beta_1)] = 4$.
\begin{proof}
	מתקיים,
	\[
		\beta_2
		= 7 - \beta_1
	\]
	ו־$7 \in \QQ$ לכן $\beta_2 \in \QQ(\beta_1)$ ובהתאם $\QQ(\beta_1, \beta_2) = \QQ(\beta_1)$ ובפרט $\QQ(\beta_1) = \QQ(\beta_2)$.

	כהיסק ישיר גם,
	\[
		[\QQ(\sqrt{\beta_1}, \sqrt{\beta_2}) : \QQ(\beta_1)]
		= [\QQ(\sqrt{\beta_1}, \sqrt{\beta_2}) : \QQ(\sqrt{\beta_1})]
		\cdot [\QQ(\sqrt{\beta_1}) : \QQ(\beta_1)]
		= 2 \cdot 2
	\]
	כמסקנה ממטלה 5 שאלה 3 סעיף ב'.
\end{proof}

\subquestion{}
נסיק ש־$L = \QQ(\sqrt{\beta_1}, \sqrt{\beta_2})$ מדרגה 8 מעל $\QQ$.
\begin{proof}
	אנו יודעים כי $[\QQ(\beta_1) : \QQ] = 2$ כשורש של פולינום אי־פריק ב־$\QQ$ מסדר 2, ולכן,
	\[
		[L : \QQ]
		= [L : \QQ(\beta_1)] \cdot [\QQ(\beta_1) : \QQ]
		= 4 \cdot 2
	\]
	וקיבלנו כי הדרגה היא $8$.
\end{proof}

\subquestion{}
נמצא את טיפוס האיזומורפיזם של $\gal(L / \QQ)$.
\begin{solution}
	אנו יודעים כי $|\gal(L / \QQ)| = 8$. \\
	אנו גם יודעים כי כל אוטומורפיזם $\sigma \in \gal(L / \QQ)$ נקבע ביחידות על־ידי המיפוי שלו ל־$\beta_1, \sqrt{\beta_1}, \sqrt{\beta_2}$, ושמיפוי זה בלתי תלוי,
	לכן נסיק ש־$\gal(L / \QQ) \simeq \HH$, חבורת הקוונטרניונים.
\end{solution}

\end{document}
