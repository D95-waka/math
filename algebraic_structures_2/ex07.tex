\input{../article_base.tex}
\title{פתרון מטלה 07 --- מבנים אלגבריים (2), 80446}

\begin{document}
\maketitle
\maketitleprint[red]

\question{}
נסמן $F = \QQ(s)$ ויהי $K$ שדה הפיצול של $x^n - s \in F[x]$. \\
נראה ש־$\aut(K / F) \simeq {(\ZZ / n \ZZ)}^\times \ltimes_{\theta} (\ZZ / n \ZZ)$, כאשר,
\[
	\theta : {(\ZZ / n \ZZ)}^\times \to \aut(\ZZ / n \ZZ),
	\qquad
	\theta(k)(n) = k n
\]
\begin{proof}
	בתרגול ראינו כי מכפלה חצי־ישרה זו היא איזומורפית לחבורת הפונקציות,
	\[
		G = \{ f : \ZZ / n \ZZ \to \ZZ / n \ZZ \mid \exists c, d,\ \forall x \in \ZZ / n \ZZ,\ f(x) = cx + d \}
	\]
	אנו נראה אם כך ש־$\aut(K / F) \simeq G$.
	%תהי $\varphi \in \aut(K / F)$, אז נגדיר $f \in G$ על־ידי $f(x) = \varphi(e)$
	בתרגול כבר ראינו כי קיים שיכון כזה, כלומר מצאנו שזהו הומומורפיזם חד־חד ערכי, ולכן עלינו רק להראות שהומומורפיזם זה הוא גם על.
	תהי $f \in G$, ונניח ש־$c, d$ קבועים כך ש־$f(x) = cx + d$.
	אנו יודעים כי $\xi_n^c \in K$ ולכן $\xi_n^m x \mapsto \xi_n^{mc + d} x^c$ הוא אוטומורפיזם תקין, כאשר ההוכחה לטענה זו זהה לזו שראינו בתרגול.
	נסמן ב־$\psi$ אוטומורפיזם זה ונקבל ש־$\aut(K / F) \ni \psi \mapsto f \in G$ בדיוק, ובכך נקבל שהומומורפיזם זה אכן על, ובהתאם הוא אוטומורפיזם.
\end{proof}

\question{}
תהי $L / \QQ$ הרחבה אלגברית נוצרת סופית.
נראה שב־$L$ יש מספר סופי של שורשי יחידה.
\begin{proof}
	נניח בשלילה שב־$L$ יש אינסוף שורשי יחידה.
	בפרט יש אינסוף שורשי יחידה פרימיטיביים, שאם לא כן יש כמות סופית של שורשי יחידה.

	TODO
\end{proof}

\end{document}
