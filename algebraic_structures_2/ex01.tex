\input{../article_base.tex}
\title{פתרון מטלה 01 --- מבנים אלגבריים (2), 80446}

\begin{document}
\maketitle
\maketitleprint{}

\question{}
תהי $L / K$ הרחבת שדות כך ש־$[L : K] = 7$.
נראה שלכל איבר $\alpha \in L \setminus K$ מתקיים $K[\alpha] = L$.
\begin{proof}
	הגדרנו את $K[\alpha]$ כשדה הנוצר על־ידי פולינום ש־$\alpha$ מאפס, בפרט $K[\alpha]$ הוא שדה כך ש־$K \subseteq K[\alpha]$ עד כדי איזומורפיזם הטלה $K[\alpha] \to K$.
	בנוסף נבחין כי $Id : K[\alpha] \to L$ הטלת זהות היא שיכון, לכן גם $K[\alpha] \subseteq L$.
	מצאנו אם כן ש־$L / K[\alpha] / K$ מגדל שדות, ולכן ממשפט האינדקס מתקיים $[L : K[\alpha]] \cdot [K[\alpha] : K] = 7$, לכן $[K[\alpha] : K] \in \{1, 7\}$.
	אבל אם אינדקס זה הוא $1$ אז נקבל ש־$\alpha \in K$ בסתירה להגדרתו, לכן $[K[\alpha] : K] = 7$, ובהתאם נובע ש־$[L : K[\alpha]] = 1$, כלומר $K[\alpha] = L$ בלבד.
\end{proof}

\question{}
יהי $\FF$ שדה סופי,
נראה שיש $p$ ראשוני ו־$n \in \NN$ כך ש־$|\FF| = p^n$.
\begin{proof}
	בהרצאה ראינו טענה שגורסת שיש תת־שדה ראשוני ב־$\FF$, אבל כמובן $|\QQ| > |\FF|$ ולכן נובע ישירות שקיים ראשוני $p$ עבורו $\FF_p \subseteq \FF$, ומהטענה נובע שראשוני זה הוא יחיד.
	נניח עתה שקיים ראשוני $q \mid |\FF|$, אז נוכל להסיק ממשפט לגרנז' נובע שקיים איבר $\alpha \in \FF^*$ כך ש־$\alpha^q = 1$, לכן המאפיין של השדה הוא $q$, אבל מאותה טענה מצאנו ש־$p = q$, ולכן רק $p$ מחלק של $\FF$.
	לבסוף נקבע $n = [\FF : \FF_p]$ ונקבל ש־$|\FF| = p^n$ ישירות מהגדרת מרחבים וקטוריים סופיים.
\end{proof}

\question{}
תהי $L / K$ הרחבת שדות ותהי $S = \{ s_i \mid 1 \le i \le m \} \subseteq L$.

\subquestion{}
נוכיח כי יש $K$־הומומורפיזם יחיד $\varphi : K[t_1, \dots, t_m] \to K[S]$ כך ש־$\varphi(t_i) = s_i$ לכל $i$.
\begin{proof}
	נניח כי $\varphi, \psi$ הומומורפיזמים המקיימים את הנתונים, אז $\forall i, \varphi(t_i) = \psi(t_i)$ מהגדרה.
	שתי ההעתקות סגורות לחיבור ולכפל, לכן אם $p_1, p_2 \in K[S]$ כך ש־$\varphi(p_j) = \psi(p_j)$ אז $\varphi(\alpha p_1 + \beta p_2) = \psi(\alpha p_1 + \beta p_2)$.
	לכן נותר שנבדוק הזדהות במונומים מתוקנים, כלומר איבר מהצורה $t_1^{\beta_1} \cdots t_m^{\beta_m}$,
	\[
		\varphi(t_1^{\beta_1} \cdots t_m^{\beta_m})
		= \varphi(t_1^{\beta_1}) \cdots \varphi(t_m^{\beta_m})
		= {\varphi(t_1)}^{\beta_1} \cdots {\varphi(t_m)}^{\beta_m}
		= {\psi(t_1)}^{\beta_1} \cdots {\psi(t_m)}^{\beta_m}
		= \psi(t_1^{\beta_1} \cdots t_m^{\beta_m})
	\]
	וקיבלנו כי אכן שתי ההעתקות מזדהות על כל התחום, כלומר $\varphi = \psi$.
\end{proof}

\subquestion{}
נפריך את הטענה כי יש $K$־הומומורפיזם יחיד $\varphi : K(t_1, \dots, t_m) \to K(S)$ כך ש־$\varphi(t_i) = s_i$ לכל $i$.
\begin{solution}
	נבחן את $\varphi : \RR(x) \to K(\{ i \}) = \CC$ עבור $K = \RR, L = \CC$.
	נניח שקיים $\varphi$ כזה, אז מתקיים $\varphi(x) = i$, ובהתאם נוכל להסיק בדומה לסעיף הקודם ש־$\varphi(f) = f(i)$ לכל פונקציה רציונלית $f \in \RR(x)$.
	לבסוף נובע $\varphi(\frac{1}{x^2 + 1}) = \frac{1}{i^2 + 1} = \frac{1}{0}$, כלומר הפונקציה לא מוגדרת היטב, וזו סתירה להנחת הקיום שלה.
\end{solution}

\question{}
יהי $\FF$ שדה ויהי $f \in \FF[x]$ פולינום מעל השדה.

\subquestion{}
נוכיח שאם $\deg f = 1$ אז $f$ ראשוני.
\begin{proof}
	$f$ הוא ראשוני אם ורק אם הוא אי־פריק, שכן חוג פולינומים הוא תחום אוקלידי.
	בהתאם נניח של־$f$ יש פירוק כלשהו, $f = g \cdot h$ עבור $g, h \in \FF[x]$.
	מטענות אודות דרגת פולינומים נסיק שמתקיים $\deg g + \deg h = \deg f$, ונובע ללא הגבלת הכלליות ש־$\deg g = 1$ וכן ש־$\deg h = 0$.
	אבל נקבל ש־$h = a x^0$, ו־$a \in \FF$, לכן קיים גם $a^{-1}$.
	נגדיר מחדש את $h' = h \cdot a^{-1} = 1$ ואת $g' = g \cdot a^{-1}$ ולכן נובע $f = g' \cdot h' = g'$, כלומר $f$ אכן אי־פריק ובהתאם ראשוני.
\end{proof}

\subquestion{}
נוכיח שאם $\deg f \in \{2, 3\}$ אז $f$ ראשוני אם ורק אם $f(\alpha) \ne 0$ לכל $\alpha \in \FF$.
\begin{proof}
	נניח ש־$f$ ראשוני ויהי $\alpha \in \FF$.
	אז $(x - \alpha) \nmid f$ מראשוניות, ונובע ישירות ש־$f(\alpha) \ne 0$.

	בכיוון ההפוך נניח שלכל $\alpha \in \FF$ מתקיים $f(\alpha) \ne 0$.
	אם $\deg f = 2$ והוא פריק נקבל בלי הגבלת הכלליות ש־$f = (x - \beta)(x - \gamma)$ עבור $\beta, \gamma \in \FF$, ונובע ישירות ש־$f(\beta) = 0$ בסתירה, לכן נוכל להסיק ש־$f$ אי־פריק ולכן ראשוני.
	נניח אם כן ש־$\deg f = 3$, ונניח שוב ש־$f$ פריק באופן לא טריוויאלי, כלומר קיימים $f = g \cdot h$ כך ש־$\deg g = 2, \deg h = 1$,
	אבל אז מהנתון נובע ש־$g$ לא מתאפס כלל וכן גם ש־$h$ לא מתאפס, וזאת סתירה להתאפסות $h$ כפולינום מהצורה $x - \beta$.
\end{proof}

\subquestion{}
נראה כי הטענה מסעיף ב' לא נכונה עבור $\deg f \ge 4$.
\begin{solution}
	נבחן את $f = x^4 + 1$ ב־$\RR$.
	לכל $x \in \RR$ ברור ש־$x^4 \ge 0$, ולכן גם $f(x) > 0$ ובפרט אין שורשים לפולינום זה.
	למרות זאת, $f = (x^2 + \sqrt{2}x + 1)(x^2 - \sqrt{2}x + 1)$, כלומר $f$ פריק.
\end{solution}

\question{}
נגדיר $\EE = \QQ[x] / (x^3 - 5)$.

\subquestion{}
נוכיח ש־$\EE$ שדה ושהוא איזומורפי לתת־השדה המינימלי של $\RR$ שמכיל את $\sqrt[3]{5}$, כלומר $\EE \simeq \QQ(\sqrt[3]{5})$.
\begin{proof}
	משאלה 4 נובע ש־$x^3 - 5$ אי־פריק מעל הרציונליים, ולכן נוכל להסיק כמסקנה מטענה מהתרגול שאכן $\EE$ שדה. \\
	נותר אם כן להוכיח ששני השדות איזומורפיים.
	נגדיר את ההומומורפיזם $\varphi : \QQ[x] \to \QQ(\sqrt[3]{5})$ הומומורפיזם ההצבה, ונבחין שמתקיים $\varphi(x^3 - 5) = 0$, כלומר $(x^3 - 5) \subseteq \ker \varphi$.
	ממשפט האיזומורפיזם הראשון נסיק שקיימים $\overline{\varphi} : \EE \to \QQ(\sqrt[3]{5}), \pi : \QQ[x] \to \EE$ כך ש־$\pi$ הטלה. \\
	מתקיים $\im \overline{\varphi} = \im \varphi$ וכן $\im \varphi = \QQ(\sqrt[3]{5})$, זאת שכן מבדיקה ישירה נוכל למצוא תצוגה פולינומיאלית על־ידי הצמדה ושימוש בפולינום ב־$\varphi$,
	לכן מלמה מהתרגול $\overline{\varphi}$ חד־חד ערכית ועל, ונוכל להסיק ש־$\EE \simeq \QQ(\sqrt[3]{5})$.
\end{proof}

\subquestion{}
נמצא $h \in \QQ[x]$ המקיים $h(\sqrt[3]{5}) = {\left(1 + 2\sqrt[3]{5} + 3 {\sqrt[3]{5}}^2\right)}^{-1}$.
\begin{solution}
	נשתמש בשיטה שהוצגה בתרגול, נסמן $f(x) = x^3 - 5$ וכן $g(x) = 1 + 2x + 3x^2$. \\
	נבחין כי אם $q_1(x) = \frac{1}{3} x - \frac{2}{9}$ וכן $r_1(x) = \frac{1}{3} (-\frac{1}{3}x + 15 - \frac{2}{3})$ אז מתקיים $f = q_1 \cdot g + r_1$. \\
	באופן דומה גם נגדיר $q_2(x) = 9x^2 + 9 \cdot 43x + 9 \cdot 43^2$ ו־$r_2(x) = 43^3 - 5$ ונקבל $f = q_2 \cdot (-r_1) + r_2$, וכן $\deg r_2 = 0$, לכן כמסקנה מהתרגול מתקיים,
	\[
		h
		= \frac{q_1(\sqrt[3]{5}) \cdot q_2(\sqrt[3]{5})}{-r_2}
	\]
\end{solution}

\end{document}
