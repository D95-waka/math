\input{../article_base.tex}
\title{פתרון מטלה 03 --- מבנים אלגבריים (2), 80446}

\begin{document}
\maketitle
\maketitleprint{}

\question{}
יהי $p$ ראשוני ו־$n \in \NN$, ויהי הפולינום הציקלוטומי מסדר $p^n$,
\[
	\frac{x^{p^n} - 1}{x^{p^{n - 1}} - 1}
	\in \QQ[x]
\]

\subquestion{}
נראה שזהו אכן פולינום.
\begin{proof}
	נראה ש־$x^{p^{n - 1}} - 1 | x^{p^n} - 1$.
	ניזכר בזהות $(1 + x + \cdots + x^{n - 1})(x - 1) = x^n - 1$ הנכונה לכל $x \in \CC$.
	לכן בפרט בהצבה ${(x^{p^n})}^k - 1 = (1 + x^{p^n} + \cdots + {(x^{p^n})}^{k - 1}) (x^{p^n} - 1)$, נוכל להציב $k = p$ ונובע,
	\[
		\frac{x^{p^{n + 1}} - 1}{x^{p^n} - 1}
		= 1 + x^{p^n} + \cdots + {(x^{p^n})}^{p - 1}
	\]
	ובפרט זהו פולינום כפי שרצינו להראות.
	נבחין שעבור $n = 0$ הטענה נובעת ישירות מהזהות.
\end{proof}

\subquestion{}
נוכיח שהפולינום הוא אי־פריק על־ידי שימוש בקריטריון אייזנשטיין.
\begin{proof}
	כלל מקדמי הפולינום הם $1$ ולכן לא נוכל להשתמש בקריטריון ישירות, נציב $x = y + 1$ ונקבל,
	\[
		\frac{x^{p^n} - 1}{x^{p^{n - 1}} - 1}
		= \frac{{(y + 1)}^{p^n} - 1}{{(y + 1)}^{p^{n - 1}} - 1}
		= 1 + {(y + 1)}^{p^n} + \cdots + {(y + 1)}^{(p - 1) p^n}
		= \sum_{i = 1}^{p - 1} \sum_{j = 0}^{i p^n} \binom{i p^n}{j} y^j
		= \sum_{j = 0}^{(p - 1) p^n} \sum_{i = j / p^n}^{p - 1} \binom{i p^n}{j} y^j
	\]
	ולכן המקדם של $y^i$ הוא $\sum_{i = j / p^n}^{p - 1} \binom{i p^n}{j}$, כאשר בפרט $a_{(p - 1)p^n} = 1$, ולכן $p \mid a_i$ לכל $i < (p - 1) p^n$ אבל $p \nmid a_{(p - 1) p^n}$.
	נבחין גם כי $a_0 = p$ ולכן $p^2 \nmid a_0$, וקריטריון אייזנשטיין חל וגורר שהפולינום הציקלוטומי מסדר $p^n$ אי־פריק.
\end{proof}

\question{}
נפרק את $f(x) = x^4 + 4 \in \QQ[x]$ לפולינומים אי־פריקים מעל $\QQ$.
\begin{solution}
	נבחין כי מעל $\CC$ ל־$f$, נסמן $\omega = e^{\frac{2 \pi i}{4}} = e^{\frac{1}{2} \pi i} = \frac{1 + i}{\sqrt{2}}$, ולכן,
	\[
		f(x)
		= (x - \sqrt{2})(x - \omega \sqrt{2})(x - \omega^2 \sqrt{2})(x - \omega^3 \sqrt{2} i)
	\]
	כלומר השורשים של $f$ הם $\omega^i \sqrt{2}$ עבור $i \in \{0, 1, 2, 3\}$.
	כל פולינום $g \in \QQ[x]$ כך ש־$g \mid f$ הוא מכפלת חלק מהגורמים הלינאריים הללו, ולכן מספיק לבדוק את $2^4 - 1 = 15$ הצירופים הללו.
	כלל הפולינומים מסדר $1$ הם מכפלות של $\sqrt{2}$ ולכן נוכל להסיק ישירות שאינם פירוק של $f$ מעל $\QQ$.
	באופן דומה לא יתכן שיהיה פולינום מחלק מדרגה 3, אחרת נקבל שאיברו החופשי הוא $2 \sqrt{2} \omega_i$ עבור $i$ כלשהו.
	נותר אם כן לבדוק את 6 הפולינומים מסדר 2.
	נוכל לפסול פולינומים שלא משלימים ל־$\omega$ בחזקה זוגית, אחרת האיבר החופשי שלהם יהיה מרוכב ובפרט לא רציונלי, ונשאר לבדוק שני פולינומים בלבד,
	\[
		(x - \omega \sqrt{2}) (x - \omega^3 \sqrt{2})
		= x^2 - \sqrt{2}(\omega + \omega^3) x + 2
	\]
	אבל מתקיים,
	\[
		\omega + \omega^3
		= \frac{i + 1 + (-1 + i)}{\sqrt{2}}
		= \sqrt{2}
	\]
	ונקבל את הפולינום $x^2 - 2x + 2$.
	מבדיקה ישירה נקבל ש־$x^4 + 4 = (x^2 - 2x + 2)(x^2 + 2x + 2)$.
	עבור המקרה השני נקבל,
	\[
		(x - \sqrt{2})(x - \omega^2 \sqrt{2})
		= x^2 - \sqrt{2} (1 + \omega^2) x + 2
	\]
	כלומר המקדם של $x$ הוא $\sqrt{2}(i + 1)$ וזהו לא מספר רציונלי.
\end{solution}

\question{}
יהיו $p_1, \ldots, p_n \in \NN$ ראשוניים שונים.
נראה ש־$[\QQ(\sqrt{p_1}, \ldots, \sqrt{p_n}) : \QQ] = 2^n$ ושהקבוצה,
\[
	\Bb
	= \left\{ \sqrt{\prod_{i \in S} p_i} \middle| S \subseteq \{1, \ldots, n\} \right\}
\]
היא בסיס ל־$\QQ(\sqrt{p_1}, \ldots, \sqrt{p_n})$ מעל $\QQ$.
\begin{proof}
	אנו יודעים שמתקיים,
	\[
		[\QQ(\sqrt{p_1}, \ldots, \sqrt{p_n}) : \QQ]
		= [\QQ(\sqrt{p_1}) : \QQ ] \cdots [\QQ(\sqrt{p_1}, \ldots, \sqrt{p_n}) : \QQ(\sqrt{p_1}, \ldots, \sqrt{p_{n - 1}}) ]
	\]
	והפולינום $f_i(x) = x^2 - p_i$ מהווה פולינום מינימלי עבור כל ראשוני כזה,
	\[
		[\QQ(\sqrt{p_1}, \ldots, \sqrt{p_i}) : \QQ(\sqrt{p_1}, \ldots, \sqrt{p_{i - 1}}) ] = 2
	\]
	לכל $i$.
	נסיק אם כך ש־$[\QQ(\sqrt{p_1}, \ldots, \sqrt{p_n}) : \QQ] = 2^n$.
\end{proof}

\end{document}
