\input{../article_base.tex}
\title{פתרון מטלה 03 --- מבנים אלגבריים (2), 80446}

\begin{document}
\maketitle
\maketitleprint{}

\question{}
יהי $p$ ראשוני ו־$n \in \NN$, ויהי הפולינום הציקלוטומי מסדר $p^n$,
\[
	\frac{x^{p^n} - 1}{x^{p^{n - 1}} - 1}
	\in \QQ[x]
\]

\subquestion{}
נראה שזהו אכן פולינום.
\begin{proof}
	נראה ש־$x^{p^{n - 1}} - 1 | x^{p^n} - 1$.
	ניזכר בזהות $(1 + x + \cdots + x^{n - 1})(x - 1) = x^n - 1$ הנכונה לכל $x \in \CC$.
	לכן בפרט בהצבה ${(x^{p^n})}^k - 1 = (1 + x^{p^n} + \cdots + {(x^{p^n})}^{k - 1}) (x^{p^n} - 1)$, נוכל להציב $k = p$ ונובע,
	\[
		\frac{x^{p^{n + 1}} - 1}{x^{p^n} - 1}
		= 1 + x^{p^n} + \cdots + {(x^{p^n})}^{p - 1}
	\]
	ובפרט זהו פולינום כפי שרצינו להראות.
	נבחין שעבור $n = 0$ הטענה נובעת ישירות מהזהות.
\end{proof}

\subquestion{}
נוכיח שהפולינום הוא אי־פריק על־ידי שימוש בקריטריון אייזנשטיין.
\begin{proof}
	כלל מקדמי הפולינום הם $1$ ולכן לא נוכל להשתמש בקריטריון ישירות, נציב $x = y + 1$ ונקבל,
	\[
		\frac{x^{p^n} - 1}{x^{p^{n - 1}} - 1}
		= \frac{{(y + 1)}^{p^n} - 1}{{(y + 1)}^{p^{n - 1}} - 1}
		= 1 + {(y + 1)}^{p^n} + \cdots + {(y + 1)}^{(p - 1) p^n}
		= \sum_{i = 1}^{p - 1} \sum_{j = 0}^{i p^n} \binom{i p^n}{j} y^j
		= \sum_{j = 0}^{(p - 1) p^n} \sum_{i = j / p^n}^{p - 1} \binom{i p^n}{j} y^j
	\]
	ולכן המקדם של $y^i$ הוא $\sum_{i = j / p^n}^{p - 1} \binom{i p^n}{j}$, כאשר בפרט $a_{(p - 1)p^n} = 1$, ולכן $p \mid a_i$ לכל $i < (p - 1) p^n$ אבל $p \nmid a_{(p - 1) p^n}$.
	נבחין גם כי $a_0 = p$ ולכן $p^2 \nmid a_0$, וקריטריון אייזנשטיין חל וגורר שהפולינום הציקלוטומי מסדר $p^n$ אי־פריק.
\end{proof}

\question{}
נפרק את $f(x) = x^4 + 4 \in \QQ[x]$ לפולינומים אי־פריקים מעל $\QQ$.
\begin{solution}
	נבחין כי מעל $\CC$ ל־$f$, נסמן $\omega = e^{\frac{2 \pi i}{4}} = e^{\frac{1}{2} \pi i} = \frac{1 + i}{\sqrt{2}}$, ולכן,
	\[
		f(x)
		= (x - \sqrt{2})(x - \omega \sqrt{2})(x - \omega^2 \sqrt{2})(x - \omega^3 \sqrt{2} i)
	\]
	כלומר השורשים של $f$ הם $\omega^i \sqrt{2}$ עבור $i \in \{0, 1, 2, 3\}$.
	כל פולינום $g \in \QQ[x]$ כך ש־$g \mid f$ הוא מכפלת חלק מהגורמים הלינאריים הללו, ולכן מספיק לבדוק את $2^4 - 1 = 15$ הצירופים הללו.
	כלל הפולינומים מסדר $1$ הם מכפלות של $\sqrt{2}$ ולכן נוכל להסיק ישירות שאינם פירוק של $f$ מעל $\QQ$.
	באופן דומה לא יתכן שיהיה פולינום מחלק מדרגה 3, אחרת נקבל שאיברו החופשי הוא $2 \sqrt{2} \omega_i$ עבור $i$ כלשהו.
	נותר אם כן לבדוק את 6 הפולינומים מסדר 2.
	נוכל לפסול פולינומים שלא משלימים ל־$\omega$ בחזקה זוגית, אחרת האיבר החופשי שלהם יהיה מרוכב ובפרט לא רציונלי, ונשאר לבדוק שני פולינומים בלבד,
	\[
		(x - \omega \sqrt{2}) (x - \omega^3 \sqrt{2})
		= x^2 - \sqrt{2}(\omega + \omega^3) x + 2
	\]
	אבל מתקיים,
	\[
		\omega + \omega^3
		= \frac{i + 1 + (-1 + i)}{\sqrt{2}}
		= \sqrt{2}
	\]
	ונקבל את הפולינום $x^2 - 2x + 2$.
	מבדיקה ישירה נקבל ש־$x^4 + 4 = (x^2 - 2x + 2)(x^2 + 2x + 2)$.
	עבור המקרה השני נקבל,
	\[
		(x - \sqrt{2})(x - \omega^2 \sqrt{2})
		= x^2 - \sqrt{2} (1 + \omega^2) x + 2
	\]
	כלומר המקדם של $x$ הוא $\sqrt{2}(i + 1)$ וזהו לא מספר רציונלי.
\end{solution}

\question{}
יהיו $p_1, \ldots, p_n \in \NN$ ראשוניים שונים.
נראה ש־$[\QQ(\sqrt{p_1}, \ldots, \sqrt{p_n}) : \QQ] = 2^n$ ושהקבוצה,
\[
	\Bb
	= \left\{ \sqrt{\prod_{i \in S} p_i} \middle| S \subseteq \{1, \ldots, n\} \right\}
\]
היא בסיס ל־$\QQ(\sqrt{p_1}, \ldots, \sqrt{p_n})$ מעל $\QQ$.
\begin{proof}
	אנו יודעים שמתקיים,
	\[
		[\QQ(\sqrt{p_1}, \ldots, \sqrt{p_n}) : \QQ]
		= [\QQ(\sqrt{p_1}) : \QQ ] \cdots [\QQ(\sqrt{p_1}, \ldots, \sqrt{p_n}) : \QQ(\sqrt{p_1}, \ldots, \sqrt{p_{n - 1}}) ]
	\]
	והפולינום $f_i(x) = x^2 - p_i$ מהווה פולינום מינימלי עבור כל ראשוני כזה,
	\[
		[\QQ(\sqrt{p_1}, \ldots, \sqrt{p_i}) : \QQ(\sqrt{p_1}, \ldots, \sqrt{p_{i - 1}}) ] = 2
	\]
	לכל $i$.
	נסיק אם כך ש־$[\QQ(\sqrt{p_1}, \ldots, \sqrt{p_n}) : \QQ] = 2^n$.

	נוכיח את כי $\Bb$ בסיס ל־$\QQ(\sqrt{p_1}, \ldots, \sqrt{p_n})$ באינדוקציה על $n$.
	עבור $n = 1$ הטענה נובעת מהגדרה, כלומר $\{ \sqrt{p_1} \}$ בסיס של $\QQ(\sqrt{p_1})$.
	נניח כי הטענה נכונה עבור $n$ ונבדוק את $n + 1$.
	יהי $\alpha \in \Sp_\QQ \Bb_n$. אילו $\alpha = 0$ אז גם $\alpha = 0 \ne p_{n + 1}$, אילו $\alpha = a \sqrt{s}$ עבור $\sqrt{s} \in \Bb_n$ אז $\alpha^2 = a^2 s \in \QQ$ ולכן לא יתכן שנגיע ל־$\sqrt{p_{n + 1}}$.
	נניח אם כך ש־$\alpha = a_1 s_1 + \cdots + a_k s_k$, כאשר $a_i \ne 0$ לכל $i$.
	$\alpha^2$ הוא חיבור של שורשי מספרים שונים ולכן לא מספר רציונלי, ונוכל להסיק שבפרט $\alpha \ne \sqrt{p_{n + 1}}$.
	לכן נובע ש־$p_{n + 1} \notin \QQ(\sqrt{p_1}, \ldots, \sqrt{p_n})$ ובהתאם $\QQ(\sqrt{p_1}, \ldots, \sqrt{p_{n + 1}}) = \QQ(\sqrt{p_1}, \ldots, \sqrt{p_n})(\sqrt{p_{n + 1}})$ וכן
	\[
		\Bb_{n + 1}
		= \left\{ \sqrt{\prod_{i \in S} p_i} \middle| S \subseteq \{1, \ldots, n + 1\} \right\}
	\]
\end{proof}

\question{}
בכל סעיף נגדיר את $\alpha$ ונחשב את $[\QQ(\alpha) : \QQ]$.

\subquestion{}
$\alpha = \sqrt{13 + 6 \sqrt{2}}$.
\begin{solution}
	ננסה למצוא פולינום שיחסום את הערך.
	\begin{align*}
		x - \sqrt{13 + 6 \sqrt{2}} = 0
		& \iff x^2 - 13 + 6 \sqrt{2} = 0 \\
		& \iff x^2 - 26 x + 169 - 72 = 0 \\
		& \iff x^2 - 26 x + 97 = 0
	\end{align*}
	ולכן $[\QQ(\alpha) : \QQ] \le 4$.
	נבחין כי $\sqrt{2} \in \QQ(\alpha)$ ולכן $\QQ(\alpha) / \QQ(\sqrt{2})$, ולכן,
	\[
		[\QQ(\alpha) : \QQ]
		= [\QQ(\alpha) : \QQ(\sqrt{2})] \cdot [\QQ(\sqrt{2}) : \QQ]
		= 2 [\QQ(\alpha) : \QQ(\sqrt{2})]
	\]
	נבחן האם $\alpha \in \QQ(\sqrt{2})$, כלומר האם קיימים $a, b \in \QQ$ כך ש־$a + b \sqrt{2} = \alpha$,
	\[
		a^2 + 2b^2 + 2ab \sqrt{2} = 13 + 6 \sqrt{2}
	\]
	ולכן $a^2 + 2b^2 = 13$ וכן $ab = 3$, נסיק,
	\[
		a^2 + 2 \frac{9}{a^2} - 13 = 0
		\iff a^4 - 13 a^2 + 18 = 0
		\iff a^2 = \frac{13 \pm \sqrt{97}}{2}
	\]
	ונסיק שלא קיימים $a, b \in \QQ$ כאלה, ולכן $2 \ge [\QQ(\alpha) : \QQ(\sqrt{2})] \ne 1$, ולכן בהכרח,
	\[
		[\QQ(\alpha) : \QQ] = 4
	\]
\end{solution}

\subquestion{}
$\alpha = \sqrt{11 + 6 \sqrt{2}}$.
\begin{solution}
	גם הפעם $\sqrt{2} \in \QQ(\alpha)$, ונרצה לבדוק האם $\alpha \in \QQ(\sqrt{2})$, נקבל מתהליך דומה לסעיף הקודם כי,
	\[
		a^2 + 2 b^2 = 11
		\qquad
		ab = 3
	\]
	ולכן
	\[
		a^2 + \frac{18}{a^2} - 11 = 0
		\iff a^2 = \frac{11 \pm \sqrt{121 - 72}}{2} = 2, 9
	\]
	ולכן $a = 3$ ונוכל להסיק שגם $b = 1$, כלומר $\alpha = 3 + \sqrt{2} \in \QQ(\sqrt{2})$ ולכן,
	\[
		[\QQ(\alpha) : \QQ]
		= [\QQ(\alpha) : \QQ(\sqrt{2})] \cdot [\QQ(\sqrt{2}) : \QQ]
		= 1 \cdot 2
	\]
\end{solution}

\question{}
נראה ש־$f(x) = x^2 + 4 \in \QQ[x]$ אי־פריק, אבל ש־$f(x + a)$ לא מקיים את קריטריון אייזנשטיין לאף $a \in \ZZ$ ולאף $p$ ראשוני.
\begin{proof}
	נבחין כי לכל $a \in \ZZ$ מתקיים $f(x + a) = x^2 + 2a x + (a^2 + 4)$.
	נבחין גם כי $a^2 + 4 = {(a + 2)}^2 - 2a$.
	על הראשוני $p$ לחלק את $2a$, לכן $p = 2$ או $2 < p \mid a$.
	אם $2 < p$ אז $p \mid 2a$ אבל $p \nmid a + 2$, ולכן בהכרח $p \nmid a^2 + 4$, ובהכרח $p = 2$.
	נבחין כי $2 \nmid 1$ ולכן התנאי עבור המקדם של המעלה הגדולה ביותר כן חל, ולכן נראה ש־$4 = p^2 \mid a^2 + 4$.
	אם $a$ מספר זוגי, אז בפרט $4 \mid a^2$ וכן גם $4 \mid a^2 + 4$, ולכן תנאי הקריטריון לא חלים.
	נניח ש־$a$ אי־זוגי, נובע שגם $a + 2$ אי־זוגי וכן גם ${(a + 2)}^2$ אי־זוגי, לעומת זאת $2a$ זוגי, ולכן $a^2 + 4 = {(a + 2)}^2 - 2a$ מספר אי־זוגי, כלומר $2 \nmid a^2 + 4$ בסתירה לדרישה ש־$p \mid a_0$.
	נסיק אם כך שתנאי הקריטריון לא חלים גם במקרה $p = 2$, לכן לא קיים $p$ עבורו התנאים מתקיימים, לכל $a$.
\end{proof}

\end{document}
