\documentclass[a4paper, 10pt]{article}

% packages
\usepackage[utf8]{inputenc}
\usepackage{fontspec}
\usepackage{amsmath}
\usepackage{amsfonts}
\usepackage{polyglossia}
\usepackage{geometry}
\usepackage{catchfile}

% style
\newcommand{\getenv}[2][]{%
  \CatchFileEdef{\temp}{"|kpsewhich --var-value #2"}{\endlinechar=-1}%
  \if\relax\detokenize{#1}\relax\temp\else\let#1\temp\fi}
\getenv[\AUTHOR]{AUTHOR}
\setotherlanguage{hebrew}
\setmainfont{Libertinus Serif}
\newfontfamily\hebrewfont{Libertinus Serif}[Script=Hebrew]
\linespread{1.5}
\setcounter{secnumdepth}{0}
\DeclareMathOperator\cis{cis}
\DeclareMathOperator\Sp{Sp}
\geometry{paper=a4paper, margin=54pt, includeheadfoot}

\title{פתרון ממ''ן 12 – אלגברה לינארית 1 (20109)}
\author{\AUTHOR}
\date{\today}

\begin{document}
\begin{hebrew}
	\maketitle
	\section{שאלה 1}
	\subsection{סעיף א'}
	נשתמש בטענה 2.6.5 לבחינת התלות הלינארית בקבוצה על־ידי יצירת מטריצה מתאימה ודרוגה:
	\[
		\begin{bmatrix}
			1 & 1 & 1 & 1 \\
			3 & 0 & 1 & 5 \\
			0 & 1 & 1 & 0 \\
			1 & 0 & 0 & 1
		\end{bmatrix}
		\xrightarrow{R_1 \leftrightarrow R_4}
		\begin{bmatrix}
			1 & 0 & 0 & 1 \\
			3 & 0 & 1 & 5 \\
			0 & 1 & 1 & 0 \\
			1 & 1 & 1 & 1
		\end{bmatrix}
		\xrightarrow[R_4 = R_4 - R_1 - R_3]{R_2 = R_2 - 3R_1}
		\begin{bmatrix}
			1 & 0 & 0 & 1 \\
			0 & 0 & 1 & 2 \\
			0 & 1 & 1 & 0 \\
			0 & 0 & 0 & 0
		 \end{bmatrix}
	 \]
	 למערכת המקדמים המצומצמת יש שורת אפסים, לכן ישנו פתרון טריוויאלי למערכת והיא תלויה לינארית.

	 \subsection{סעיף ב'}
	 באופן דומה לסעיף הראשון נבנה מטריצה מהווקטורים ונדרגה:
	 \[
		\begin{bmatrix}
			1 & 1 & 1 & 0 \\
			1 & 2 & 1 & 2 \\
			2 & 1 & 1 & 2 \\
			2 & 2 & 2 & 2
		\end{bmatrix}
		\xrightarrow[R_3 = R_3 + R_1, R_4 = R_4 + R_1]{R_2 = R_2 - R_1}
		\begin{bmatrix}
			1 & 1 & 1 & 0 \\
			0 & 1 & 0 & 2 \\
			0 & 2 & 2 & 2 \\
			0 & 0 & 0 & 2
		\end{bmatrix}
		\xrightarrow[R_3 = R_3 - R_4]{R_2 = R_2 - R_4}
		\begin{bmatrix}
			1 & 1 & 1 & 0 \\
			0 & 1 & 0 & 0 \\
			0 & 2 & 2 & 0 \\
			0 & 0 & 0 & 2
		\end{bmatrix}
		\xrightarrow{R_3 = R_3 + R_2}
		\begin{bmatrix}
			1 & 1 & 1 & 0 \\
			0 & 1 & 0 & 0 \\
			0 & 0 & 2 & 0 \\
			0 & 0 & 0 & 2
		\end{bmatrix}
	\]
	מטריצת המקדמים עבור קבוצת הווקטורים לא שקולת שורה למטריצה בעלת שורת אפסים, לכן היא בלתי תלויה לינארית לפי טענה 2.6.5.

	\section{שאלה 2}
	נוכיח כי יש אינסוף פתרונות למשוואה
	$x_1v_1 + x_2v_2 + \cdots + x_n v_n = 0$. \\*
	אם אין פתרון למשוואה $x_1v_1 + x_2v_2 + \cdots + x_n v_n = w$
	אז לפי משפט 1.12.1 מטריצת המקדמים השקולה למשוואה שקולת שורה
	למטריצה בה יש שורה מהתצורה $[0, \ldots, 0, a] (a \ne 0)$.
	לפיכך ניתן לדעת כי מטריצת המקדמים המצומצמת של המשוואה
	$x_1v_1 + x_2v_2 + \cdots + x_n v_n = 0$
	מכילה שורת אפס.
	לפי שורת האפס אנו יודעים כי למשוואה יש משתנה חופשי אחד לפחות,
	ולכן לפי משפט 1.12.2 יש לה אינסוף פתרונות.

	\section{שאלה 3}
	\subsection{סעיף א'}
	תהיינה המטריצות $A_{m \times n}, B_{n \times m}$
	המקיימות $(*)AB=I_m$.
	נוכיח כי למערכת ההומוגנית $B\underline{x} = \underline{0}$
	יש פתרון יחיד. \\*
	נראה כי עבור המשוואה מתקיים:
	\[
		\begin{aligned}
			& B \underline{x} = \underline{0} & A \cdot \\
			& A B \underline{x} = A \underline{0} & (*) \\
			& I \underline{x} = \underline{0} \\
			& \underline{x} = \underline{0} \\
		\end{aligned}
	\]
	אנו רואים כי ממערכת המשוואות הנתונה נובע כי
	$\underline{x} = 0$ בלבד, וזהו הפתרון היחיד למערכת.

	\subsection{סעיף ב'}
	נוכיח כי $m \le n$. \\*
	ידוע כי למערכת $Bx = 0$ ישנו פתרון בודד, לכן אין במערכת משתנה חופשי.
	לפי משפט 2.6.5 וקטורי העמודה המרכיבים את המטריצה $B$
	הם בלתי תלויים לינארית.
	ישנם $m$ וקטורים כאלה,
	ככמות השורות ב־$B$, וכל אחד מהם שייך ל־$F^n$ כאשר $F$
	הוא השדה עליו מוגדרות המטריצות.
	לפי מסקנה 2.6.7 מתקיים $m \le n$.

	\subsection{סעיף ג'}
	תהיה מטריצה $X$ כך ש־$BX = I_n$. נוכיח כי $X = A$ ו־$n = m$. \\*
	נשתמש בדרך ההוכחה של סעיף ב’ על $B$ ו־$X$
	כמחליפות של $A$ ו־$B$ בהתאמה,
	ונראה כי $n \le m$, אבל $m \le n$, לכן $n = m$. \\*
	שתי המטריצות $A, B$ הן מטריצות ריבועיות.
	לפי טענה 3.10.3 מטריצה היא הפיכה אם ורק אם העמודות שלה בלתי
	תלויות לינארית. ראינו בסעיף א’ כי אכן זהו המצב ב־$B$ ולכן
	היא מטריצה הופכית. \\*
	מתקיים עבור המטריצה ההופכית $B$ לפי הגדרה 3.8.2 כי $AB = BA = I$,
	אך $BA = BX$, לכן לפי טענה  3.8.3 מתקיים $A = X$.

	\section{שאלה 4}
	תהיינה מטריצות ריבועיות $A, B$ מסדר $n$,
	נוכיח כי אם $AB^2 - A$ הפיכה, אז $BA + A$ הפיכה. \\*
	לפי שאלה 3.10.2 שתי מכפלת מטריצות הפיכה אם ורק אם
	המטריצות המוכפלות הפיכות גם הן,
	ניתן לראות כי $AB^2 - A = A(B^2 - I)$ לפי הפילוג על כפל מטריצות.
	לכן $A$ הפיכה ו־$B^2 - I$ הפיכה גם היא.
	נוסיף ונפרק $B^2 - I = (B-I)(B+I)$,
	ולכן גם $B - I$ הפיכה, וכמוה גם $B + I$. \\*
	נגדיר $M = {(AB^2 - A)}^{-1}$,
	נראה כי על־פי החילופיות בכפל של מטריצות הפיכות מתקיים:
	\[
		\begin{aligned}
			& M(AB^2 - A) = I \\
			& MA(B^2 - I) = I \\
			& MA(B+I)(B-I) = I \\
			& M(B - I)(AB + A) = I
		\end{aligned}
	\]
	אנו רואים כי המטריצה $AB + A$ היא הפיכה,
	שכן מכפלתה ב־$M(B - I)$ מובילה למטריצת היחידה.

	\section{שאלה 5}
	\subsection{סעיף א'}
	תהיה מטריצה אנטיסימטרית $A$, כאשר $I + 2A$ הפיכה.
	נוכיח כי גם $I - 2A$ הפיכה. \\*
	נראה תחילה כי חיבור מטריצות משוחלפות שקול לשחלוף חיבורי המטריצות המקוריות,
	תהנייה מטריצות ${[a_{ij}]}_{n \times m}, {[b_{ij}]}_{n \times m}$ מתקיים:
	\[
		{[a_{ij}]}^t + {[b_{ij}]}^t =
		[a_{ji} + b_{ji}] =
		{([a_{ij}] + [b_{ij}])}^t
	\]
	ניתן לראות באותה הצורה כי כפל בסקלר ניתן להוצאה מפעולת השחלוף. \\*
	נראה גם כי מטריצה הפיכה משוחלפת היא עדיין הפיכה,
	על־פי טענה 3.4.5 $(\#)$ עבור מטריצה הפיכה $B$ מתקיים
	${(B^{-1}B)}^t = B^t {(B^{-1})}^t$,
	לכן לפי הגדרת הפיכות 3.8.2 כל מטריצה הפיכה משוחלפת שומרת על הפיכותה. \\*
	נשתמש בטענות הללו ונראה כי
	${(I + 2A)}^t = I^t + 2A^t = I - 2A$,
	לכן המטריצה $I - 2A$ הפיכה.

	\subsection{סעיף ב'}
	נוכיח כי המטריצה $C = {(I-2A)(I+2A)}^{-1}$
	מקיימת $C^t C = I$. \\*
	תחילה נוכיח כי עבור כל מטריצה הפיכה $D$ מתקיים
	${(D^t)}^{-1} = {(D^{-1})}^t$.
	$I$ היא מטריצה סימטרית, לכן מתקיים $I^t = I$.
	נראה כי עבור כל מטריצה הופכית $D$ מתקיים:
	\[
		\begin{aligned}
			& I = D D^{-1} \\
			& I^t = {(DD^{-1})}^t \\
			& I^t = {(D^{-1})}^t D^t
		\end{aligned}
	\]
	ידוע כי $D^t$ היא הופכית, נגדיר $D' = D^t$, אז $D'$ הופכית:
	\[
		I = D'^{-1} D' \rightarrow
		I = {(D^t)}^{-1} D^t
	\]
	נשלב את שתי המשוואות לפי $I$
	\[
		\begin{aligned}
			& {(D^t)}^{-1} D^t = I = I^t = {(D^{-1})}^t D^t \\
			& {(D^t)}^{-1} D^t = {(D^{-1})}^t D^t \\
			& {(D^t)}^{-1} = {(D^{-1})}^t & (*)
		\end{aligned}
	\]
	המטריצה $C$ הפיכה שכן היא מכפלה של מטריצות הפיכות. נחשב את ערך $C^t$:
	\[
		\begin{aligned}
			C^t & = {((I-2A){(I+2A)}^{-1})}^t \\
			& = {(I-2A)}^t {({(I+2A)}^{-1})}^t & (\#) \\
			& = (I+2A){(I-2A)}^{-1} & (*)
		\end{aligned}
	\]
	נציב:
	\[
		\begin{aligned}
			C^t C & = (I+2A){(I-2A)}^{-1} (I-2A){(I+2A)}^{-1} \\
			& = (I+2A){(I+2A)}^{-1} {(I-2A)}^{-1} (I-2A) \\
			& = II \\
			& = I
		\end{aligned}
	\]
	הוכחנו כי המטריצה $C$ מקיימת $C^t C = I$.

	\section{שאלה 6}
	\subsection{סעיף א'}
	קיימת מטריצה $C$ כך שמתקיים $B = CA$
	משום שהמטריצות $A, B$ שקולות שורה.
	נבצע פעולות אלמנטריות אשר מתבטאות בפעולות שורה כדי להגיע מ־$A$ ל־$B$
	כדי להוכיח טענה זו:
	\[
		\begin{aligned}
			& \begin{bmatrix}
				2 & 1 & 2 \\
				4 & 0 & 3 \\
				0 & 3 & 5
			\end{bmatrix}
			\xrightarrow{R_1 = R_1 + R_3}
			\begin{bmatrix}
				2 & 4 & 0 \\
				4 & 0 & 3 \\
				0 & 3 & 5
			\end{bmatrix}
			\xrightarrow{R_1 \leftrightarrow R_2}
			\begin{bmatrix}
				4 & 0 & 3 \\
				2 & 4 & 0 \\
				0 & 3 & 5
			\end{bmatrix}
			\xrightarrow{R_3 = R_3 + R_1}
			\begin{bmatrix}
				4 & 0 & 3 \\
				2 & 4 & 0 \\
				4 & 3 & 1
			\end{bmatrix} \\
			& \xrightarrow{R_3 = R_3 + 3R_2}
			\begin{bmatrix}
				4 & 0 & 3 \\
				2 & 4 & 0 \\
				3 & 1 & 1
			\end{bmatrix}
			\xrightarrow{R_1 = R_1 + R_3}
			\begin{bmatrix}
				0 & 1 & 4 \\
				2 & 4 & 0 \\
				3 & 1 & 1
			\end{bmatrix}
			\xrightarrow{R_1 = R_1 + 4R_2}
			\begin{bmatrix}
				1 & 3 & 4 \\
				2 & 4 & 0 \\
				3 & 1 & 1
			\end{bmatrix}
		\end{aligned}
	\]
	לפי משפט 3.9.4, נוכל לפרק את פעולות השורה שביצענו
	במעבר כמכפלות של מטריצות אלמנטריות,
	מכפלה זו בהכפלה ב־$A$ שווה ל־$B$, לכן מטריצה זו היא $C$.
	נחשב את ערך המטריצה לפי הפעולות האלמנטריות על מטריצת היחידה:
	\[
		C= \begin{bmatrix}
			0 & 2 & 1 \\
			1 & 0 & 1 \\
			3 & 1 & 4
		\end{bmatrix}
	\]

	\subsection{סעיף ב'}
	נרשום את $C$ כמכפלה של המטריצות האלמנטריות המתקבלות מהפעולות
	האלמנטריות שבוצעו במעבר בסעיף הקודם:
	\[
		C = 
		\begin{bmatrix}
			0 & 2 & 1 \\
			1 & 0 & 1 \\
			3 & 1 & 4
		\end{bmatrix}
		=
		\begin{bmatrix}
			1 & 0 & 1 \\
			0 & 1 & 0 \\
			0 & 0 & 1 \\
		\end{bmatrix}
		\begin{bmatrix}
			0 & 1 & 0 \\
			1 & 0 & 0 \\
			0 & 0 & 1 \\
		\end{bmatrix}
		\begin{bmatrix}
			1 & 0 & 0 \\
			0 & 1 & 0 \\
			1 & 0 & 1 \\
		\end{bmatrix}
		\begin{bmatrix}
			1 & 0 & 0 \\
			0 & 1 & 0 \\
			0 & 3 & 1 \\
		\end{bmatrix}
		\begin{bmatrix}
			1 & 0 & 1 \\
			0 & 1 & 0 \\
			0 & 0 & 1 \\
		\end{bmatrix}
		\begin{bmatrix}
			1 & 4 & 0 \\
			0 & 1 & 0 \\
			0 & 0 & 1 \\
		\end{bmatrix}
	\]

\end{hebrew}
\end{document}
