\documentclass[a4paper, 10pt]{article}

% packages
\usepackage[utf8]{inputenc}
\usepackage{fontspec}
\usepackage{amsmath}
\usepackage{amsfonts}
\usepackage{polyglossia}
\usepackage{geometry}
\usepackage{catchfile}

% style
\newcommand{\getenv}[2][]{%
  \CatchFileEdef{\temp}{"|kpsewhich --var-value #2"}{\endlinechar=-1}%
  \if\relax\detokenize{#1}\relax\temp\else\let#1\temp\fi}
\getenv[\AUTHOR]{AUTHOR}
\setotherlanguage{hebrew}
\setmainfont{Libertinus Serif}
\newfontfamily\hebrewfont{Libertinus Serif}[Script=Hebrew]
\linespread{1.5}
\setcounter{secnumdepth}{0}
\DeclareMathOperator\cis{cis}
\DeclareMathOperator\Sp{Sp}
\geometry{paper=a4paper, margin=54pt, includeheadfoot}

\title{פתרון ממ''ן 11 – אלגברה לינארית 1 (20109)}
\author{\AUTHOR}
\date{\today}

\begin{document}
\begin{hebrew}
	\maketitle
	\section{שאלה 1}
	\subsection{סעיף א'}
נמצא את ערכו של $\alpha$:
	\[
		\begin{aligned}
			& \alpha = 2 \cdot 4 - 3 \cdot {1 \over 2} \\
			& \alpha = 8 - 3 \cdot 6 \\
			& \alpha = 8 - 18 \\
			& \alpha = 1
		\end{aligned}
	\]
	נמצא את ערכו של $\beta$:
	\[
		\begin{aligned}
			& \beta = {2 \over 3} - {3 \over 4} \\
			& \beta = 8 - 9 \\
			& \beta = 10
		\end{aligned}
	\]

	\subsection{סעיף ב'}
	1.
	\[
		\begin{aligned}
			& 3x^2 = 6 \\
			& x^2 = 2 \\
			& x = 3, 4
		\end{aligned}
	\]
	2.
	\[
		\begin{aligned}
			& 6x^2 + {1 \over 2} = 0 \\
			& 6x^2 + 4 = 0 \\
			& x^2 + {2 \over 3} = 0 \\
			& x^2 = 4 \\
			& x = 2, 5
		\end{aligned}
	\]
	3.
	\[
		\begin{aligned}
			& 5x + 4y + z = 0 \\
			& z = -5x - 4y = 2x + 3y \\
			& \{ t, s \in \mathbb{Z}_7 \mid (t, s, 2t + 3s) \}
		\end{aligned}
	\]

	\section{שאלה 2}
	\subsection{סעיף א'}
	$(A, \oplus, *)$ הוא אכן שדה.
	ניתן להוכיח כי זהו שדה לפי הגדרה 1.2.1,
	אך נעשה זאת בהישענות על עובדת היות $\mathbb{R}$ שדה.
	נגדיר את הפונקציה $f: A \to \mathbb{R}$
	כך ש־$f(x, 1) = x$.
	הפונקציה $f$ היא חד־חד ערכית שכן עבור כל $x, y$ מתקיים
	$f(x, 1) = x = y = f(y, 1)$ אם ורק אם $x = y$.
	פונקציה זו גם על, שכן כל מספר ממשי מוכל בתמונת $f$.
	ידוע לנו כי $\mathbb{R}$ הוא אכן שדה,
	וניתן לראות כי בהגדרת החיבור והכפל והרכבת הפונקציה $f$
	מתקבלות הגדרות החיבור והכפל בהתאמה של השדה $\mathbb{R}$,
	לכן בסך־הכול ניתן להסתמך על תכונות השדה $\mathbb{R}$
	והפונקציה $f$ ולראות כי גם $(A, \oplus, *)$ שדה.

	\subsection{סעיף ב'}
	1. הפעולה $*$ היא חילופית, שכן עבור כל $a, b \in \mathbb{R}$ מתקיים:
	\[
		a * b = a + b - 2 = b + a - 2 = b * a
	\]
	הפעולה גם קיבוצית, שכן עבור כל $a, b, c \in \mathbb{R}$ מתקיים:
	\[
		(a * b) * c = (a + b - 2) + c - 2 = a + (b + c - 2) - 2 = a * (b * c)
	\]
	2. נוכיח כי קיים איבר ניטרלי עבור הפעולה. אם איבר הוא ניטרלי,
	אז בביצוע הפעולה איתו הערך המקורי לא משתנה.
	ניצור משוואה מתאימה: $a * e = 0$, כאשר $e$ מייצג את האיבר הניטרלי.
	נפתור:
	\[
		\begin{aligned}
			& a * e = a \\
			& a + e - 2 = a \\
			& e = 2
		\end{aligned}
	\]
	לכן קיים איבר ניטרלי לפעולה וערכו הוא $2$.

	\subsection{סעיף ג'}
	נוכיח כי $\mathbb{Z}_9$ איננו שדה.
	על־פי משפט 1.2.6, בכל שדה אם מתקיים $ab = 0$,
	אז $a = 0$ או $b = 0$.
	ב־$\mathbb{Z}_9$ מתקיים $3 \cdot 3 = 0$,
	וזוהי סתירה להנחה כי הוא שדה,
	לכן $\mathbb{Z}_9$ איננו שדה.

	\section{שאלה 3}
	\subsection{סעיף א'}
	נמיר את מערכת המשוואות למטריצת מקדמים ונדרגה:
	\[
		\begin{aligned}
			& \begin{bmatrix}
			1 & 2 & 1 & 1 & \vline & 1 \\
			1 & 1 & 2 & 1 & \vline & 2 \\
			1 & 1 & 1 & 0 & \vline & 2 \\
			2 & 1 & 1 & 1 & \vline & 8
			\end{bmatrix}
			\xrightarrow{R_1 \leftrightarrow R_3}
			\begin{bmatrix}
			1 & 1 & 1 & 0 & \vline & 2 \\
			1 & 1 & 2 & 1 & \vline & 2 \\
			1 & 2 & 1 & 1 & \vline & 1 \\
			2 & 1 & 1 & 1 & \vline & 8
			\end{bmatrix}
			\xrightarrow[R_3 = R_3 - R_1, R_4 = R_4 - 2R_1] {R_2 = R_2 - R_1}
			\begin{bmatrix}
			1 & 1 & 1 & 0 & \vline & 2 \\
			0 & 0 & 1 & 1 & \vline & 0 \\
			0 & 1 & 0 & 1 & \vline & -1 \\
			0 & -1 & -1 & 1 & \vline & 4
			\end{bmatrix} \\
			& \xrightarrow{R_2 \leftrightarrow R_3}
			\begin{bmatrix}
			1 & 1 & 1 & 0 & \vline & 2 \\
			0 & 1 & 0 & 1 & \vline & -1 \\
			0 & 0 & 1 & 1 & \vline & 0 \\
			0 & -1 & -1 & 1 & \vline & 4
			\end{bmatrix}
			\xrightarrow{R_4 = R_4 + R_3 + R_2}
			\begin{bmatrix}
			1 & 1 & 1 & 0 & \vline & 2 \\
			0 & 1 & 0 & 1 & \vline & -1 \\
			0 & 0 & 1 & 1 & \vline & 0 \\
			0 & 0 & 0 & 3 & \vline & 3
			\end{bmatrix}
		\end{aligned}
	\]
	נשתמש במטריצת המדרגות המפושטת ונבצע הצבה לאחור:
	\[
		\begin{aligned}
			& 3t = 3 \rightarrow t = 1 \\
			& z + t = 0 \rightarrow z = -1 \\
			& y + t = -1 \rightarrow y = -2 \\
			& x + y + z = 2 \rightarrow x = 5
		\end{aligned}
	\]
	הפתרון היחיד למערכת המשוואות הוא
	$x = 5, y = -2, z = -1, t = 1$.

	\subsection{סעיף ב'}
	על־פי למה 5.2.6 חיבור וכפל מודולו ניתנים לביצוע
	על תוצאות ומכפלות מספרים משדה $\mathbb{Z}$.
	ננצל זאת כדי להשתמש במטריצת המקדמים המצומצמת שחישבנו
	בסעיף הקודם ונחיל עליה את השדה $\mathbb{Z}_3$ על־ידי ביצוע מודולו
	לכל אחד מן הסקלרים במטריצה:
	\[
		\begin{bmatrix}
			1 & 1 & 1 & 0 & \vline & 2 \\
			0 & 1 & 0 & 1 & \vline & -1 \\
			0 & 0 & 1 & 1 & \vline & 0 \\
			0 & 0 & 0 & 3 & \vline & 3
		\end{bmatrix}
		\equiv_3
		\begin{bmatrix}
			1 & 1 & 1 & 0 & \vline & 2 \\
			0 & 1 & 0 & 1 & \vline & 2 \\
			0 & 0 & 1 & 1 & \vline & 0 \\
			0 & 0 & 0 & 0 & \vline & 0
		\end{bmatrix}
	\]
	הפעם ניתן לראות כי הגענו לשורת $0$,
	לכן ישנו משתנה אחד חופשי,
	ממטריצת המדרגות אנו רואים כי משתנה זה הוא $t$.
	נבצע הצבה לאחור:
	\[
		\begin{aligned}
			& z + t = 0 \rightarrow z = -t \\
			& y + t = 2 \rightarrow y = 2 - t \\
			& x + y + z = 2 \rightarrow x + 2 - t - t = 2 \rightarrow x = 2t
		\end{aligned}
	\]
	קבוצת פתרונות מערכת המשוואות היא:
	\[
		\{ s \in \mathbb{Z}_3 \mid (2s, 2 - s, -s, s) \}
	\]
	בשל היותו של השדה סופי, יש שלושה פתרונות למשוואה:
	$(0, 2, 0, 0), (2, 1, 2, 1), (1, 0, 1, 2)$

	\section{שאלה 4}
	תחילה, נמיר את מערכת המשוואות למטריצת מקדמים ונדרגה:
	\[
		\begin{aligned}
			& \begin{bmatrix}
			1 & 2 & a & \vline & -3 - a \\
			1 & 2 - a & -1 & \vline & 2 - a \\
			a & a & 1 & \vline & 7
			\end{bmatrix}
			\xrightarrow[R_3 = R_3 - aR_1]{R_2 = R_2 - R_1}
			\begin{bmatrix}
			1 & 2 & a & \vline & -3 - a \\
			0 & -a & -1 - a & \vline & 5 \\
			0 & -a & 1 - a^2 & \vline & 7 + 3a + a^2
			\end{bmatrix}
			\xrightarrow{R_3 = R_3 - R_2} \\
			& \begin{bmatrix}
			1 & 2 & a & \vline & -3 - a \\
			0 & -a & -1 - a & \vline & 5 \\
			0 & 0 & 2 + a - a^2 & \vline & 2 + 3a + a^2
			\end{bmatrix}
			=
			\begin{bmatrix}
			1 & 2 & a & \vline & -3 - a \\
			0 & -a & -1 - a & \vline & 5 \\
			0 & 0 & -(a + 1)(a - 2) & \vline & (a + 1)(a + 2)
			\end{bmatrix}
		\end{aligned}
	\]
	(i) למערכת המשוואות יהיה פתרון יחיד אם אף שורה
	לא תתאפס במטריצת המקדמים המדורגת.
	ניתן לראות שעבור $a = -1$ השורה השלישית מתאפסת,
	ועבור $a = 2$ השורה היא שורת סתירה,
	לכן כדי שיהיה פתרון בודד $a \ne -1, 2$. \\*
	(ii) כאמור, הערך היחיד עבורו ישנו איפוס באחת השורות הוא כאשר $a = -1$,
	במקרה זה השורה השלישית מתאפסת, ו־$z$ הופך למשתנה חופשי.
	נציב לאחור:
	\[
		\begin{aligned}
			& -ay - (a + 1)z = 5 \rightarrow y = 5 \\
			& x + 2y + az = -3 - a \rightarrow x + 10 - z = -2
			\rightarrow x = z - 12
		\end{aligned}
	\]
	הפתרון הכללי הוא
	$\{ t \in \mathbb{R} \mid (t - 12, 5, t) \}$. \\*
	(iii) לא יהיה פתרון למערכת המשוואות במקרה
	שקיימת שורת סתירה במטריצת המקדמים המדורגת,
	אנו רואים כי רק במקרה ש־$a = 2$ השורה השלישית הופכת לשורת סתירה.

	\section{שאלה 5}
	נמיר את מערכת המשוואות למטריצה ונדרגה:
	\[
		\begin{aligned}
			\begin{bmatrix}
				1 & a & a & a - b & \vline & b + 1 \\
				1 & a + 1 & a + b & 2a - b & \vline & a + b + 1 \\
				3 & 3a & 3a + b & 3a - b & \vline & 4b + 3 \\
				1 & a & a & 0 & \vline & 2b
			\end{bmatrix}
			\xrightarrow{R_1 \leftrightarrow R_4}
			\begin{bmatrix}
				1 & a & a & 0 & \vline & 2b \\
				1 & a + 1 & a + b & 2a - b & \vline & a + b + 1 \\
				3 & 3a & 3a + b & 3a - b & \vline & 4b + 3 \\
				1 & a & a & a - b & \vline & b + 1
			\end{bmatrix}
			\xrightarrow[R_3 = R_3 - 3R_1, R_4 = R_4 - R_1]{R_2 = R_2 - R_1} \\
			\begin{bmatrix}
				1 & a & a & 0 & \vline & 2b \\
				0 & 1 & b & 2a - b & \vline & a - b + 1 \\
				0 & 0 & b & 3a - b & \vline & -2b + 3 \\
				0 & 0 & 0 & a - b & \vline & -b + 1
			\end{bmatrix}
			\xrightarrow[R_4 = R_4 - 3R_4]{R_2 = R_2 - 2R_4}
			\begin{bmatrix}
				1 & a & a & 0 & \vline & 2b \\
				0 & 1 & b & b & \vline & a + b - 1 \\
				0 & 0 & b & 2b & \vline & b \\
				0 & 0 & 0 & a - b & \vline & -b + 1
			\end{bmatrix}
			\xrightarrow{R_2 = R_2 - R_3}
			\begin{bmatrix}
				1 & a & a & 0 & \vline & 2b \\
				0 & 1 & 0 & -b & \vline & a - 1 \\
				0 & 0 & b & 2b & \vline & b \\
				0 & 0 & 0 & a - b & \vline & -b + 1
			\end{bmatrix}
		\end{aligned}
	\]
	תחילה נראה עבור אילו ערכים של $a$ ו־$b$ יש למערכת המשוואות פתרון יחיד.
	למערכת יהיה פתרון יחיד כאשר למטריצת המדרגות לא
	תהיה שורת אפסים ושורת סתירה כלל.
	בשל הערך $1$ המופיע בשתי השורות הראשונות במטריצה,
	שתי שורות אלה לא יכולות להשפיע על קיומו של פתרון יחיד.
	השורה השלישית תלויה כולה ב־$b$,
	לכן איפוסו יוביל לשורת אפסים ולאינסוף פתרונות.
	בשורה הרביעית ניתן לראות כי כאשר $a=b$ אז השורה תהיה שורת סתירה או
	שורת אפסים. לכן כדי שיהיה פתרון יחיד למערכת המשוואות חייב להתקיים
	$a \ne b, a \ne 0, b \ne 0$. \\*
	למערכת המשוואות אין פתרון כאשר במטריצת המדרגות יש שורת סתירה.
	כאמור, שורת סתירה תיתכן בשורה הרביעית בלבד.
	ניתן לראות שמצב זה יקרה כאשר $a = b$ וגם
	$-b +1 \ne 0 \rightarrow b \ne 0$.
	דהינו לא יהיה פתרון למערכת המשוואות במצב שבו $a = b \ne 1$. \\*
	כפי שראינו, אם $b = 0$ או $a = b = 1$ אז יהיו אינסוף פתרונות
	עבור מערכת המשוואות.
	תחילה נמצא את הפתרון הכללי כאשר $a \ne 0, b = 0$ על־ידי הצבה לאחור:
	\[
		\begin{aligned}
			& (a - 0)w = -0 + 1 \rightarrow w = {1 \over a} \\
			& y = a - 1 \\
			& x + ay + az = 0 \rightarrow x = -a^2 + a - az
		\end{aligned}
	\]
	לכן במקרה זה הפתרון הכללי יהיה
	$\{ t \in \mathbb{R} \mid (-a^2 + a - at, a - 1, t, {1 \over a}) \}$. \\*
	במקרה שבו $a = b = 0$ אנו כבר יודעים כי אין פתרון למערכת המשוואות.
	נציב לאחור במקרה זה:
	\[
		\begin{aligned}
			& z + 2w = 1 \rightarrow z = 1 - 2w \\
			& y - w = 0 \rightarrow y = w \\
			& x + y + z = 2 \rightarrow x = 1 + w
		\end{aligned}
	\]
	לכן במקרה זה הפתרון הכללי הוא
	$\{ t \in \mathbb{R} \mid (1 + t, t, 1 - 2t, t) \}$.

\end{hebrew}
\end{document}
