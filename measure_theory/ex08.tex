\input{../article_base.tex}
\title{פתרון מטלה 8 --- תורת המידה, 80517}

\DeclareMathOperator{\esssup}{ess\,sup}
\DeclareMathOperator{\essinf}{ess\,inf}

\begin{document}
\maketitle
\maketitleprint[blue]

\question{}
נניח ש־$(X, \Aa)$ מרחב מדיד ו־$T : X \to X$ מדידה.
הגדרנו מידה על $X$ להיות $T$־אינווריאנטית אם מתקיים $T_* \mu = \mu$.
נאמר גם שמידה $T$־אינווריאנטית היא ארגודית אם לכל $A \in \Aa$ המקיימת $T^{-1}(A) = A$ מתקיים ש־$A$ היא $\mu$־טריוויאלית, כלומר $\mu(A) = 0 \lor \mu(A^C) = 0$.

\subquestion{}
תהי $A \in \Aa$ ונגדיר את סדרת הקבוצות $A_1 = A$ ו־$A_{n + 1} = T^{-1}(A_n)$. \\
נראה ש־$A^- = \liminf A_n, A^+ = \limsup A_n$ הן $T$־אינווריאנטיות.
\begin{proof}
	מהגדרה עלינו להראות ש־$T^{-1}(A^-) = A^-, T^{-1}(A^+) = A^+$.
	\[
		T^{-1}(A^-)
		= T^{-1}\left(\bigcup_{n = 1}^\infty \bigcap_{k = n}^\infty A_k\right)
		\overset{(1)}{=}  \bigcup_{n = 1}^\infty T^{-1}\left( \bigcap_{k = n}^\infty A_k \right)
		\overset{(2)}{=}  \bigcup_{n = 1}^\infty \bigcap_{k = n}^\infty T^{-1}\left( A_k \right)
		= \bigcup_{n = 1}^\infty \bigcap_{k = n}^\infty A_{k + 1}
		= A^-
	\]
	כאשר $(1), (2)$ נובעים מתכונות תמונה הפוכה והמעבר האחרון נובע מתכונות הגבול התחתון.
	המהלך עבור $A^+$ שקול.
\end{proof}

\subquestion{}
נניח ש־$\mu$ מידת הסתברות ארגודית.
נראה שאם $T^{-1}(A) = A$ כמעט תמיד אז $\mu(A) \in \{0, 1\}$.
\begin{proof}
	נתון ש־$\mu$ ארגודית, כלומר אם $E \in \Aa$ וכן $E$ היא $T$־אינווריאנטית אז $\mu(A) \in \{0, 1\}$.
	נגדיר $A_1 = A, A_{n + 1} = T^{-1}(A_n)$, מתקיים $A_1 = A_2$ כמעט תמיד, וכן אם $A_1 = A_n$ כמעט תמיד אז גם $A_n = A_{n + 1}$ כמעט תמיד ולכן $A_1 = A_{n + 1}$ כמעט תמיד, נסיק ש־$A = A_n$ כמעט תמיד לכל $n$.
	מהסעיף הקודם $A^- = \liminf A_n, A^+ = \limsup A_n$ הן $T$־אינווריאנטיות ולכן $\mu(A^+), \mu(A^-) \in \{0, 1\}$.
	\[
		\mu(A^-)
		= \mu(\liminf A_n)
		\le \liminf \mu(A_n)
		= \liminf \mu(A)
		= \mu(A)
	\]
	אבל $\mu(A^-) \in \{0, 1\}$ ואם $\mu(A^-) = 1$ אז נקבל $\mu(A) = 1$ וסיימנו, לכן נניח ש־$\mu(A^-) = 0$.
	המרחב הוא מרחב הסתברות ולכן,
	\[
		\mu(A^+)
		= \mu(\limsup A_n)
		\ge \limsup \mu(A_n)
		= \mu(A)
	\]
	אם $\mu(A^+) = 0$ אז שוב נקבל $\mu(A) = 0$ ולכן נניח ש־$\mu(A^+) = 1$.
	אם $A = A_n$ כמעט תמיד אז גם $A = A \cap A = A_n \cap A_{n + 1}$ כמעט תמיד ובאינדוקציה $A = \bigcap_{n = 1}^k A_n$ כמעט תמיד, נסיק שגם $\mu(A) = \mu(\lim_{n \to \infty} \bigcap_{n = 1}^k A_n)$.
	\[
		0
		= \mu(A^-)
		= \mu(\bigcup_{n = 1}^\infty \bigcap_{k = n}^\infty A_k)
		= \lim_{n \to \infty} \mu(\bigcap_{k = n}^\infty A_k)
	\]
	ולכן נקבל שבמצב זה $\mu(A) = 0$ כפי שרצינו.
\end{proof}

\subquestion{}
נאמר שפונקציה $f : X \to \RR$ היא $T$־אינווריאנטית אם $f = f \circ T$. \\
נראה שמידה $T$־אינווריאנטית $\mu$ היא ארגודית אם ורק אם כל הפונקציות ה־$T$־אינווריאנטיות המדידות שוות לפונקציה קבועה כלשהי כמעט בכל מקום.
\begin{proof}
	נניח ש־$\mu$ ארגודית ותהי $f : X \to \RR$ פונקציה מדידה ו־$T$־אינווריאנטית כלשהי, נראה ש־$f$ קבועה כמעט תמיד.
	נניח בשלילה שלא, כלומר קיימים $a, b \in \RR$ מתקיים $\mu(f^{-1}(a)) > 0, \mu(f^{-1}(b)) > 0$.
	נבחין כי $T^{-1}(f^{-1}(a)) = f^{-1}(a)$ ולכן $f^{-1}(a) \subseteq X$ קבוצה $T$־אינווריאנטית ובהתאם $\mu(f^{-1}(a)) = 0$ או $\mu(X \setminus f^{-1}(a)) = 0$, אבל $\mu(f^{-1}(a)) > 0$ ולכן $\mu(X \setminus f^{-1}(a)) = 0$.
	אבל $f^{-1}(b) \subseteq X \setminus f^{-1}(a)$ ולכן $0 < \mu(f^{-1}(b)) \le \mu(X \setminus f^{-1}(a)) = 0$ בסתירה.

	נניח עתה שכל $f$ כזו קבועה כמעט תמיד ונראה ש־$\mu$ ארגודית.
	תהי $A \in \Aa$ המקיימת $T^{-1}(A) = A$, נראה ש־$\mu(A) = 0$ או $\mu(A^C) = 0$.
	נגדיר $g = \indicator_A$ ולכן $g(x) = 1 \iff x \in A \iff x \in T^{-1}(A) \iff T(x) \in A \iff g(T(x)) = 1$, אז $g$ היא $T$־אינווריאנטית ולכן קבועה.
	אבל אז נובע $g =_{\mu} 0$ או $g =_{\mu} 1$ ובפרט $\mu(A) = 0$ או $g(A^C) = 0$ כמו שרצינו.
\end{proof}

\question{}
נוכיח בשלבים את המשפט הבא:
יהי $X$ מרחב מדיד ו־$T : X \to X$ מדידה, אז כל שתי מידות הסתברות $T$־אינווריאנטיות ארגודיות שונות הן סינגולריות אחת לשנייה, ונניח ש־$T$ הפיכה וכן ש־$T^{-1}$ מדידה.

\subquestion{}
תהיינה שתי מידות הסתברות $\mu, \nu$ כך שהן $T$־אינווריאנטיות.
נסמן את הפירוק של $\nu$ לפי $\mu$ על־פי משפט לבג־רדון־ניקודים ב־$\nu = \nu_a + \nu_s$.
נראה ש־$\nu_a, \nu_s$ הן גם $T$־אינווריאנטיות.
\begin{proof}
	מהגדרת $T$־אינווריאנטיות נקבל ש־$T_* \mu = \mu$ ו־$T_* \nu = \nu$.
	לכן אם $T_* \nu = \lambda_a + \lambda_s$ פירוק של $T_* \nu$ לפי $T_* \mu$ אז למעשה $\nu = \lambda_a + \lambda_s$.
	אבל $T_* \mu = \mu$ ולכן $\lambda_a + \lambda_s$ פירוק רדון־ניקודים של $\nu$ מעל $\mu$ ומיחידות הפירוק נקבל $\lambda_s = \nu_s, \lambda_a = \nu_a$.
	אז נקבל $T_* \nu = T_* (\nu_a + \nu_s) = T_* \nu_a + T_* \nu_s = \lambda_a + \lambda_s = \nu_a + \nu_s = \nu$ ובהתאם $\nu_a, \nu_s$ שתיהן $T$־אינווריאנטיות.
\end{proof}

\subquestion{}
נסיק שאם $\nu$ ארגודית אז $\nu = \nu_a$ או $\nu = \nu_s$.
\begin{proof}
	נניח בשלילה שלא, לכן קיימת קבוצה מדידה $A$ שעבורה $\nu_s(A), \nu_a(A) > 0$.
	משאלה 1 סעיף א' נוכל לקחת את $\limsup T^{n}(A)$ ולכן נניח בלי הגבלת הכלליות ש־$A$ היא קבוצה $T$־אינווריאנטית.
	אם מתקיים $\nu(A) = 0$ אז נקבל סתירה להנחה ולכן נניח ש־$\nu(A^C) = 0$.
	נשים לב ש־$\nu_a  \ll \mu$ ולכן $\mu(E) = 0 \implies \nu_a(E) = 0$ ולכן באופן שקול גם $\nu_a(E) > 0 \implies \mu(E) > 0$ ונסיק ש־$\mu(A) > 0$.
	אבל $\nu_s \perp \mu$, כלומר קיימות $B, C$ מדידות וזרות כך ש־$\mu(B^C) = 0, \nu_s(C^C) = 0$, אבל $\mu(A) > 0$ ולכן $A \subseteq B$ ונסיק ש־$\nu_s(A) = 0$ בסתירה.
\end{proof}

\subquestion{}
נניח ש־$\mu$ ארגודית וש־$\nu = \nu_a$.
נגדיר את $h$ להיות נגזרת רדון־ניקודים $\frac{d \nu_a}{d \mu}$. \\
נראה שמתקיים $\int_A h\ d \mu = \int_A h \circ T\ d \mu$ לכל $A$ מדידה, ונסיק ש־$h =_{\mu} h \circ T$.
\begin{proof}
	הפונקציה $h$ מקיימת את התכונה $\nu_a(E) = \int_E h\ d \mu$.
	אבל מסעיף א' נובע ש־$\nu_a$ היא $T$־אינווריאנטית ולכן $\nu_a(E) = T_* \nu_a(E) = \nu_a(T^{-1}(E))$ ולכן,
	\[
		\int_E h\ d \mu
		= \nu_a(E)
		= \nu_a(T^{-1}(E))
		= \int_{T^{-1}(E)} h\ d \mu
		= \int_{T^{-1}(E)} h\ d T_* \mu
		= \int_E h \circ T\ d \mu
	\]
	כפי שרצינו.

	נניח בשלילה שלא $h =_{\mu} h \circ T$ ולכן $E = \{ x \in X \mid h(x) \ne h(T(x)) \} = \{ x \mid h(x) < h(T(x)) \} \uplus \{ x \mid h(x) > h(T(x)) \}$ היא קבוצה ממידה חיובית.
	נניח בלי הגבלת הכלליות ש־$E = \{ h(x) < h(T(x)) \}$ היא ממידה חיובית ולכן נקבל ש־$\int_E h \circ T - h\ d \mu > 0$ בסתירה, לכן $h =_{\mu} h \circ T$.
\end{proof}

\subquestion{}
נראה ש־$h =_{\mu} 1$ ונסיק ש־$\nu = \mu$.
\begin{proof}
	נזכור כי $h$ היא פונקציה מדידה, ובסעיף הקודם הראינו שגם $T$־אינווריאנטית כמעט תמיד, בלי הגבלת הכלליות תמיד על־ידי הגדרת הפונקציה,
	\[
		x \mapsto \begin{cases}
			x & h(x) = h(T(x)) \\
			0 & \text{otherwise}
		\end{cases}
	\]
	ולכן משאלה 1 סעיף ג' והעובדה ש־$\mu$ מידת הסתברות ארגודית נסיק ש־$h$ היא קבועה כמעט תמיד.
	נתון כי $1 = \nu(X) = \nu_a(X) = \int h\ d \mu = h \int 1\ d \mu = h \cdot 1$ ובהתאם $h =_{\mu} 1$.
	אז נקבל $\nu(E) = \nu_a(E) = \int_E h\ d \mu = \int_E 1\ d \mu = \mu(E)$ ובהתאם $\mu = \nu$.
\end{proof}

\question{}
\subquestion{}
תהיינה $f, g \in L^1(\RR)$ פונקציות אינטגרביליות לבג.
נגדיר,
\[
	\mu(E) = \int_E f\ d \lambda,
	\quad
	\nu(E) = \int_E g\ d \lambda,
\]
המידות המתקבלות כאינטגרל על הפונקציות הללו. \\
נמצא תנאי מספיק והכרחי לכך ש־$\mu \perp \nu$.
\begin{proof}
	מתקיים $\mu \perp \nu$ אם ורק אם $A, B \subseteq \RR$ מדידות לבג וזרות כך ש־$\mu(A^C) = \nu(B^C) = 0$.
	לכן בהתאם נקבל,
	\[
		0
		= \int_{A^C} f\ d \lambda
	\]
	אבל $f \ge 0$ ולכן $f \restriction A^C =_{\lambda} 0$.
	נסיק אם כך ש־$\supp f \subseteq A, \supp g \subseteq B$ ובפרט ש־$\supp f \cap \supp g = \emptyset$.

	מהצד השני נניח ש־$\supp f \cap \supp g = \emptyset$ ולכן נסמן את הקבוצות הללו בהתאמה $A, B$ ונקבל ש־$A \cap B = \emptyset$ וכן,
	\[
		\mu(A^C)
		= \int_{A^C} f\ d \mu
		= 0
	\]
	כלומר $\mu \perp \nu$ לפי הגדרה.
\end{proof}

\subquestion{}
נניח ש־$\mu, \nu$ מידות כך ש־$d \nu = h\ d \mu$ כאשר $h$ פונקציה מדידה חיובית ממש.
נראה ש־$\mu$ ו־$\nu$ שקולות.
\begin{proof}
	נזכיר כי $\mu, \nu$ שקולות כאשר $\forall E \in \Aa,\ \mu(E) = 0 \iff \nu(E) = 0$.
	מהגדרה מתקיים,
	\[
		\nu(E)
		= \int_E h\ d \mu
	\]
	נניח ש־$E \in \Aa$ מדידה כלשהי וכן ש־$\mu(E) = 0$.
	אז מתקיים גם $\int_E f\ d \mu = 0$ מהגדרת אינטגרל לבג ולכל $f$ מדידה, בפרט הטענה נכונה גם ל־$h$ ולכן $\nu(E) = 0$.

	נניח בכיוון השני ש־$\nu(E) = 0$ ונניח בשלילה ש־$\mu(E) > 0$, אז מחיוביות בהחלט של $h$ נקבל ש־$\int_E h\ d \mu > 0$ בסתירה.
\end{proof}

\question{}
תהיינה $\mu, \nu_1, \nu_2, \ldots$ מידות על $X$ ונגדיר $\nu = \sum_{i = 1}^\infty \nu_i$.

\subquestion{}
נראה שאם $\nu_i \perp \mu$ לכל $i \ge 1$ אז $\nu \perp \mu$.
\begin{proof}
	לכל $i \ge 1$ נגדיר $A_i, B_i$ מדידות זרות כך ש־$\mu(A_i^C) = \nu_i(B_i^C) = 0$.
	נגדיר $A = \bigcap_{i = 1}^\infty A_i$.
	אז $A$ מדידה אף היא וכן,
	\[
		\mu(A^C)
		= \mu(\bigcup_{i = 1}^\infty A_i^C)
		\le \sum_{i = 1}^\infty \mu(A_i^C) = 0
	\]
	ולכן $A, B_i$ הן קבוצות זרות ומדידות המעידות על $\mu \perp \nu_i$.
	נגדיר גם $B = \bigcup_{i = 1}^\infty B_i$ ולכן,
	\[
		\nu(B^C)
		= \sum_{i = 1}^\infty \nu_i(B^C)
		= \sum_{i = 1}^\infty \nu_i\left(\bigcap_{i = 1}^\infty B_i^C\right)
		\le \sum_{i = 1}^\infty \nu_i(B_i^C)
		\le \sum_{i = 1}^\infty 0
		= 0
	\]
	כלומר מצאנו ש־$\mu(A^C) = 0, \nu(B^C) = 0$ ונותר להראות ש־$A, B$ זרות כדי לקבל שהן מעידות על $\mu \perp \nu$, אבל זה נובע ישירות מהגדרתן כאיחודים וחיתוכים של קבוצות זרות.
\end{proof}

\subquestion{}
נראה שאם $v_i \ll \mu$ לכל $i \ge 1$ אז $\nu \ll \mu$.
\begin{proof}
	נתון כי $\mu(E) = 0 \implies \nu_i(E) = 0$ לכל $E$ מדידה.
	אז נובע גם $\nu(E) = \sum_{i = 1}^\infty \nu_i(E) = \sum_{i = 1}^\infty 0 = 0$ ובהתאם $\nu \ll \mu$.
\end{proof}

\end{document}
