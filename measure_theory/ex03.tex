\input{../article_base.tex}
\title{פתרון מטלה 3 --- תורת המידה, 80517}

\begin{document}
\maketitle
\maketitleprint[blue]

\question{}
נניח ש־$(X, \Aa, \mu)$ מרחב מידה.

\subquestion{}
נוכיח שאם $0 \le f \le g$ מדידות אז $\int_E f\ d \mu \le \int_E g\ d \mu$ לכל $E \in \Aa$.
\begin{proof}
	בלי הגבלת הכלליות $X = E$ אחרת נוכל לקחת $f \cdot \indicator_E, g \cdot \indicator_E$. \\
	מהגדרה מתקיים,
	\[
		\int f\ d \mu
		= \sup \{ \int s\ d \mu \mid s \le f, s \text{ is simple} \}
	\]
	אבל לכל $s$ כזאת נקבל ש־$s \le f \le g$, כלומר,
	\[
		\{ \int s\ d \mu \mid s \le f, s \text{ is simple} \}
		\subseteq \{ \int s\ d \mu \mid s \le g, s \text{ is simple} \}
	\]
	ולכן נקבל שגם,
	\[
		\sup \{ \int s\ d \mu \mid s \le f, s \text{ is simple} \}
		\le \sup \{ \int s\ d \mu \mid s \le g, s \text{ is simple} \}
	\]
	ובהתאם נסיק $\int f\ d \mu \le \int g\ d \mu$.
\end{proof}

\subquestion{}
נוכיח שאם $A \subseteq B$ מדידות ו־$0 \le f$ אז,
\[
	\int_A f\ d \mu
	\le \int_B f\ d \mu
\]
\begin{proof}
	נניח ש־$B = X$ שכן נוכל לקחת כמו בסעיף הקודם מציינים בהתאם. \\
	נניח ש־$s \le f$ פשוטה ולכן,
	\[
		s(x)
		= \sum_{n = 1}^N \alpha_n \indicator_{E_n}
	\]
	עבור $\alpha_n \in \RR^+, E_n \in \Aa$, ובהתאם להגדרה,
	\[
		\int_A s\ d \mu
		= \sum_{n = 1}^N \alpha_n \cdot \mu(E_n \cap A)
		\le \sum_{n = 1}^N \alpha_n \cdot \mu(E_n)
		= \int s\ d \mu
	\]
	ולכן נסיק שגם $\int_A f\ d \mu \le \int f\ d \mu$.
\end{proof}

\subquestion{}
יהי $0 \le c \le \infty$ ו־$0 \le f$, אז נראה ש־$\int_E cf\ d \mu = c \int_E f\ d \mu$.
\begin{proof}
	נניח ש־$E = X$ וכן תהי $s \le f$ פשוטה המסומנת כבסעיף הקודם.
	אז מתקיים,
	\[
		\int cs\ d \mu
		= \sum_{n = 1}^N c \alpha \cdot \mu(E_n)
		= c \sum_{n = 1}^N \alpha \cdot \mu(E_n)
		= c \int s\ d \mu
	\]
	כלומר התכונה מתקיימת לכל $s$ פשוטה, ובפרט נסיק,
	\begin{align*}
		\int cf\ d \mu
		& = \sup\{ \int cs\ d \mu \mid s \le f, s \text{ is simple} \} \\
		& = \sup\{ c \int s\ d \mu \mid s \le f, s \text{ is simple} \} \\
		& = c \sup\{ \int s\ d \mu \mid s \le f, s \text{ is simple} \} \\
		& = c \int f\ d \mu
	\end{align*}
	כלומר הטענה מתקיימת.
\end{proof}

\question{}
יהי $(X, \Aa, \mu)$ מרחב מידה ונניח ש־$N \subseteq X$ מוכלת בקבוצה ממידה אפס, וכן ש־$f : N^C \to \CC$ פונקציה.
נראה שאם $f_1, f_2 : X \to \CC$ הרחבות מדידות של $f$ אז מתקיים,
\[
	\int f_1\ d \mu
	= \int f_2\ d \mu
\]
\begin{proof}
	נניח בלי הגבלת הכלליות ש־$N \in \Aa$ ממידה אפס, כלומר נבחר את הקבוצה שמעידה ש־$N$ מוכלת בקבוצה כזו.
	נבחן גם את $f : N^C \to [0, \infty]$, שכן אם הטענה נכונה עבורה, אז מהגדרת האינטגרל על פונקציה מרוכבת, הטענה נובעת ישירות.
	נניח ש־$s \le f_1$ פשוטה כך ש־$s = \sum_{n = 1}^N \alpha_n \indicator_{E_n}$, אז מתקיים,
	\[
		\int s\ d \mu
		= \sum_{n = 1}^N \alpha_n \mu(E_n)
		= \sum_{n = 1}^N \alpha_n \mu((E_n \cap N^C) \cupdot (E_n \cap N))
		= \sum_{n = 1}^N \alpha_n \mu(E_n \cap N^C)
		= \int_{N^C} s\ d \mu
	\]
	ולכן נסיק שגם,
	\[
		\int f_1\ d \mu
		= \sup\{ \int s\ d \mu \mid s \le f_1, s \text{ is simple} \}
		= \sup\{ \int_{N^C} s\ d \mu \mid s \le f_1, s \text{ is simple} \}
		= \int_{N^C} f_1\ d \mu
	\]
	אבל נתון כי $f_1 \equiv f_2$ על $N^C$, ולכן נובע $\int f_1\ d \mu = \int f_2\ d \mu$.
\end{proof}

\question{}
נעסוק עתה בדחיפה קדימה של מידה.
יהי $(X, \Aa, \mu)$ מרחב מידה ונניח ש־$(Y, \Bb)$ מרחב מדיד.
ותהי $\rho : X \to Y$ העתקה מדידה.
נגדיר את הדחיפה קדימה של $\mu$ על־ידי $\rho$ על־ידי,
\[
	\forall E \in \Bb,\ \rho_* \mu(E) = \mu(\rho^{-1}(E))
\]

\subquestion{}
נראה ש־$\rho_* \mu$ היא מידה.
\begin{proof}
	נניח ש־${\{ E_n \}}_{n = 1}^\infty \subseteq \Bb$ סדרת מדידות זרות.
	אז מהגדרת תמונה הפוכה מתקיים $\rho^{-1}(\biguplus_{n = 1}^\infty E_n) = \biguplus_{n = 1}^\infty \rho^{-1}(E_n)$.
	מהגדרת $\mu$ כמידה נובע $\mu(\bigcup \rho^{-1}(E_n)) = \sum_{n = 1}^\infty \mu(\rho^{-1}(E_n))$, כלומר קיבלנו שבדיוק $\rho_* \mu(\bigcup E_n) = \sum_{n = 1}^\infty \rho_* \mu(E_n)$ כפי שרצינו.
\end{proof}

\subquestion{}
נראה שלכל $f : Y \to [0, \infty]$ מדידה מתקיים,
\[
	\int_X (f \circ \rho)\ d \mu
	= \int_Y f\ d \rho_* \mu
\]
\begin{proof}
	תהי $s \le f$ פשוטה ונניח ש־$s = \sum_{n = 1}^N \alpha_n \indicator_{E_n}$, אז מתקיים,
	\[
		\int_Y s\ d\ \rho_* \mu
		= \sum_{n = 1}^N \alpha_n \rho_* \mu(E_n)
		= \sum_{n = 1}^N \alpha_n \mu(\rho^{-1}(E_n))
		= \int_X (s \circ \rho)\ d \mu
	\]
	כלומר הטענה מתקיימת עבור פונקציות פשוטות, ולכן מהגדרת הסופרימום והאינטגרל נסיק שהטענה נכונה גם עבור $f$ מדידה כללית.
\end{proof}

\subquestion{}
נניח ש־$X = [0, 2 \pi]$ ו־$Y = S^1$ עם מרחבים מדידים בורל, ונניח שגם $\rho : X \to Y$ מוגדרת על־ידי $\rho(x) = e^{ix}$.
נניח שקיימת מידת לבג על $X$, מידה המחזירה עבור קטע את אורכו, ונסמן אותה ב־$\lambda$.
נתאר את $\rho_* \lambda$.
\begin{solution}
	המידה $\rho_* \lambda$ היא מידה על $S^1$, ונניח ש־$E \subseteq Y$ קבוצה קשירה מסילתית, אז $\rho^{-1}(E)$ הוא קטע ולכן $\lambda(\rho^{-1}(E))$ הוא אורך הקטע המושרה מהקבוצה.
	כלומר גאומטרית $\rho_* \lambda$ מחזירה עבור קשת במעגל היחידה את האורך שלה.
\end{solution}

\question{}
יהי $(X, \Aa, \mu)$ מרחב מידה סופי.
נראה שלכל $f : X \to \RR_+$ מדידה וחסומה מתקיים,
\[
	S_+ 
	= \int f\ d \mu
	= \inf\left\{ \int \varphi\ d \mu \mid f \le \varphi, \varphi \text{ is simple} \right\}
	= S_-
\]
\begin{proof}
	נבחין תחילה כי לכל $s \ge f$ פשוטה מתקיים מהנתון $0 \le \int s\ d \mu < \infty$ ולכן $S_-$ מוגדר ובפרט $0 \le S_- < \infty$. \\
	נרצה להראות ש־$S_+ \le S_-$.
	תהיינה $s \le f \le \varphi$ פשוטות.
	בלי הגבלת הכלליות קיימים ${\{ E_n \}}_{n = 1}^N \subseteq \Aa$ כך ש־$s \restriction E_n, \varphi \restriction E_n$ קבועות לכל $n$, אחרת נבחר את התחומים של שתי הפונקציות ונחתוך אותם חיתוכים סופיים.
	נסמן $s = \sum_{n = 1}^N \alpha_n \indicator_{E_n}, \varphi = \sum_{n = 1}^N \beta_n \indicator_{E_n}$, אז מתקיים,
	\[
		S_- - S_+
		= \int \varphi\ d \mu - \int s\ d \mu
		= \int \varphi - s\ d \mu
		= \int \sum_{n = 1}^N E_n (\beta_n - \alpha_n)\ d \mu
		= \sum_{n = 1}^N \mu(E_n) (\beta_n - \alpha_n)
		\ge 0
	\]
	ולכן נובע ש־$S_- \ge S_+$. \\
	נעבור להראות שגם $S_- \le S_+$.
	מטענה שקולה לקיום סדרת פשוטות מתכנסת נקודתית נסיק שקיימת סדרת פשוטות מונוטונית יורדת $\varphi_n : X \to \RR_+$ המתכנסת נקודתית ל־$f$.
	יהי $\varepsilon > 0$, אז מתקיים ש־$\{ n \in \NN \mid {\lVert f - \varphi_n \rVert}_{\infty} < \varepsilon \}$ אינסופית לכל $\varepsilon$, נבחין כי זהו ניסוח שקול של התכנסות נקודתית.
	בהתאם נובע גם שקיימת $n$ כך ש־$\lvert \int f\ d \mu - \int \varphi_n\ d \mu \rvert < \varepsilon$ ולכן בהכרח $\inf_{n \in \NN} \int \varphi_n\ d \mu \le S_+$ ובפרט $S_- = S_+$.
\end{proof}

\question{}
נמצא דוגמה למרחב מידה $(X, \Aa, \mu)$, פונקציה מדידה $f : X \to (0, \infty)$ וסדרת פונקציות אי־שליליות $f_n \to f$ כך שהסדרה $\int f_n\ d \mu$ לא מתכנסת ל־$\int f\ d \mu$.
\begin{solution}
	נגדיר את $X = [0, 1], \Aa = \Bb X$ וכן את המידה $\mu = \lambda$ מידת לבג שרנו מניחים שקיימת. \\
	נגדיר $f_n = n \cdot \indicator_{(0, \frac{1}{n})}$, וכן את $f = 0$.
	אז לכל $x \in X$ קיים $n \in \NN$ כך ש־$\frac{1}{n} < x$ ולכן לכל $m > n$ מתקיים $f_n(x) = 0$, נסיק ש־$f_n \to f$.
	בנוסף $f_n$ פשוטה לכל $n$ ומתקיים,
	\[
		\int f_n\ d \mu
		= n \cdot \frac{1}{n}
		= 1
	\]
	אבל $f$ פשוטה אף היא ומתקיים $\int f\ d \mu = 0 \ne 1$, ובפרט $1 \not\to 0$.
\end{solution}

\end{document}
